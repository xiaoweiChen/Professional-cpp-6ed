通过解决下面的练习,可以练习本章讨论的内容。所有练习的解决方案都可以在本书的网站\url{www.wiley.com/go/proc++6e}下载到源码。然而,若在练习中卡住了,在从网站上寻找解决方案之前,可以考虑先重读本章的部分内容,试着自己找到答案。

代码注释和编码风格是主观的,以下练习并没有一个完美的答案,网站上提供的解决方案提供了这些练习的许多可能正确答案之一。

\begin{itemize}
\item
\textbf{练习 3-1}: 第 1 章讨论了一个员工记录系统。该系统有一个数据库,数据库的一个成员函数是 displayCurrent()。下面是这个成员函数的实现,包括一些注释:


\begin{cpp}
void Database::displayCurrent() const // The displayCurrent() member function
{
    for (const auto& employee : m_employees) { // For each employee...
        if (employee.isHired()) { // If the employee is hired
            employee.display(); // Then display that employee
        }
    }
}
\end{cpp}

你觉得这些注释有什么不对吗?为什么?能提出更好的意见吗?


\item
\textbf{练习 3-2}: 第 1 章中的员工记录系统包含一个 Database 类。下面是该类的片段,其中只有三个成员函数。为这个代码片段添加 JavaDoc 风格的注释。查阅第 1 章,复习这些成员函数的确切功能。

\begin{cpp}
class Database
{
    public:
    Employee& addEmployee(const std::string& firstName,
    const std::string& lastName);
    Employee& getEmployee(int employeeNumber);
    Employee& getEmployee(const std::string& firstName,
    const std::string& lastName);
    // Remainder omitted...
};
\end{cpp}

\item
\textbf{练习 3-3}: 下面的类存在一些命名问题。能找出它们并建议更好的名称吗?

\begin{cpp}
class xrayController
{
    public:
        // Gets the active X-ray current in μA.
        double getCurrent() const;
        // Sets the current of the X-rays to the given current in μA.
        void setIt(double Val);
        // Sets the current to 0 μA.
        void 0Current();
        // Gets the X-ray source type.
        const std::string& getSourceType() const;
        // Sets the X-ray source type.
        void setSourceType(std::string_view _Type);
    private:
        double d; // The X-ray current in μA.
        std::string m_src__type; // The type of the X-ray source.
};
\end{cpp}

\item
\textbf{练习 3-4}: 给定下面的代码片段,请重新格式化这个片段三次:首先将大括号放在它们自己的行上,然后缩进大括号本身,最后对于单语句代码块,移除大括号。这个练习可以帮助你体验不同的格式化风格,以及它们对代码可读性的影响。

\begin{cpp}
Employee& Database::getEmployee(int employeeNumber)
{
    for (auto& employee : m_employees) {
        if (employee.getEmployeeNumber() == employeeNumber) {
            return employee;
        }
    }
    throw logic_error { "No employee found." };
}
\end{cpp}
\end{itemize}