The C++ language provides a number of stylistic tools without any formal guidelines on how to use them. Ultimately, any style convention is measured by how widely it is adopted and how much it benefits the readability of the code. When coding as part of a team, you should raise issues of style early in the process as part of the discussion of what language and tools to use.
The most important point about style is to appreciate that it is an important aspect of programming.

Teach yourself to check over the style of your code before you make it available to others. Recognize good style in the code you interact with, and adopt the conventions that you and your organization find useful.
To conclude this chapter, keep the following in mind:

\begin{center}
\textit
Always code as if the guy who ends up maintaining your code will be a violent psychopath who knows where you live. Code for readability.
\end{center}

\begin{flushright}
\textit
John F. Woods, sep 24, 1991, comp.lang.c++
\end{flushright}

This chapter concludes the first part of this book. The next part discusses software design on a high level.