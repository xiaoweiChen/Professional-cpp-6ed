
编程中,文档通常指源文件中包含的注释。注释是展示开发者在编写代码时的所思所想,可以表达一些代码无法表达的东西。

\mySubsubsection{3.2.1.}{为什么要写注释?}

写注释似乎是一个好方法,但各位是否停下来想过,为什么需要对代码进行注释?有时开发者承认注释的重要性,但没有完全理解注释为什么重要,其中有几个原因。

\mySamllsection{解释用法}

使用注释原因之一是,解释应如何与代码交互。通常,开发者能够仅基于函数的名称、返回值的类型,以及其参数的名称和类型来理解函数的作用,但并非所有的内容都能在代码中表达。函数的前置条件和后置条件(前置条件是客户端代码在调用函数之前必须满足的条件,后置条件是函数执行完毕后必须满足的条件),以及函数可能抛出的异常,那些需要在注释中解释的事情。在我看来,只有当注释真正有用,才应该添加注释。尽管如此,函数没有前置或后置条件的情况还是很罕见,是否需要为函数添加注释应由开发者决定。经验丰富的开发者在这方面不会有问题,但经验较少的开发者可能无法做出正确的决定。这就是为什么有些公司规定,模块或头文件中,至少每个公开可访问的函数或成员函数都应该有一个注释,解释其作用、参数是什么、返回什么值、需要满足哪些前置和后置条件,以及它可能抛出哪些异常。

注释提供了用自然语言陈述无法使用代码表达的机会。例如,C++中没有办法指示数据库对象的saveRecord()成员函数,若在调用openDatabase()之前使用,会抛出异常,而注释是注明这种限制的理想之地:

\begin{cpp}
// Throws:
//    DatabaseNotOpenedException if openDatabase() has not been called yet.
int saveRecord(Record& record);
\end{cpp}

saveRecord()成员函数接受对非const Record对象的引用。用户可能想知道为什么不是对const的引用,所以这需要在注释中解释:

\begin{cpp}
// Parameters:
//    record: If the given record doesn't yet have a database ID, then saveRecord()
//    modifies the record object to store the ID assigned by the database.
// Throws:
//    DatabaseNotOpenedException if openDatabase() has not been called yet.
int saveRecord(Record& record);
\end{cpp}

C++语言强制指定函数的返回类型,但不提供方法来说明返回值实际表示什么。saveRecord()的声明表明返回一个int(本节将进一步讨论一个糟糕的设计决策),但是使用者阅读该声明时,并不知道int是什么意思。一条注释就足以解释其含义:

\begin{cpp}
// Saves the given record to the database.
//
// Parameters:
//    record: If the given record doesn't yet have a database ID, then saveRecord()
//    modifies the record object to store the ID assigned by the database.
// Returns: int
//    An integer representing the ID of the saved record.
// Throws:
//    DatabaseNotOpenedException if openDatabase() has not been called yet.
int saveRecord(Record& record);
\end{cpp}

前面的注释以正式的方式记录了关于saveRecord()的所有信息,包括描述该成员函数作用的语句。一些公司要求这种正式和详尽的文档,但我不推荐一直使用这种注释风格。例如,第一行相当无用,因为函数的名称已经自解释了。参数的描述以及关于异常的注释很重要,所以这些肯定应该保留。

由于这个版本的saveRecord()返回一个通用的int类型,因此需要记录返回类型确切代表什么。而更好的设计是返回一个RecordID,而非一个普通的int,这样可以省去为返回类型添加任何注释的需求。RecordID可以是一个只有一个int数据成员的简单类,但它传达了更多信息,并且需要的话,可以添加更多的数据成员,所以下面是一个更好的saveRecord():

\begin{cpp}
// Parameters:
//    record: If the given record doesn't yet have a database ID, then saveRecord()
//    modifies the record object to store the ID assigned by the database.
// Throws:
//    DatabaseNotOpenedException if openDatabase() has not been called yet.
RecordID saveRecord(Record& record);
\end{cpp}

\begin{myNotic}{NOTE}
当公司的编码指南没有强制为函数编写正式的注释时,在写注释时请使用常识。根据函数名、返回类型,及其参数的名称和类型,只在注释中说明一些不明显的内容。
\end{myNotic}

有时,函数的参数和返回类型是通用的,可以用来传递各种信息,需要清楚地记录确切传递的是什么类型。Windows中的消息处理程序接受两个参数,LPARAM和WPARAM,并且可以返回一个LRESULT。这三个参数几乎可以用来传递任何东西,但不能改变它们的类型。通过使用类型转换,可以用来传递简单的整数或指向某些对象的指针。注释写起来可能像这样:

\begin{cpp}
// Parameters:
//    WPARAM wParam: (WPARAM)(int): An integer representing...
//    LPARAM lParam: (LPARAM)(string*): A string pointer representing...
// Returns: (LRESULT)(Record*)
//    nullptr in case of an error, otherwise a pointer to a Record object
//    representing...
LRESULT handleMessage(WPARAM wParam, LPARAM lParam);
\end{cpp}

公开文档应该描述代码的行为,而不是实现细节。行为包括输入、输出、错误条件和处理、预期用途和性能保证。描述生成单个随机数接口的公开文档,应说明这个接口不接受任何参数,返回一个在之前指定范围内的整数,并应列出当出现问题时可能抛出的所有异常。公开文档不应该解释实际生成数字的线性同余算法细节。针对用户的注释中提供过多的实现细节,可能是编写注释时最常见的错误了。

\mySamllsection{解释复杂的代码}

实际的代码中,良好的注释也同样重要。一个简单的用户处理输入,并将结果写入控制台的程序中,可能很容易阅读并理解。但在专业领域,经常需要编写在复杂算法,或者过于深奥而无法一目了然的代码。

看看下面的代码。它写得很好,但可能不会立即明白它在做什么。若以前见过这个算法,可能会认出它,但新手可能并不理解这些代码在干嘛。

\begin{cpp}
void sort(int data[], std::size_t size)
{
    for (int i { 1 }; i < size; ++i) {
        int element { data[i] };
        int j { i };
        while (j > 0 && data[j - 1] > element) {
            data[j] = data[j - 1];
            j--;
        }
        data[j] = element;
    }
}
\end{cpp}

一个好的方法是在注释中描述函数的参数、使用的算法以及(循环)不变量。不变量是在代码执行期间必须为true的条件,例如循环迭代。在下面修改后的函数中,顶部的注释解释了两个参数的含义,函数开始处的详细注释从高层次解释了算法,内联注释解释了可能令人困惑的代码:

\begin{cpp}
// Implements the "insertion sort" algorithm.
// data is an array containing the elements to be sorted.
// size contains the number of elements in the data array.
void sort(int data[], std::size_t size)
{
    // The insertion sort algorithm separates the array into two parts--the
    // sorted part and the unsorted part. Each element, starting at position
    // 1, is examined. Everything earlier in the array is in the sorted part,
    // so the algorithm shifts each element over until the correct position
    // is found to insert the current element. When the algorithm finishes
    // with the last element, the entire array is sorted.

    // Start at position 1 and examine each element.
    for (int i { 1 }; i < size; ++i) {
        // Loop invariant:
        //   All elements in the range 0 to i-1 (inclusive) are sorted.

        int element { data[i] };
        // j marks the position in the sorted part where element will be inserted.
        int j { i };
        // As long as the value in the slot before the current slot in the sorted
        // array is higher than element, shift values to the right to make room
        // for inserting element (hence the name, "insertion sort") in the correct
        // position.
        while (j > 0 && data[j - 1] > element) {
            // invariant: elements in the range j+1 to i are > element.
            data[j] = data[j - 1];
            // invariant: elements in the range j to i are > element.
            j--;
        }
        // At this point the current position in the sorted array
        // is *not* greater than the element, so this is its new position.
        data[j] = element;
    }
}
\end{cpp}

新的代码确实更加冗长,但是包含注释后,不熟悉排序算法的读者更有可能理解这段代码所做的事情。

\mySamllsection{传递元信息}

\begin{myWarning}{WARNING}
本节中提到的元信息都是过去实践的做法。这种元信息是强烈不推荐使用的,如第28章中讨论的,需要强制性使用版本控制系统。这样的解决方案提供了一个带修订日期、作者姓名的注释变更历史,若使用得当,还包括伴随每次修改的注释,包括对变更请求和错误报告的引用。应该分别提交每个变更请求或错误修复,并附上描述性注释。有了版本控制系统,就不需要在源代码文件中手动跟踪此类元信息。
\end{myWarning}

旧的历史代码库中,可能会遇到用于提供比代码本身更高层次信息的注释。这种元信息提供了关于代码创建的详细信息,不涉及其行为的具体细节,例如每个函数的原始作者、编写代码的日期、函数针对的具体功能、与代码行对应的错误编号、稍后重新检查代码中可能存在的问题的提醒、变更日志等。这里有一个例子:

\begin{cpp}
// Date       | Change
//------------+--------------------------------------------------
// 2001-04-13 | REQ #005: <marcg> Do not normalize values.
// 2001-04-17 | REQ #006: <marcg> use nullptr instead of NULL.

// Author: marcg
// Date: 110412
// Feature: PRD version 3, Feature 5.10
RecordID saveRecord(Record& record)
{
    if (!m_databaseOpen) { throw DatabaseNotOpenedException { }; }
    RecordID id { getDB()->saveRecord(record) };
    if (id != -1) { // Added to address bug #142 – jsmith 110428
        record.setId(id);
    }
    // TODO: What if setId() throws an exception? – akshayr 110501
    return id;
}
\end{cpp}

\mySamllsection{版权声明}

另一种类型的元信息是版权声明。一些公司要求在每个源文件的开头都有这样的版权声明,尽管自从1886年的伯尔尼公约以来,并不需要明确写出版权声明才能拥有版权。

\mySubsubsection{3.2.2.}{注释风格}

每个组织对代码注释都有不同的方法,有些组织会强制要求使用特定的风格,以便为文档提供共同的代码标准。其他时候,注释的数量和风格则由开发者自己决定。以下示例展示了注释代码的几种方法。

\mySamllsection{每行注释}

避免文档不足的一种方法是过度文档化,为每一行代码都包含一个注释。为每一行代码添加注释可确保写的每一样东西都有具体的原因。实际上,如此大规模的沉重注释很不方便,并且有些混乱,甚至是无趣的!例如:

\begin{cpp}
int result; // Declare an integer to hold the result.
result = doodad.getResult(); // Get the doodad's result.
if (result % 2 == 0) { // If the result modulo 2 is 0 ...
    logError(); // then log an error,
} else { // otherwise ...
    logSuccess(); // log success.
} // End if/else.
return result; // Return the result.
\end{cpp}

这段代码中的注释将每一行描述为一个易于阅读的英文故事。假设读者具有基本的C++技能,这些注释就完全没有必要。这些注释并没有为代码添加真正有用的信息,特别是这行代码:

\begin{cpp}
if (result % 2 == 0) { // If the result modulo 2 is 0 ...
\end{cpp}

这个注释只是代码的英文翻译,没有说明开发者为什么对结果使用模运算符并取值为2。下面将是一个更好的注释:

\begin{cpp}
if (result % 2 == 0) { // If the result is even ...
\end{cpp}

修改后的注释虽然对大多数开发者来说仍然相当明显,并提供了关于代码的有用信息。使用模运算符2,是因为代码需要检查结果是否为偶数。

更进一步的,若某些表达式做的事情不明显,建议将其转换为一个具有适当名称的函数。这使得代码自解释,在使用函数的地方不需要编写注释,并且产生了一块可重用的代码。例如,可以定义一个函数isEven():

\begin{cpp}
bool isEven(int value) { return value % 2 == 0; }
\end{cpp}

然后像这样使用,无需注释:

\begin{cpp}
if (isEven(result)) {
\end{cpp}

尽管过度注释可能会导致冗长和多余,但在代码难以理解的情况下,重度注释可能会有所帮助。以下代码也为每一行添加了注释,但这些注释实际上很有用:

\begin{cpp}
// Calculate the doodad. The start, end, and offset values come from the
// table on page 96 of the "Doodad API v1.6."
result = doodad.calculate(Start, End, Offset);
// To determine success or failure, we need to bitwise AND the result with
// the processor-specific mask (see "Doodad API v1.6," page 201).
result &= getProcessorMask();
// Set the user field value based on the "Marigold Formula."
// (see "Doodad API v1.6", page 136)
setUserField((result + MarigoldOffset) / MarigoldConstant + MarigoldConstant);
\end{cpp}

虽然这段代码脱离了上下文,但注释很好地解释了每一行的作用。没有这些注释,涉及 \& 和神秘“Marigold Formula”的计算将难以解读。

\begin{myNotic}{NOTE}
通常没有必要为每一行代码添加注释,若代码复杂这种程度,请不要只是将代码翻译成自然语言,而是说明代码所做事情的真实意图。
\end{myNotic}

\mySamllsection{前缀注释}

有的团队会决定以一个标准注释开始所有源文件,这是一个记录有关程序和特定文件的重要信息的机会。每个文件顶部可能包含的信息示例包括以下内容:

\begin{itemize}
\item
版权信息

\item
文件/类的简要描述

\item
不完整的功能 *

\item
已知的问题 *
\end{itemize}

* 这些项目通常由问题和特性跟踪系统处理(参见第30章)。

下面列出了一些永远不应该包含在注释中的信息的例子,这些信息应由版本控制系统处理(参见第28章)。

\begin{itemize}
\item
最后修改日期

\item
原作者

\item
更改日志

\item
文件所寻址的特性ID
\end{itemize}

下面是一个前缀注释的例子:

\begin{cpp}
// Implements the basic functionality of a watermelon. All units are expressed
// in terms of seeds per cubic centimeter. Watermelon theory is based on the
// white paper "Algorithms for Watermelon Processing."
//
// The following code is (c) copyright 2023, FruitSoft, Inc. ALL RIGHTS RESERVED
\end{cpp}

\mySamllsection{固定注释}

编写可以由外部文档构建器解析的标准格式注释是一种流行的编程实践。Java可以编写符合标准格式的注释,这使得一个名为JavaDoc的工具能够自动为项目创建超链接的文档。对于C++,通过Doxygen(可在\url{doxygen.org}获取)的免费工具可以解析注释来自动构建HTML文档、类图、UNIX手册和其他文档。Doxygen甚至能够识别并解析C++程序中的JavaDoc风格注释。下面的代码展示了Doxygen能够识别的JavaDoc注释:

\begin{cpp}
/**
* Implements the basic functionality of a watermelon
* TODO: Implement updated algorithms!
*/
export class Watermelon
{
    public:
        /**
         * @param initialSeeds The starting number of seeds, must be > 5.
         * @throws invalid_argument if initialSeeds <= 5.
         */
        Watermelon(std::size_t initialSeeds);
        /**
         * Computes the seed ratio, using the Marigold algorithm.
         * @param slow Whether or not to use long (slow) calculations.
         * @return The marigold ratio.
         */
        double calculateSeedRatio(bool slow);
};
\end{cpp}

Doxygen识别C++语法和特殊的注释指令,如@param、@return和@throws,以生成可定制的输出。图3.1展示了Doxygen生成的HTML类参考的一个例子。

即使使用工具自动生成文档,也应避免编写无用的注释。前面代码中的Watermelon构造函数,其注释省略了描述,只描述了参数和它抛出的异常。如下面的例子所示:

\begin{cpp}
/**
 * The Watermelon constructor.
 * @param initialSeeds The starting number of seeds, must be > 5.
 * @throws invalid_argument if initialSeeds <= 5.
 */
Watermelon(std::size_t initialSeeds);
\end{cpp}

\myGraphic{0.5}{content/part1/chapter3/images/1.png}{图3.1}

如图3.1所示的自动生成的文档在开发过程中很有帮助,允许开发人员浏览类及其关系的高级描述。团队可以轻松地自定义一个像 Doxygen 这样的工具,以适应相应的注释风格。理想情况下,团队会对文档进行每日构建。

\mySamllsection{临时注释}

大多数时候,仅根据需要使用注释。以下是在代码中使用注释的一些指南:

\begin{itemize}
\item
添加注释之前,首先考虑是否可以通过重写代码来使注释变得多余。通过重命名变量、函数和类;重新排序代码中的步骤;引入中间的命名变量等。

\item
试想其他人正在阅读您的代码时,若有些微妙之处不是很清晰,应该记录下来。

\item
不用初始化代码,版本控制系统会自动跟踪这类信息。

\item
若正在使用的API不是很明显,请在该API的文档中包含对其作用的说明。

\item
更新代码时更新注释,没有什么比有错误注释的代码更令人困惑的了。

\item
若使用注释将一个函数分成几个部分,请考虑是否可以将该函数分解为多个更小的函数。

\item
避免使用有冒犯性的文字,因为永远不知道有一天谁会看到这些代码。

\item
讲笑话和调侃通常无伤大雅,必要的时候请和你的经理商量一下是否要写这些东西。
\end{itemize}

\mySamllsection{自解释代码}

良好的代码通常不需要大量注释。若发现自己每行都在添加注释,考虑一下代码是否可以重写,以更好地与注释中的内容相匹配,为函数、参数、变量、类等使用描述性名称。正确使用const;若一个变量不应该修改,就将其标记为const。重新排序函数中的步骤,使其更清楚地知道它在做什么。引入中间的好名字变量,使算法更容易理解。记住C++是一种语言,其主要目的是告诉计算机做什么,但语言的语义也可以用来向读者解释其含义。

编写自解释代码的另一种方法是,将代码分解为更小的片段。这就是下一节的主题了。

\begin{myNotic}{NOTE}
好的代码自然易读,只需要注释来提供有用的信息。
\end{myNotic}



















