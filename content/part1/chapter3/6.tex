
许多编程团队会因为代码格式化的争论而分裂,甚至有些友谊会因此破裂。在我的大学时代,我有一个朋友就是因为if语句中空格的使用与一个同学发生了激烈的辩论,以至于有人路过时都要停下来确认是否一切安好。

若您的组织已经制定了代码格式化的标准,那么应该感到庆幸。您可能不喜欢他们制定的某些标准,但至少不必为此争论。

若组织中没有代码格式化的标准,我建议您引入这样的标准。标准化的编码指南确保能团队中的所有开发者遵循相同的命名约定、格式化规则等,这使得代码更加统一且易于理解。

有一些自动化的工具,可以在将代码提交到版本控制系统之前根据特定的规则格式化代码。一些集成开发环境(IDE)内置了这样的工具,例如,在保存文件时自动格式化代码。

若团队成员都以自己的方式编写代码,请尽量保持宽容。

一些方式仅仅是个人偏好而已,而其他一些则实际上会使得团队协作变得困难。

\mySubsubsection{3.6.1.}{关于花括号对齐的争论}

或许最具争议的要点是放置表示代码块的大括号的位置,大括号的使用有几种风格。本书中,大括号放在引领语句的同一行,类、函数或成员函数除外。这种风格在下面的代码中展示(并且贯穿全书):

\begin{cpp}
void someFunction()
{
    if (condition()) {
        println("condition was true");
    } else {
        println("condition was false");
    }
}
\end{cpp}

这种风格在节省垂直空间的同时,通过缩进显示代码块。一些开发者可能会辩称,在实际编码中保存垂直空间并不重要。下面展示了一个更冗长的风格:

\begin{cpp}
void someFunction()
{
    if (condition())
    {
        println("condition was true");
    }
    else
    {
        println("condition was false");
    }
}
\end{cpp}

有些开发者甚至会这样写代码:

\begin{cpp}
void someFunction()
{
    if (condition())
        {
            println("condition was true");
        }
    else
        {
            println("condition was false");
        }
}
\end{cpp}

另一个争论点是是否在单个语句周围加大括号,例如:

\begin{cpp}
void someFunction()
{
    if (condition())
        println("condition was true");
    else
        println("condition was false");
}
\end{cpp}

显然,我不会推荐任何特定的风格,因为我不想有人给我寄刀片。就我个人而言,我总是使用大括号,即使是对单个语句,这可以防止某些编写糟糕的 C 风格宏(参见第 11 章),并且在未来添加语句时更安全。

\begin{myNotic}{NOTE}
选择表示代码块的风格时,重要的考虑因素是您是否能够仅通过查看代码就能清楚地看到哪个代码块属于哪个条件。
\end{myNotic}

\mySubsubsection{3.6.2.}{关于空格和括号的讨论}

代码单行的格式化也可能是一个争议的源头。同样,我不会提倡特定的方法,但阅读本书的各位可能会遇到这里展示的几种风格。

本书中,我在任何关键字后使用空格,在任何运算符前后使用空格,在参数列表或调用中的每个逗号后使用空格,并使用括号来表明运算的顺序,如下所示:

\begin{cpp}
if (i == 2) {
    j = i + (k / m);
}
\end{cpp}

另一种风格,如下所示,将 if 关键字视为函数,关键字和左括号之间没有空格。同时,由于 if 语句中没有语义上的相关性,省略了用于表面运算顺序的括号。

\begin{cpp}
if( i == 2 ) {
    j = i + k / m;
}
\end{cpp}

这种差异很微妙,哪种更好取决于读者,但我不能绕过这个问题而不指出 ,if 不是一个函数。

\mySubsubsection{3.6.3.}{空格、制表符和换行符}

使用空格和制表符不仅仅是风格上的偏好。若团队没有就空格和制表符的使用达成一致,当开发者合作工作时,则会有大问题。最明显的是,当 Alice 使用四个空格的制表符缩进代码而 Bob 使用五个空格的制表符时,当他们一起工作在同一个文件上时,谁也无法正确显示代码。更糟糕的问题是,当 Bob 同时格式化代码以使用制表符时,Alice 正在编辑同一代码;许多版本控制系统可能无法合并 Alice 的更改。

大多数,编辑器都有可配置的空格和制表符设置。一些环境甚至可以适应代码的格式,或者总是以空格保存,即使使用 Tab 键进行编写。若有一个灵活的环境,就有更好的机会能够与他人的代码一起工作。只需记住,制表符和空格不同,因为制表符可以是任意长度,而空格始终是空格。

最后,并非所有平台都以相同的方式表示行结束。例如,Windows 使用 \verb|\|r\verb|\|n 作为行结束符,而基于 Linux 的平台通常使用 \verb|\|n。若您在公司中使用多个平台,则需要同意使用哪种行结束风格。这里,您的集成开发环境(IDE)很可能会配置需要的行结束风格,或使用自动化工具自动修复行结束符,例如,在将代码提交到版本控制系统时。

