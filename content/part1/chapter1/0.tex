\noindent
\textbf{WHAT’S IN THIS CHAPTER?}

\begin{itemize}
\item
A brief overview of the most important parts and syntax of the C++ language and the C++ Standard Library

\item
How to write a basic class

\item
How scope resolution works

\item
What uniform initialization is

\item
The use of const

\item
What pointers, references, exceptions, and type aliases are

\item
Basics of type inference
\end{itemize}

\noindent
\textbf{WILEY.COM DOWNLOADS FOR THIS CHAPTER}

Please note that all the code examples for this chapter are available as a part of the chapter’s code download on this book’s website at \url{www.wiley.com/go/proc++6e} on the Download Code tab.

The goal of this chapter is to cover briefly the most important parts of C++ so that you have a foundation of knowledge before embarking on the rest of this book. This chapter is not a comprehensive lesson in the C++ programming language or the Standard Library. Certain basic points, such as what a program is and what recursion is, are not covered. Esoteric points, such as the definition of a union, or the volatile keyword, are also omitted. Certain parts of the C language that are less relevant in C++ are also left out, as are parts of C++ that get in-depth coverage in later chapters.

This chapter aims to cover the parts of C++ that programmers encounter every day. For example, if you’re fairly new to C++ and don’t understand what a reference variable is, you’ll learn about that kind of variable here. You’ll also learn the basics of how to use the functionality available in the Standard Library, such as vector containers, optional values, string objects, and more. These modern constructs from the Standard Library are briefly introduced in this chapter so that they can be used throughout examples in this book from the beginning.

If you already have significant experience with C++, skim this chapter to make sure that there aren’t any fundamental parts of the language on which you need to brush up. If you’re new to C++, read this chapter carefully and make sure you understand the examples. If you need additional introductory information, consult the titles listed in Appendix B, “Annotated Bibliography.”








