
C++语言常视为“更好的C语言”或“C语言的超集”,其设计为面向对象的C,所以称之为“带类的C”。后来,C语言的许多令人烦恼和粗糙的问题得到了解决。因为C++是基于C的,若是一个经验丰富的C语言开发者,那肯定会在本节中看到的一些熟悉的语法。当然,这两种语言也有不同之处。C23标准规范文档的篇幅不到800页,而C++23标准规范文档的篇幅超过2000页。所以,若是一名C语言开发者,或是熟悉其他语言(如Java、C\#、Python等)的开发者,请留意自己不熟悉的语法!

\mySubsubsection{1.1.1.}{必不可少的“Hello, World”}

下面的代码是最简单的C++程序。若使用的是旧版本的C++,import  std;std::println()会无法识,将需要使用替代方案(稍后会讨论)。

\begin{cpp}
// 01_helloworld.cpp
import std;

int main()
{
    std::println("Hello, World!");
    return 0;
}
\end{cpp}

这段代码会在屏幕上输出“Hello, World!”。这是一个简单的程序,并展示了关于C++程序格式的相关概念:

\begin{itemize}
\item
注释

\item
导入模块

\item
main()函数

\item
屏幕输出

\item
从函数返回
\end{itemize}

这些概念将在接下来的章节中进行简要解释。

\mySamllsection{注释}

程序的第一行是注释,这是一条仅为开发者存在的信息,编译器会忽略它。C++中,有两种描述注释的方法。可以使用两个正斜杠来表示该行后面的内容是注释:

\begin{cpp}
// 01_helloworld.cpp
\end{cpp}

同样的行为(也就是说,没有)可以通过使用多行注释来实现。多行注释以/*开始,以*/结束。下面的代码展示了一个多行注释的作用:

\begin{cpp}
/* This is a multiline comment.
   The compiler will ignore it.
*/
\end{cpp}

注释将在第3章“编码风格”中详细介绍。

\mySamllsection{导入模块}

模块是C++20的四大新特性之一,取代了旧的以头文件为机制的方式。若想使用某个模块的功能,只需简单地导入该模块,可以通过导入声明来实现。从C++23开始,可以通过导入一个单一的标准命名模块来访问整个C++标准库,这个标准模块叫做std。Hello, World的第一行导入了这个标准模块:

\begin{cpp}
import std;
\end{cpp}

若没有导入该模块,将无法执行进行屏幕输出。

若没有C++23的标准命名模块支持,必须显式导入代码所需的头文件。由于标准库中有超过100个头文件,并不总是清楚需要导入哪个特定的头文件才能使用某个特定功能。作为一个参考,附录C“标准库头文件”中,列出了C++标准库的所有头文件,包括它们内容的简短描述。例如,“Hello, World”中,可以不导入标准命名模块std,只导入代码需要的头文件即可。这个例子中,代码只需要导入<print>来获取文本的输出功能。请注意,导入命名模块std时,不使用尖括号,但在导入单个头文件时,需要使用尖括号:

\begin{cpp}
import <print>;
\end{cpp}

因为本书是一本关于C++23的书,所以在所有地方都使用了模块。C++标准库提供的所有功能都包含在定义良好的头文件中。本书中的大多数示例只是导入std命名模块,而不是单个头文件,但会指出功能是哪个头文件提供的。

模块不仅限于标准库,我们也可以编写自己的模块来提供自定义类型和功能。

\begin{myNotic}{NOTE}
若正在使用的编译器尚未完全支持模块,可以将头文件导入声明替换为\#include预处理指令。
\end{myNotic}

\mySamllsection{编译器如何处理源码}

简而言之,构建C++程序需要三个步骤。从技术上讲,编译过程中还有更多的阶段,现阶段这个简化的视图就足够了。

\begin{enumerate}
\item
首先,代码通过预处理运行,预处理器识别有关代码的元信息并处理预处器指令,例如\#include指令。处理所有预处理指令的源文件称为\textbf{翻译单元}。

\item
接下来,所有翻译单元独立编译或翻译成机器可读的文件,其中对函数等的引用尚未定义。

\item
解析这些引用在最后阶段由链接器完成,其会将所有的目标文件链接到最终的可执行文件中。
\end{enumerate}

\CXXTwentythreeLogo{-40}{-50}
\begin{myNotic}{NOTE}
从C++23开始,标准规定C++编译器必须接受使用UTF-8编码的源代码文件。第21章“字符串本地化和正则表达式”讨论了不同的编码,包括UTF-8。我建议您配置工具链以使用UTF-8。这将提高您的文件在不同平台之间的可移植性,并允许源文件中使用非英文字符。

为了在Microsoft Visual C++中启用UTF-8支持,请在项目属性 -> 配置属性 -> C/C++ -> 命令行 下的附加选项设置中添加 /utf-8 选项。对于GCC,请使用命令行选项 -finput-charset=UTF-8。Clang默认认为所有文件都是UTF-8编码。
\end{myNotic}

\mySamllsection{预处理指令}

若编译器尚不支持模块,需要用\#include头文件来代替导入模块或头文件。也就是说,显式的导入声明,如import <print>;,需要用\#include预处理指令替换:

\begin{cpp}
#include <print>
\end{cpp}

预处理的指令以\#字符开头,例如\#include。\#include指令告诉预处理器从<print>头文件中获取所有内容并将其复制到当前文件中。<print>头文件提供了将文本打印到屏幕的功能。

第11章“模块、头文件和其他主题”对预处理指令进行了更详细的讨论。

\mySamllsection{main()函数}

当然,main()是程序的起始点。main()的返回类型是int,表示程序的结果状态。main()函数可以没有任何参数,或者两个参数:

\begin{cpp}
int main(int argc, char** argv)
\end{cpp}

argc给出传递给程序的参数数量,而argv包含这些参数。其中,argv[0]可能是程序名称,也可能是空字符串,所以不要依赖它;相反,应该使用特定于平台的函数来获取程序名称。另外,包含在argv中的实际参数,需要从索引1开始计算。

\CXXTwentythreeLogo{-40}{-50}

\mySamllsection{屏幕输出}

C++23前,可使用I/O流将文本输出到屏幕。流会在下一节和第13章“揭开I/O的神秘面纱”中进行了简要介绍。C++23引入了一个新的、更易于使用的机制来输出文本到屏幕,这在本书的每个代码片段中都使用到了:std::print()和println(),都在<print>中定义。

第2章“使用字符串和字符串视图”详细讨论了std::print()和println(),用于将字符串格式化和输出到屏幕上。其基本使用的方式非常直接,其最基本的形式中,println()可以用来打印一行文本,该文本输出后会自动换行:

\begin{cpp}
std::println("Hello, World!");
\end{cpp}

println()的第一个参数是字符串格式的,其中可以包含替换字段,这些字段将由作为第二个和后续参数传递的值替换。可以在每个字段旁边包含大括号\{\}来指示替换字段的位置。例如:

\begin{cpp}
std::println("There are {} ways I love you.", 219);
\end{cpp}

这个例子中,数字219将插入到字符串中,所以输出为:

\begin{shell}
There are 219 ways I love you
\end{shell}

可以根据需要设置任意多的替换字段,例如:

\begin{cpp}
std::println("{} + {} = {}", 2, 4, 6);
\end{cpp}

这个例子中,每个字段都是按顺序应用的,所以结果输出为:

\begin{shell}
2 + 4 = 6
\end{shell}

关于替换字段的格式还有很多要说的,但这是第2章的内容。

若使用print()而不是println(),输出的文本将不会以换行。

\mySamllsection{输入输出流}

若编译器还不支持C++23,在使用std::print()和println()函数,则必须使用I/O流重新实现。

输入输出流会在第13章中深入的介绍,但其基本概念很简单。可以将输出流想象成数据槽通道,投入其中的东西都会输出。std::cout是与用户控制台或标准输出对应的通道。还有其他的通道,包括std::cerr,输出到错误控制台。<{}<运算符将数据投入通道中。输出流允许多种类型的数据在单行代码中顺序发送到通道中。以下代码输出文本,然后是一个数字,最后是文本:

\begin{cpp}
std::cout << "There are " << 219 << " ways I love you." << std::endl;
\end{cpp}

从C++20开始,推荐使用std::format(),其定义在<format>中,对字符串进行格式化。format()函数使用与print()和println()相同的替换字段概念,这会在第2章中详细讨论。用其来重写前面的语句也很简单:

\begin{cpp}
std::cout << std::format("There are {} ways I love you.", 219) << std::endl;
\end{cpp}

若编译器尚不支持print()和println(),可以使用cout、format()和endl。假设有以下代码:


\begin{cpp}
std::println("{} + {} = {}", 2, 4, 6);
\end{cpp}

将println()替换为format(),将结果流式传输到cout,并添加endl输出:

\begin{cpp}
std::cout << std::format("{} + {} = {}", 2, 4, 6) << std::endl;
\end{cpp}

std::endl表示行结束。当输出流遇到std::endl时,会将输出到目前为止已发送到通道中的所有内容,并移动到下一行。表示行结束的另一种方法是使用\verb|\|n字符。\verb|\|n字符是一个转义字符,是一个换行字符。转义字符可以在任何带引号的文本字符串中使用。下表显示了最常见的几种:

% Please add the following required packages to your document preamble:
% \usepackage{longtable}
% Note: It may be necessary to compile the document several times to get a multi-page table to line up properly
\begin{longtable}{|l|l|}
\hline
\textbf{转义字符}         & \textbf{含义}                                             \\ \hline
\endfirsthead
%
\endhead
%
\textbackslash{}n                & 换行:将光标移到下一行的开头 \\ \hline
\textbackslash{}r &
\begin{tabular}[c]{@{}l@{}}回车:将光标移到当前行的开头,但不能换行\end{tabular} \\ \hline
\textbackslash{}t                & 制表符                                                          \\ \hline
\textbackslash{}\textbackslash{} & 反斜杠字符                                          \\ \hline
\textbackslash{}"                & 引号                                               \\ \hline
\end{longtable}

\begin{myWarning}{WARNING}
请记住,endl会将新行插入到流中,并将当前在其缓冲区中的所有内容输出到通道中。所以不建议过度使用endl,例如在循环中,因为这会影响性能。另一方面,在流中插入\textbackslash{}n也会插入新行,但不会自动刷新缓冲区。
\end{myWarning}

默认情况下,print()和println()将文本输出到标准输出控制台std::cout。也可以输出到错误控制台std::cerr:

\begin{cpp}
std::println(std::cerr, "Error: {}", 6);
\end{cpp}

流也可以用来接受来自用户的输入。最简单的方法是对输入流使用操作符>{}>。cin输入流可接受来自用户的键盘输入:

\begin{cpp}
import std;
int main()
{
    int value;
    std::cin >> value;
    std::println("You entered {}", value);
}
\end{cpp}

操作符>{}>在读取值后遇到空格字符时停止输入,所以不能使用操作符读取包含空格的文本。此外,用户输入可能很棘手,因为不知道用户将输入什么样的数据。第13章详细讨论了输入流,包括如何读取带有空格的文本。

若是C++新手,并且具有C背景,可能会想知道是否可以使用printf()和scanf()函数。虽然这些函数仍然可以在C++中使用,但强烈建议使用现代的print()、println()和format()函数和流库,其主要原因是因为printf()和scanf()系列函数不提供类型安全的保障。

\mySamllsection{从函数返回}

“Hello, World”程序的最后一行:

\begin{cpp}
return 0;
\end{cpp}

由于这是main()函数,从它返回将控制权返回给操作系统。当这样做时,其传递值0,这通常向操作系统发出信号,表明在执行程序时没有错误。对于错误情况,可以返回非零值。

main()中的return语句是可选的。若不写一个,编译器将隐式地添加一个return 0;。

\mySubsubsection{1.1.2.}{命名空间}

命名空间解决了不同代码段之间的命名冲突问题。例如,正在编写一些代码,其中包含一个名为foo()的函数。某天,决定开始使用第三方库,该库也有一个foo()函数,这时编译器无法知道代码中引用的是哪个版本的foo()。若不能更改库的函数名,就要更改自己的函数名,这是一件非常痛苦的事情。

这种情况下,命名空间可以发挥作用,因为可以定义定义名称的上下文。将代码放置在命名空间,请将其包含在命名空间块中:

\begin{cpp}
namespace mycode {
    void foo()
    {
        std::println("foo() called in the mycode namespace");
    }
}
\end{cpp}

将您自己的foo()放在命名空间mycode中,可以将其与第三方库提供的foo()函数隔离。要调用支持命名空间的foo()版本,请使用::将命名空间添加到函数名之前,也称为作用域解析操作符:

\begin{cpp}
mycode::foo(); // Calls the "foo" function in the "mycode" namespace
\end{cpp}

属于mycode命名空间块的代码都可以调用同一命名空间内的其他代码,而无需显式添加命名空间。这种隐式命名空间的做法,有助于提高代码的可读性。也可以使用using指令来消除命名空间的前缀。该指令告诉编译器,后续代码正在使用指定命名空间中的名称,所以下面的代码隐藏了命名空间:

\begin{cpp}
using namespace mycode;

int main()
{
    foo(); // Implies mycode::foo();
}
\end{cpp}

\begin{myNotic}{NOTE}
main()函数绝不能放在命名空间中,其必须在全局命名空间中。
\end{myNotic}

单个源文件可以包含多个using指令,但不要过度使用这种方式。极端情况下,若声明使用人类已知的所有所以空间,实际上就等于消除了命名空间!若使用包含相同名称的两个名称空间,将名称冲突会继续。知道代码在哪个名称空间中运行也很重要,这样就不会意外地调用错误的函数。

之前已经了解了命名空间的语法,在“Hello, World”程序中我们使用了它,其中println是在std名称空间中定义的名称。可以用using指令写“Hello, World”,如下所示:

\begin{cpp}
import std;

using namespace std;

int main()
{
    println("Hello, World!");
}
\end{cpp}

\begin{myNotic}{NOTE}
本书中的大多数代码片段都假定std命名空间有一个using指令,C++标准库中的所有内容都可以使用,而不需要用std::来限定。
\end{myNotic}

using声明可用于引用命名空间中的特定项。只想在std命名空间中使用非限定的print,可以使用下面的using声明:

\begin{cpp}
using std::print;
\end{cpp}

后续代码可以在不添加名称空间的情况下引用print,但std命名空间中的其他项,如println,仍然需要显式限定:

\begin{cpp}
using std::print;
print("Hello, ");
std::println("World!");
\end{cpp}

\begin{myWarning}{WARNING}
永远不要在全局作用域的头文件中放置using指令或using声明;否则,将强制每个包含头文件的源文件执行此操作。可以将其放在较小的作用域中(例如命名空间或类作用域中),即使在头文件中也是如此。把using指令或声明放在模块接口文件中也完全没问题,只要不导出。因为我认为模块使接口更容易理解,所以本书完全限定模块接口文件中的所有类型。模块接口文件和从模块导出将在后续章节中进行解释。
\end{myWarning}

\mySamllsection{嵌套命名空间}

嵌套命名空间是指一个命名空间在另一个命名空间内。每个命名空间由双冒号分隔。这里有一个例子:

\begin{cpp}
namespace MyLibraries::Networking::FTP {
    /* ... */
}
\end{cpp}

这种紧凑的语法在C++17之前不可用。若是编译器不支持C++17,可以使用如下方法:

\begin{cpp}
namespace MyLibraries {
    namespace Networking {
        namespace FTP {
            /* ... */
        }
    }
}
\end{cpp}

\mySamllsection{命名空间的别名}

命名命名空间别名可用于为另一个命名空间提供一个新的、可能更短的名称:

\begin{cpp}
namespace MyFTP = MyLibraries::Networking::FTP;
\end{cpp}

\mySubsubsection{1.1.3.}{字面量}

字面量用于在代码中编写数字或字符串,C++支持一些标准字面量。整数可以用以下字面值表示(示例表示相同的数字123):

\begin{itemize}
\item
十进制字面量, 123

\item
八进制字面量, 0173 (以0开头)

\item
十六进制字面量, 0x7B (以0x开头)

\item
二进制字面量, 0b1111011 (以0b开头)
\end{itemize}

\begin{myWarning}{WARNING}
永远不要在数字字面值前面放一个0,除非它是八进制字面量!
\end{myWarning}

C++中使用字面量的其他例子:

\begin{itemize}
\item
浮点值 (例如 3.14f)

\item
双精度浮点值 (例如 3.14)

\item
十六进制浮点字面值 (例如 0x3.ABCp-10 和 0Xb.cp12l)

\item
单个字符 (例如 'a')

\item
以零结尾的字符数组 (例如 "character array")
\end{itemize}

字面量可以带有后缀,例如3.14f中的f,以强制使用某种类型。本例中,3.14f的结果是float,而3.14的结果是double。

单引号字符可以用作数字字面量中的数字分隔符。例如:

\begin{itemize}
\item
23'456'789

\item
2'34'56'789

\item
0.123'456f
\end{itemize}

仅用空格分隔的多个字符串字面值,可以自动连接成单个字符串。例如:

\begin{cpp}
std::println("Hello, "
             "World!");
\end{cpp}

等价于:

\begin{cpp}
std::println("Hello, World!");
\end{cpp}

也可以自定义字面值,这是第15章“重载C++操作符”中一项高级特性。

\mySubsubsection{1.1.4.}{变量}

In C++, variables can be declared just about anywhere in your code and can be used anywhere in the current block below the line where they are declared. Variables can be declared without being given a value. These uninitialized variables generally end up with a semi-random value based on whatever is in memory at that time, and they are therefore the source of countless bugs. Variables in C++ can alternatively be assigned an initial value when they are declared. The code that follows shows both flavors of variable declaration, both using ints, which represent integer values:

\begin{cpp}
int uninitializedInt;
int initializedInt { 7 };
println("{} is a random value", uninitializedInt);
println("{} was assigned as an initial value", initializedInt);
\end{cpp}

\begin{myNotic}{NOTE}
Most compilers will issue a warning or an error when code is using uninitialized variables. Some compilers will generate code that will report an error at run time.
\end{myNotic}

The initializedInt variable is initialized using the uniform initialization syntax. You can also use the following assignment syntax for initializing variables:

\begin{cpp}
int initializedInt = 7;
\end{cpp}

Uniform initialization was introduced with the C++11 standard in 2011. It is recommended to use uniform initialization instead of the old assignment syntax, so that’s the syntax used in this book. The section “Uniform Initialization” later in this chapter goes deeper in on the benefits and why it is recommended.

Variables in C++ are strongly typed; that is, they always have a specific type. C++ comes with a whole set of built-in types that you can use out of the box. The following table shows the most common types:

% Please add the following required packages to your document preamble:
% \usepackage{longtable}
% Note: It may be necessary to compile the document several times to get a multi-page table to line up properly
\begin{longtable}{|l|l|l|}
\hline
\textbf{Type} &
\textbf{Description} &
\textbf{Usage} \\ \hline
\endfirsthead
%
\endhead
%
\begin{tabular}[c]{@{}l@{}}(signed) int\\ signed\end{tabular} &
\begin{tabular}[c]{@{}l@{}}Positive and negative integers;\\ the range depends on the\\ compiler (usually 4 bytes)\end{tabular} &
\begin{tabular}[c]{@{}l@{}}int i \{-7\};\\ signed int i \{-6\};\\ signed i \{-5\};\end{tabular} \\ \hline
(signed) short (int) &
Short integer (usually 2 bytes) &
\begin{tabular}[c]{@{}l@{}}short s \{13\};\\ short int s \{14\};\\ signed short s \{15\};\\ signed short int s \{16\};\end{tabular} \\ \hline
(signed) long (int) &
Long integer (usually 4 bytes) &
long l \{-7L\}; \\ \hline
(signed) long long (int) &
\begin{tabular}[c]{@{}l@{}}Long long integer; the range\\ depends on the compiler but\\ is at least the same as for long\\ (usually 8 bytes)\end{tabular} &
long long ll \{14LL\}; \\ \hline
\begin{tabular}[c]{@{}l@{}}unsigned (int)\\ unsigned short (int)\\ unsigned long (int)\\ unsigned long long (int)\end{tabular} &
\begin{tabular}[c]{@{}l@{}}Limits the preceding types to\\ values \textgreater{}= 0\end{tabular} &
\begin{tabular}[c]{@{}l@{}}unsigned int i \{2U\};\\ unsigned j \{5U\};\\ unsigned short s \{23U\};\\ unsigned long l \{54UL\};\\ unsigned long long ll \{140ULL\};\end{tabular} \\ \hline
float &
\begin{tabular}[c]{@{}l@{}}Single precision floating-point\\ numbers\end{tabular} &
float f \{7.2f\}; \\ \hline
double &
\begin{tabular}[c]{@{}l@{}}Double precision floating-point\\ numbers; precision is at least the\\ same as for float\end{tabular} &
double d \{7.2\}; \\ \hline
long double &
\begin{tabular}[c]{@{}l@{}}Long double precision floating\\ point numbers; precision is at\\ least the same as for double\end{tabular} &
long double d \{16.98L\}; \\ \hline
\begin{tabular}[c]{@{}l@{}}char\\ unsigned char\\ signed char\end{tabular} &
A single character &
char ch \{'m'\}; \\ \hline
\begin{tabular}[c]{@{}l@{}}char8\_t\\ char16\_t\\ char32\_t\end{tabular} &
\begin{tabular}[c]{@{}l@{}}A single n-bit UTF-n-encoded\\ Unicode character where n can\\ be 8, 16, or 32\end{tabular} &
\begin{tabular}[c]{@{}l@{}}char8\_t c8 \{u8'm'\};\\ char16\_t c16 \{u'm'\};\\ char32\_t c32 \{U'm'\};\end{tabular} \\ \hline
wchar\_t &
\begin{tabular}[c]{@{}l@{}}A single wide character; the size\\ depends on the compiler\end{tabular} &
wchar\_t w \{L'm'\}; \\ \hline
bool &
\begin{tabular}[c]{@{}l@{}}A Boolean type that can have\\ one of two values: true or false\end{tabular} &
bool b \{true\}; \\ \hline
\end{longtable}

The range of signed and unsigned integer and char types is as follows:

% Please add the following required packages to your document preamble:
% \usepackage{longtable}
% Note: It may be necessary to compile the document several times to get a multi-page table to line up properly
\begin{longtable}{|l|l|l|}
\hline
\textbf{Type}                                             & \textbf{Signed}                 & \textbf{Unsigned}  \\ \hline
\endfirsthead
%
\endhead
%
char                                                      & -128 to 127                     & 0 to 255           \\ \hline
\begin{tabular}[c]{@{}l@{}}2-byte\\ integers\end{tabular} & -32,768 to 32,767               & 0 to 65,535        \\ \hline
\begin{tabular}[c]{@{}l@{}}4-byte\\ integers\end{tabular} & -2,147,483,648 to 2,147,483,647 & 0 to 4,294,967,295 \\ \hline
\begin{tabular}[c]{@{}l@{}}8-byte\\ integers\end{tabular} &
\begin{tabular}[c]{@{}l@{}}-9,223,372,036,854,775,808 to\\ 9,223,372,036,854,775,807\end{tabular} &
0 to 18,446,744,073,709,551,615 \\ \hline
\end{longtable}

Type char is a different type compared to both the signed char and unsigned char types. It should be used only to represent characters. Depending on your compiler, it can be either signed or unsigned, so you should not rely on it being signed or unsigned.

The range and precision of floating-point types is discussed in the section “Floating-Point Numbers” later in this chapter.

Related to char, <cstddef> provides the std::byte type representing a single byte. Before C++17, a char or unsigned char was used to represent a byte, but those types make it look like you are working with characters. std::byte on the other hand clearly states your intention, that is, a single byte of memory. A byte can be initialized as follows:

\begin{cpp}
std::byte b { 42 };
\end{cpp}

\begin{myNotic}{NOTE}
C++ does not provide a basic string type. However, a standard implementation of a string is provided as part of the Standard Library, as briefly discussed later in this chapter and in detail in Chapter 2.
\end{myNotic}

\mySamllsection{Numerical Limits}

C++ provides a standard way to obtain information about numeric limits, such as the maximum possible value for an integer on the current platform. In C, you could access constants, such as INT\_MAX. While those are still available in C++, it’s recommended to use the std::numeric\_limits class template defined in <limits>. Class templates are discussed later in this book, but those details are not important to understand how to use numeric\_limits. For now, you just need to know that, since it is a class template, you have to specify the type you are interested in between a set of angle brackets. For example, to get numeric limits for integers, you write std::numeric\_limits<int>. Consult a Standard Library reference (see Appendix B) to learn exactly what kind of information you can query using numeric\_limits.

Here are a few examples:

\begin{cpp}
println("int:");
println("Max int value: {}", numeric_limits<int>::max());
println("Min int value: {}", numeric_limits<int>::min());
println("Lowest int value: {}", numeric_limits<int>::lowest());

println("\ndouble:");
println("Max double value: {}", numeric_limits<double>::max());
println("Min double value: {}", numeric_limits<double>::min());
println("Lowest double value: {}", numeric_limits<double>::lowest());
\end{cpp}

The output of this code snippet on my system is as follows:

\begin{shell}
int:
Max int value: 2147483647
Min int value: -2147483648
Lowest int value: -2147483648
double:
Max double value: 1.7976931348623157e+308
Min double value: 2.2250738585072014e-308
Lowest double value: -1.7976931348623157e+308
\end{shell}

Note the differences between min() and lowest(). For an integer, the minimum value equals the lowest value. However, for floating-point types, the minimum value is the smallest positive value that can be represented, while the lowest value is the most negative value representable, which equals -max().

\mySamllsection{Zero Initialization}

Variables can be initialized to zero with \{0\}, or with a zero initializer, \{\}. Zero initialization initializes primitive integer types (such as char, int, and so on) to zero, primitive floating-point types to 0.0, pointer types to nullptr, and constructs objects with the default constructor (discussed later).

Here is an example of zero initializing a float and an int:

\begin{cpp}
float myFloat {};
int myInt {};
\end{cpp}

\mySamllsection{Casting}

Variables can be converted to other types by casting them. For example, a float can be cast to an int. C++ provides three ways to explicitly change the type of a variable. The first method is a holdover from C; it is not recommended but, unfortunately, still commonly used. The second method is rarely used. The third method is the most verbose but is also the cleanest one and is therefore recommended.

\begin{cpp}
float myFloat { 3.14f };
int i1 { (int)myFloat }; // method 1
int i2 { int(myFloat) }; // method 2
int i3 { static_cast<int>(myFloat) }; // method 3
\end{cpp}

The resulting integer will be the value of the floating-point number with the fractional part truncated. Chapter 10, “Discovering Inheritance Techniques,” describes the different casting methods in more detail. In some contexts, variables can be automatically cast, or coerced. For example, a short can be automatically converted into a long because a long represents the same type of data with at least the same precision:

\begin{cpp}
long someLong { someShort }; // no explicit cast needed
\end{cpp}

When automatically casting variables, you need to be aware of the potential loss of data. For example, casting a float to an int throws away the fractional part of the number, and the resulting integer can even be completely wrong if the floating-point value represents a number bigger than the maximum representable integer value. Most compilers will issue a warning or even an error if you assign a float to an int without an explicit cast. If you are certain that the left-hand side type is fully compatible with the right-hand side type, it’s OK to cast implicitly.

\mySamllsection{Floating-Point Numbers}

Working with floating-point numbers can be more complicated than working with integral types. You need to keep a few things in mind. Calculations with floating-point values that are orders of magnitude different can cause errors. Furthermore, calculating the difference between two floating-point numbers that are almost identical will cause the loss of precision. Also keep in mind that a lot of decimal values cannot be represented exactly as floating-point numbers. However, going deeper in on the numerical problems with using floating-point numbers and how to write numerical stable floatingpoint algorithms is outside the scope of this book, as these topics warrant a whole book on their own.

There are several special floating-point numbers:

\begin{itemize}
\item
+/-infinity: Represents positive and negative infinity, for example the result of dividing a nonzero number by zero

\item
NaN: Abbreviation for not-a-number, for example the result of dividing zero by zero, a mathematically undefined result
\end{itemize}

To check whether a given floating-point number is not-a-number, use std::isnan(). To check for infinity, use std::isinf(). Both functions are defined in <cmath>.

To obtain one of these special floating-point values, use numeric\_limits, for example std::numeric\_limits<double>::infinity().

\CXXTwentythreeLogo{-40}{-50}

\mySamllsection{Extended Floating-Point Types}

As mentioned in the section on variables earlier, C++ provides the following standard floating-point types: float, double, and long double.

C++23 introduces the following extended floating-point types that have become popular in certain domains. Support for these is optional, and not all compilers provide these types.

% Please add the following required packages to your document preamble:
% \usepackage{longtable}
% Note: It may be necessary to compile the document several times to get a multi-page table to line up properly
\begin{longtable}{|l|l|l|}
\hline
\textbf{Type}    & \textbf{Description}                               & \textbf{Literal suffix} \\ \hline
\endfirsthead
%
\endhead
%
std::float16\_t  & 16-bit format from the IEEE 754 standard.          & F16 or f16              \\ \hline
std::float32\_t  & 32-bit format from the IEEE 754 standard.          & F32 or f32              \\ \hline
std::float64\_t  & 64-bit format from the IEEE 754 standard.          & F64 or f64              \\ \hline
std::float128\_t & 128-bit format from the IEEE 754 standard.         & F128 or f128            \\ \hline
std::bfloat16\_t & Brain floating point.
(Developed by Google Brain, an artificial intelligence group at Google. It is used in AI processors and supported in hardware on the latest NVIDIA GPUs.) Used in certain AI domains. & BF16 or bf16            \\ \hline
\end{longtable}

Most of the time, the standard types, float, double, and long double, are enough. From these, double should be your default type. Using float can trigger loss of precision, and, depending on your use case, this might or might not be acceptable.

\mySamllsection{Range and Accuracy of Floating-Point Types}

Floating-point types have a limited range and a limited precision. The following table gives detailed specifications of all standard and extended floating-point types supported by C++. However, the specifications of the standard types, float, double, and long double, are not specified exactly by the C++ standard. The standard says only that long double should have at least the same precision as double, and double should have at least the same precision as float. For these three types, the table shows values commonly used by compilers.

% Please add the following required packages to your document preamble:
% \usepackage{longtable}
% Note: It may be necessary to compile the document several times to get a multi-page table to line up properly
\begin{longtable}{|l|l|l|l|l|l|l|}
\hline
\textbf{Type} &
\textbf{Name} &
\textbf{Mantissa bits} &
\textbf{Decimal digits} &
\textbf{Exponent bit} &
\textbf{Min} &
\textbf{Max} \\ \hline
\endfirsthead
%
\endhead
%
float           & \begin{tabular}[c]{@{}l@{}}Single\\ precision\end{tabular}   & 24 & 7.22  & 8  & $1.18$x$10^{-38}$   & $3.40$x$10^{38}$   \\ \hline
double          & \begin{tabular}[c]{@{}l@{}}Double\\ precision\end{tabular}   & 53 & 15.95 & 11 & $2.23$x$10^{-308}$  & $1.80$x$10^{308}$  \\ \hline
long double     & \begin{tabular}[c]{@{}l@{}}Extended\\ precision\end{tabular} & 64 & 19.27 & 15 & $3.36$x$10^{-4932}$ & $1.19$x$10^{4932}$ \\ \hline
std::float16\_t & \begin{tabular}[c]{@{}l@{}}Half\\ precision\end{tabular}     & 11 & 3.31  & 5  & $6.10$x$10^{-5}$    & $65504$       \\ \hline
std::float32\_t & \begin{tabular}[c]{@{}l@{}}Single\\ precision\end{tabular}   & 24 & 7.22  & 8  & $1.18$x$10^{-38}$   & $3.40$x$10^{38}$   \\ \hline
std::float64\_t & \begin{tabular}[c]{@{}l@{}}Double\\ precision\end{tabular}   & 53 & 15.95 & 11 & $2.23$x$10^{-308}$  & $1.80$x$10^{308}$  \\ \hline
std::float128\_t &
\begin{tabular}[c]{@{}l@{}}Quadruple\\ precision\end{tabular} &
113 &
34.02 &
15 &
$3.36$x$10^{-4932}$ &
$1.19$x$10^{4932}$ \\ \hline
std::bfloat16\_t &
\begin{tabular}[c]{@{}l@{}}Brain\\ floating\\ point\end{tabular} &
8 &
2.41 &
8 &
$1.18$x$10^{-38}$ &
$3.40$x$10^{38}$ \\ \hline
\end{longtable}

\mySubsubsection{1.1.5.}{Operators}

What good is a variable if you don’t have a way to change it? The following table shows common operators used in C++ and sample code that makes use of them. Operators in C++ can be binary (operate on two expressions), unary (operate on a single expression), or even ternary (operate on three expressions). There is only one ternary operator in C++, and it is explained in the section “The Conditional Operator” later in this chapter. Furthermore, Chapter 15, “Overloading C++ Operators,” is reserved for operators and explains how you can add support for these operators to your own custom types.

% Please add the following required packages to your document preamble:
% \usepackage{longtable}
% Note: It may be necessary to compile the document several times to get a multi-page table to line up properly
\begin{longtable}{|l|l|l|}
\hline
\textbf{Operator} &
\textbf{Description} &
\textbf{Usage} \\ \hline
\endfirsthead
%
\endhead
%
= &
\begin{tabular}[c]{@{}l@{}}Binary operator to assign the value on the right to the\\ expression on the left.\end{tabular} &
\begin{tabular}[c]{@{}l@{}}int i;\\ i = 3;\\ int j;\\ j = i;\end{tabular} \\ \hline
! &
\begin{tabular}[c]{@{}l@{}}Unary operator to complement the true/false (non-0/0)\\ status of an expression.\end{tabular} &
\begin{tabular}[c]{@{}l@{}}bool b \{!true\};\\ bool b2 \{!b\};\end{tabular} \\ \hline
+ &
Binary operator for addition. &
\begin{tabular}[c]{@{}l@{}}int i \{3 + 2\};\\ int j \{i + 5\};\\ int k \{i + j\};\end{tabular} \\ \hline
\begin{tabular}[c]{@{}l@{}}-\\ *\\ /\end{tabular} &
Binary operators for subtraction, multiplication, and division. &
\begin{tabular}[c]{@{}l@{}}int i \{5 – 1\};\\ int j \{5 * 2\};\\ int k \{j / i\};\end{tabular} \\ \hline
\% &
\begin{tabular}[c]{@{}l@{}}Binary operator for the remainder of a division operation.\\ This is also referred to as the mod or modulo operator. For\\ example: 5\%2=1.\end{tabular} &
int rem \{5 \% 2\}; \\ \hline
++ &
\begin{tabular}[c]{@{}l@{}}Unary operator to increment an expression by 1. If the\\ operator occurs after the expression, or post-increment, the\\ result of the expression is the unincremented value. If the\\ operator occurs before the expression, or pre-increment, the\\ result of the expression is the new value.\end{tabular} &
\begin{tabular}[c]{@{}l@{}}i++;\\ ++i;\end{tabular} \\ \hline
-- &
Unary operator to decrement an expression by 1. &
\begin{tabular}[c]{@{}l@{}}i--;\\ --i;\end{tabular} \\ \hline
\begin{tabular}[c]{@{}l@{}}+=\\ -=\\ *=\\ /+\\ \%=\end{tabular} &
\begin{tabular}[c]{@{}l@{}}Shorthand syntax for:\\ i = i + (j);\\ i = i - (j);\\ i = i * (j);\\ i = i / (j);\\ i = i \% (j);\end{tabular} &
\begin{tabular}[c]{@{}l@{}}i += j;\\ i -= j;\\ i *= j;\\ i /= j;\\ i \%= j;\end{tabular} \\ \hline
\begin{tabular}[c]{@{}l@{}}\&\\ \&=\end{tabular} &
\begin{tabular}[c]{@{}l@{}}Takes the raw bits of one expression and performs a bitwise\\ AND with the other expression.\end{tabular} &
\begin{tabular}[c]{@{}l@{}}i = j \& k;\\ j \&= k;\end{tabular} \\ \hline
\begin{tabular}[c]{@{}l@{}}|\\ |=\end{tabular} &
\begin{tabular}[c]{@{}l@{}}Takes the raw bits of one expression and performs a bitwise\\ OR with the other expression.\end{tabular} &
\begin{tabular}[c]{@{}l@{}}i = j | k;\\ j |= k;\end{tabular} \\ \hline
\begin{tabular}[c]{@{}l@{}}\textless{}\textless\\ \textgreater{}\textgreater\\ \textless{}\textless{}=\\ \textgreater{}\textgreater{}=\end{tabular} &
\begin{tabular}[c]{@{}l@{}}Takes the raw bits of an expression and “shifts” each bit left\\ (\textless{}\textless{}) or right (\textgreater{}\textgreater{}) the specified number of places.\end{tabular} &
\begin{tabular}[c]{@{}l@{}}i = i \textless{}\textless 1;\\ i = i \textgreater{}\textgreater 4;\\ i \textless{}\textless{}= 1;\\ i \textgreater{}\textgreater{}= 4;\end{tabular} \\ \hline
\begin{tabular}[c]{@{}l@{}}\textasciicircum\\ \textasciicircum{}=\end{tabular} &
\begin{tabular}[c]{@{}l@{}}Performs a bitwise exclusive or, also called XOR operation,\\ on two expressions.\end{tabular} &
\begin{tabular}[c]{@{}l@{}}i = i \textasciicircum j;\\ i \textasciicircum{}= j;\end{tabular} \\ \hline
\end{longtable}

Operators of the form op=, e.g., +=, are called compound assignment operators.

When a binary operator is applied to two operands of different types, the compiler inserts an implicit conversion to convert one of them to the other before applying the operator. You can also use explicit conversions to convert one type to another using static\_cast().

For implicit conversions, the compiler has certain rules to decide which type it converts to which other type. For example, for a binary operation with a small integer type and a larger integer type, the smaller type will be converted to the larger one. However, the results might not always be as you would expect. Thus, I recommend being careful with implicit conversions and using explicit conversions to make sure the compiler does what you intend.

The following code snippet shows the most common variable types and operators in action. It also shows explicit conversions and explains why they are necessary. If you are unsure about how variables and operators work, try to figure out what the output of this program will be, and then run it to confirm your answer.

\begin{cpp}
int someInteger { 256 };
short someShort;
long someLong;
float someFloat;
double someDouble;

someInteger++;
someInteger *= 2;
// Conversion from larger integer type to smaller integer type
// can cause a warning or error, hence static_cast() is required.
someShort = static_cast<short>(someInteger);
someLong = someShort * 10000;
someFloat = someLong + 0.785f;
// To make sure the division is performed with double precision,
// someFloat is explicitly converted to double first.
someDouble = static_cast<double>(someFloat) / 100000;
println("{}", someDouble);
\end{cpp}

The C++ compiler has a recipe for the order in which expressions are evaluated. If you have a complicated expression with many operators, the order of execution may not be obvious. For that reason, it’s probably better to break up a complicated expression into several smaller expressions, or explicitly group subexpressions by using parentheses. For example, the following line of code might be confusing unless you happen to know the exact evaluation order of the operators:

\begin{cpp}
int i { 34 + 8 * 2 + 21 / 7 % 2 };
\end{cpp}

Adding parentheses makes it clear which operations are happening first:

\begin{cpp}
int i { 34 + (8 * 2) + ( (21 / 7) % 2 ) };
\end{cpp}

For those of you playing along at home, both approaches are equivalent and end up with i equal to 51. If you assumed that C++ evaluated expressions from left to right, your answer would have been 1. C++ evaluates /, *, and \% first (in left-to-right order), followed by addition and subtraction, then bitwise operators. Parentheses let you explicitly tell the compiler that a certain operation should be evaluated first.

Formally, the evaluation order of operators is expressed by their precedence. Operators with a higher precedence are executed before operators with a lower precedence. The following list shows the precedence of the operators from the previous table. Operators higher in the list have higher precedence and hence are executed before operators lower in the list.

\begin{itemize}
\item
++ \enspace −− (postfix)

\item
! \enspace ++ \enspace −− (prefix)

\item
* \enspace / \enspace \%

\item
+ \enspace −

\item
<{}< \enspace >{}>

\item
\&

\item
\textasciicircum

\item
|

\item
= \enspace += \enspace -= \enspace *= \enspace /= \enspace \%= \enspace \&= \enspace |= \enspace \textasciicircum= \enspace <{}<= \enspace >{}>=

\end{itemize}

This is only a selection of the available C++ operators. Chapter 15 gives a complete overview of all available operators, including their precedence.


\mySubsubsection{1.1.6.}{Enumerations}

An integer really represents a single value from a larger set of values—the entire sequence of integral numbers. Enumerations are types that let you define your own sequences so that you can declare variables with values in that sequence. For example, in a chess program, you could represent each piece as an int, with constants for the piece types, as shown in the following code. The integers representing the types are marked const to indicate that they can never change.

\begin{cpp}
const int PieceTypeKing { 0 };
const int PieceTypeQueen { 1 };
const int PieceTypeRook { 2 };
const int PieceTypePawn { 3 };
//etc.
int myPiece { PieceTypeKing };
\end{cpp}

This representation can become dangerous. Since a piece is just an int, what would happen if another programmer added code to increment the value of a piece? By adding 1, a king becomes a queen, which really makes no sense. Worse still, someone could come in and give a piece a value of -1, which has no corresponding constant.

Strongly typed enumerations solve these problems by tightly defining the range of values for a variable. The following code declares a new type, PieceType, which has four possible values, called enumerators, representing four of the chess pieces:

\begin{cpp}
enum class PieceType { King, Queen, Rook, Pawn };
\end{cpp}

This new type can be used as follows:

\begin{cpp}
PieceType piece { PieceType::King };
\end{cpp}

Behind the scenes, an enumeration is just an integer value. The underlying values for King, Queen, Rook, and Pawn are 0, 1, 2, and 3, respectively. It’s possible to specify the integer values for enumerators yourself. The syntax is as follows:

\begin{cpp}
enum class PieceType
{
    King = 1,
    Queen,
    Rook = 10,
    Pawn
};
\end{cpp}

If you do not assign a value to an enumerator, the compiler automatically assigns it a value that is the previous enumerator incremented by 1. If you do not assign a value to the first enumerator, the compiler assigns it the value 0. So, in this example, King has the integer value 1, Queen has the value 2 assigned by the compiler, Rook has the value 10, and Pawn has the value 11 assigned automatically by the compiler.

Even though enumerators are internally represented by integer values, they are not automatically converted to integers, which means the following is illegal:

\begin{cpp}
int underlyingValue { piece };
\end{cpp}

\CXXTwentythreeLogo{-40}{-35}

Starting with C++23, you can use std::to\_underlying(). For example:

\begin{cpp}
int underlyingValue { to_underlying(piece) };
\end{cpp}

By default, the underlying type of an enumerator is an integer, but this can be changed as follows:

\begin{cpp}
enum class PieceType : unsigned long
{
    King = 1,
    Queen,
    Rook = 10,
    Pawn
};
\end{cpp}

For an enum class, the enumerator names are not automatically exported to the enclosing scope. This means they cannot clash with other names already defined in the parent scope. As a result, different strongly typed enumerations can have enumerators with the same name. For example, the following two enumerations are perfectly legal:

\begin{cpp}
enum class State { Unknown, Started, Finished };
enum class Error { None, BadInput, DiskFull, Unknown };
\end{cpp}

A big benefit of this is that you can give short names to the enumerators, for example, Unknown instead of UnknownState and UnknownError. However, it also means that you either have to fully qualify enumerators, or use a using enum or using declaration. Here’s an example of a using enum declaration:

\begin{cpp}
using enum PieceType;
PieceType piece { King };
\end{cpp}

A using declaration can be used if you want to avoid having to fully qualify specific enumerators. For example, in the following code snippet, King can be used without full qualification, but other enumerators still need to be fully qualified:

\begin{cpp}
using PieceType::King;
PieceType piece { King };
piece = PieceType::Queen;
\end{cpp}

\begin{myWarning}{WARNING}
Even though C++ allows you to avoid fully qualifying enumerators, I recommend using this feature judiciously. At least try to minimize the scope of the using enum or using declaration because if this scope is too big, you risk reintroducing name clashes. The section on the switch statement later in this chapter shows a properly scoped use of a using enum declaration.
\end{myWarning}

\mySamllsection{Old-Style Enumerations}

New code should always use the strongly typed enumerations explained in the previous section. However, in legacy code bases, you might find old-style enumerations, also known as unscoped enumerations: enum instead of enum class. Here is the previous PieceType defined as an old-style enumeration:

\begin{cpp}
enum PieceType { PieceTypeKing, PieceTypeQueen, PieceTypeRook, PieceTypePawn };
\end{cpp}

The enumerators of such old-style enumerations are exported to the enclosing scope. This means that in the parent scope you can use the names of the enumerators without fully qualifying them, for example:

\begin{cpp}
PieceType myPiece { PieceTypeQueen };
\end{cpp}

This of course also means that they can clash with other names already defined in the parent scope resulting in a compilation error. Here’s an example:

\begin{cpp}
bool ok { false };
enum Status { error, ok };
\end{cpp}

This code snippet does not compile because the name ok is first defined to be a Boolean variable, and later the same name is used as the name of an enumerator. Visual C++ 2022 emits the following error:

\begin{shell}
error C2365: 'ok': redefinition; previous definition was 'data variable'
\end{shell}

Hence, you should make sure such old-style enumerations have enumerators with unique names, such as PieceTypeQueen, instead of simply Queen.

These old-style enumerations are not strongly typed, meaning they are not type safe. They are always interpreted as integers, and thus you can inadvertently compare enumerators from completely different enumerations, or pass an enumerator of the wrong enumeration to a function.

\begin{myWarning}{WARNING}
Always use strongly typed enum class enumerations instead of oldstyle, unscoped, type-unsafe enum enumerations.
\end{myWarning}

\mySubsubsection{1.1.7.}{Structs}

Structs let you encapsulate one or more existing types into a new type. The classic example of a struct is a database record. If you are building a personnel system to keep track of employee information, you might want to store the first initial, last initial, employee number, and salary for each employee. A struct that contains all of this information is shown in the employee.cppm module interface file that follows. This is your first self-written module in this book. Module interface files usually have .cppm as extension. The first line in the module interface file is a module declaration and states that this file is defining a module called employee. Furthermore, a module needs to explicitly state what it exports, i.e., what will be visible when this module is imported somewhere else. Exporting a type from a module is done with the export keyword in front of, for example, a struct.

\begin{cpp}
export module employee;

export struct Employee {
    char firstInitial;
    char lastInitial;
    int employeeNumber;
    int salary;
};
\end{cpp}

A variable declared with type Employee has all of these fields built in. The individual fields of a struct can be accessed by using the . operator. The example that follows creates and then outputs the record for an employee. Just as with the standard named module std, you don’t use angle brackets when importing custom modules.

\begin{cpp}
import std;
import employee; // Import our employee module

using namespace std;

int main()
{
    // Create and populate an employee.
    Employee anEmployee;
    anEmployee.firstInitial = 'J';
    anEmployee.lastInitial = 'D';
    anEmployee.employeeNumber = 42;
    anEmployee.salary = 80000;
    // Output the values of an employee.
    println("Employee: {}{}", anEmployee.firstInitial,
        anEmployee.lastInitial);
    println("Number: {}", anEmployee.employeeNumber);
    println("Salary: ${}", anEmployee.salary);
}
\end{cpp}

\mySubsubsection{1.1.8.}{Conditional Statements}

Conditional statements let you execute code based on whether something is true. As shown in the following sections, there are two main types of conditional statements in C++: if/else statements and switch statements.

\mySamllsection{if/else Statements}

The most common conditional statement is the if statement, which can be accompanied by an else. If the condition given inside the if statement is true, the line or block of code is executed. If not, execution continues with the else case if present or with the code following the conditional statement. The following code shows a cascading if statement, a fancy way of saying that the if statement has an else statement that in turn has another if statement, and so on:

\begin{cpp}
if (i > 4) {
    // Do something.
} else if (i > 2) {
    // Do something else.
} else {
    // Do something else.
}
\end{cpp}

The expression between the parentheses of an if statement must be a Boolean value or evaluate to a Boolean value. A value of 0 evaluates to false, while any non-zero value evaluates to true. For example, if(0) is equivalent to if(false). Logical evaluation operators, described later, provide ways of evaluating expressions to result in a true or false Boolean value.

\mySamllsection{Initializers for if Statements}

C++ allows you to include an initializer inside an if statement using the following syntax:

\begin{cpp}
if (<initializer>; <conditional_expression>) {
    <if_body>
} else if (<else_if_expression>) {
    <else_if_body>
} else {
    <else_body>
}
\end{cpp}

Any variable introduced in the <initializer> is available only in the <conditional\_expression>, in the <if\_body>, in all <else\_if\_expression>s and <else\_if\_body>s, and in the <else\_body>. Such variables are not available outside the if statement.

It is too early in this book to give a useful example of this feature, but here is an example of how it could be employed:

\begin{cpp}
if (Employee employee { getEmployee() }; employee.salary > 1000) { ... }
\end{cpp}

In this example, the initializer gets an employee by calling the getEmployee() function. Functions are discussed later in this chapter. The condition checks whether the salary of the retrieved employee exceeds 1000. Only in that case is the body of the if statement executed. More concrete examples will be given throughout this book.

\mySamllsection{switch Statements}

The switch statement is an alternate syntax for performing actions based on the value of an expression. In C++, the expression of a switch statement must be of an integral type, a type convertible to an integral type, an enumeration, or a strongly typed enumeration, and must be compared to constants. Each constant value represents a “case.” If the expression matches the case, the subsequent lines of code are executed until a break statement is reached. You can also provide a default case, which is matched if none of the other cases matches. The following pseudocode shows a common use of the switch statement:

\begin{cpp}
switch (menuItem) {
    case OpenMenuItem:
        // Code to open a file
        break;
    case SaveMenuItem:
        // Code to save a file
        break;
    default:
        // Code to give an error message
        break;
}
\end{cpp}

A switch statement can always be converted into if/else statements. The previous switch statement can be converted as follows:

\begin{cpp}
if (menuItem == OpenMenuItem) {
    // Code to open a file
} else if (menuItem == SaveMenuItem) {
    // Code to save a file
} else {
    // Code to give an error message
}
\end{cpp}

switch statements are generally used when you want to do something based on more than one specific value of an expression, as opposed to some test on the expression. In such a case, the switch statement avoids cascading if/else statements. If you need to inspect only one value, an if or if/else statement is fine.

Once a case expression matching the switch condition is found, all statements that follow it are executed until a break statement is reached. This execution continues even if another case expression is encountered, which is called fallthrough. In the following example, a single set of statements is executed for both Mode::Standard and Default. If mode is Custom, then value is first changed from 42 to 84, after which the Standard and Default statements are executed. In other words, the Custom case falls through until it eventually reaches a break statement or the end of the switch statement. This code snippet also shows a nice example of using a properly scoped using enum declaration to avoid having to write Mode::Custom, Mode::Standard, and Mode::Default for the different case labels.

\begin{cpp}
enum class Mode { Default, Custom, Standard };

int value { 42 };
Mode mode { /* ... */ };
switch (mode) {
    using enum Mode;

    case Custom:
        value = 84;
    case Standard:
    case Default:
        // Do something with value ...
        break;
}
\end{cpp}

Fallthrough can be a source of bugs, for example if you accidentally forget a break statement. Because of this, some compilers give a warning if a fallthrough is detected in a switch statement, unless the case is empty. In the previous example, no compiler will give a warning that the Standard case falls through to the Default case, but a compiler might give a warning for the Custom case fallthrough. To prevent this warning and to make it clear to a reader and the compiler that the fallthrough is intentional, you can use a [[fallthrough]] attribute as follows:

\begin{cpp}
switch (mode) {
    using enum Mode;

    case Custom:
        value = 84;
        [[fallthrough]];
    case Standard:
    case Default:
        // Do something with value ...
        break;
}
\end{cpp}

Surrounding the statements following a case expression with braces is often optional, but sometimes necessary, for example, when defining variables. Here is an example:

\begin{cpp}
switch (mode) {
    using enum Mode;
    case Custom:
        {
            int someVariable { 42 };
            value = someVariable * 2;
            [[fallthrough]];
        }
    case Standard:
    case Default:
        // Do something with value ...
        break;
}
\end{cpp}

When using a switch statement for enumerations, most compilers issue a warning when you don’t handle all different enumerators, either by explicitly writing cases for each enumerator or by writing cases for only a selection of the enumerators in combination with a default case. However, it’s recommended not to include a default case in a switch statement switching on enumerations. Instead, you should explicitly list all enumerators. The reason is that this makes the code less error prone for when you later add more enumerators to the enumeration. In that case, if you forget to add any new enumerator to specific switch statements, the compiler will issue a warning instead of silently handling the new enumerator using the default case.

\mySamllsection{Initializers for switch Statements}

Just as for if statements, you can use initializers with switch statements. The syntax is as follows:

\begin{cpp}
switch (<initializer>; <expression>) { <body> }
\end{cpp}

Any variables introduced in the <initializer> are available only in the <expression> and in the <body>. They are not available outside the switch statement.

\mySubsubsection{1.1.9.}{The Conditional Operator}

C++ has one operator that takes three arguments, known as a ternary operator. It is used as a shorthand conditional expression of the form “if [something] then [perform action], otherwise [perform some other action].” The conditional operator is represented by a ? and a :. The following code outputs “yes” if the variable i is greater than 2, and “no” otherwise:

\begin{cpp}
println("{}", (i > 2) ? "yes" : "no");
\end{cpp}

The parentheses around i > 2 are optional, so the following is equivalent:

\begin{cpp}
println("{}", i > 2 ? "yes" : "no");
\end{cpp}

The advantage of the conditional operator is that it is an expression, not a statement like the if and switch statements. Hence, a conditional operator can occur within almost any context. In the preceding example, the conditional operator is used within code that performs output. A convenient way to remember how the syntax is used is to treat the question mark as though the statement that comes before it really is a question. For example, “Is i greater than 2? If so, the result is ‘yes’; if not, the result is ‘no.’”

\mySubsubsection{1.1.10.}{Logical Evaluation Operators}

You have already seen a logical evaluation operator without a formal definition. The > operator compares two values. The result is true if the value on the left is greater than the value on the right. All logical evaluation operators follow this pattern—they all result in a true or false.

The following table shows common logical evaluation operators:

% Please add the following required packages to your document preamble:
% \usepackage{longtable}
% Note: It may be necessary to compile the document several times to get a multi-page table to line up properly
\begin{longtable}{|l|l|l|}
\hline
\textbf{op} &
\textbf{Description} &
\textbf{Usage} \\ \hline
\endfirsthead
%
\endhead
%
\begin{tabular}[c]{@{}l@{}}\textless\\ \textless{}=\\ \textgreater\\ \textgreater{}=\end{tabular} &
\begin{tabular}[c]{@{}l@{}}Determines if the left-hand side is less than, less\\ than or equal to, greater than, or greater than or\\ equal to the right-hand side.\end{tabular} &
\begin{tabular}[c]{@{}l@{}}if (i \textless 0) \{\\   print("i is negative");\\ \}\end{tabular} \\ \hline
== &
\begin{tabular}[c]{@{}l@{}}Determines if the left-hand side equals the\\ right-hand side. Don’t confuse this with the =\\ (assignment) operator!\end{tabular} &
\begin{tabular}[c]{@{}l@{}}if (i == 3) \{\\   print("i is 3");\\ \}\end{tabular} \\ \hline
!= &
\begin{tabular}[c]{@{}l@{}}Not equals. The result of the statement is true if\\ the left-hand side does not equal the right-hand\\ side.\end{tabular} &
\begin{tabular}[c]{@{}l@{}}if (i != 3) \{\\   print("i is not 3");\\ \}\end{tabular} \\ \hline
\textless{}=\textgreater{} &
\begin{tabular}[c]{@{}l@{}}Three-way comparison operator, also called the\\ spaceship operator. Explained in more detail in\\ the next section.\end{tabular} &
result = i \textless{}=\textgreater 0; \\ \hline
! &
\begin{tabular}[c]{@{}l@{}}Logical NOT. This complements the true/false\\ status of a Boolean expression. This is a unary\\ operator.\end{tabular} &
\begin{tabular}[c]{@{}l@{}}if (!bool1) \{\\   print("bool1 is false");\\ \}\end{tabular} \\ \hline
\&\& &
\begin{tabular}[c]{@{}l@{}}Logical AND. The result is true if both parts of\\ the expression are true.\end{tabular} &
\begin{tabular}[c]{@{}l@{}}if (bool1 \&\& bool2) \{\\   print("both are true");\\ \}\end{tabular} \\ \hline
|| &
\begin{tabular}[c]{@{}l@{}}Logical OR. The result is true if either part of\\ the expression is true.\end{tabular} &
\begin{tabular}[c]{@{}l@{}}if (bool1 || bool2) \{\\   print("at least one is true");\\ \}\end{tabular} \\ \hline
\end{longtable}

C++ uses short-circuit logic when evaluating logical expressions. That means that once the final result is certain, the rest of the expression won’t be evaluated. For example, if you are performing logical OR operations of several Boolean expressions, as shown in the following code, the result is known to be true as soon as one of them is found to be true. The rest won’t even be checked.

\begin{cpp}
bool result { bool1 || bool2 || (i > 7) || (27 / 13 % i + 1) < 2 };
\end{cpp}

In this example, if bool1 is found to be true, the entire expression must be true, so the other parts aren’t evaluated. In this way, the language saves your code from doing unnecessary work. It can, however, be a source of hard-to-find bugs if the later subexpressions in some way influence the state of the program (for example, by calling a separate function).

The following code shows a statement using \&\& that short-circuits after the second term because 0 always evaluates to false:

\begin{cpp}
bool result { bool1 && 0 && (i > 7) && !done };
\end{cpp}

Short-circuiting can be beneficial for performance. You can put less resource intensive tests first so that more expensive tests are not even executed when the logic short-circuits. It is also useful in the context of pointers to avoid parts of the expression to be executed when a pointer is not valid. Pointers and short-circuiting with pointers are discussed later in this chapter.

\mySubsubsection{1.1.11.}{Three-Way Comparisons}

The three-way comparison operator can be used to determine the order of two values. It is also called the spaceship operator because its sign, <=>, resembles a spaceship. With a single expression, it tells you whether a value is equal, less than, or greater than another value. Because it has to return more than just true or false, it cannot return a Boolean type. Instead, it returns an enumeration-like type, defined in <compare> in the std namespace. If the operands are integral types, the result is a strong ordering and can be one of the following:

\begin{itemize}
\item
strong\_ordering::less: First operand less than second

\item
strong\_ordering::greater: First operand greater than second

\item
strong\_ordering::equal: First operand equal to second
\end{itemize}

Here is an example of its use:

\begin{cpp}
int i { 11 };
strong_ordering result { i <=> 0 };
if (result == strong_ordering::less) { println("less"); }
if (result == strong_ordering::greater) { println("greater"); }
if (result == strong_ordering::equal) { println("equal"); }
\end{cpp}

Certain types don’t have a total ordering. For example, not-a-number floating-point values are never equal, less than, or greater than any other floating-point value. Thus, such comparisons result in a partial ordering:

\begin{itemize}
\item
partial\_ordering::less: First operand less than second

\item
partial\_ordering::greater: First operand greater than second

\item
partial\_ordering::equivalent: First operand equivalent to second, meaning !(a<b) \&\& !(b<a); for example, -0.0 is equivalent to +0.0, but they are not equal

\item
partial\_ordering::unordered: If one or both of the operands is not-a-number
\end{itemize}

If you really need a strong ordering of your floating-point values, e.g., if you know they are never not-a-number, you can use std::strong\_order(), which always produces an std::strong\_ordering result.

There is also a weak ordering, which is an additional ordering type that you can choose from to implement three-way comparisons for your own types. With a weak ordering, all values are ordered, i.e., there is no unordered result, but the ordering is not strong, meaning there can be non-equal values that are equivalent. An example is ordering strings with case-insensitive comparisons. In that case, the strings “Hello World” and “hello world” are certainly not equal, but they are equivalent. Here are the different results of a weak ordering:

\begin{itemize}
\item
weak\_ordering::less: First operand less than second

\item
weak\_ordering::greater: First operand greater than second

\item
weak\_ordering::equivalent: First operand equivalent to second
\end{itemize}

The three different types of ordering support certain implicit conversions. A strong\_ordering can be converted implicitly to a partial\_ordering or a weak\_ordering. A weak\_ordering can be converted implicitly to a partial\_ordering.

For primitive types, using the three-way comparison operator doesn’t gain you much compared to just performing individual comparisons using the ==, <, and > operators. However, it becomes useful with objects that are more expensive to compare. With the three-way comparison operator, such objects can be ordered with a single operator, instead of potentially having to call two individual comparison operators, triggering two expensive comparisons. Chapter 9, “Mastering Classes and Objects,” explains how to add support for three-way comparisons to your own types.

Finally, <compare> provides named comparison functions to interpret the result of an ordering. These functions are std::is\_eq(), is\_neq(), is\_lt(), is\_lteq(), is\_gt(), and is\_gteq() returning true if an ordering represents ==, !=, <, <=, >, or >= respectively, false otherwise. Here is an example:

\begin{cpp}
int i { 11 };
strong_ordering result { i <=> 0 };
if (is_lt(result)) { println("less"); }
if (is_gt(result)) { println("greater"); }
if (is_eq(result)) { println("equal"); }
\end{cpp}

\mySubsubsection{1.1.12.}{Functions}

For programs of any significant size, placing all the code inside of main() is unmanageable. To make programs easier to understand, you need to break up, or decompose, code into concise functions.

In C++, you first declare a function to make it available for other code to use. If the function is used only inside a particular file, you generally declare and define the function in that source file. If the function is for use by other modules or files, you export a declaration for the function from a module interface file, while the function’s definition can be either in the same module interface file or in a module implementation file (discussed later).

\begin{myNotic}{NOTE}
Function declarations are often called function prototypes or function headers to emphasize that they represent how the function can be accessed, but not the code behind it. The term function signature is used to denote the combination of the function name and its parameter list, but without the return type.
\end{myNotic}

A function declaration is shown in the following code. This example has a return type of void, indicating that the function does not provide a result to the caller. The caller must provide two arguments for the function to work with—an integer and a character.

\begin{cpp}
void myFunction(int i, char c);
\end{cpp}

Without an actual definition to match this function declaration, the link stage of the compilation process will fail because code that makes use of the function will be calling nonexistent code. The following definition prints the values of the two parameters:

\begin{cpp}
void myFunction(int i, char c)
{
    println("The value of i is {}.", i);
    println("The value of c is {}.", c);
}
\end{cpp}

Elsewhere in the program, you can make calls to myFunction() and pass in arguments for the two parameters. Some sample function calls are shown here:

\begin{cpp}
int someInt { 6 };
char someChar { 'c' };
myFunction(8, 'a');
myFunction(someInt, 'b');
myFunction(5, someChar);
\end{cpp}

\begin{myNotic}{NOTE}
In C++, unlike C, a function that takes no parameters just has an empty parameter list. It is not necessary to use void to indicate that no parameters are taken. However, you must still use void to indicate when no value is returned.
\end{myNotic}

C++ functions can also return a value to the caller. The following function adds two numbers and returns the result:

\begin{cpp}
int addNumbers(int number1, int number2)
{
    return number1 + number2;
}
\end{cpp}

This function can be called as follows:

\begin{cpp}
int sum { addNumbers(5, 3) };
\end{cpp}

\mySamllsection{Function Return Type Deduction}

You can ask the compiler to figure out the return type of a function automatically. To make use of this functionality, just specify auto as the return type.

\begin{cpp}
auto addNumbers(int number1, int number2)
{
    return number1 + number2;
}
\end{cpp}

The compiler deduces the return type based on the expressions used for the return statements in the body of the function. There can be multiple return statements, but they must all resolve to exactly the same type as the compiler will never insert any implicit conversions to deduce the return type of a function. Such a function can even include recursive calls (calls to itself), but the first return statement in the function must be a non-recursive call.

\mySamllsection{Current Function’s Name}

Every function has a local predefined variable \_\_func\_\_ containing the name of the current function. One use of this variable could be for logging purposes.

\begin{cpp}
int addNumbers(int number1, int number2)
{
    println("Entering function {}", __func__);
    return number1 + number2;
}
\end{cpp}

\mySamllsection{Function Overloading}

Overloading a function means providing several functions with the same name but with a different set of parameters. Only specifying different return types is not enough, as the returned value can be ignored when calling the function; instead, the number and/or types of the parameters must be different.

Suppose you want to provide versions of addNumbers() that work with integers and with doubles. Without overloading, you would have to come up with unique names, for example:

\begin{cpp}
int addNumbersInts(int a, int b) { return a + b; }
double addNumbersDoubles(double a, double b) { return a + b; }
\end{cpp}

With function overloading, you don’t need to come up with different names for the different versions of a function. The following code snippet defines two functions called addNumbers(), one defined for integers, the other defined for doubles:

\begin{cpp}
int addNumbers(int a, int b) { return a + b; }
double addNumbers(double a, double b) { return a + b; }
\end{cpp}

When calling addNumbers(), the compiler automatically selects the correct function overload based on the provided arguments. This process is called overload resolution.

\begin{cpp}
println("{}", addNumbers(1, 2)); // Calls the integer version
println("{}", addNumbers(1.11, 2.22)); // Calls the double version
\end{cpp}

\mySubsubsection{1.1.13.}{Attributes}

Attributes are a mechanism to add optional and/or vendor-specific information into source code. Before attributes were standardized in C++, vendors decided how to specify such information. Examples are \_\_attribute\_\_, \_\_declspec, and so on. Since C++11, there is standardized support for attributes by using the double square brackets syntax [[attribute]].

Earlier in this chapter, the [[fallthrough]] attribute is introduced to prevent a compiler warning when fallthrough in a switch case statement is intentional. The C++ standard defines a couple more standard attributes.

\mySamllsection{[[nodiscard]]}

The [[nodiscard]] attribute can be used on a function that returns a value. The compiler will then issue a warning if the return value from that function is not used by the calling function. Here is an example:

\begin{cpp}
[[nodiscard]] int func() { return 42; }

int main()
{
    func();
}
\end{cpp}

The compiler issues a warning similar to the following:

\begin{shell}
warning C4834: discarding return value of function with 'nodiscard' attribute
\end{shell}

This feature can, for example, be used for functions that return error codes. By adding the [[nodiscard]] attribute to such functions, the error codes returned from them cannot be ignored.

More general, the [[nodiscard]] attribute can be used on classes, structs, functions, and enumerations. An example of applying the attribute to an entire class is when you have a class representing error conditions. By applying [[nodiscard]] to such a class, the compiler will issue a warning for every function call that returns such an error condition and where the caller doesn’t do anything with it.

A reason can be provided for the [[nodiscard]] attribute in the form of a string. This reason is then displayed in the warning messages generated by the compiler if the returned value is ignored by the caller of the function. Here is an example:

\begin{cpp}
[[nodiscard("Some explanation")]] int func();
\end{cpp}

\mySamllsection{[[maybe\_unused]]}

The [[maybe\_unused]] attribute can be used to suppress the compiler from issuing a warning when something is unused, as in this example:

\begin{cpp}
int func(int param1, int param2)
{
    return 42;
}
\end{cpp}

If the compiler warning level is set high enough, this function definition results in two compiler warnings. For example, Microsoft Visual C++ gives these warnings:

\begin{shell}
warning C4100: 'param2': unreferenced formal parameter
warning C4100: 'param1': unreferenced formal parameter
\end{shell}

By using the [[maybe\_unused]] attribute, you can suppress such warnings:

\begin{cpp}
int func(int param1, [[maybe_unused]] int param2)
{
    return 42;
}
\end{cpp}

In this case, the second parameter is marked with the attribute suppressing its warning. The compiler now only issues a warning for param1:

\begin{shell}
warning C4100: 'param1': unreferenced formal parameter
\end{shell}

The [[maybe\_unused]] attribute can be used on classes, structs, non-static data members, unions, typedefs, type aliases, variables, functions, enumerations, and enumerators. Some of these terms you might not know yet but are discussed later in this book.

\mySamllsection{[[noreturn]]}

Adding a [[noreturn]] attribute to a function means that it never returns control to the caller. Typically, the function either causes some kind of termination (process termination or thread termination) or throws an exception. Exceptions are discussed later in this chapter. With this attribute, the compiler can avoid giving certain warnings or errors because it now knows more about the intent of the function. Here is an example:

\begin{cpp}
import std;
using namespace std;

[[noreturn]] void forceProgramTermination()
{
    exit(1); // Defined in <cstdlib>
}

bool isDongleAvailable()
{
    bool isAvailable { false };
    // Check whether a licensing dongle is available...
    return isAvailable;
}

bool isFeatureLicensed(int featureId)
{
    if (!isDongleAvailable()) {
        // No licensing dongle found, abort program execution!
        forceProgramTermination();
    } else {
        // Dongle available, perform license check of the given feature...
        bool isLicensed { featureId == 42 };
        return isLicensed;
    }
}

int main()
{
    bool isLicensed { isFeatureLicensed(42) };
    println("{}", isLicensed);
}
\end{cpp}

This code snippet compiles fine without any warnings or errors. However, if you remove the [[noreturn]] attribute, the compiler generates the following warning (output from Visual C++):

\begin{shell}
warning C4715: 'isFeatureLicensed': not all control paths return a value
\end{shell}

\mySamllsection{[[deprecated]]}

[[deprecated]] can be used to mark something as deprecated, which means you can still use it, but its use is discouraged. This attribute accepts an optional argument that can be used to explain the reason for the deprecation, as in this example:

\begin{cpp}
[[deprecated("Unsafe function, please use xyz")]] void func();
\end{cpp}

If you use this deprecated function, you’ll get a compilation error or warning. For example, GCC gives the following warning:

\begin{shell}
warning: 'void func()' is deprecated: Unsafe function, please use xyz
\end{shell}

\mySamllsection{[[likely]] and [[unlikely]]}

The likelihood attributes [[likely]] and [[unlikely]] can be used to help the compiler in optimizing code. These attributes can, for example, be used to mark branches of if and switch statements according to how likely it is that a branch will be taken. However, these attributes are rarely required. Compilers and hardware these days have powerful branch prediction to figure it out themselves, but in certain cases, such as performance critical code, you might have to help the compiler. The syntax is as follows:

\begin{cpp}
int value { /* ... */ };
if (value > 11) [[unlikely]] { /* Do something ... */ }
else { /* Do something else ... */ }

switch (value)
{
    [[likely]] case 1:
        // Do something ...
        break;
    case 2:
        // Do something ...
        break;
    [[unlikely]] case 12:
        // Do something ...
        break;
}
\end{cpp}

\CXXTwentythreeLogo{-40}{-50}

\mySamllsection{[[assume]]}

The [[assume]] attribute allows the compiler to assume that certain expressions are true without evaluating them at run time. The compiler can use such assumptions to better optimize the code. As an example, let’s look at the following function:

\begin{cpp}
int divideBy32(int x)
{
    return x / 32;
}
\end{cpp}

The function accepts a signed integer, so the compiler has to produce code to make sure the division works for both positive and negative numbers. If you are sure that x will never be negative, and for some reason you cannot make x of type unsigned, you can add an assumption as follows:

\begin{cpp}
int divideBy32(int x)
{
    [[assume(x >= 0)]];
    return x / 32;
}
\end{cpp}

With this assumption in place, the compiler can omit any code to handle negative numbers and optimize the division into a single instruction, a simple right shift of five bits.

\mySubsubsection{1.1.14.}{C-Style Arrays}

\begin{myWarning}{WARNING}
This section briefly explains C-style arrays, as you will encounter them in legacy code. However, in C++, it is best to avoid C-style arrays and instead use Standard Library functionality, such as std::array and vector, discussed in the following two sections.
\end{myWarning}

Arrays hold a series of values, all of the same type, each of which can be accessed by its position in the array. In C++, you must provide the size of the array when the array is declared. You cannot give a variable as the size—it must be a constant, or a constant expression (constexpr). Constant expressions are discussed in Chapter 9. The code that follows shows the declaration of an array of three integers followed by three lines to initialize the elements to 0:

\begin{cpp}
int myArray[3];
myArray[0] = 0;
myArray[1] = 0;
myArray[2] = 0;
\end{cpp}

\begin{myWarning}{WARNING}
In C++, the first element of an array is always at position 0, not position 1! The last position of the array is always the size of the array minus 1!
\end{myWarning}

The “Loops” section later in this chapter discusses how you could use loops to initialize each element of an array. However, instead of using loops or the previous initialization mechanism, you can also accomplish zero initialization with the following one-liner:

\begin{cpp}
int myArray[3] = { 0 };
\end{cpp}

You can even drop the 0.

\begin{cpp}
int myArray[3] = {};
\end{cpp}

Finally, the equal sign is optional as well, so you can write this:

\begin{cpp}
int myArray[3] {};
\end{cpp}

An array can be initialized with an initializer list, in which case the compiler deduces the size of the array automatically. Here’s an example:

\begin{cpp}
int myArray[] { 1, 2, 3, 4 }; // The compiler creates an array of 4 elements
\end{cpp}

If you do specify the size of the array and the initializer list has fewer elements than the given size, the remaining elements are set to 0. For example, the following code sets only the first element in the array to the value 2 and sets all others to 0:

\begin{cpp}
int myArray[3] { 2 };
\end{cpp}

To get the size of a stack-based C-style array, you can use the std::size() function, defined in <array>. It returns an std::size\_t, which is an unsigned integer type defined in <cstddef>. Here is an example:

\begin{cpp}
std::size_t arraySize { std::size(myArray) };
\end{cpp}

\begin{myNotic}{NOTE}
In legacy code, you might see size\_t being used without the std namespace qualification, without a using namespace std directive, and without a using std::size\_t declaration. This does not work any longer when you use import std, as that imports everything into the std namespace. Hence, you need either to use std::size\_t or to use a proper using directive or declaration. Alternatively, Chapter 11 explains that you can import the named module std.compat instead of std, but this is not recommended for new code.
\end{myNotic}

\CXXTwentythreeLogo{-40}{-50}

\begin{myNotic}{NOTE}
C++23 introduces a literal suffix uz for type std::size\_t, for example, 42uz.
\end{myNotic}

An older trick to get the size of a stack-based C-style array was to use the sizeof operator. The sizeof operator returns the size of its argument in bytes. To get the number of elements in a stackbased array, you divide the size in bytes of the array by the size in bytes of the first element. Here’s an example:

\begin{cpp}
std::size_t arraySize { sizeof(myArray) / sizeof(myArray[0]) };
\end{cpp}

The preceding examples show a one-dimensional array of integers, which you can think of as a line of integers, each with its own numbered compartment. C++ allows multidimensional arrays. You might think of a two-dimensional array as a checkerboard, where each location has a position along the x-axis and a position along the y-axis. Three-dimensional arrays can be pictured as a cube, while higher-dimensional arrays are harder to visualize. The following code shows the syntax for creating a two-dimensional array of characters for a tic-tac-toe board and then putting an “o” in the center square:

\begin{cpp}
char ticTacToeBoard[3][3];
ticTacToeBoard[1][1] = 'o';
\end{cpp}

Figure 1.1 shows a visual representation of this board with the position of each square.

\mySubsubsection{1.1.15.}{std::array}

The arrays discussed in the previous section come from C and still work in C++. However, C++ has a special type for fixed-size containers called std::array, defined in <array>. It’s basically a thin wrapper around C-style arrays.

\myGraphic{0.6}{content/part1/chapter1/images/1.png}{FIGURE 1.1}

There are a number of advantages to using std::arrays instead of C-style arrays. They always know their own size, are not automatically cast to a pointer to avoid certain types of bugs, and have iterators to easily loop over the elements. Iterators are discussed in detail in Chapter 17, “Understanding Iterators and the Ranges Library.” The following example demonstrates how to use the array container. The array type is a class template accepting a number of class template parameters that allow you to specify how many elements you want to store in the container and their type. You provide class template arguments for class template parameters by specifying them between the angle brackets after array, as in array<int,3>.

Chapter 12, “Writing Generic Code with Templates,” discusses templates in detail. However, for now, just remember that you have to specify two arguments between the angle brackets; the first represents the type of the elements in the array, and the second represents the size of the array.

\begin{cpp}
array<int, 3> arr { 9, 8, 7 };
println("Array size = {}", arr.size());
println("2nd element = {}", arr[1]);
\end{cpp}

C++ supports class template argument deduction (CTAD), as discussed in detail in Chapter 12. For now, it’s enough to remember that this allows you to avoid having to specify the template arguments between angle brackets for certain class templates. CTAD works only when using an initializer because the compiler uses this initializer to automatically deduce the template arguments. This works for std::array, allowing you to define the previous array as follows:

\begin{cpp}
array arr { 9, 8, 7 };
\end{cpp}

\begin{myNotic}{NOTE}
C-style arrays and std::arrays have a fixed size, which must be known at compile time. They cannot grow or shrink at run time.
\end{myNotic}

If you want an array with a dynamic size, it is recommended to use std::vector, as explained in the next section. A vector automatically increases in size when you add new elements to it.

\mySubsubsection{1.1.16.}{std::vector}

The C++ Standard Library provides a number of different non-fixed-size containers that can be used to store information. std::vector, declared in <vector>, is an example of such a container. The vector class replaces the concept of C-style arrays with a much more flexible and safer mechanism.

As a user, you need not worry about memory management, as a vector automatically allocates enough memory to hold its elements. A vector is dynamic, meaning that elements can be added and removed at run time. Chapter 18, “Standard Library Containers,” goes into more detail regarding containers, but the basic use of a vector is straightforward, which is why it’s introduced in the beginning of this book so that it can be used in examples. The following code demonstrates the basic functionality of vector:

\begin{cpp}
// Create a vector of integers.
vector<int> myVector { 11, 22 };

// Add some more integers to the vector using push_back().
myVector.push_back(33);
myVector.push_back(44);

// Access elements.
println("1st element: {}", myVector[0]);
\end{cpp}

myVector is declared as vector<int>. The angle brackets are required to specify the template arguments, just as with std::array. A vector is a generic container. It can contain almost any type of object, but all elements in a vector must be of the same type. This type is specified between the angle brackets. Templates are discussed in detail in Chapter 12 and Chapter 26, “Advanced Templates.”

Just as std::array, the vector class template supports CTAD, allowing you to define myVector as follows:

\begin{cpp}
vector myVector { 11, 22 };
\end{cpp}

Again, an initializer is required for CTAD to work. The following is illegal:

\begin{cpp}
vector myVector;
\end{cpp}

To add elements to a vector, you can use the push\_back() member function. Individual elements can be accessed using a similar syntax as for arrays, i.e., operator[].

\mySubsubsection{1.1.17.}{std::pair}

The std::pair class template is defined in <utility>. It groups together two values of possibly different types. The values are accessible through the first and second public data members. Here is an example:

\begin{cpp}
pair<double, int> myPair { 1.23, 5 };
println("{} {}", myPair.first, myPair.second);
\end{cpp}

pair also supports CTAD, so you can define myPair as follows:

\begin{cpp}
pair myPair { 1.23, 5 };
\end{cpp}

\begin{myNotic}{NOTE}
While you could write a function returning an std::pair, it is recommended to write a small struct or class containing the two values and return that from the function. The downside of returning a pair is that client code must use first and second to access the two values. By returning a proper struct or class, you can give more meaningful names to the two values.
\end{myNotic}

\mySubsubsection{1.1.18.}{std::optional}

std::optional, defined in <optional>, holds a value of a specific type, or nothing. It is introduced already in this first chapter as it is a useful type to use in some of the examples throughout the book.

Basically, optional can be used for parameters of a function if you want to allow for values to be optional. It is also often used as a return type from a function if the function can either return something or not. This removes the need to return “special” values from functions such as nullptr, -1, EOF, and so on. It also removes the need to write the function as returning a Boolean, representing success or failure, while storing the actual result of the function in an argument passed to the function as an output parameter (a parameter of type reference-to-non-const, discussed later in this chapter).

The optional type is a class template, so you have to specify the actual type that you need between angle brackets, as in optional<int>. This syntax is similar to how you specify the type stored in a vector, for example vector<int>.

Here is an example of a function returning an optional:

\begin{cpp}
optional<int> getData(bool giveIt)
{
    if (giveIt) {
        return 42;
    }
    return nullopt; // or simply return {};
}
\end{cpp}

You can call this function as follows:

\begin{cpp}
optional<int> data1 { getData(true) };
optional<int> data2 { getData(false) };
\end{cpp}

To determine whether an optional has a value, use the has\_value() member function, or simply use the optional in an if statement:

\begin{cpp}
println("data1.has_value = {}", data1.has_value());
if (!data2) {
    println("data2 has no value.");
}
\end{cpp}

If an optional has a value, you can retrieve it with value() or with the dereferencing operator *. This operator is discussed in detail later in this chapter in the context of pointers.

\begin{cpp}
println("data1.value = {}", data1.value());
println("data1.value = {}", *data1);
\end{cpp}

If you call value() on an empty optional, an std::bad\_optional\_access exception is thrown. Exceptions are introduced later in this chapter.

value\_or() can be used to return either the value of an optional or another value when the optional is empty:

\begin{cpp}
println("data2.value = {}", data2.value_or(0));
\end{cpp}

You cannot store a reference (discussed later in this chapter) in an optional, so optional<T\&> does not work. Instead, you can store a pointer in an optional.

\mySubsubsection{1.1.19.}{Structured Bindings}

A structured binding allows you to declare multiple variables that are initialized with elements from a data structure such as an array, struct, or pair.

Assume you have the following std::array:

\begin{cpp}
array values { 11, 22, 33 };
\end{cpp}

You can declare three variables, x, y, and z, initialized with the three values from the array as follows. You have to use the auto keyword for structured bindings, i.e., you cannot, for example, specify int instead of auto.

\begin{cpp}
auto [x, y, z] { values };
\end{cpp}

The number of variables declared with the structured binding has to match the number of values in the expression on the right.

Structured bindings also work with structs if all non-static members are public. Here’s an example:

\begin{cpp}
struct Point { double m_x, m_y, m_z; };
Point point;
point.m_x = 1.0; point.m_y = 2.0; point.m_z = 3.0;
auto [x, y, z] { point };
\end{cpp}

As a final example, the following code snippet decomposes the elements of a pair into separate variables:

\begin{cpp}
pair myPair { "hello", 5 };
auto [theString, theInt] { myPair }; // Decompose using structured bindings.
println("theString: {}", theString);
println("theInt: {}", theInt);
\end{cpp}

It is also possible to create a set of references-to-non-const or references-to-const using the structured bindings syntax, by using auto\& or const auto\& instead of auto. Both references-to-nonconst and references-to-const are discussed later in this chapter.

\mySubsubsection{1.1.20.}{Loops}

Computers are great for doing the same thing over and over. C++ provides four looping mechanisms: the while loop, do/while loop, for loop, and range-based for loop.

\mySamllsection{The while Loop}

The while loop lets you perform a block of code repeatedly as long as an expression evaluates to true. For example, the following completely silly code prints “This is silly.” five times:

\begin{cpp}
int i { 0 };
while (i < 5) {
    println("This is silly.");
    ++i;
}
\end{cpp}

The keyword break can be used within a loop to immediately get out of the loop and resume execution of the program starting at the line of code following the loop. The keyword continue can be used to return to the top of the loop and reevaluate the while expression. However, using continue in loops is often considered poor style because it causes the execution of a program to jump around somewhat haphazardly, so use it sparingly.

\mySamllsection{The do/while Loop}

C++ has a variation on the while loop called do/while. It works similarly to the while loop, except that the code to be executed comes first, and the conditional check for whether to continue happens at the end. In this way, you can use a loop when you want a block of code to always be executed at least once and possibly additional times based on some condition. The example that follows prints the statement, “This is silly.” once, even though the condition ends up being false:

\begin{cpp}
int i { 100 };
do {
    println("This is silly.");
    ++i;
} while (i < 5);
\end{cpp}

\mySamllsection{The for Loop}

The for loop provides another syntax for looping. Any for loop can be converted to a while loop, and vice versa. However, the for loop syntax is often more convenient because it looks at a loop in terms of a starting expression, an ending condition, and a statement to execute at the end of every iteration. In the following code, i is initialized to 0; the loop continues as long as i is less than 5; and at the end of every iteration, i is incremented by 1. This code does the same thing as the while loop example earlier but is more readable because the starting value, ending condition, and per-iteration statements are all visible on one line.

\begin{cpp}
for (int i { 0 }; i < 5; ++i) {
    println("This is silly.");
}
\end{cpp}

\mySamllsection{The Range-Based for Loop}

The range-based for loop is the fourth looping mechanism. It allows for easy iteration over elements of a container. This type of loop works for C-style arrays, initializer lists (discussed later in this chapter), and any type that supports begin() and end() functions returning iterators (see Chapter 17), such as std::array, vector, and all other Standard Library containers discussed in Chapter 18, “Standard Library Containers.”

The following example first defines an array of four integers. The range-based for loop then iterates over a copy of every element in this array and prints out each value. To iterate over the elements themselves without making copies, use a reference variable, as discussed later in this chapter.

\begin{cpp}
array arr { 1, 2, 3, 4 };
for (int i : arr) { println("{}", i); }
\end{cpp}

\mySamllsection{Initializers for Range-Based for Loops}

You can use initializers with range-based for loops, similar to initializers for if and switch statements. The syntax is as follows:

\begin{cpp}
for (<initializer>; <range-declaration> : <range-expression>) { <body> }
\end{cpp}

Any variables introduced in the <initializer> are available only in the <range-declaration>, the <range-expression> and in the <body>. They are not available outside the range-based for loop.
Here is an example:

\begin{cpp}
for (array arr { 1, 2, 3, 4 }; int i : arr) { println("{}", i); }
\end{cpp}

\mySubsubsection{1.1.21.}{Initializer Lists}

Initializer lists are defined in <initializer\_list> and make it easy to write functions that can accept a variable number of arguments. The std::initializer\_list type is a class template, and so it requires you to specify the type of elements in the list between angle brackets, similar to how you specify the type of elements stored in a vector. The following example shows how to use an initializer list:

\begin{cpp}
import std;
using namespace std;

int sum(initializer_list<int> values)
{
    int total { 0 };
    for (int value : values) {
        total += value;
    }
    return total;
}
\end{cpp}

By accepting an initializer list of integers as a parameter, the function sum() can be called with a braced initializer of integers as argument. The body of the function uses a range-based for loop to accumulate the total sum. This function can be used as follows:

\begin{cpp}
int a { sum({ 1, 2, 3 }) };
int b { sum({ 10, 20, 30, 40, 50, 60 }) };
\end{cpp}

Initializer lists are type safe. All elements in such a list must be of the same type. For the sum() function shown here, all elements of the initializer list must be integers. Trying to call it with a double, as shown next, results in a compilation error or warning stating that converting from double to int requires narrowing.

\begin{cpp}
int c { sum({ 1, 2, 3.0 }) };
\end{cpp}

\mySubsubsection{1.1.22.}{Strings in C++}

There are two ways to work with strings in C++:

\begin{itemize}
\item
The C style: Representing strings as arrays of characters

\item
The C++ style: Wrapping a C-style representation in an easier-to-use and safer string type
\end{itemize}

Chapter 2 provides a detailed discussion. For now, the only thing you need to know is that the C++ std::string type is defined in <string> and that you can use a C++ string almost like a basic type. The following example shows that strings can be used just like character arrays:

\begin{cpp}
string myString { "Hello, World" };
println("The value of myString is {}", myString);
println("The second letter is {}", myString[1]);
\end{cpp}

\mySubsubsection{1.1.23.}{C++ as an Object-Oriented Language}

If you are a C programmer, you may have viewed the features covered so far in this chapter as convenient additions to the C language. As the name C++ implies, in many ways the language is just a “better C.” There is one major point that this view overlooks: unlike C, C++ is an objectoriented language.

Object-oriented programming (OOP) is a different, arguably more natural, way to write code. If you are used to procedural languages such as C or Pascal, don’t worry. Chapter 5, “Designing with Classes,” covers all the background information you need to know to shift your mindset to the objectoriented paradigm. If you already know the theory of OOP, the rest of this section will get you up to speed (or refresh your memory) on basic C++ object syntax.

\mySamllsection{Defining Classes}

A class defines the characteristics of an object. In C++, classes are usually defined and exported from a module interface file (.cppm), while their definitions can either be directly in the same module interface file or in a corresponding module implementation file (.cpp). Chapter 11 discusses modules in depth.

A basic class definition for an airline ticket class is shown in the following example. The class can calculate the price of the ticket based on the number of miles in the flight and whether the customer is a member of the Elite Super Rewards Program.

The definition begins by declaring the class name. Inside a set of curly braces, the data members (properties) of the class and its member functions (behaviors) are declared. Each data member and member function is associated with a particular access level: public, protected, or private. These labels can occur in any order and can be repeated. Members that are public can be accessed from outside the class, while members that are private cannot be accessed from anywhere outside the class. Members that are protected can be accessed by derived classes, explained in detail in Chapter 10 in the context of inheritance. It’s recommended to make all your data members private, and if needed, to give access to them with public or protected getters to retrieve data from an object and public or protected setters to set data for an object. This way, you can easily change the representation of your data while keeping the public/protected interface the same.

Remember, when writing a module interface file, don’t forget to use an export module declaration to specify which module you are writing, and don’t forget to explicitly export the types you want to make available to users of your module.

\begin{cpp}
export module airline_ticket;

import std;

export class AirlineTicket
{
    public:
        AirlineTicket();
        ˜AirlineTicket();
        double calculatePriceInDollars();
        std::string getPassengerName();
        void setPassengerName(std::string name);
        int getNumberOfMiles();
        void setNumberOfMiles(int miles);
        bool hasEliteSuperRewardsStatus();
        void setHasEliteSuperRewardsStatus(bool status);
    private:
        std::string m_passengerName;
        int m_numberOfMiles;
        bool m_hasEliteSuperRewardsStatus;
};
\end{cpp}

This book follows the convention to prefix each data member of a class with a lowercase m followed by an underscore, such as m\_passengerName.

The member function that has the same name as the class with no return type is a constructor. It is automatically called when an object of the class is created. The member function with a tilde (~) character followed by the class name is a destructor. It is automatically called when an object is destroyed.

The .cppm module interface file defines the class, while the implementations of the member functions in this example are in a .cpp module implementation file. This source file starts with the following module declaration to tell the compiler that this is a source file for the airline\_ticket module:

\begin{cpp}
module airline_ticket;
\end{cpp}

There are several ways to initialize data members of a class. One way is to use a constructor initializer, which follows a colon after the constructor header. Here is the AirlineTicket constructor with a constructor initializer:

\begin{cpp}
AirlineTicket::AirlineTicket()
    : m_passengerName { "Unknown Passenger" }
    , m_numberOfMiles { 0 }
    , m_hasEliteSuperRewardsStatus { false }
{}
\end{cpp}

A second option is to put the initializations in the body of the constructor, as shown here:

\begin{cpp}
AirlineTicket::AirlineTicket()
{
    // Initialize data members.
    m_passengerName = "Unknown Passenger";
    m_numberOfMiles = 0;
    m_hasEliteSuperRewardsStatus = false;
}
\end{cpp}

However, if the constructor is only initializing data members without doing anything else, then there is actually no real need for a constructor because data members can be initialized directly inside a class definition, also known as in-class initializers. For example, instead of writing an AirlineTicket constructor, you can modify the data members in the class definition to initialize them as follows:

\begin{cpp}
private:
    std::string m_passengerName { "Unknown Passenger" };
    int m_numberOfMiles { 0 };
    bool m_hasEliteSuperRewardsStatus { false };
\end{cpp}

If your class additionally needs to perform some other types of initializations, such as opening a file, allocating memory, and so on, then you still need to write a constructor to handle those initializations.

Here is the destructor for the AirlineTicket class:

\begin{cpp}
AirlineTicket::~AirlineTicket()
{
    // Nothing to do in terms of cleanup
}
\end{cpp}

This destructor doesn’t do anything and can simply be removed from this class. It is just shown here so you know the syntax of destructors. Destructors are required if you need to perform some cleanup, such as closing files, freeing memory, and so on. Chapters 8, “Gaining Proficiency with Classes and Objects,” and 9 discuss destructors in more detail.

The definitions of the other AirlineTicket class member functions are shown here:

\begin{cpp}
double AirlineTicket::calculatePriceInDollars()
{
    if (hasEliteSuperRewardsStatus()) {
        // Elite Super Rewards customers fly for free!
        return 0;
    }
    // The cost of the ticket is the number of miles times 0.1.
    // Real airlines probably have a more complicated formula!
    return getNumberOfMiles() * 0.1;
}

string AirlineTicket::getPassengerName() { return m_passengerName; }
void AirlineTicket::setPassengerName(string name) { m_passengerName = name; }

int AirlineTicket::getNumberOfMiles() { return m_numberOfMiles; }
void AirlineTicket::setNumberOfMiles(int miles) { m_numberOfMiles = miles; }

bool AirlineTicket::hasEliteSuperRewardsStatus()
{
    return m_hasEliteSuperRewardsStatus;
}

void AirlineTicket::setHasEliteSuperRewardsStatus(bool status)
{
    m_hasEliteSuperRewardsStatus = status;
}
\end{cpp}

As mentioned in the beginning of this section, it’s also possible to put the member function implementations directly in the module interface file. The syntax is as follows:

\begin{cpp}
export class AirlineTicket
{
    public:
        double calculatePriceInDollars()
        {
            if (hasEliteSuperRewardsStatus()) { return 0; }
            return getNumberOfMiles() * 0.1;
        }

        std::string getPassengerName() { return m_passengerName; }
        void setPassengerName(std::string name) { m_passengerName = name; }

        int getNumberOfMiles() { return m_numberOfMiles; }
        void setNumberOfMiles(int miles) { m_numberOfMiles = miles; }

        bool hasEliteSuperRewardsStatus() { return m_hasEliteSuperRewardsStatus; }
        void setHasEliteSuperRewardsStatus(bool status)
        {
            m_hasEliteSuperRewardsStatus = status;
        }
    private:
        std::string m_passengerName { "Unknown Passenger" };
        int m_numberOfMiles { 0 };
        bool m_hasEliteSuperRewardsStatus { false };
};
\end{cpp}

\mySamllsection{Using Classes}

To use the AirlineTicket class, you first need to import its module:

\begin{cpp}
import airline_ticket;
\end{cpp}

The following sample program makes use of the class. This example shows the creation of a stackbased AirlineTicket object:

\begin{cpp}
AirlineTicket myTicket;
myTicket.setPassengerName("Sherman T. Socketwrench");
myTicket.setNumberOfMiles(700);
double cost { myTicket.calculatePriceInDollars() };
println("This ticket will cost ${}", cost);
\end{cpp}

The AirlineTicket example exposes you to the general syntax for creating and using classes. Of course, there is much more to learn, and that’s the topic of Chapters 8, 9, and 10.

\mySubsubsection{1.1.24.}{Scope Resolution}

As a C++ programmer, you need to familiarize yourself with the concept of a scope, which defines where an item is visible. Every name in your program, including variable, function, and class names, is in a certain scope. You create scopes with namespaces, function definitions, blocks delimited by curly braces, and class definitions. Variables that are initialized in the initialization statement of for loops and range-based for loops are scoped to that for loop and are not visible outside the for loop. Similarly, variables initialized in an initializer for if or switch statements are scoped to that if or switch statement and are not visible outside that statement. When you try to access a variable, function, or class, the name is first looked up in the nearest enclosing scope, then the parent scope, and so forth, up to the global scope. Any name not in a namespace, function, block delimited by curly braces, or class is assumed to be in the global scope. If it is not found in the global scope, at that point the compiler generates an undefined symbol error.

Sometimes names in scopes hide identical names in other scopes. Other times, the scope you want is not part of the default scope resolution from that particular line in the program. If you don’t want the default scope resolution for a name, you can qualify the name with a specific scope using the scope resolution operator ::. The following example demonstrates this. The example defines a class Demo with a get() member function, a get() function that is globally scoped, and a get() function that is in the NS namespace.

\begin{cpp}
class Demo
{
    public:
    int get() { return 5; }
};

int get() { return 10; }

namespace NS
{
    int get() { return 20; }
}
\end{cpp}

The global scope is unnamed, but you can access it specifically by using the scope resolution operator by itself (with no name prefix). The different get() functions can be called as follows. In this example, the code itself is in the main() function, which is always in the global scope:

\begin{cpp}
int main()
{
    Demo d;
    println("{}", d.get()); // prints 5
    println("{}", NS::get()); // prints 20
    println("{}", ::get()); // prints 10
    println("{}", get()); // prints 10
}
\end{cpp}

If the earlier namespace called NS is defined as an unnamed / anonymous namespace, that is, a namespace without a name as follows:

\begin{cpp}
namespace
{
    int get() { return 20; }
}
\end{cpp}

then the following line will cause a compilation error about ambiguous name resolution because you would have a get() defined in the global scope, and another get() defined in the unnamed namespace.

\begin{cpp}
println("{}", get());
\end{cpp}

The same error occurs if you add the following using directive right before the main() function:

\begin{cpp}
using namespace NS;
\end{cpp}

\mySubsubsection{1.1.25.}{Uniform Initialization}

Before C++11, initialization of types was not always uniform. For example, take the following definitions of a circle, once as a structure, and once as a class:

\begin{cpp}
struct CircleStruct
{
    int x, y;
    double radius;
};

class CircleClass
{
    public:
        CircleClass(int x, int y, double radius)
            : m_x { x }, m_y { y }, m_radius { radius } {}
    private:
        int m_x, m_y;
        double m_radius;
};
\end{cpp}

In pre-C++11, initialization of a variable of type CircleStruct and a variable of type CircleClass looked different:

\begin{cpp}
CircleStruct myCircle1 = { 10, 10, 2.5 };
CircleClass myCircle2(10, 10, 2.5);
\end{cpp}

For the structure version, you can use the \{...\} syntax. However, for the class version, you needed to call the constructor using function notation: (...).

Since C++11, you can more uniformly use the \{...\} syntax to initialize types, as follows:

\begin{cpp}
CircleStruct myCircle3 = { 10, 10, 2.5 };
CircleClass myCircle4 = { 10, 10, 2.5 };
\end{cpp}

The definition of myCircle4 automatically calls the constructor of CircleClass. Even the use of the equal sign is optional, so the following are identical:

\begin{cpp}
CircleStruct myCircle5 { 10, 10, 2.5 };
CircleClass myCircle6 { 10, 10, 2.5 };
\end{cpp}

As another example, in the section “Structs” earlier in this chapter, an Employee structure is initialized as follows:

\begin{cpp}
Employee anEmployee;
anEmployee.firstInitial = 'J';
anEmployee.lastInitial = 'D';
anEmployee.employeeNumber = 42;
anEmployee.salary = 80'000;
\end{cpp}

With uniform initialization, this can be rewritten as follows:

\begin{cpp}
Employee anEmployee { 'J', 'D', 42, 80'000 };
\end{cpp}

Uniform initialization is not limited to structures and classes. You can use it to initialize almost anything in C++. For example, the following code initializes all four variables with the value 3:

\begin{cpp}
int a = 3;
int b(3);
int c = { 3 }; // Uniform initialization
int d { 3 }; // Uniform initialization
\end{cpp}

Uniform initialization can be used to perform zero-initialization of variables; you just specify an empty set of curly braces, as shown here:

\begin{cpp}
int e { }; // Uniform initialization, e will be 0
\end{cpp}

This syntax can also be used with structures. If you create an instance of the Employee struct as follows, then its data members are default initialized, which, for primitive types such as char and int, means they’ll contain whatever random data is left in memory:

\begin{cpp}
Employee anEmployee;
\end{cpp}

However, if you create the instance as follows, then all data members are zero initialized:

\begin{cpp}
Employee anEmployee { };
\end{cpp}

A benefit of using uniform initialization is that it prevents narrowing. When using the old-style assignment syntax to initialize variables, C++ implicitly performs narrowing, as shown here:

\begin{cpp}
int main()
{
    int x = 3.14;
}
\end{cpp}

For the statement in main(), C++ automatically truncates 3.14 to 3 before assigning it to x. Some compilers might issue a warning about this narrowing, while others won’t. In any case, narrowing conversions should not go unnoticed, as they might cause subtle or not so subtle bugs. With uniform initialization, the assignment to x must generate a compilation error if your compiler fully conforms to the C++11 standard:

\begin{cpp}
int x { 3.14 }; // Error because narrowing
\end{cpp}

If a narrowing cast is what you need, I recommend using the gsl::narrow\_cast() function available in the Guidelines Support Library (GSL).

Uniform initialization can also be used in the constructor initializer to initialize arrays that are members of a class.

\begin{cpp}
class MyClass
{
    public:
        MyClass()
            : m_array { 0, 1, 2, 3 }
        {
        }
    private:
        int m_array[4];
};
\end{cpp}

Uniform initialization can be used with the Standard Library containers as well—such as std::vector, already demonstrated earlier in this chapter.

\begin{myNotic}{NOTE}
Considering all these benefits, it is recommended to use uniform initialization over using the assignment syntax to initialize variables. Hence, this book uses uniform initialization wherever possible.
\end{myNotic}

\mySamllsection{Designated Initializers}

Designated initializers initialize data members of aggregates using their name. An aggregate type is an object of an array type, or an object of a structure or class that satisfies the following restrictions: only public data members, no user-declared or inherited constructors, no virtual functions (see Chapter 10), and no virtual, private, or protected base classes (see Chapter 10). A designated initializer starts with a dot followed by the name of a data member. Designated initializers must be in the same order as the declaration order of the data members. Mixing designated initializers and nondesignated initializers is not allowed. Any data members that are not initialized using a designated initializer are initialized with their default values, which means the following:

\begin{itemize}
\item
Data members that have an in-class initializer will get that value.

\item
Data members that do not have an in-class initializer are zero initialized.
\end{itemize}

Let’s take a look at a slightly modified Employee structure. This time the salary data member has a default value of 75,000.

\begin{cpp}
struct Employee {
    char firstInitial;
    char lastInitial;
    int employeeNumber;
    int salary { 75'000 };
};
\end{cpp}

Earlier in this chapter, such an Employee structure is initialized using a uniform initialization syntax as follows:

\begin{cpp}
Employee anEmployee { 'J', 'D', 42, 80'000 }
\end{cpp}

Using designated initializers, this can be written as follows:

\begin{cpp}
Employee anEmployee {
    .firstInitial = 'J',
    .lastInitial = 'D',
    .employeeNumber = 42,
    .salary = 80'000
};
\end{cpp}

A benefit of using such designated initializers is that it’s much easier to understand what a designated initializer is initializing compared to using the uniform initialization syntax.

With designated initializers, you can skip initialization of certain members if you are satisfied with their default values. For example, when creating an employee, you could skip initializing employeeNumber, in which case employeeNumber is zero initialized as it doesn’t have an in-class initializer:

\begin{cpp}
Employee anEmployee {
    .firstInitial = 'J',
    .lastInitial = 'D',
    .salary = 80'000
};
\end{cpp}

With the uniform initialization syntax, this is not possible, and you have to specify 0 for the employee number as follows:

\begin{cpp}
Employee anEmployee { 'J', 'D', 0, 80'000 };
\end{cpp}

If you skip initializing the salary data member as follows, then salary gets its default value, which is its in-class initialization value, 75,000:

\begin{cpp}
Employee anEmployee {
    .firstInitial = 'J',
    .lastInitial = 'D'
};
\end{cpp}

A final benefit of using designated initializers is that when members are added to the data structure, existing code using designated initializers keeps working. The new data members will just be initialized with their default values.

\mySubsubsection{1.1.26.}{Pointers and Dynamic Memory}

Dynamic memory allows you to build programs with data that is not of fixed size at compile time. Most nontrivial programs make use of dynamic memory in some form.

\mySamllsection{The Stack and the Free Store}

\myGraphic{0.4}{content/part1/chapter1/images/2.png}{FIGURE 1.2}

Memory in your C++ application is divided into two parts—the stack and the free store. One way to visualize the stack is as a deck of cards. The current top card represents the current scope of the program, usually the function that is currently being executed. All variables declared inside the current function will take up memory in the top stack frame, the top card of the deck. If the current function, which I’ll call foo(), calls another function bar(), a new card is put on the deck so that bar() has its own stack frame to work with. Any parameters passed from foo() to bar() are copied from the foo() stack frame into the bar() stack frame. Figure 1.2 shows what the stack might look like during the execution of a hypothetical function foo() that has declared two integer values.

Stack frames are nice because they provide an isolated memory workspace for each function. If a variable is declared inside the foo() stack frame, calling the bar() function won’t change it unless you specifically tell it to. Also, when the foo() function is done running, the stack frame goes away, and all of the variables declared within the function no longer take up memory. Variables that are stackallocated do not need to be deallocated (deleted) by the programmer; it happens automatically.

The free store is an area of memory that is completely independent of the current function or stack frame. You can put variables on the free store if you want them to exist even when the function in which they were created has completed. The free store is less structured than the stack. You can think of it as just a pile of bits. Your program can add new bits to the pile at any time or modify bits that are already on the pile. You have to make sure that you deallocate (delete) any memory that you allocated on the free store. This does not happen automatically, unless you use smart pointers, which are discussed in detail in Chapter 7, “Memory Management.”

\begin{myWarning}{WARNING}
Pointers are introduced here because you will encounter them, especially in legacy code bases. In new code, however, such raw/naked pointers are allowed only if there is no ownership involved. Otherwise, you should use one of the smart pointers explained in Chapter 7.
\end{myWarning}

\mySamllsection{Working with Pointers}

You can put anything on the free store by explicitly allocating memory for it. For example, to put an integer on the free store, you need to allocate memory for it, but first you need to declare a pointer:

\begin{cpp}
int* myIntegerPointer;
\end{cpp}

The * after the int type indicates that the variable you are declaring refers or points to some integer memory. Think of the pointer as an arrow that points at the dynamically allocated free store memory. It does not yet point to anything specific because you haven’t assigned it to anything; it is an uninitialized variable. Uninitialized variables should be avoided at all times, and especially uninitialized pointers because they point to some random place in memory. Working with such pointers will most likely make your program crash. That’s why you should always declare and initialize your pointers at the same time! You can initialize them to a null pointer (nullptr—for more information, see the “Null Pointer Constant” section) if you don’t want to allocate memory right away:

\begin{cpp}
int* myIntegerPointer { nullptr };
\end{cpp}

A null pointer is a special default value that no valid pointer will ever have and converts to false when used in a Boolean expression. Here’s an example:

\begin{cpp}
if (!myIntegerPointer) { /* myIntegerPointer is a null pointer. */ }
\end{cpp}

You use the new operator to allocate the memory:

\begin{cpp}
myIntegerPointer = new int;
\end{cpp}

In this case, the pointer points to the address of just a single integer value. To access this value, you need to dereference the pointer. Think of dereferencing as following the pointer’s arrow to the actual value on the free store. To set the value of the newly allocated free store integer, you would use code like the following:

\begin{cpp}
*myIntegerPointer = 8;
\end{cpp}

Notice that this is not the same as setting myIntegerPointer to the value 8. You are not changing the pointer; you are changing the memory that it points to. If you were to reassign the pointer value, it would point to the memory address 8, which is probably random garbage that will eventually make your program crash.

After you are finished with your dynamically allocated memory, you need to deallocate the memory using the delete operator. To prevent the pointer from being used after having deallocated the memory it points to, it’s recommended to set it to nullptr:

\begin{cpp}
delete myIntegerPointer;
myIntegerPointer = nullptr;
\end{cpp}

\begin{myWarning}{WARNING}
A pointer must be valid before it is dereferenced. Dereferencing a null pointer or an uninitialized pointer causes undefined behavior. Your program might crash, but it might just as well keep running and start giving strange results.
\end{myWarning}

Pointers don’t always point to free store memory. You can declare a pointer that points to a variable on the stack, even another pointer. To get a pointer to a variable, you use the \& (“address of”) operator:

\begin{cpp}
int i { 8 };
int* myIntegerPointer { &i }; // Points to the variable with the value 8
\end{cpp}

C++ has a special syntax for dealing with pointers to structures or classes. Technically, if you have a pointer to a structure or a class, you can access its fields by first dereferencing it with *, and then using the normal . syntax, as in the code that follows. This code snippet also demonstrates how to dynamically allocate and deallocate an Employee instance.

\begin{cpp}
Employee* anEmployee { new Employee { 'J', 'D', 42, 80'000 } };
println("{}", (*anEmployee).salary);
delete anEmployee; anEmployee = nullptr;
\end{cpp}

This syntax is a little messy. The -> (arrow) operator lets you perform both the dereference and the field access in one step. The following statement is equivalent to the previous println() call but is easier to read:

\begin{cpp}
println("{}", anEmployee->salary);
\end{cpp}

Remember the concept of short-circuiting logic, discussed earlier in this chapter? This can be useful in combination with pointers to avoid using an invalid pointer, as in the following example:

\begin{cpp}
bool isValidSalary { anEmployee && anEmployee->salary > 0 };
\end{cpp}

Or, here it is a little bit more verbose:

\begin{cpp}
bool isValidSalary { anEmployee != nullptr && anEmployee->salary > 0 };
\end{cpp}

anEmployee is dereferenced to get the salary only if it is a valid pointer. If it is a null pointer, the logical operation short-circuits, and the anEmployee pointer is not dereferenced.

\mySamllsection{Dynamically Allocated Arrays}

The free store can also be used to dynamically allocate arrays. You use the new[] operator to allocate memory for an array.

\begin{cpp}
int arraySize { 8 };
int* myVariableSizedArray { new int[arraySize] };
\end{cpp}

This allocates enough memory to hold arraySize integers. Figure 1.3 shows what the stack and the free store both look like after this code is executed. As you can see, the pointer variable still resides on the stack, but the array that was dynamically created lives on the free store.

Now that the memory has been allocated, you can work with myVariableSizedArray as though it were a regular stack-based array:

\begin{cpp}
myVariableSizedArray[3] = 2;
\end{cpp}

When your code is done with the array, it should remove the array from the free store so that other variables can use the memory. In C++, you use the delete[] operator to do this:

\begin{cpp}
delete[] myVariableSizedArray;
myVariableSizedArray = nullptr;
\end{cpp}

The brackets after delete indicate that you are deleting an array!

\myGraphic{1.0}{content/part1/chapter1/images/3.png}{FIGURE 1.3}

\begin{myNotic}{NOTE}
If you do need dynamically allocated memory, avoid using malloc() and free() from C. Instead, use new and delete, or new[] and delete[] from C++. However, in modern C++, the goal is to avoid new, delete, new[], and delete[] altogether, and use more modern constructs such as Standard Library containers, e.g., std::vector, and smart pointers, discussed in Chapter 7.
\end{myNotic}

\begin{myWarning}{WARNING}
To prevent memory leaks, every call to new should be paired with a call to delete, and every call to new[] should be paired with a call to delete[]. Not calling delete or delete[], or mismatching calls, results in memory leaks or worse. All these intricacies are discussed in Chapter 7.
\end{myWarning}

\mySamllsection{Null Pointer Constant}

Before C++11, the constant NULL, defined in <cstddef>, was used for null pointers. You cannot get access to this constant using any import declaration; instead, you must use \#include <cstddef>. NULL is simply defined as the constant 0, which can cause problems. Take the following example:

\begin{cpp}
#include <cstddef>

void func(int i) { /* ... */ }

int main()
{
    func(NULL);
}
\end{cpp}

The code defines a function func() with a single integer parameter. The main() function calls func() with argument NULL, which is supposed to be a null pointer constant. However, since NULL is not a real pointer, but identical to the integer 0, it triggers a call to func(int). This might be unexpected behavior. Hence, some compilers even give a warning about this.

This problem is avoided by using a real null pointer constant, nullptr. The following code uses this real null pointer constant and causes a compilation error because there is no overload of func() accepting a pointer:

\begin{cpp}
func(nullptr);
\end{cpp}

\mySubsubsection{1.1.27.}{The Use of const}

The keyword const can be used in a few different ways in C++. Its uses are related, but there are subtle differences. The subtleties of const make for excellent interview questions! Basically, the keyword const is short for “constant” and specifies that something remains unchanged.

The compiler enforces this requirement by marking any attempt to change it as an error. Furthermore, when optimizations are enabled, the compiler can take advantage of this knowledge to produce better code.

\mySamllsection{const as a Qualifier for a Type}

If you assumed that the keyword const has something to do with constants, you have correctly uncovered one of its uses. In C, programmers often use the preprocessor \#define mechanism, see Chapter 11, to declare symbolic names for values that won’t change during the execution of the program, such as the version number. In C++, programmers are encouraged to avoid \#define in favor of using const to define constants. Defining a constant with const is just like defining a variable, except that the compiler guarantees that code cannot change the value. Here are some examples:

\begin{cpp}
const int versionNumberMajor { 2 };
const int versionNumberMinor { 1 };
const std::string productName { "Super Hyper Net Modulator" };
const double PI { 3.141592653589793238462 };
\end{cpp}

You can mark any variable const, including global variables and class data members.

\mySamllsection{const Member Functions}

When a variable contains one or more levels of indirection via a pointer, applying const becomes trickier. Consider the following lines of code:

\begin{cpp}
int* ip;
ip = new int[10];
ip[4] = 5;
\end{cpp}

Suppose that you decide to apply const to ip. Set aside your doubts about the usefulness of doing so for a moment, and consider what it means. Do you want to prevent the ip variable itself from being changed, or do you want to prevent the values to which it points from being changed? That is, do you want to prevent the second statement or the third statement?

To prevent the pointed-to values from being modified (as in the third statement), you can add the keyword const to the declaration of ip like this:

\begin{cpp}
const int* ip;
ip = new int[10];
ip[4] = 5; // DOES NOT COMPILE!
\end{cpp}

Now you cannot change the values to which ip points. An alternative but semantically equivalent way to write this is as follows:

\begin{cpp}
int const* ip;
ip = new int[10];
ip[4] = 5; // DOES NOT COMPILE!
\end{cpp}

Putting the const before or after the int makes no difference in its functionality.

If you instead want to mark ip itself const (not the values to which it points), you need to write this:

\begin{cpp}
int* const ip { nullptr };
ip = new int[10]; // DOES NOT COMPILE!
ip[4] = 5; // Error: dereferencing a null pointer
\end{cpp}

Now that ip itself cannot be changed, the compiler requires you to initialize it when you declare it, either with nullptr as in the preceding code or with newly allocated memory as follows:

\begin{cpp}
int* const ip { new int[10] };
ip[4] = 5;
\end{cpp}

You can also mark both the pointer and the value to which it points const like this:

\begin{cpp}
int const* const ip { nullptr };
\end{cpp}

Here is an alternative but equivalent syntax:

\begin{cpp}
const int* const ip { nullptr };
\end{cpp}

Although this syntax might seem confusing, there is actually a simple rule: the const keyword applies to whatever is directly to its left. Consider this line again:

\begin{cpp}
int const* const ip { nullptr };
\end{cpp}

From left to right, the first const is directly to the right of the word int. Thus, it applies to the int to which ip points. Therefore, it specifies that you cannot change the values to which ip points. The second const is directly to the right of the *. Thus, it applies to the pointer to the int, which is the ip variable. Therefore, it specifies that you cannot change ip (the pointer) itself.

The reason this rule becomes confusing is an exception. That is, the first const can go before the variable like this:

\begin{cpp}
const int* const ip { nullptr };
\end{cpp}

This “exceptional” syntax is used much more commonly than the other syntax.

You can extend this rule to any number of levels of indirection, as in this example:

\begin{cpp}
const int * const * const * const ip { nullptr };
\end{cpp}

\begin{myNotic}{NOTE}
Here is another easy-to-remember rule to figure out complicated variable declarations: read from right to left. For example, int* const ip reads from right to left as “ip is a const pointer to an int.” Further, int const* ip reads as “ip is a pointer to a const int,” and const int* ip reads as “ip is a pointer to an int constant.”
\end{myNotic}

\mySamllsection{const to Protect Parameters}

In C++, you can cast a non-const variable to a const variable. Why would you want to do this? It offers some degree of protection from other code changing the variable. If you are calling a function that a co-worker of yours is writing and you want to ensure that the function doesn’t change the value of an argument you pass in, you can tell your co-worker to have the function take a const parameter. If the function attempts to change the value of the parameter, it will not compile.

In the following code, a string* is automatically cast to a const string* in the call to mysteryFunction(). If the author of mysteryFunction() attempts to change the value of the passed string, the code will not compile. There are ways around this restriction, but using them requires conscious effort. C++ only protects against accidentally changing const variables.

\begin{cpp}
void mysteryFunction(const string* someString)
{
    *someString = "Test"; // Will not compile
}

int main()
{
    string myString { "The string" };
    mysteryFunction(&myString); // &myString is a string*
}
\end{cpp}

You can also use const on primitive-type parameters to prevent accidentally changing them in the body of the function. For example, the following function has a const integer parameter. In the body of the function, you cannot modify the param integer. If you do try to modify it, the compiler will generate an error.

\begin{cpp}
void func(const int param) { /* Not allowed to change param... */ }
\end{cpp}

\mySamllsection{const Member Functions}

A second use of the const keyword is to mark class member functions as const, preventing them from modifying data members of the class. The AirlineTicket class introduced earlier can be modified to mark all read-only member functions as const. The const must be added to both the member function declaration and its definition. If any const member function tries to modify one of the AirlineTicket data members, the compiler will emit an error.

\begin{cpp}
export class AirlineTicket
{
    public:
        double calculatePriceInDollars() const;

        std::string getPassengerName() const;
        void setPassengerName(std::string name);

        int getNumberOfMiles() const;
        void setNumberOfMiles(int miles);

        bool hasEliteSuperRewardsStatus() const;
        void setHasEliteSuperRewardsStatus(bool status);
    private:
        std::string m_passengerName { "Unknown Passenger" };
        int m_numberOfMiles { 0 };
        bool m_hasEliteSuperRewardsStatus { false };
};

std::string AirlineTicket::getPassengerName() const
{
    return m_passengerName;
}
// Other member functions omitted...
\end{cpp}

\begin{myNotic}{NOTE}
To follow the const-correctness principle, it’s recommended to declare member functions that do not change any data members of the object as being const. These member functions are also called inspectors, compared to mutators for non-const member functions.
\end{myNotic}

\mySubsubsection{1.1.28.}{References}

Professional C++ code, including much of the code in this book, uses references extensively. A reference in C++ is an alias for another variable. All modifications to the reference change the value of the variable to which it refers. You can think of references as implicit pointers that save you the trouble of taking the address of variables and dereferencing the pointer. Alternatively, you can think of references as just another name for the original variable. You can create stand-alone reference variables, use reference data members in classes, accept references as parameters to functions, and return references from functions.

\mySamllsection{Reference Variables}

Reference variables must be initialized as soon as they are created, like this:

\begin{cpp}
int x { 3 };
int& xRef { x };
\end{cpp}

Attaching \& to a type indicates that the variable is a reference. It is still used as though it was a normal variable, but behind the scenes, it is really a pointer to the original variable. Both the variable x and the reference variable xRef point to exactly the same value; i.e., xRef is just another name for x. If you change the value through either one of them, the change is visible through the other one as well. For example, the following code sets x to 10 through xRef:

\begin{cpp}
xRef = 10;
\end{cpp}

You cannot declare a reference variable outside of a class definition without initializing it.

\begin{cpp}
int& emptyRef; // DOES NOT COMPILE!
\end{cpp}

\begin{myWarning}{WARNING}
A reference variable must always be initialized when it’s created.
\end{myWarning}

\mySamllsection{Modifying References}

A reference always refers to the same variable to which it is initialized; references cannot be changed once they are created. The syntax might be confusing for beginning C++ programmers. If you assign a variable to a reference when the reference is declared, the reference refers to that variable. However, if you assign a variable to a reference after that, the variable to which the reference refers is changed to the value of the variable being assigned. The reference is not updated to refer to that variable. Here is a code example:

\begin{cpp}
int x { 3 }, y { 4 };
int& xRef { x };
xRef = y; // Changes value of x to 4. Doesn't make xRef refer to y.
\end{cpp}

You might try to circumvent this restriction by taking the address of y when you assign it:

\begin{cpp}
xRef = &y; // DOES NOT COMPILE!
\end{cpp}

This code does not compile. The address of y is a pointer, but xRef is declared as a reference to an int, not a reference to a pointer.

Some programmers go even further in their attempts to circumvent the intended semantics of references. What if you assign a reference to a reference? Won’t that make the first reference refer to the variable to which the second reference refers? You might be tempted to try this code:

\begin{cpp}
int x { 3 }, z { 5 };
int& xRef { x };
int& zRef { z };
zRef = xRef; // Assigns values, not references
\end{cpp}

The final statement does not change to what zRef refers to. Instead, it sets the value of z to 3, because xRef refers to x, which is 3.

\begin{myWarning}{WARNING}
Once a reference is initialized to refer to a specific variable, you cannot change the reference to refer to another variable; you can change only the value of the variable the reference refers to.
\end{myWarning}

\mySamllsection{References-to-const}
References-to-const
const applied to references is usually easier than const applied to pointers for two reasons. First, references are const by default, in that you can’t change to what they refer. So, there is no need to mark them const explicitly. Second, you can’t create a reference to a reference, so there is usually only one level of indirection with references. The only way to get multiple levels of indirection is to create a reference to a pointer.

Thus, when C++ programmers refer to a reference-to-const, they mean something like this:

\begin{cpp}
int z;
const int& zRef { z };
zRef = 4; // DOES NOT COMPILE
\end{cpp}

By applying const to the int\&, you prevent assignment to zRef, as shown. Similar to pointers, const int\& zRef is equivalent to int const\& zRef. Note, however, that marking zRef const has no effect on z. You can still modify the value of z by changing it directly instead of through the reference.

You cannot create a reference to an unnamed value, such as an integer literal, unless the reference is to a const value. In the following example, unnamedRef1 does not compile because it is a referenceto-non-const referring to a constant. That would mean you could change the value of the constant, 5, which doesn’t make sense. unnamedRef2 works because it’s a reference-to-const, so you cannot for example write unnamedRef2 = 7.

\begin{cpp}
int& unnamedRef1 { 5 }; // DOES NOT COMPILE
const int& unnamedRef2 { 5 }; // Works as expected
\end{cpp}

The same holds for temporary objects. You cannot create a reference-to-non-const to a temporary object, but a reference-to-const is fine. For example, suppose you have the following function returning an std::string object:

\begin{cpp}
string getString() { return "Hello world!"; }
\end{cpp}

You can create a reference-to-const to the result of calling getString(), and that reference keeps the temporary std::string object alive until the reference goes out of scope:

\begin{cpp}
string& string1 { getString() }; // DOES NOT COMPILE
const string& string2 { getString() }; // Works as expected
\end{cpp}

\mySamllsection{References to Pointers and Pointers to References}

You can create references to any type, including pointer types. Here is an example of a reference to a pointer to int:

\begin{cpp}
int* intP { nullptr };
int*& ptrRef { intP };
ptrRef = new int;
*ptrRef = 5;
delete ptrRef; ptrRef = nullptr;
\end{cpp}

The syntax is a little strange: you might not be accustomed to seeing * and \& right next to each other. However, the semantics are straightforward: ptrRef is a reference to intP, which is a pointer to int. Modifying ptrRef changes intP. References to pointers are rare but can occasionally be useful, as discussed in the “Reference Parameters” section later in this chapter.

Taking the address of a reference gives the same result as taking the address of the variable to which the reference refers. Here is an example:

\begin{cpp}
int x { 3 };
int& xRef { x };
int* xPtr { &xRef }; // Address of a reference is pointer to value.
*xPtr = 100;
\end{cpp}

This code sets xPtr to point to x by taking the address of a reference to x. Assigning 100 to *xPtr changes the value of x to 100. Writing the comparison xPtr == xRef will not compile because of a type mismatch; xPtr is a pointer to an int, while xRef is a reference to an int. The comparisons xPtr == \&xRef and xPtr == \&x both compile without errors and are both true.

Finally, note that you cannot declare a reference to a reference or a pointer to a reference. For example, neither int\&\& nor int\&* is allowed.

\mySamllsection{Structured Bindings and References}

Structured bindings are introduced earlier in this chapter. One of the examples given was the following:

\begin{cpp}
pair myPair { "hello", 5 };
auto [theString, theInt] { myPair }; // Decompose using structured bindings
\end{cpp}

Now that you know about references and const variables, it’s time to learn that both can be combined with structured bindings as well. Here’s an example:

\begin{cpp}
auto& [theString, theInt] { myPair }; // Decompose into references-to-non-const
const auto& [theString, theInt] { myPair }; // Decompose into references-to-const
\end{cpp}

\mySamllsection{Reference Data Members}

Data members of classes can be references. As discussed earlier, a reference cannot exist without referring to some other variable, and it is not possible to change where a reference refers to. Thus, reference data members cannot be initialized inside the body of a class constructor, but they must be initialized in the constructor initializer. Syntax-wise, a constructor initializer immediately follows the constructor header and starts with a colon. The following is a quick example with the constructor initializer highlighted. Chapter 9 goes in much more detail.

\begin{cpp}
class MyClass
{
    public:
        MyClass(int& ref) : m_ref { ref } { /* Body of constructor */ }
    private:
        int& m_ref;
};
\end{cpp}

\begin{myWarning}{WARNING}
A reference must always be initialized when it’s created. Usually, references are created when they are declared, but reference data members need to be initialized in a constructor initializer for the containing class.
\end{myWarning}

\mySamllsection{Reference Parameters}

C++ programmers do not often use stand-alone reference variables or reference data members. The most common use of references is for parameters to functions. The default parameter-passing semantics is pass-by-value: functions receive copies of their arguments. When those parameters are modified, the original arguments remain unchanged. Pointers to stack variables are often used in C to allow functions to modify variables in other stack frames. By dereferencing the pointer, the function can change the memory that represents the variable even though that variable isn’t in the current stack frame. The problem with this approach is that it brings the messiness of pointer syntax into what is really a simple task.

Instead of passing pointers to functions, C++ offers a better mechanism, called pass-by-reference, where parameters are references instead of pointers. The following are two implementations of an addOne() function. The first one has no effect on the variable that is passed in because it is passed by value, and thus the function receives a copy of the value passed to it. The second one uses a reference and thus changes the original variable.

\begin{cpp}
void addOne(int i)
{
    i++; // Has no real effect because this is a copy of the original
}

void addOne(int& i)
{
    i++; // Actually changes the original variable
}
\end{cpp}

The syntax for the call to the addOne() function with an integer reference is no different than if the function just took an integer.

\begin{cpp}
int myInt { 7 };
addOne(myInt);
\end{cpp}

\begin{myNotic}{NOTE}
There is a subtle difference between the two addOne() implementations. The version using pass-by-value accepts literals without a problem; for example, addOne(3); is legal. However, doing the same with the pass-by-reference version of addOne() will result in a compilation error. This can be solved by using reference-to-const parameters, discussed in the next section.
\end{myNotic}

Here is another example where pass-by-reference comes in handy; it’s a simple swap function to swap the values of two ints:

\begin{cpp}
void swap(int& first, int& second)
{
    int temp { first };
    first = second;
    second = temp;
}
\end{cpp}

You can call it like this:

\begin{cpp}
int x { 5 }, y { 6 };
swap(x, y);
\end{cpp}

When swap() is called with the arguments x and y, the first parameter is initialized to refer to x, and the second parameter is initialized to refer to y. When swap() modifies first and second, x and y are actually changed.

A common quandary arises when you have a pointer to something that you need to pass to a function that takes a reference. You can “convert” a pointer to a reference in this case by dereferencing the pointer. This action gives you the value to which the pointer points, which the compiler then uses to initialize the reference parameter. For example, you can call swap() like this:

\begin{cpp}
int x { 5 }, y { 6 };
int *xp { &x }, *yp { &y };
swap(*xp, *yp);
\end{cpp}

Finally, if you have a function that needs to return an object of a class that is expensive to copy, you’ll often see the function accepting an output parameter of type reference-to-non-const to such a class that the function then modifies, instead of directly returning such an object. Developers thought that this was the recommended way to prevent any performance penalties with creating copies when returning objects from functions. However, even back then, compilers were usually smart enough to avoid any redundant copies. So, we have the following rule:

\begin{myWarning}{WARNING}
The recommended way to return objects from a function is to return them by value, instead of using output parameters.
\end{myWarning}

\mySamllsection{Pass-by-Reference-to-const}

The main value in reference-to-const parameters is efficiency. When you pass a value into a function, an entire copy is made. When you pass a reference, you are really just passing a pointer to the original so the computer doesn’t need to make a copy. By passing a reference-to-const, you get the best of both worlds: no copy is made, and the original variable cannot be changed. References-toconst become more important when you are dealing with objects because they can be large and making copies of them can have unwanted side effects. The following example shows how to pass an std::string to a function as a reference-to-const:

\begin{cpp}
import std;
using namespace std;
void printString(const string& myString) { println("{}", myString); }

int main()
{
    string someString { "Hello World" };
    printString(someString);
    printString("Hello World"); // Passing literals works.
}
\end{cpp}

\mySamllsection{Pass-by-Reference vs. Pass-by-Value}

Pass-by-reference is required when you want to modify the parameter and see those changes reflected in the variable passed to the function. However, you should not limit your use of pass-by-reference to only those cases. Pass-by-reference avoids copying the arguments to the function, providing two additional benefits:

\begin{itemize}
\item
Efficiency: Large objects could take a long time to copy. Pass-by-reference passes only a reference to the object into the function.

\item
Support: Not all classes allow pass-by-value.
\end{itemize}

If you want to leverage these benefits but do not want to allow the original objects to be modified, you should mark the parameters const, giving you pass-by-reference-to-const.

\begin{myNotic}{NOTE}
These benefits of pass-by-reference imply that you should use pass-by-value only for simple built-in types such as int and double for which you don’t need to modify the arguments. If you need to pass an object to a function, prefer to pass it by reference-to-const instead of by value. This prevents unnecessary copying. Pass it by reference-to-non-const if the function needs to modify the object. Chapter 9 slightly modifies this rule after the introduction of move semantics, allowing passby-value of objects in certain cases.
\end{myNotic}

\mySamllsection{Reference Return Values}

You can return a reference from a function. Of course, you can use this technique only if the variable to which the returned reference refers to continues to exist following the function termination.

\begin{myWarning}{WARNING}
From a function, never return a reference to a variable that is locally scoped to that function, such as an automatically allocated variable on the stack that will be destroyed when the function ends.
\end{myWarning}

One of the main reasons to return a reference is if you want to be able to assign to the return value directly as an lvalue (the left-hand side of an assignment statement). Several overloaded operators commonly return references, such as operators =, +=, and so on. Chapter 15 goes into more details on how to write such overloaded operators yourself.

Another reason to return a reference from a function is if the return type is expensive to copy. By returning a reference or reference-to-const, the copying is avoided, but keep the earlier warning in mind. This is often used to return objects by reference-to-const from class member functions, as demonstrated later in this chapter.

\mySamllsection{Deciding Between References and Pointers}

References in C++ could be considered redundant: everything you can do with references, you can accomplish with pointers. For example, you could write the earlier shown swap() function like this:

\begin{cpp}
void swap(int* first, int* second)
{
    int temp { *first };
    *first = *second;
    *second = temp;
}
\end{cpp}

However, this code is more cluttered than the version with references. References make your programs cleaner and easier to understand. They are also safer than pointers: it’s impossible to have a null reference, and you don’t explicitly dereference references, so you can’t encounter any of the dereferencing errors associated with pointers. Of course, these arguments about references being safer are valid only in the absence of any pointers. For example, take the following function that accepts a reference to an int:

\begin{cpp}
void refcall(int& t) { ++t; }
\end{cpp}

You could declare a pointer and initialize it to point to some random place in memory. Then you could dereference this pointer and pass it as the reference argument to refcall(), as in the following code. This code compiles fine, but it is undefined what will happen when executed. It could for example cause a crash.

\begin{cpp}
int* ptr { (int*)8 };
refcall(*ptr);
\end{cpp}

Most of the time, you can use references instead of pointers. References to objects also support polymorphism, discussed in detail in Chapter 10, in the same way as pointers to objects. However, there are some use cases in which you need to use a pointer. One example is when you need to change the location to which it points. Recall that you cannot change the variable to which a reference refers. For example, when you dynamically allocate memory, you need to store a pointer to the result in a pointer rather than a reference. A second use case in which you need to use a pointer is when the pointer is optional, that is, when it can be nullptr. Yet another use case is if you want to store polymorphic types (discussed in Chapter 10) in a container.

A long time ago, and in legacy code, a way to distinguish between appropriate use of pointers and references in parameters and return types was to consider who owns the memory. If the code receiving the variable became the owner and thus became responsible for releasing the memory associated with an object, it had to receive a pointer to the object. If the code receiving the variable didn’t have to free the memory, it received a reference. Nowadays, however, raw pointers should be avoided in favor of smart pointers (see Chapter 7), which is the recommended way to transfer ownership.

\begin{myNotic}{NOTE}
Prefer references over pointers; that is, use a pointer only if a reference is not possible.
\end{myNotic}

Consider a function that splits an array of ints into two arrays: one of even numbers and one of odd numbers. The function doesn’t know how many numbers in the source array will be even or odd, so it should dynamically allocate the memory for the destination arrays after examining the source array. It should also return the sizes of the two new arrays. Altogether, there are four items to return: pointers to the two new arrays and the sizes of the two new arrays. Obviously, you must use pass-by-reference. The canonical C way to write the function looks like this:

\begin{cpp}
void separateOddsAndEvens(const int arr[], size_t size, int** odds,
    size_t* numOdds, int** evens, size_t* numEvens)
{
    // Count the number of odds and evens.
    *numOdds = *numEvens = 0;
    for (size_t i = 0; i < size; ++i) {
        if (arr[i] % 2 == 1) {
            ++(*numOdds);
        } else {
            ++(*numEvens);
        }
    }

    // Allocate two new arrays of the appropriate size.
    *odds = new int[*numOdds];
    *evens = new int[*numEvens];

    // Copy the odds and evens to the new arrays.
    size_t oddsPos = 0, evensPos = 0;
    for (size_t i = 0; i < size; ++i) {
        if (arr[i] % 2 == 1) {
            (*odds)[oddsPos++] = arr[i];
        } else {
            (*evens)[evensPos++] = arr[i];
        }
    }
}
\end{cpp}

The final four parameters to the function are the “reference” parameters. To change the values to which they refer, separateOddsAndEvens() must dereference them, leading to some ugly syntax in the function body. Additionally, when you want to call separateOddsAndEvens(), you must pass the address of two pointers so that the function can change the actual pointers, and pass the address of two size\_ts so that the function can change the actual size\_ts. Note also that the caller is responsible for deleting the two arrays created by separateOddsAndEvens()!

\begin{cpp}
int unSplit[] { 1, 2, 3, 4, 5, 6, 7, 8, 9, 10 };
int* oddNums { nullptr };
int* evenNums { nullptr };
size_t numOdds { 0 }, numEvens { 0 };

separateOddsAndEvens(unSplit, std::size(unSplit),
    &oddNums, &numOdds, &evenNums, &numEvens);

// Use the arrays...

delete[] oddNums; oddNums = nullptr;
delete[] evenNums; evenNums = nullptr;
\end{cpp}

If this syntax annoys you (which it should), you can write the same function by using references to obtain true pass-by-reference semantics:

\begin{cpp}
void separateOddsAndEvens(const int arr[], size_t size, int*& odds,
    size_t& numOdds, int*& evens, size_t& numEvens)
{
    numOdds = numEvens = 0;
    for (size_t i { 0 }; i < size; ++i) {
        if (arr[i] % 2 == 1) {
            ++numOdds;
        } else {
            ++numEvens;
        }
    }

    odds = new int[numOdds];
    evens = new int[numEvens];

    size_t oddsPos { 0 }, evensPos { 0 };
    for (size_t i { 0 }; i < size; ++i) {
        if (arr[i] % 2 == 1) {
            odds[oddsPos++] = arr[i];
        } else {
            evens[evensPos++] = arr[i];
        }
    }
}
\end{cpp}

In this case, the odds and evens parameters are references to int*s. separateOddsAndEvens() can modify the int*s that are used as arguments to the function (through the reference), without any explicit dereferencing. The same logic applies to numOdds and numEvens, which are references to size\_ts. With this version of the function, you no longer need to pass the addresses of the pointers or size\_ts; the reference parameters handle it for you automatically:

\begin{cpp}
separateOddsAndEvens(unSplit, std::size(unSplit),
    oddNums, numOdds, evenNums, numEvens);
\end{cpp}

Even though using reference parameters is already much cleaner than using pointers, it is recommended that you avoid dynamically allocated arrays as much as possible. For example, by using the Standard Library vector container, the separateOddsAndEvens() function can be rewritten to be much safer, shorter, more elegant, and much more readable, because all memory allocations and deallocations happen automatically.

\begin{cpp}
void separateOddsAndEvens(const vector<int>& arr,
    vector<int>& odds, vector<int>& evens)
{
    for (int i : arr) {
        if (i % 2 == 1) {
            odds.push_back(i);
        } else {
            evens.push_back(i);
        }
    }
}
\end{cpp}

This version can be used as follows:

\begin{cpp}
vector<int> vecUnSplit { 1, 2, 3, 4, 5, 6, 7, 8, 9, 10 };
vector<int> odds, evens;
separateOddsAndEvens(vecUnSplit, odds, evens);
\end{cpp}

Note that you don’t need to deallocate the odds and evens containers; the vector class takes care of this. This version is much easier to use than the versions using pointers or references.

The version using vectors is already much better than the versions using pointers or references, but as I recommended earlier, output parameters should be avoided as much as possible. If a function needs to return something, it should just return it instead of using output parameters! Since C++17, a compiler is not allowed to perform any copying or moving of objects for statements of the form return object; where object is a nameless temporary. This is called mandatory elision of copy/ move operations and means that there’s no performance penalty at all by returning object by value. If object is a local variable that is not a function parameter, non-mandatory elision of copy/move operations is allowed, an optimization also known as named return value optimization (NRVO). This optimization is not guaranteed by the standard. Some compilers perform this optimization only for release builds but not for debug builds. With mandatory and non-mandatory elision, compilers can avoid any copying of objects that are returned from functions. This results in zero-copy pass-by-value semantics. Note that for NRVO, even though the copy/move constructors won’t be called, they still need to be accessible; otherwise, the program is ill-formed according to the standard. Copy/move operations and constructors are discussed in Chapter 9, but those details are not important for the current discussion.

The following version of separateOddsAndEvens() returns a simple struct of two vectors, instead of accepting two output vectors as parameters, and uses designated initializers.

\begin{cpp}
struct OddsAndEvens { vector<int> odds, evens; };

OddsAndEvens separateOddsAndEvens(const vector<int>& arr)
{
    vector<int> odds, evens;
    for (int i : arr) {
        if (i % 2 == 1) {
            odds.push_back(i);
        } else {
            evens.push_back(i);
        }
    }
    return OddsAndEvens { .odds = odds, .evens = evens };
}
\end{cpp}

With these changes, the code to call separateOddsAndEvens() becomes compact yet easy to read and understand:

\begin{cpp}
vector<int> vecUnSplit { 1, 2, 3, 4, 5, 6, 7, 8, 9, 10 };
auto oddsAndEvens { separateOddsAndEvens(vecUnSplit) };
// Do something with oddsAndEvens.odds and oddsAndEvens.evens...
\end{cpp}

\begin{myNotic}{NOTE}
Avoid output parameters. If a function needs to return something, just return it by value.
\end{myNotic}

\mySubsubsection{1.1.29.}{const\_cast()}

In C++ every variable has a specific type. It is possible in certain situations to cast a variable of one type to a variable of another type. To that end, C++ provides five types of casts: const\_cast(), static\_cast(), reinterpret\_cast(), dynamic\_cast(), and std::bit\_cast(). This section discusses const\_cast(). The second type of cast, static\_cast(), is briefly introduced earlier in this chapter and discussed in more detail in Chapter 10. The other remaining casts are also discussed in Chapter 10.

const\_cast() is the most straightforward of the different casts available. You can use it to add const-ness to a variable or cast away const-ness of a variable. It is the only cast of the five that is allowed to cast away const-ness. Theoretically, of course, there should be no need for a const cast. If a variable is const, it should stay const. In practice, however, you sometimes find yourself in a situation where a function is specified to take a const parameter, which it must then pass to a function that takes a non-const parameter, and you are absolutely sure that the latter function will not modify its non-const argument. The “correct” solution would be to make const consistent in the program, but that is not always an option, especially if you are using third-party libraries. Thus, you sometimes need to cast away the const-ness of a variable, but again you should do this only when you are sure the function you are calling will not modify the object; otherwise, there is no other option than to restructure your program. Here is an example:

\begin{cpp}
void thirdPartyLibraryFunction(char* str);

void f(const char* str)
{
    thirdPartyLibraryFunction(const_cast<char*>(str));
}
\end{cpp}

Additionally, the Standard Library provides a helper function called std::as\_const(), defined in <utility>, which returns a reference-to-const of its reference parameter. Basically, as\_const(obj) is equivalent to const\_cast<const T\&>(obj), where T is the type of obj. Using as\_const() results in shorter and more readable code compared to using const\_cast(). Concrete use cases for as\_const() are coming later in this book, but its basic use is as follows:

\begin{cpp}
string str { "C++" };
const string& constStr { as_const(str) };
\end{cpp}

\mySubsubsection{1.1.30.}{Exceptions}

C++ is a flexible language, but it does allow you to do unsafe things. For example, the compiler will let you write code that scribbles on random memory addresses or tries to divide by zero (computers don’t deal well with infinity). One language feature that attempts to add a degree of safety is exceptions.

An exception is an exceptional situation, that is, a situation that you don’t expect or want in the normal flow of execution of a program. For example, if you are writing a function that retrieves a web page, several things could go wrong. The Internet host that contains the page might be down, the page might come back blank, or the connection could be lost. One way you could handle this situation is by returning a special value from the function, such as nullptr or an error code. Exceptions provide a much better mechanism for dealing with problems.

Exceptions come with some new terminology. When a piece of code detects an exceptional situation, it throws an exception. Another piece of code catches the exception and takes appropriate action. The following example shows a function, divideNumbers(), that throws an exception if the caller passes in a denominator of zero. The std::invalid\_argument exception is defined in <stdexcept>.

\begin{cpp}
double divideNumbers(double numerator, double denominator)
{
    if (denominator == 0) {
        throw invalid_argument { "Denominator cannot be 0." };
    }
    return numerator / denominator;
}
\end{cpp}

When the throw statement is executed, the function immediately ends without returning a value. If the caller surrounds the function call with a try/catch block, as shown in the following code, it receives the exception and is able to handle it. Chapter 14, “Handling Errors,” goes into much more detail on exception handling, but for now, just remember that it is recommended to catch exceptions by reference-to-const, such as const invalid\_argument\& in the following example. Also note that all Standard Library exception classes have a member function called what(), which returns a string containing a brief explanation of the exception.

\begin{cpp}
try {
    println("{}", divideNumbers(2.5, 0.5));
    println("{}", divideNumbers(2.3, 0));
    println("{}", divideNumbers(4.5, 2.5));
} catch (const invalid_argument& exception) {
    println("Exception caught: {}", exception.what());
}
\end{cpp}

The first call to divideNumbers() executes successfully, and the result is printed on the screen. The second call throws an exception. No value is returned, and the only output is the error message that is printed when the exception is caught. The third call is never executed because the second call throws an exception, causing the program to jump to the catch block. The output for the preceding code snippet is as follows:

\begin{shell}
5
Exception caught: Denominator cannot be 0.
\end{shell}

Exceptions can get tricky in C++. To use exceptions properly, you need to understand what happens to the stack variables when an exception is thrown, and you have to be careful to properly catch and handle the necessary exceptions. Also, if you need to include more information about an error in an exception, you can write your own exception types. Lastly, the C++ compiler doesn’t force you to catch every exception that might occur. If your code never catches any exceptions but an exception is thrown, the program will be terminated. These trickier aspects of exceptions are covered in detail in Chapter 14.

\mySubsubsection{1.1.31.}{Type Aliases}

A type alias provides a new name for an existing type declaration. You can think of a type alias as syntax for introducing a synonym for an existing type declaration without creating a new type. The following gives a new name, IntPtr, to the int* type declaration:

\begin{cpp}
using IntPtr = int*;
\end{cpp}

You can use the new type name and the definition it aliases interchangeably. For example, the following two lines are valid:

\begin{cpp}
int* p1;
IntPtr p2;
\end{cpp}

Variables created with the new type name are completely compatible with those created with the original type declaration. So, it is perfectly valid, given these definitions, to write the following, because they are not just compatible types; they are the same type:

\begin{cpp}
p1 = p2;
p2 = p1;
\end{cpp}

The most common use for type aliases is to provide manageable names when the real type declarations become too unwieldy. This situation commonly arises with templates. An example from the Standard Library itself is std::basic\_string<T> to represent strings. It’s a class template where T is the type of each character in the string, for example char. You have to specify the template type parameter any time you want to refer to such a type. For declaring variables, specifying function parameters, and so on, you would have to write basic\_string<char>:

\begin{cpp}
void processVector(const vector<basic_string<char>>& vec) { /* omitted */ }

int main()
{
    vector<basic_string<char>> myVector;
    processVector(myVector);
}
\end{cpp}

Since basic\_string<char> is used that frequently, the Standard Library provides the following type alias as a shorter, more meaningful name:

\begin{cpp}
using string = basic_string<char>;
\end{cpp}

With this type alias, the previous code snippet can be written more elegantly:

\begin{cpp}
void processVector(const vector<string>& vec) { /* omitted */ }

int main()
{
    vector<string> myVector;
    processVector(myVector);
}
\end{cpp}

\mySubsubsection{1.1.32.}{typedefs}

Type aliases were introduced in C++11. Before C++11, you had to use typedefs to accomplish something similar but in a more convoluted way. This old mechanism is still explained here because you will come across it in legacy code bases.

Just as a type alias, a typedef provides a new name for an existing type declaration. For example, take the following type alias:

\begin{cpp}
using IntPtr = int*;
\end{cpp}

This can be written as follows with a typedef:

\begin{cpp}
typedef int* IntPtr;
\end{cpp}

As you can see, it’s much less readable. The order is reversed, which causes a lot of confusion, even for professional C++ developers. Other than being more convoluted, a typedef behaves the same as a type alias. For example, the typedef can be used as follows:

\begin{cpp}
IntPtr p;
\end{cpp}

Type aliases and typedefs are not entirely equivalent, though. Compared to typedefs, type aliases are more powerful when used with templates, but that is a topic covered in Chapter 12 because it requires more details about templates.

\begin{myNotic}{NOTE}
Always prefer type aliases over typedefs.
\end{myNotic}

\mySubsubsection{1.1.33.}{Type Inference}

Type inference allows the compiler to automatically deduce the type of an expression. There are two keywords for type inference: auto and decltype.

\mySamllsection{The auto Keyword}

The auto keyword has a number of different uses:

\begin{itemize}
\item
To deduce a function’s return type, as explained earlier in this chapter

\item
To define structured bindings, as explained earlier in this chapter

\item
To deduce the type of an expression, as discussed in this section

\item
To deduce the type of non-type template parameters; see Chapter 12

\item
To define abbreviated function templates; see Chapter 12

\item
To use with decltype(auto); see Chapter 12

\item
To write functions using the alternative function syntax; see Chapter 12

\item
To write generic lambda expressions; see Chapter 19, “Function Pointers, Function Objects, and Lambda Expressions”
\end{itemize}

auto can be used to let the compiler automatically deduce the type of a variable at compile time. The following statement shows the simplest use of the auto keyword in that context:

\begin{cpp}
auto x { 123 }; // x is of type int.
\end{cpp}

In this example, you don’t win much by typing auto instead of int; however, it becomes useful for more complicated types. Suppose you have a function called getFoo() that has a complicated return type. If you want to assign the result of calling getFoo() to a variable, you can spell out the complicated type, or you can simply use auto and let the compiler figure it out:

\begin{cpp}
auto result { getFoo() };
\end{cpp}

This has the added benefit that you can easily change the function’s return type without having to update all the places in the code where that function is called.

\mySamllsection{The auto\& Syntax}

Using auto to deduce the type of an expression strips away reference and const qualifiers. Suppose you have the following:

\begin{cpp}
const string message { "Test" };
const string& foo() { return message; }
\end{cpp}

You can call foo() and store the result in a variable with the type specified as auto, as follows:

\begin{cpp}
auto f1 { foo() };
\end{cpp}

Because auto strips away reference and const qualifiers, f1 is of type string, and thus a copy is made! If you want a reference-to-const, you can explicitly make it a reference and mark it const, as follows:

\begin{cpp}
const auto& f2 { foo() };
\end{cpp}

Earlier in this chapter, the as\_const() utility function is introduced. It returns a reference-to-const version of its reference parameter. Be careful when using as\_const() in combination with auto. Since auto strips away reference and const qualifiers, the following result variable has type string, not const string\&, and hence a copy is made:

\begin{cpp}
string str { "C++" };
auto result { as_const(str) };
\end{cpp}

\begin{myWarning}{WARNING}
Always keep in mind that auto strips away reference and const qualifiers and thus creates a copy! If you do not want a copy, use auto\& or const auto\&.
\end{myWarning}

\mySamllsection{The auto* Syntax}

The auto keyword can also be used for pointers. Here’s an example:

\begin{cpp}
int i { 123 };
auto p { &i };
\end{cpp}

The type of p is int*. There is no danger here to accidentally make a copy, unlike when working with references as discussed in the previous section. However, when working with pointers, I do recommend using the auto* syntax as it more clearly states that pointers are involved, for example:

\begin{cpp}
auto* p { &i };
\end{cpp}

Additionally, using auto* versus just auto does resolve a strange behavior when using auto, const, and pointers together. Suppose you write the following:

\begin{cpp}
const auto p1 { &i };
\end{cpp}

Most of the time, this is not doing what you expect it to do!

Often, when you use const, you want to protect the thing to which the pointer is pointing to. You would think that p1 is of type const int*, but in fact, the type is int* const, so it’s a const pointer to a non-const integer! Putting the const after the auto as follows doesn’t help; the type is still int* const:

\begin{cpp}
auto const p2 { &i };
\end{cpp}

When you use auto* in combination with const, then it is behaving as you would expect. Here’s an example:

\begin{cpp}
const auto* p3 { &i };
\end{cpp}

Now p3 is of type const int*. If you really want a const pointer instead of a const integer, you put the const at the end:

\begin{cpp}
auto* const p4 { &i };
\end{cpp}

p4 has type int* const.

Finally, with this syntax you can make both the pointer and the integer constant:

\begin{cpp}
const auto* const p5 { &i };
\end{cpp}

p5 is of type const int* const. You cannot achieve this if you omit the *.

\mySamllsection{Copy List vs. Direct List Initialization}

There are two types of initializations that use braced initializer lists:

\begin{itemize}
\item
Copy list initialization: T obj = \{arg1, arg2, ...\};

\item
Direct list initialization: T obj \{arg1, arg2, ...\};
\end{itemize}

In combination with auto type deduction, there is an important difference between copy- and direct list initialization. Here is an example:

\begin{cpp}
// Copy list initialization
auto a = { 11 }; // initializer_list<int>
auto b = { 11, 22 }; // initializer_list<int>

// Direct list initialization
auto c { 11 }; // int
auto d { 11, 22 }; // Error, too many elements.
\end{cpp}

For copy list initialization, all the elements in the braced initializer must be of the same type. For example, the following does not compile:

\begin{cpp}
auto b = { 11, 22.33 }; // Compilation error
\end{cpp}

\mySamllsection{The decltype Keyword}

The decltype keyword takes an expression as argument and computes the type of that expression, as shown here:

\begin{cpp}
int x { 123 };
decltype(x) y { 456 };
\end{cpp}

In this example, the compiler deduces the type of y to be int because that is the type of x.

The difference between auto and decltype is that decltype does not strip reference and const qualifiers. Take, again, a function foo() returning a reference-to-const string. Defining f2 using decltype as follows results in f2 being of type const string\&, and thus no copy is made:

\begin{cpp}
decltype(foo()) f2 { foo() };
\end{cpp}

On first sight, decltype doesn’t seem to add much value. However, it is powerful in the context of templates, discussed in Chapters 12 and 26.

\mySubsubsection{1.1.34.}{The Standard Library}

C++ comes with a Standard Library, which contains a lot of useful classes that can readily be used in your code. The benefit of using these classes is that you don’t need to reinvent their functionality, and you don’t need to waste time on implementing things that have already been implemented for you. Another benefit is that the classes available in the Standard Library are heavily tested and verified for correctness by thousands of users. The Standard Library classes are also optimized for performance, so using them will most likely result in better performance compared to making your own implementation.

A lot of functionality is provided by the Standard Library. Chapters 16 to 24 provide more details; however, when you start working with C++, it is good to have an idea of what the Standard Library can do for you from the beginning. This is especially important if you are a C programmer. As a C programmer, you might try to solve problems in C++ the same way you would solve them in C, but in C++ there is probably an easier and safer solution to the problem that involves using Standard Library classes.

That is the reason why this chapter already introduces some Standard Library classes, such as std::string, array, vector, pair, and optional. These are used throughout examples in this book from the beginning, to make sure you get into the habit of using Standard Library classes. Many more classes are introduced in Chapters 16 to 24.






