

\mySubsubsection{2.2.1.}{Format Strings}



\mySubsubsection{2.2.2.}{Argument Indices}



\mySubsubsection{2.2.3.}{Printing to Different Destinations}


\CXXTwentythreeLogo{-35}{-59}
\mySubsubsection{2.2.4.}{Compile-Time Verification of Format Strings}


\mySamllsection{Non-Compile-Time Constant Format Strings}

\mySamllsection{Handling Errors in Non-Compile-Time Constant Format Strings}



\mySubsubsection{2.2.5.}{Format Specifiers}


\CXXTwentythreeLogo{-35}{-59}
\mySubsubsection{2.2.6.}{Formatting Escaped Characters and Strings}

\mySamllsection{width}

\mySamllsection{[fill]align}

\mySamllsection{sign}

\mySamllsection{\#}

\mySamllsection{type}

\mySamllsection{precision}

\mySamllsection{0}

\mySamllsection{L}

The optional L specifier enables locale-specific formatting. This option is valid only for arithmetic types, such as integers, floating-point types, and Booleans. When used with integers, the L option specifies that the locale-specific digit group separator character must be used. For floating-point types, it means to use the locale-specific digit group and decimal separator characters. For Boolean types output in textual form, it means to use the locale-specific representation of true and false.

When using the L specifier, you have to pass an std::locale instance as the first parameter to std::format(). This works only with format(), not with print() and println(). Here is an example that formats a floating-point number using the nl locale:

\begin{cpp}
float f { 1.2f };
cout << format(std::locale{ "nl" }, "|{:Lg}|\n", f); // |1,2|
\end{cpp}

Locales are discussed in Chapter 21.

\CXXTwentythreeLogo{-35}{-49}
\mySubsubsection{2.2.7.}{Formatting Ranges}




\mySubsubsection{2.2.8.}{Support for Custom Types}


