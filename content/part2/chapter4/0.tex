\noindent
\textbf{WHAT’S IN THIS CHAPTER?}

\begin{itemize}
\item
The definition of programming design

\item
The importance of programming design

\item
The aspects of design that are unique to C++

\item
The two fundamental themes for effective C++ design: abstraction and reuse

\item
The different types of code available for reuse

\item
The advantages and disadvantages of code reuse

\item
Guidelines for choosing a library to reuse

\item
Open-source libraries

\item
The C++ Standard Library
\end{itemize}

Before writing a single line of code in your application, you should design your program. What data structures will you use? What classes will you write? This plan is especially important when you program in groups. Imagine sitting down to write a program with no idea what your co-worker, who is working on the same program, is planning! In this chapter, you’ll learn how to use the Professional C++ approach to C++ design.

Despite the importance of design, it is probably the most misunderstood and underused aspect of the software-engineering process. Too often, programmers jump into applications without a clear plan: they design as they code. This approach can lead to convoluted and overly complicated designs. It also makes development, debugging, and maintenance tasks more difficult.

Although it seems counterintuitive, investing extra time at the beginning of a project to design it properly actually saves time over the life of the project.












