\noindent
\textbf{内筒概要}

\begin{itemize}
\item
编程设计的定义

\item
编程设计的重要性

\item
C++独有的设计

\item
高效的C++设计有两个基本主题:抽象和重用

\item
可重用的不同类型的代码

\item
代码重用的优点和缺点

\item
选择库的指导原则

\item
开源库

\item
C++标准库
\end{itemize}

写应用程序的单行代码之前,应该设计你的程序。会使用哪些数据结构?会编写哪些类?这个计划在团队合作编程时尤为重要。想象一下,没有了解任何计划有就开始编写程序,对与你合作同事来说,是件多么可怕的事情呀!本章中,将学习如何使用专业的C++方法来进行编程设计。

尽管设计非常重要,但它是软件工程过程中最容易误解和最未充分利用的方面。通常,开发者在没有明确计划的情况下开始编写应用程序:边编码,边设计。这种方法可能导致设计变得复杂和过于混乱。这也使得开发、调试和维护任务变得更加困难。

虽然这看似违反直觉,但投入更多时间在项目开始时设计,实际上会节省项目生命周期中的时间,即“磨刀不误砍柴工”。












