In this chapter, you learned about the professional C++ approach to design. I hope that it convinced you that software design is an important first step in any programming project. In fact, it is not just the first step, but designs need to be kept up-to-date with each incremental improvement of the code.

You learned about some of the aspects of C++ that make design difficult, including the multiparadigm capabilities of the language supporting both object-oriented and procedural designs, its large feature set and Standard Library, and its facilities for writing generic code. With this information, you are better prepared to tackle C++ design.

This chapter introduced two design themes. The first theme, the concept of abstraction, or separating interface from implementation, permeates this book and should be a guideline for all your design work.

The second theme, the notion of reuse, both of code and designs, also arises frequently in real-world projects, and in this book. You learned that your C++ designs should include both reuse of code, in the form of libraries and frameworks, and reuse of ideas and designs, in the form of techniques and patterns. You should write your code to be as reusable as possible. Also remember about the tradeoffs and about specific guidelines for reusing code, including understanding the capabilities and limitations, the performance, licensing and support models, the platform limitations, prototyping, and where to find help. You also learned about performance analysis and big-O notation. Now that you understand the importance of design and the basic design themes, you are ready for the rest of Part II.

Chapter 5 describes strategies for using the object-oriented aspects of C++ in your design.