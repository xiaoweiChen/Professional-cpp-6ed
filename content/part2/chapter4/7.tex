本章中,了解了专业的 C++ 设计方法。在任何一个编程项目中,软件设计都是一个重要的第一步。事实上,它不仅仅是第一步,设计还需要随着代码的每次增量改进保持最新。

了解到 C++ 的一些使设计变得困难的方面,包括语言支持面向对象和过程设计的多范式能力,其庞大的功能集和标准库,以及编写泛型代码的设施。有了这些信息,就能更好地应对 C++ 设计。

本章介绍了两个设计主题。第一个主题是抽象的概念,或者说是将接口与实现分离,这一主题贯穿了整本书,应该是设计工作的指导原则。

第二个主题是重用的概念,包括代码和设计的重用,这在实际项目和本书中经常出现。了解到, C++ 设计应该包括代码的重用,形式为库和框架,以及想法和设计的重用,形式为技术和模式。应该尽可能地编写可重用的代码,同时也要记住关于权衡以及重用代码的具体指导原则,包括理解功能和能力、性能、许可和支持模型、平台限制、原型制作以及在哪里寻求帮助。还学习了性能分析和大 O 符号。现在了解了设计的重要性和基本设计主题,已经准备好学习第二部分的其他内容。

第 5 章描述了在设计中,使用 C++ 面向对象的策略。