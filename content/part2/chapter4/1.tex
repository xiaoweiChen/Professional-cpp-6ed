The first step when starting a new program, or a new feature for an existing program, is to analyze the requirements. This involves having discussions with your stakeholders. A vital outcome of this analysis phase is a functional requirements document describing what exactly the new piece of code has to do, but it does not explain how it has to do it. Requirement analysis can also result in a nonfunctional requirements document describing how the final system should be, compared to what it should do. Examples of non-functional requirements are that the system needs to be secure, extensible, satisfy certain performance criteria, and so on.

Once all requirements have been collected, the design phase of the project can start. Your program design, or software design, is the specification of the architecture that you will implement to fulfill all functional and non-functional requirements of the program. Informally, the design is how you plan to write the program. You should generally write your design in the form of a design document. Although every company or project has its own variation of a desired design document format, most design documents share the same general layout, which includes two main parts:

\begin{itemize}
\item
The gross subdivision of the program into subsystems, including interfaces and dependencies between the subsystems, data flow between the subsystems, input and output to and from each subsystem, and a general threading model

\item
The details of each subsystem, including subdivision into classes, class hierarchies, data structures, algorithms, specific threading models, and error-handling specifics
\end{itemize}

The design documents usually include diagrams and tables showing subsystem interactions and class hierarchies. The Unified Modeling Language (UML) is the industry standard for such diagrams and is used for diagrams in this and subsequent chapters. See Appendix D, “Introduction to UML,” for a brief introduction to the UML syntax. With that being said, the exact format of the design document is less important than the process of thinking about your design.

\begin{myNotic}{NOTE}
The point of designing is to think about your program before you write it.
\end{myNotic}

You should generally try to make your design as good as possible before you begin coding. The design should provide a map of the program that any reasonable programmer could follow in order to implement the application. Of course, it is inevitable that the design will need to be modified once you begin coding and you encounter issues that you didn’t think of earlier. Software-engineering processes have been designed to give you the flexibility to make these changes. Scrum, an agile software development methodology, is one example of such an iterative process whereby the application is developed in cycles, known as sprints. With each sprint, designs can be modified, and new requirements can be taken into account. Chapter 28, “Maximizing Software Engineering Methods,” describes various software-engineering process models in more detail.






























