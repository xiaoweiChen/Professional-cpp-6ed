You should design code that both you and other programmers can reuse. This rule applies not only to libraries and frameworks that you specifically intend for other programmers to use but also to any class, subsystem, or component that you design for a program. You should always keep in mind the following mottos:

\begin{itemize}
\item
“Write once, use often.”

\item
“Try to avoid code duplication.”

\item
“DRY—Don’t Repeat Yourself.”
\end{itemize}

There are several reasons for this:

\begin{itemize}
\item
Code is rarely used in only one program. You can be sure that your code will be used again somehow, so design it correctly to begin with.

\item
Designing for reuse saves time and money. If you design your code in a way that precludes future use, you ensure that you or your partners will spend time reinventing the wheel later when you encounter a need for a similar piece of functionality.

\item
Other programmers in your group must be able to use the code that you write. You are probably not working alone on a project. Your co-workers will appreciate your efforts to offer them well-designed, functionality-packed libraries and pieces of code to use. Designing for reuse can also be called cooperative coding.

\item
Lack of reuse leads to code duplication; code duplication leads to a maintenance nightmare. If a bug is found in duplicated code, it has to be fixed in all places where it got duplicated. Whenever you find yourself copy-pasting a piece of code, you have to at least consider moving it out to a helper function or class.

\item
You will be the primary beneficiary of your own work. Experienced programmers never throw away code. Over time, they build a personal library of evolving tools. You never know when you will need a similar piece of functionality in the future.
\end{itemize}

\begin{myWarning}{WARNING}
When you design or write code as an employee of a company, the company, not you, generally owns the intellectual property rights. It is often illegal to retain copies of your designs or code when you terminate your employment with the company. The same is also true when you are self-employed and working for clients.
\end{myWarning}







