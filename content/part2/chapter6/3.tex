By reading this chapter, you learned how you should design reusable code. You read about the philosophy of reuse, summarized as “write once, use often,” and learned that reusable code should be both general purpose and easy to use. You also discovered that designing reusable code requires you to use abstraction, to structure your code appropriately, and to design good interfaces.

This chapter presented specific tips for structuring your code: to avoid combining unrelated or logically separate concepts, to use templates for generic data structures and algorithms, to provide appropriate checks and safeguards, and to design for extensibility.

This chapter also presented six strategies for designing interfaces: to follow familiar ways of doing things, to not omit required functionality, to present uncluttered interfaces, to provide documentation, to provide multiple ways to perform the same functionality, and to provide customizability. It also discussed how to reconcile the often-conflicting demands of generality and ease of use.

The chapter concluded with SOLID, an easy-to-remember acronym that describes the most important design principles discussed in this and other chapters.

This is the last chapter of the second part of the book, which focuses on discussing design themes at a higher level. The next part delves into the implementation phase of the software engineering process, with details of C++ coding.