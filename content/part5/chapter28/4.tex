
不可能有任何书籍或工程理论完全符合项目或组织的需求。我建议尽可能多地学习各种方法,并设计自己的流程。将不同方法论的概念结合起来,可能比想象的要容易。例如,RUP可选地支持类似XP的方法。以下是构建软件工程流程的一些提示。

\mySubsubsection{28.4.1.}{开放接受新想法}

一些工程技术乍一看似乎很疯狂,或者不太可能奏效。将软件工程方法论中的新创新视为改进现有流程的一种方式,有机会时尝试一下。如果XP听起来很有趣,但不确定它是否适合您的组织,看看是否可以逐渐引入它,一次尝试几个原则,或者在一个较小的试点项目中尝试。

\mySubsubsection{28.4.2.}{将新想法放在桌面上}

工程团队很可能由来自不同背景的人组成,您可能有创业老手、长期顾问、新毕业生和博士。每个人都有不同的经验和不同的想法,认为软件项目应该如何进行。有时最好的流程是将这些不同背景思想进行融合。

\mySubsubsection{28.4.3.}{识别哪些有效,哪些无效}

项目结束时(最好是Scrum方法论中的冲刺回顾期间),让团队一起评估流程。有时有一个主要问题,没有人注意到,直到整个团队停下来思考。也许有一个大家都知道的问题,但没有人讨论。

考虑哪些地方出了问题,看看这些部分如何得到修复,有些组织要求在提交源代码之前进行正式的代码审查。如果代码审查太长太无聊,以至于没有人能做得很好,作为一个团队讨论代码审查技术。

同时考虑哪些地方做得好,这些部分如何得到扩展。例如,维护特性任务作为一个团队可编辑的维基百科有效,则可以花些时间使网站变得更好。

\mySubsubsection{28.4.4.}{不要逃避}

无论流程是由经理强制执行,还是由团队自定义,都有其存在的理由。如果您认为流程有缺陷或过于复杂,可以与经理谈谈。不要避开流程,因为它最终会对您造成困扰。


