\noindent
\textbf{WHAT’S IN THIS CHAPTER?}

\begin{itemize}
\item
What a software life cycle model is, with examples of the waterfall model, the sashimi model, spiral-like models, and agile

\item
What software engineering methodologies are, with examples of the Unified Process, Rational Unified Process, Scrum, Extreme Programming, and Software Triage

\item
What version control means
\end{itemize}

\noindent
\textbf{WILEY.COM DOWNLOADS FOR THIS CHAPTER}

This chapter starts the last part of this book, which is about software engineering. This part describes software engineering methods, code efficiency, testing, debugging, design techniques, design patterns, and how to target multiple platforms.

When you first learned how to program, you were probably on your own schedule. You were free to do everything at the last minute if you wanted to, and you could radically change your design during implementation. When coding in the professional world, however, programmers rarely have such flexibility. Even the most liberal engineering managers admit that some amount of process is necessary. These days, knowing the software engineering process is as important as knowing how to code.

In this chapter, I will survey various approaches to software engineering. I will not go into great depth on any one approach—there are plenty of excellent books on software engineering processes. My intention is to cover some different types of processes in broad strokes so you can compare and contrast them. I will try not to advocate or discourage any particular methodology. Rather, I hope that by learning about the tradeoffs of several different approaches, you’ll be able to construct a process that works for you and the rest of your team. Whether you’re a contractor working alone on projects or your team consists of hundreds of engineers on several continents, understanding different approaches to software development will help you with your job on a daily basis.

The final part of this chapter discusses version control systems that make it easy to manage source code and keep track of its history. A version control system is mandatory in every company to avoid a source code maintenance nightmare. Even for one-person projects, I highly recommend you use such a system.







