通过解决下面的练习,可以练习本章讨论的内容。所有练习的解决方案都可以在本书的网站\url{www.wiley.com/go/proc++6e}下载到源码。然而,若在练习中卡住了,在从网站上寻找解决方案之前,可以考虑先重读本章的部分内容,试着自己找到答案。

\begin{itemize}
\item
Exercise 30-1: What are the three types of testing?

\item
Exercise 30-2: Make a list of unit tests that you can think of for the following piece of code:

\begin{cpp}
export class Foo
{
    public:
        // Constructs a Foo. Throws invalid_argument if a >= b.
        explicit Foo(int a, int b) : m_a { a }, m_b { b }
        {
            if (a >= b) {
                throw std::invalid_argument { "a should be less than b." };
            }
        }
        int getA() const { return m_a; }
        int getB() const { return m_b; }
    private:
        int m_a { 0 };
        int m_b { 0 };
};
\end{cpp}

\item
Exercise 30-3: If you are using Visual C++, implement the unit tests that you’ve listed in Exercise 30-2 using the Visual C++ Testing Framework.

\item
Exercise 30-4: Suppose you have written a function to calculate the factorial of a number.
The factorial of a number n, written as n!, is the product of all numbers 1 to n. For example, 3! equals 1×2×3. You decide to follow the advice given in this chapter and to write unit tests for your code. You run the code to calculate 10!; it produces 36288000. You write a unit test that verifies that the code produces 36288000 when asked to calculate the factorial of 10. What do you think of such a unit test?
\end{itemize}
