
单元测试是防止缺陷的第一道防线,但它们只是更大测试过程中的一个部分。更高层次的测试专注于产品部件如何协同工作,而不是单元测试相对狭窄的关注点。某种程度上,更高层次的测试更难以编写,因为不太清楚需要编写哪些测试,但只有测试了部件如何协同工作,才能说程序运行正常。

\mySubsubsection{30.4.1.}{集成测试}

集成测试覆盖组件相遇的区域。与单元测试不同,集成测试通常涉及两个或更多类。集成测试擅长测试两个组件之间的交互,通常由两个不同的开发者编写。实际上,编写集成测试的过程揭示了设计之间的不兼容性。

\mySamllsection{示例集成测试}

由于没有硬性规则来确定应该编写哪些集成测试,一些示例可能有助于了解何时使用集成测试。以下场景描述了适合集成测试的情况,但它们并不涵盖所有可能的情况。就像单元测试一样,随着时间的推移,将逐渐培养出对有用集成测试的直觉。

\mySamllsection{基于JSON的文件序列化}

假设项目包括一个持久层,用于将某些类型的对象保存到磁盘并从磁盘读取。使用JSON格式是当前流行的数据序列化方式,组件的逻辑分解可能包括一个位于自定义文件API之上的JSON转换层,这两个组件都可以进行彻底的单元测试。JSON层可以有单元测试,确保不同类型的对象正确转换为JSON并从JSON填充。文件API可以有测试,用于在磁盘上读取、写入、更新和删除文件。

当这些模块开始协同工作时,集成测试就适用了。应该有一个集成测试,该测试通过JSON层将对象保存到磁盘,然后读取它并将其与原始对象进行比较。因为该测试覆盖了两个模块,所以它是一个基本的集成测试。

\mySamllsection{共享资源的读者和写者}

想象一个包含共享数据结构的程序。例如,一个股票交易程序可以有一个买入和卖出请求的队列。与接收股票交易请求相关的组件可以向队列添加订单,与执行股票交易相关的组件可以从队列中获取数据。可以对队列类进行大量单元测试,但在与实际使用它的组件一起测试之前,不知道设计时的假设是否正确。

好的集成测试使用股票请求组件和股票交易组件作为队列类的客户端。可以编写一些示例订单,并确保它们通过客户端组件成功进入和退出队列。

\mySamllsection{第三方库的包装器}

集成测试并不总是需要发生在自己的代码的集成点上。很多时候,集成测试是为了测试与第三方库之间的交互。

例如,可能正在使用一个数据库连接库与关系型数据库系统进行通信。也许为该库构建了一个面向对象的包装器,该包装器增加了连接缓存的支持或提供了更友好的界面。这是一个重要的集成点,尽管包装器可能为数据库提供了更有用的接口,但它可能引入了对原始库的潜在误用。

换句话说,编写包装器是一件好事,但如果引入了错误的包装器,那将是一场灾难。

\mySamllsection{集成测试技术}

实际编写集成测试时,集成测试与单元测试之间的界限往往很模糊。如果一个单元测试被修改以触及另一个组件,它是否突然变成了一个集成测试?从某种程度上说,答案无关紧要,因为好的测试就是好的测试,无论测试类型如何。我建议将集成测试和单元测试的概念作为两种测试方法,但避免陷入给每个测试贴上类别的困境。

实现方面,集成测试通常使用单元测试框架编写,进一步模糊了它们的区别。事实上,单元测试框架提供了一种编写是/否测试,并产生有用结果的简单方法。无论测试是关注单一功能单元还是两个组件的交集,从框架的角度来看,这几乎没有区别。

然而,出于性能或组织原因,可能希望尝试将单元测试与集成测试分开。例如,团队可能决定每个人都必须在提交新代码之前运行集成测试,但对无关的单元测试可以稍微宽松一些,将两种类型的测试分开也增加了结果的价值。如果JSON类测试中发生测试失败,将清楚地表明是该类中的错误,而不是该类与文件API之间的交互错误。

\mySubsubsection{30.4.2.}{系统测试}

系统测试在集成测试之上,甚至更高一个层次。这些测试审视程序的整体,系统测试经常使用虚拟用户来模拟人类与程序的交互。当然,虚拟用户必须编写一个动作脚本来执行,其他系统测试依赖于脚本或一组固定的输入和预期输出。

与单元测试和集成测试一样,单个系统测试执行特定的测试并期望特定的结果。通常,系统测试用于确保不同的特性以组合的方式进行工作。

理论上,一个完全系统测试的程序将,包含对每个特性的每个可能性的测试。这种方法很快就会变得难以处理,但仍然应该努力测试许多特性的组合。例如,一个图形程序可能有一个系统测试,该测试导入一个图像,对其进行旋转,应用模糊滤镜,转换为黑白,然后保存。该测试将保存的图像与包含预期结果的文件进行比较。

不幸的是,关于系统测试,很难制定具体的规则,它们高度依赖于实际应用程序。对于没有用户交互处理文件的程序,系统测试可以像单元测试和集成测试一样编写。对于图形程序,虚拟用户方法可能是最佳选择。对于服务器应用程序,可能需要构建模拟网络流量客户端的测试桩。重要的是,实际上是在测试程序的真实使用情况,而不仅仅是其一部分。

\mySubsubsection{30.4.3.}{回归测试}

回归测试更像是一个测试概念,而不是特定类型的测试。其思想是,当特性工作正常,开发者往往会将其搁置一旁,并假设它将继续正常工作。不幸的是,新特性和其他代码更改往往会导致之前正常工作的功能出现问题。

回归测试通常作为对那些基本完成且工作的特性的“常识”检查。如果回归测试编写得很好,当引入导致特性故障的更改时,测试将不再通过。

如果公司有一支质量保证测试团队,回归测试可能以手动测试的形式出现。测试人员像用户一样进行一系列步骤,逐渐测试以前版本中工作的每个特性。如果仔细执行,这种方法既彻底又准确,但不可扩展。

另一个极端,可以构建一个完全自动化的系统,该系统以虚拟用户的方式执行每个功能。这将是脚本编写的一个挑战,但有一些商业和非商业软件包可以简化各种类型应用程序的脚本编写。

一个中间方法称为冒烟测试。一些测试只会测试最重要的功能子集,这些功能应该工作。如果某个部分出现问题,应该立即显现。如果冒烟测试通过,可以用更严格的手动或自动化测试所取代。冒烟测试的概念最早是在电子学领域引入的。在构建一个电路,包括不同组件如真空管、电阻器等之后,问题就是:“它组装正确吗?”一个解决方案是:“插上电源,打开,看看是否冒烟。”如果冒烟(短路产生高温),设计可能错误,或者组装可能错误。通过观察哪个部分冒烟,从而判断错误。

一些缺陷就像噩梦一样:既可怕又反复出现。反复出现的缺陷令人沮丧,是对工程资源的浪费。为了防止缺陷反复出现,应该为修复的缺陷编写回归测试。通过为修复的缺陷编写测试,既证明了缺陷已经修复,又设置了一个警报,如果缺陷再次出现,会触发这个警报——例如,的更改回滚或以其他方式撤销,或者两个分支没有正确合并到主开发分支。当一个已经修复的缺陷的回归测试失败时,修复起来应该很容易,因为回归测试可以引用原始的缺陷编号,并描述第一次是如何修复的。






