作为软件工程师,在测试中的角色可能从基本的单元测试责任,到完全管理自动化测试系统不等。由于测试角色和风格变化很大,以下是一些基于我经验的建议:

\begin{itemize}
\item
花些时间设计自动化测试系统,一个全天候运行的系统可以快速检测到失败。一个在失败时自动向工程师发送电子邮件或在房间中播放音乐剧的系统,可以提高问题的可见性。

\item
不要忘记压力测试。即使数据库访问类的完整套件单元测试通过了,当被几十个线程同时使用时,也可能出现问题。应该在产品可能面临的最极端的真实环境中测试产品。

\item
各种平台上进行测试,或者在与客户系统紧密相似的平台上进行测试。在多个平台上进行的测试,可在同一物理机器上运行几个不同平台的虚拟机环境。

\item
可以编写一些测试来故意向系统中注入错误。例如,可以编写一个在文件读取时删除文件的测试,或者模拟网络操作期间的网络故障。

\item
错误和测试密切相关,错误修复应通过编写回归测试来验证。带有测试的注释可以引用原始错误编号。

\item
不要删除失败的测试。当同事在调试错误时,如果发现删除了测试,可能会来找您。
\end{itemize}

\begin{myWarning}{WARNING}
最重要的建议是记住测试是软件开发的一部分。如果同意这一点,并在开始编码之前接受它,当功能完成后,仍然需要做更多工作来证明它的工作情况。
\end{myWarning}