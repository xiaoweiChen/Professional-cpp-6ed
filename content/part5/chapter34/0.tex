\noindent
\textbf{内筒概要}

\begin{itemize}
\item
如何编写在多个平台上运行的代码

\item
如何将不同的编程语言混合在一起
\end{itemize}

本章的所有代码示例都可以在\url{https://github.com/Professional-CPP/edition-6}获得。

C++程序可以被编译以在多种计算平台上运行,并且已经严格定义该语言,确保在一个平台上用C++编程与在另一个平台上用C++编程相似。尽管语言已经标准化,但在用C++编写专业质量的程序时,平台差异最终还是会显现出来。即使开发工作被限制在特定的平台上,编译器之间的小差异也可能引发大问题。本章探讨了在存在多个平台和多种编程语言的世界中编程的必要复杂性。

本章的第一部分概述了C++程序员遇到的与平台相关的问题,平台是构成开发和/或运行时系统的所有细节的集合。例如,平台可能是运行在Windows 11上的Intel Core i7处理器上的Microsoft Visual C++ 2022编译器,或可能是运行在PowerPC处理器上的Linux上的GCC 13.2编译器。这两个平台都能够编译和运行C++程序,但它们之间存在显著差异。

本章的第二部分探讨了C++如何与其他编程语言交互。虽然C++是一种通用编程语言,但它并不总是完成工作的最佳工具。通过多种机制,可以将C++代码与其他语言集成。


