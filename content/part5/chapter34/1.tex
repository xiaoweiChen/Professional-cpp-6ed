
C++语言遇到平台问题的原因有几个。C++是一种高级语言,标准没有指定某些低级细节,对象在内存中的布局是由编译器决定的,而不是由标准指定的(从C++23开始,成员的顺序由标准定义,必须与类定义中声明的顺序相同——旧标准允许编译器重新排序成员)。不同的编译器可以为对象使用不同的内存布局。

C++还面临着在不具有标准实现的情况下,提供标准语言和标准库的挑战。C++编译器和库供应商对规范的不同解释,可能导致在从一个系统转移到另一个系统时出现问题。

C++在选择语言提供哪些功能作为标准方面是有选择性的。尽管存在标准库,程序通常需要语言或标准库不提供的功能。这种功能通常来自第三方库或平台本身,可能会有很大的差异。

\mySubsubsection{34.1.1.}{架构问题}

架构一词通常指的是程序运行的处理器或处理器族,运行Windows或Linux的标准PC通常运行在x86或x64架构上,而较旧的macOS版本通常在PowerPC架构上。作为一种高级语言,C++封装了这些架构之间的差异。例如,Core i7处理器可能有一个指令执行与六个PowerPC指令相同的功能。作为C++开发者,不需要知道这种差异是什么,甚至不需要知道其存在。使用高级语言的一个优点是,编译器负责将代码转换成处理器的本地汇编代码格式。

然而,处理器差异有时会上升到C++代码层面。首先讨论整数的大小,对于编写跨平台代码是很重要的。其他的除非正在进行特别低层的工作,否则不会经常遇到,但仍应该知道它们的存在。

\mySamllsection{整数大小}

C++标准没有定义整数类型的确切大小,标准只在[basic.fundamental]部分说了以下内容:

\begin{myNotic}{NOTE}
有五种标准的有符号整数类型:signed char、short int、int、long int 和 long long int。这个列表中,每种类型至少提供与列表中它前面的类型一样多的存储空间。
\end{myNotic}

标准确实对这些类型的大小提供了一些提示,但从未给出确切的大小。实际大小取决于编译器。如果想编写真正的跨平台代码,不能依赖于这些类型。本章末尾的一个练习要求进一步调研这个问题。

除了这些核心语言的整数类型之外,C++标准库确实定义了许多具有明确指定大小的类型,所有这些类型都定义在<Cstdint>中的std命名空间中,其中一些类型是可选的。以下是一个概述:

% Please add the following required packages to your document preamble:
% \usepackage{longtable}
% Note: It may be necessary to compile the document several times to get a multi-page table to line up properly
\begin{longtable}{|l|l|}
\hline
\textbf{类型} & \textbf{描述}                                                                                                                                \\ \hline
\endfirsthead
%
\endhead
%
\begin{tabular}[c]{@{}l@{}}int8\_t\\ int16\_t\\ int32\_t\\ int64\_t\end{tabular} &
\begin{tabular}[c]{@{}l@{}}大小恰好为8、16、32或64位的有符号整数。在特殊平台上,这些类型中的一些\\可能不存在。例如,特殊平台不支持8位类型,int8\_t将根本不存在。\end{tabular} \\ \hline
\begin{tabular}[c]{@{}l@{}}int\_fast8\_t\\ int\_fast16\_t\\ int\_fast32\_t\\ int\_fast64\_t\end{tabular} &
\begin{tabular}[c]{@{}l@{}}大小至少为8、16、32或64位的有符号整数。对于这些,编译器应使用满足要求\\的最快的整数类型。\end{tabular} \\ \hline
\begin{tabular}[c]{@{}l@{}}int\_least8\_t\\ int\_least16\_t\\ int\_least32\_t\\ int\_least64\_t\end{tabular} &
\begin{tabular}[c]{@{}l@{}}大小至少为8、16、32或64位的有符号整数——存在最小的此类类型。即使是在\\特殊平台上,这些类型也保证存在。例如,一个24位字节的平台将把int\_least8\_t\\和int\_least16\_t都别名为它的24位char类型。\end{tabular} \\ \hline
intmax\_t     & 编译器支持的最大尺寸的整数类。                                                                                    \\ \hline
intptr\_t     & \begin{tabular}[c]{@{}l@{}}足够大以存储指针的整数类型。这个类型也是可选的,大多数编译器都支持。\end{tabular} \\ \hline
\end{longtable}

还有可用的无符号变体,如uint8\_t、uint\_fast8\_t等。

\begin{myNotic}{NOTE}
当编写跨平台代码时,我建议使用<cstdint>类型,而不是基本整数类型。
\end{myNotic}

\mySamllsection{二进制兼容性}

不能将一个为Core i7计算机编译的程序,在基于PowerPC的Mac上运行。这两个平台不兼容,两个处理器使用不同的指令集。编译一个C++程序时,源代码转换成计算机执行的二进制指令。这个二进制格式是由平台定义的,而不是由C++语言定义的。

支持非二进制兼容平台的一个解决方案是,使用目标平台上的编译器分别构建每个版本。

另一个解决方案是交叉编译。在平台X上进行开发,但希望程序在平台Y和Z上运行时,可以在平台X上使用交叉编译器为平台Y和Z生成二进制代码。

也可以开源程序,将源代码提供给最终用户时,可以在自己的系统上本地编译,并在自己的机器上构建正确二进制格式的程序版本。如第4章中所述,开源软件变得越来越流行。主要原因是允许其他开发者协作开发软件,增加可以在其上运行的平台数量。

\mySamllsection{地址大小}

当有人描述一个架构为64位时,很可能是指地址大小为64位,即8字节。一般来说,具有较大地址大小的系统可以处理更多内存,并且可能更快地处理复杂程序。

指针是内存地址,所以与地址大小固有地联系在一起。有时,开发者会认为指针总是8字节,但这是不正确的。例如,考虑以下代码,它输出一个指针的大小:

\begin{cpp}
int *ptr { nullptr };
println("ptr size is {} bytes", sizeof(ptr));
\end{cpp}

如果这个程序在32位x86系统上编译和执行,输出将是:

\begin{shell}
ptr size is 4 bytes
\end{shell}

如果在x86-64系统上编译和运行它,输出将是:

\begin{shell}
ptr size is 8 bytes
\end{shell}

从开发者的角度来看,指针大小不同的后果是不能将指针等同于4或8字节。更一般地说,需要意识到大多数大小不是由C++标准规定的。标准只说long至少和int一样长,int至少和short一样长等。

指针的大小也不一定与整数的大小相同。在64位平台上,指针是64位的,但整数可能是32位的。将64位指针转换为32位整数会导致丢失32个关键位!标准在<cstdint>中定义了一个std::uintptr\_t类型别名,是一个至少足够大以容纳指针的整数类型。这个类型的定义是可选的,但所有编译器都支持。

\begin{myWarning}{WARNING}
永远不要假设指针是32位或64位。除非使用std::uintptr\_t,否则永远不要将指针转换为整数。
\end{myWarning}

\mySamllsection{字节序}

所有现代计算机都以二进制形式存储数字,但在两个平台上的同一数字的表示可能并不相同。这听起来似乎很矛盾,但表示数字有两种方法,都有其合理性。

现今的大多数计算机都是按字节寻址的,内存中的每个字节都有一个唯一的内存地址。C++中的数字类型通常占用多个字节,一个short可能占用2个字节。想象一下,程序包含以下代码:

\begin{cpp}
short myShort { 513 };
\end{cpp}

二进制中,数字513是0000 0010 0000 0001。这个数字包含16个1和0,或者16位。因为每个字节有8位,所以计算机需要2个字节来存储这个数字。每个单独的内存地址包含1个字节,所以需要将数字分成多个字节。假设short是2个字节,数字分成两部分。数字的高位被放入高阶字节,数字的低位放入低阶字节。这时,高阶字节是0000 0010,低阶字节是0000 0001。

现在数字已经分成内存大小的部分,剩下的问题是如何将它们存储在内存中。需要两个字节,但字节的顺序不确定,这取决于所讨论系统的架构。

表示数字的一种方法是将高阶字节首先放入内存,然后是低阶字节。这种策略称为大端序,因为数字的大部分先出现。PowerPC和SPARC处理器使用大端序方法。其他一些处理器,如x86,将字节以相反的顺序排列,将低阶字节首先放入内存。这种方法称为小端序,因为数字的小部分先出现。一个架构可以选择其中一种方法,通常基于向后兼容性。对于好奇的人来说,大端序和小端序这两个术语比现代计算机早了几百年。乔纳森·斯威夫特在18世纪的小说《格列佛游记》中创造了这些术语,用来描述关于从哪一端敲开鸡蛋辩论中的对立阵营。

无论特定架构使用的是大端序还是小端序,程序都可以继续使用数字值,而不必注意机器是大端序还是小端序。只有当数据在架构之间迁移时,字节序才变得重要。例如,正在通过网络发送二进制数据,可能需要考虑另一系统的字节顺序。一个解决方案是使用标准网络字节顺序,终是大端序。在通过网络发送数据之前,将其转换为大端序,每当从网络接收数据时,将其从大端序转换为系统的本地字节序。

类似地,如果正在将二进制数据写入文件,可能需要考虑当该文件在字节序相反的系统上打开时会发生什么。

标准库包括在<bit>中定义的std::endian枚举,可以用来确定当前系统是大端还是小端系统。以下代码可输出系统的本地字节序:

\begin{cpp}
switch (endian::native)
{
    case endian::little:
        println("Native ordering is little-endian.");
        break;
    case endian::big:
        println("Native ordering is big-endian.");
        break;
}
\end{cpp}

\mySubsubsection{34.1.2.}{实现问题}

当编写C++编译器时,它是由试图遵循C++标准的人设计的。C++标准超过2000页,并且用散文、伪代码和示例混合编写。根据这样的标准实现编译器的两个人,可能会以不同的方式解释每一条规定信息,或者没有捕捉到每一个边缘情况,所以编译器会有错误。

\mySamllsection{编译器的怪癖和扩展}

没有简单的规则来寻找或避免编译器错误,能做的最好的事情是跟上编译器更新的步伐,也许订阅编译器的邮件列表或新闻组。如果怀疑遇到了编译器错误,对错误信息或见证的条件进行简单的网络搜索,可能会发现一个变通方案或补丁。

编译器在处理最近标准更新中添加的语言特性时,容易出现问题。最近几年,主要编译器的供应商在添加对最新特性的支持方面相当迅速。

另一个需要注意的问题是,编译器经常包含它们自己的语言扩展,而不向开发者明确指出。可变长度堆栈数组(VLAs)不是C++语言的一部分,但它们是C语言的一部分。一些编译器支持C和C++标准,并允许在C++代码中使用VLAs。这样的一个编译器是g++。以下代码在g++编译器中编译和运行如预期:

\begin{cpp}
int i { 4 };
char myStackArray[i]; // Not a standard language feature!
\end{cpp}

一些编译器扩展可能很有用,但如果可能在某个时候更换编译器,应该看看你的编译器是否有避免使用此类扩展的严格模式。例如,将-pedantic标志传递给g++编译前面的代码会产生以下警告:

\begin{shell}
warning: ISO C++ forbids variable length array 'myStackArray' [-Wvla]
\end{shell}

C++规范允许通过\#pragma机制进行某种类型的编译器语言扩展。\#pragma是一个预处理指令,其行为由实现定义。如果实现不理解该指令,就会忽略它,一些编译器允许开发者使用\#pragma临时关闭编译器的警告。

\mySamllsection{库实现}

编译器很可能包括C++标准库的一个实现,但因为标准库用C++编写,所以没有必要使用与编译器捆绑的实现。可以使用一个第三方的标准库,例如:一个为了速度优化的标准库,或者可以编写自己的标准库。

当然,标准库实现者面临的问题与编译器编写者相同:标准容易受到理解能力的影响。某些实现可能做出了与需求不兼容的权衡,一个实现可能优化了速度,而另一个实现可能专注于优化能够在运行时捕获误用的能力。

使用标准库实现或任何第三方库时,考虑设计者在开发过程中所做的权衡非常重要。第4章包含了使用库涉及的更详细讨论。

\mySamllsection{处理不同的实现}

如前几节所述,并不是所有的编译器和标准库实现都表现得完全一样。进行跨平台开发时,作为开发者,很可能使用的是单一的工具链,即单一的编译器与单一的标准库实现。亲自验证所有代码更改与产品必须构建的所有工具链不太可能,解决方案是持续集成和自动化测试。

应该设置一个持续集成环境,自动在所有需要支持的工具链上构建所有代码更改。当在其中一个工具链上构建失败,应该自动通知开发者导致构建失败的原因。

并不是所有的开发环境都使用相同的项目文件,来描述所有的源文件、编译器开关等。如果需要支持多个环境,为每个环境手动维护独立的项目文件是一个维护噩梦。最好是使用单一类型的项目文件或单一集合的构建脚本,然后自动转换为特定工具链的具体项目文件或具体构建脚本。有这样一个工具叫做CMake。源文件、编译器开关、要链接的库等的集合在CMake配置文件中描述,这些文件也支持脚本编写。然后CMake工具自动生成项目文件,例如:用于Windows上的Visual C++开发,或用于macOS上的Xcode开发。

当持续集成环境产生了构建,就应该触发自动化测试。这应该运行一系列测试脚本,以验证产生的可执行文件的正确行为。这一步中,如果出现问题,也应该自动通知开发者。

\mySubsubsection{34.1.3.}{平台特定的功能}

C++是一种极好的通用编程语言。得益于标准库,C++语言包含了如此多的功能,以至于普通的开发者可以在不超出内置功能的情况下,快乐地用C++编程。然而,专业程序需要更多工具。本节列出了一些重要的功能,这些功能由平台或第三方库提供,而不是由C++语言或C++标准库提供:

\begin{itemize}
\item
图形用户界面:大多数商业程序运行在具有图形用户界面的操作系统中,包含可点击的按钮、可移动的窗口和分层菜单等元素。C++和C语言一样,没有这些元素的概念。要用C++编写图形应用程序,可以使用绘制窗口、通过鼠标接受输入和其他图形任务的特定平台库。更好的选择是使用第三方库,如wxWidgets (\url{wxwidgets.org})、Qt (\url{qt.io})、Uno (\url{platform.uno})等,这些库为构建图形应用程序提供了抽象层。这些库通常支持许多不同的目标平台。

\item
网络:互联网改变了我们编写应用程序的方式。如今大多数应用程序通过网络检查更新,游戏提供网络多人模式。C++还没有提供网络机制,尽管存在几个标准库,编写网络软件的最常见方式是通过称为套接字的抽象。大多数平台上都可以找到套接字库的实现,提供了简单的面向过程的在网络间传输数据的方法。一些平台支持像C++中的I/O流那样的基于流的网络系统。还有提供网络抽象层的第三方网络库可用,这些库通常支持许多不同的目标平台。选择支持IP独立的网络库比仅支持IPv4的库要好,因为IPv6已经在使用了。

\item
操作系统事件和应用程序交互:纯C++代码中,与周围操作系统和其他应用程序的交互很少。没有平台扩展的标准C++程序中,几乎只能获得命令行参数。例如,复制和粘贴(与操作系统的“剪贴板”交互)在C++中并不直接支持。可以使用平台提供的库或使用支持多个平台的第三方库。例如,wxWidgets和Qt都是抽象剪贴板操作并支持多个平台的库的例子。

\item
底层文件:第13章解释了C++中的标准I/O,包括文件的读写。许多操作系统提供自己的文件API,这些API通常与C++标准库中的文件类不兼容。这些库通常提供操作系统特定的文件工具,获取当前用户的主目录的机制。

\item
线程:在C++03或更早的版本中,并不直接支持单个程序内的并发线程执行。自从C++11以来,线程支持库已经包含在标准库中,如第27章所解释的,C++17增加了并行算法,如第20章所讨论的。如果需要比标准库提供的更强大的线程功能,那么需要使用第三方库,两个选项是Intel的线程构建模块(TBB)和STE||AR Group的高性能并行X(HPX)库。
\end{itemize}

\begin{myNotic}{NOTE}
如果正在进行跨平台开发,并且需要C++语言或C++标准库未提供的功能,应该尝试找到提供所需功能的第三方跨平台库。如果直接开始使用特定平台的API,那么将复杂化跨平台代码,因为需要为每个需要支持的平台实现该功能。
\end{myNotic}

\begin{myNotic}{NOTE}
使用第三方库时,请获取这些库的源代码,并使用需要的确切工具链自行构建。
\end{myNotic}

















