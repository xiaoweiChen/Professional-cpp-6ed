If you take away one point from this chapter, it should be that C++ is a flexible language. It exists in the sweet spot between languages that are too tied to a particular platform, and languages that are too high-level and generic. Rest assured that when you develop code in C++, you aren’t locking yourself into the language forever. C++ can be mixed with other technologies and has a solid history and code base that will help guarantee its relevance in the future.

In Part V of this book, I discussed software engineering methods, writing efficient C++, testing and debugging techniques, design techniques and patterns, and cross-platform and cross-language application development. This is a terrific way to end your journey through professional C++ programming because these topics help good C++ programmers become great C++ programmers. By thinking through your designs, experimenting with different approaches in object-oriented programming, selectively adding new techniques to your coding repertoire, and practicing testing and debugging techniques, you’ll be able to take your C++ skills to the professional level.

