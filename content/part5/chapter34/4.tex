通过解决下面的练习,可以练习本章讨论的内容。所有练习的解决方案都可以在本书的网站\url{www.wiley.com/go/proc++6e}下载到源码。若在练习中卡住了,可以考虑先重读本章的部分内容,试着自己找到答案,再在从网站上寻找解决方案。

\begin{itemize}
\item
\textbf{练习34-1}: 编写一个程序,输出所有标准C++整数类型的大小。尝试在不同的编译器和不同的平台上编译并执行。

\item
\textbf{练习34-2}: 本章介绍了整数值的大端和小端编码概念。还解释说,在网络上,总是建议使用大端编码,并按需要进行转换。编写一个程序,可以双向转换16位无符号整数的小端和大端编码。注意使用的数据类型,编写一个main()函数来测试你的函数。

\item
\textbf{加分练习}:能为32位整数做同样的事情吗?

\item
\textbf{练习34-3}: “转变范式”部分中的网络示例展示了,如何使用C风格的API与C++一起使用,可能有点抽象。它没有提供一个完整的实现,因为这将需要网络代码,而这些代码既不包含在C标准库中,也不包含在C++标准库中。这个练习中,我们看看一个更小的C风格库,可能会在自己的C++代码中使用它。这个C风格库基本上由两个函数组成。第一个函数,reverseString(),分配一个新的字符串并初始化为给定源字符串的反转。第二个函数,freeString(),释放reverseString()分配的内存。以下是声明,带有描述性注释:

\begin{cpp}
/// <summary>
/// Allocates a new string and initializes it with the reverse of a given string.
/// </summary>
/// <param name="string">The source string to reverse.</param>
/// <returns>A newly allocated buffer filled with the reverse of the
/// given string.
/// The returned memory needs to be freed with freeString().</returns>
char* reverseString(const char* string);

/// <summary>Frees the memory allocated for the given string.</summary>
/// <param name="string">The string to deallocate.</param>
void freeString(char* string);
\end{cpp}

如何在C++代码中使用这个“库”?

\item
\textbf{练习34-4}: 本章中关于在C和C++代码中混合的例子都是关于从C++调用C代码,当限制自己使用C已知的数据类型时,也可以相反。这个练习中,将结合两个方向。编写一个名为writeTextFromC(const char*)的C函数,调用一个名为writeTextFromCpp(const char*)的C++函数,该函数使用std::println()将给定的字符串打印到标准输出。为了测试代码,用C++编写一个main()函数,并调用C函数writeTextFromC()。
\end{itemize}
