
程序应该包含一些功能,以便在不可避免的错误出现时更容易进行调试。本节描述了一些此类功能的示例,并在适当的情况下提供可以并入程序的示例实现。

\mySubsubsection{31.4.1.}{错误日志}

设想以下场景:刚刚发布了旗舰产品的新版本,第一个用户报告说程序“停止工作”。我们试图从用户那里获取更多信息,最终发现程序在操作过程中死亡。用户记不清他在做什么或是否有错误消息。这时应该如何调试这个问题?

同样的场景,除了从用户那里获取有限的信息外,还能够检查用户计算机上的错误日志。日志中,看到程序的这样一条消息:“错误:无法打开config.xml文件。”查看生成该错误消息的位置附近的代码,发现没有检查文件是否成功打开的情况下,就从文件中进行了读取。现在,已经找到了错误的原因!

错误日志是将错误消息写入持久存储的过程,以便在应用程序或甚至机器崩溃后能够访问。尽管有上述示例场景,可能仍对此策略持怀疑态度。如果程序遇到错误,其行为难道不会很明显吗?如果出了问题,用户不会注意到吗?正如前面的例子所示,用户的报告并不总是准确或完整的。许多程序,如操作系统内核和Unix上的长时间运行的守护进程(如inetd(互联网服务守护进程)或syslogd),都不是交互式的,在机器上无人值守运行,这些程序与用户沟通的唯一方式就是通过错误日志。许多情况下,程序可能还希望从某些错误中自动恢复,并隐藏这些错误。尽管如此,拥有这些错误的日志对于提高程序的整体稳定性非常有价值。

程序应该在遇到错误时记录错误,如果用户报告了一个错误,将能够检查机器上的日志文件,以查看程序在遇到错误之前是否报告了错误。不幸的是,错误日志是与平台相关的:C++不包含标准的日志机制。平台特定日志机制示例包括Unix中的syslog设施和Windows中的事件报告API,应该查阅开发平台的文档。还有一些开源的跨平台日志框架实现,这里有两个例子:

\begin{itemize}
\item
log4cpp :\url{log4cpp.sourceforge.net}

\item
Boost.Log :\url{boost.org}
\end{itemize}

现在已经确信日志是程序中的一个很好的功能,能很想在代码的每几行中记录消息,以便在错误发生时,能够追踪执行的代码路径。这类日志消息称为跟踪日志。

然而,由于两个原因,不应该将这些跟踪写入日志文件。首先,写入持久存储很慢。即使在异步写入日志的系统上,记录这么多信息也会减慢程序执行的速度。其次,也是最重要的,想在跟踪中放入的大部分信息不适合最终用户看到。这只会混淆用户,导致不必要的开销。在正确的情境下,跟踪日志是一种重要的调试技术,如下节所述。

以下是一些应该记录的错误类型的具体指南:

\begin{itemize}
\item
不可恢复的错误,如系统调用意外失败。

\item
管理员可以采取措施的错误,如内存不足、数据文件格式错误、无法写入磁盘或网络连接中断。

\item
意外的错误,如未预料到的代码路径或变量值意外。另一方面,代码应该“预期”用户输入无效数据,并适当处理。意外的错误表示程序中存在错误。

\item
潜在的安全漏洞,如未经授权的地址尝试网络连接,或尝试过多的网络连接(拒绝服务)。
\end{itemize}

日志错误中应该包含什么?至少,每个日志错误应包含错误发生的时间和日期,以及错误消息本身。在多线程应用程序中,记录导致错误的线程ID也很有用。从C++23开始,标准库包括std::stacktrace(第14章中讨论过),允许在程序的执行过程中时刻获取堆栈跟踪,可以将这样的堆栈跟踪或至少部分跟踪与放在每个日志错误中。

记录警告或可恢复的错误也很有用,给读者一个调查,确定是否有可以避免记录它们的选项。

大多数日志API允许指定一个日志级别或错误级别,通常至少有“错误”、“警告”或“信息”。可以在比“错误”更严重的日志级别下,记录非错误条件。例如,可能想要记录应用程序中的重大状态变化,或程序的启动和关闭。还可以考虑在运行时给用户提供一种调整程序日志级别的方法,以便可以自定义生成的日志量。

\mySubsubsection{31.4.2.}{调试跟踪}

调试复杂问题时,公共错误信息通常不包含足够的信息。经常需要完整的代码路径跟踪,或者在错误出现之前的变量值。除了基本消息之外,有时在调试跟踪中包含以下信息也很有帮助:

\begin{itemize}
\item
如果是多线程程序,则线程ID

\item
生成跟踪的函数名称

\item
该函数所在的源文件名称
\end{itemize}

可以通过特殊的调试模式或在环形缓冲区中,向程序添加这种跟踪。这两种技术都在以下各节中详细解释。注意,多线程程序中,必须确保你的跟踪日志线程安全。有关多线程编程的详细信息,请参见第27章。

\begin{myWarning}{WARNING}
跟踪文件可以写成文本格式,如果这样做,不要记录过多的细节。不想通过日志文件泄露知识产权!如果黑客能够获取跟踪输出,记录过多的实现细节也可能很危险!
\end{myWarning}

\mySamllsection{调试模式}

添加调试跟踪的第一种技术是,为程序提供一个调试模式。调试模式下,程序将跟踪输出写入标准错误或文件,并在执行过程中可能进行检查。有几种方法可以将调试模式添加到程序中,以下所有示例都是以文本格式编写的跟踪。

\mySamllsection{启动时调试模式}

启动时调试模式允许应用程序根据命令行参数选择,是否以调试模式运行。这种策略包括在“发布”二进制文件中的调试代码,并允许在客户现场启用调试模式。这确实要求用户重新启动程序以在调试模式下运行,这可能会阻止开发者获取有关某些错误的信息。

以下示例是一个实现启动时调试模式的简单程序,这个程序并没有做什么有用的事情;只是为了演示这个技术。

所有的日志功能都封装在一个Logger类中。这个类有两个静态数据成员:日志文件的名称和一个表示日志启用或禁用的布尔值。这个类有一个公共的静态成员函数模板log(),像这样的可变参数模板在第26章中讨论过。每次调用log()时,日志文件都会打开、刷新和关闭。这可能会稍微降低性能,但确实保证了正确的日志记录,这更为重要。

\begin{cpp}
class Logger
{
    public:
        static void enableLogging(bool enable) { ms_loggingEnabled = enable; }
        static bool isLoggingEnabled() { return ms_loggingEnabled; }
        template <typename... Args>
        static void log(const Args&... args)
        {
            if (!ms_loggingEnabled) { return; }
            ofstream logfile { ms_debugFilename, ios_base::app };
            if (logfile.fail()) {
                println(cerr, "Unable to open debug file!");
                return;
            }
            print(logfile, "{:L} UTC: ", chrono::system_clock::now());
            // Use a fold-expression; see Chapter 26.
            (logfile << ... << args);
            logfile << endl;
        }
    private:
        static inline const string ms_debugFilename { "debugfile.out" };
        static inline bool ms_loggingEnabled { false };
};
\end{cpp}

以下是一个辅助宏,用于简化记录某些内容。这是一个接受可变数量参数的宏。可以使用\_\_VA\_ARGS\_\_来访问这些参数,这个宏使用了\_\_func\_\_,这是C++标准定义的一个预定义变量,包含当前函数的名称。

\begin{cpp}
#define LOG(...) Logger::log(__func__, "(): ", __VA_ARGS__)
\end{cpp}

这个宏将代码中每个对LOG()的调用替换为对Logger::log()的调用。宏自动将函数名作为第一个参数包含到Logger::log()中。例如,这样调用宏:

\begin{cpp}
LOG("The value is: ", value);
\end{cpp}

LOG()宏将这个调用替换为以下内容:

\begin{cpp}
Logger::log(__func__, "(): ", "The value is: ", value);
\end{cpp}

启动时调试模式需要解析命令行参数,以确定是否应该启用调试模式。遗憾的是,C++中没有用于解析命令行参数的标准功能。这个程序使用一个简单的isDebugSet()函数,来检查所有命令行参数中是否有调试标志,但是一个解析所有命令行参数的函数更复杂。

\begin{cpp}
bool isDebugSet(int argc, char** argv)
{
    auto parameters { views::counted(argv, argc) };
    return ranges::contains(parameters, string_view { "-d" });
}
\end{cpp}

这个示例中,使用了一些测试代码来练习调试模式。定义了两个类,ComplicatedClass和UserCommand。这两个类都定义了一个operator<{}<,将它们的实例写入流。Logger类使用这个运算符将对象转储到日志文件中。

\begin{cpp}
class ComplicatedClass { /* ... */ };
ostream& operator<<(ostream& outStream, const ComplicatedClass& src)
{
    outStream << "ComplicatedClass";
    return outStream;
}

class UserCommand { /* ... */ };
ostream& operator<<(ostream& outStream, const UserCommand& src)
{
    outStream << "UserCommand";
    return outStream;
}
\end{cpp}

这里有一些带有多个日志调用的测试代码:

\begin{cpp}
UserCommand getNextCommand(ComplicatedClass* obj)
{
    UserCommand cmd;
    return cmd;
}

void processUserCommand(const UserCommand& cmd)
{
    // Details omitted for brevity.
}

void trickyFunction(ComplicatedClass* obj)
{
    LOG("given argument: ", *obj);
    for (size_t i { 0 }; i < 100; ++i) {
        UserCommand cmd { getNextCommand(obj) };
        LOG("retrieved cmd ", i, ": ", cmd);
        try {
            processUserCommand(cmd);
        } catch (const exception& e) {
            LOG("exception from processUserCommand(): ", e.what());
        }
    }
}

int main(int argc, char** argv)
{
    Logger::enableLogging(isDebugSet(argc, argv));
    if (Logger::isLoggingEnabled()) {
        // Print the command-line arguments to the trace.
        for (size_t i { 0 }; i < argc; ++i) {
            LOG("Argument: ", argv[i]);
        }
    }
    ComplicatedClass obj;
    trickyFunction(&obj);
    // Rest of the function not shown.
}
\end{cpp}

运行此应用程序有两种方式:

\begin{shell}
> STDebug
> STDebug -d
\end{shell}

在命令行中指定了 -d 参数时,才会激活调试模式。

\begin{myWarning}{WARNING}
C++中的宏应该尽可能避免使用,因为它们可能很难调试。出于日志记录的目的,使用一个简单的宏是可以接受的,它使得使用日志代码变得更加容易。即便如此,使用第14章讨论的std::source\_location,可以修改示例以避免使用宏。这是本章末尾一个练习。
\end{myWarning}

\mySamllsection{编译时调试模式}

除了通过命令行参数启用或禁用调试模式之外,还可以使用如DEBUG\_MODE这样的预处理器符号以及\#ifdefs来选择性地将调试代码编译到程序中。要生成此程序的调试版本,需要定义符号DEBUG\_MODE进行编译。编译器允许在编译过程中定义符号;详情请参阅编译器的文档。例如,GCC允许通过命令行指定 –Dsymbol。Microsoft Visual C++通过Visual Studio IDE或在使用Visual C++命令行工具时指定 /D symbol。除了使用自定义的DEBUG\_MODE符号之外,还可以使用NDEBUG符号,该符号由编译器为发行版定义,而不为调试版定义。

这种技术的优点是调试代码不会编译到“发行”二进制中,因此不会增加其大小。缺点是在客户现场进行测试或在发现错误后无法进行调试。

下载源码库中的CTDebug.cpp,里面提供了一个示例实现。关于这个实现的一个重要说明是,包含了对LOG()宏的定义:

\begin{cpp}
#ifdef DEBUG_MODE
    #define LOG(...) Logger::log(__func__, "(): ", __VA_ARGS__)
#else
    #define LOG(...) (void)0
#endif
\end{cpp}

也就是说,如果未定义DEBUG\_MODE,则所有对LOG()的调用都将替换为无操作,即所谓的no-ops。

\begin{myWarning}{WARNING}
请注意,不要将必须执行以确保程序正常运行的代码放在LOG()调用中。例如,以下代码行可能会引起麻烦:

\begin{cpp}
LOG("Result: ", calculateResult());
\end{cpp}

如果未定义DEBUG\_MODE,预处理器将所有LOG()调用替换为no-ops,所以calculateResult()的调用也将移除!
\end{myWarning}

由于如果未定义DEBUG\_MODE,则会移除日志代码,这可能导致某些变量未使用,从而触发编译器警告。使用[[maybe\_unused]]属性(参见第1章)可以避免此类警告。例如:

\begin{cpp}
int main([[maybe_unused]] int argc, [[maybe_unused]] char** argv)
{
#ifdef DEBUG_MODE
    // Print the command-line arguments to the trace.
    for (size_t i { 0 }; i < argc; ++i) { LOG("Argument: ", argv[i]); }
#endif
    ComplicatedClass obj;
    trickyFunction(&obj);
    // Rest of the function not shown.
}
\end{cpp}

\mySamllsection{运行时调试模式}

提供调试模式的最灵活方式是,在运行时启用或禁用。提供此功能的一种方式是提供一个控制调试模式的异步接口,这个接口可以是一个异步命令,对应用程序进行进程间调用(例如,使用套接字、信号或远程过程调用)。这个接口也可以是用户界面中的一个菜单命令。C++没有提供执行进程间通信或实现用户界面的标准方式,因此不展示这种技术的示例。

\mySamllsection{环形缓冲区}

调试模式对于调试可复现的问题和运行测试非常有用,但错误经常在程序以非调试模式运行时出现,等到客户启用调试模式时,已经来不及获取有关错误的信息。解决这个问题的一个办法是在程序中始终启用跟踪,通常只需要最近的跟踪来调试程序,所以应该只存储程序执行过程中最近时刻的跟踪,实现的一种方式是通过日志文件的轮换。

但出于性能考虑,最好不要将这些跟踪持续记录到磁盘上,而应该将它们存储在内存中,并提供一种机制,以便在需要时,将所有跟踪消息转储到标准错误或日志文件中。

一种常见的技术是使用环形缓冲区(或循环缓冲区)来存储固定数量的消息,或固定内存量中的消息。当缓冲区填满时,从缓冲区的开始处再次写入消息,覆盖旧消息。这个循环可以无限次重复。以下部分提供了环形缓冲区的实现,并展示如何在程序中使用。

\mySamllsection{环形缓冲区的接口}

以下RingBuffer类提供了一个简单的环形缓冲区来存储消息。客户端在构造函数中指定条目数,并使用addEntry()成员函数添加消息。当条目数超过允许的数量,新的条目就会覆盖缓冲区中最旧的条目。缓冲区还提供了在消息添加到缓冲区时将条目输出到流的选项。客户端可以在构造函数中指定一个输出流,并使用setOutput()成员函数重置它。最后,operator<{}<将整个缓冲区输出到一个输出流,这个实现使用了第26章的可变参数模板成员函数。

\begin{cpp}
export class RingBuffer final
{
    public:
        // Constructs a ring buffer with space for numEntries.
        // Entries are written to *outStream as they are queued (optional).
        explicit RingBuffer(std::size_t numEntries = DefaultNumEntries,
            std::ostream* outStream = nullptr);
        // Adds an entry to the ring buffer, possibly overwriting the
        // oldest entry in the buffer (if the buffer is full).
        template <typename... Args>
        void addEntry(const Args&... args)
        {
            std::ostringstream oss;
            std::print(oss, "{:L} UTC: ", std::chrono::system_clock::now());
            // Use a fold-expression; see Chapter 26.
            (oss << ... << args);
            addStringEntry(std::move(oss).str());
        }

        // Streams the buffer entries, separated by newlines, to outStream.
        friend std::ostream& operator<<(std::ostream& outStream, RingBuffer& rb);

        // Streams entries as they are added to the given stream.
        // Specify nullptr to disable this feature.
        // Returns the old output stream.
        std::ostream* setOutput(std::ostream* newOutStream);

    private:
        std::vector<std::string> m_entries;
        std::vector<std::string>::iterator m_next;

        std::ostream* m_outStream { nullptr };
        bool m_wrapped { false };

        static constexpr std::size_t DefaultNumEntries { 500 };

        void addStringEntry(std::string entry);
};
\end{cpp}

\mySamllsection{环形缓冲区的实现}

这个环形缓冲区的实现存储了一个固定数量的字符串对象。这种方法当然不是最有效的解决方案,其他的可能性是为缓冲区提供固定数量的字节数。然而,除非正在编写一个高性能应用程序,否则这个实现应该够用了。

对于多线程程序,将线程ID添加到每个跟踪条目中是非常有用的。当然,在使用多线程应用程序之前,必须使环形缓冲区线程安全。有关多线程编程,请参见第27章。

下面是这些实现的代码:

\begin{cpp}
// Initialize the vector to hold exactly numEntries. The vector size
// does not need to change during the lifetime of the object.
// Initialize the other members.
RingBuffer::RingBuffer(size_t numEntries, ostream* outStream)
    : m_entries { numEntries }, m_outStream { outStream }, m_wrapped { false }
{
    if (numEntries == 0) {
        throw invalid_argument { "Number of entries must be > 0." };
    }
    m_next = begin(m_entries);
}

// The addStringEntry algorithm is pretty simple: add the entry to the next
// free spot, then reset m_next to indicate the free spot after
// that. If m_next reaches the end of the vector, it starts over at 0.
//
// The buffer needs to know if the buffer has wrapped or not so
// that it knows whether to print the entries past m_next in operator<<.
void RingBuffer::addStringEntry(string entry)
{
    // If there is a valid m_outStream, write this entry to it.
    if (m_outStream) { *m_outStream << entry << endl; }
    // Move the entry to the next free spot and increment
    // m_next to point to the free spot after that.
    *m_next = move(entry);
    ++m_next;
    // Check if we've reached the end of the buffer. If so, we need to wrap.
    if (m_next == end(m_entries)) {
        m_next = begin(m_entries);
        m_wrapped = true;
    }
}

// Set the output stream.
ostream* RingBuffer::setOutput(ostream* newOutStream)
{
    return exchange(m_outStream, newOutStream);
}

// operator<< uses an ostream_iterator to "copy" entries directly
// from the vector to the output stream.
//
// operator<< must print the entries in order. If the buffer has wrapped,
// the earliest entry is one past the most recent entry, which is the entry
// indicated by m_next. So, first print from entry m_next to the end.
//
// Then (even if the buffer hasn't wrapped) print from beginning to m_next-1.
ostream& operator<<(ostream& outStream, RingBuffer& rb)
{
    if (rb.m_wrapped) {
        // If the buffer has wrapped, print the elements from
        // the earliest entry to the end.
        copy(rb.m_next, end(rb.m_entries),
        ostream_iterator<string>{ outStream, "\n" });
    }
    // Now, print up to the most recent entry.
    // Go up to m_next because the range is not inclusive on the right side.
    copy(begin(rb.m_entries), rb.m_next,
    ostream_iterator<string>{ outStream, "\n" });
    return outStream;
}
\end{cpp}

\mySamllsection{使用环形缓冲区}

要使用环形缓冲区,可以创建一个实例,并开始向其中添加消息。当想打印缓冲区时,只需使用operator<{}<将缓冲区打印到适当的ostream。以下是之前启动时调试模式程序的修改版本,改为使用环形缓冲区。更改已突出显示。ComplicatedClass和UserCommand类的定义,以及getNextCommand()、processUserCommand()和trickyFunction()函数未显示。

\begin{cpp}
RingBuffer debugBuffer;

#define LOG(...) debugBuffer.addEntry(__func__, "(): ", __VA_ARGS__)

int main(int argc, char** argv)
{
    // Log the command-line arguments.
    for (size_t i { 0 }; i < argc; ++i) {
        LOG("Argument: ", argv[i]);
    }

    ComplicatedClass obj;
    trickyFunction(&obj);

    // Print the current contents of the debug buffer to cout.
    cout << debugBuffer;
}
\end{cpp}

\mySamllsection{显示环形缓冲区内容}

将跟踪调试消息存储在内存中是一个很好的开始,但要使其有用,需要一种方法来访问这些跟踪以进行调试。

程序应该提供一个“钩子”来告诉它导出消息,这个钩子可以与运行时启用调试时使用的接口类似。此外,如果程序遇到致命错误导致其退出,可以在退出前自动将环形缓冲区导出到日志文件。

另一种获取这些消息的方法是获取程序的内存转储。每个平台处理内存转储的方式不同,因此应该查阅您的平台的参考资料或专家意见。

\mySubsubsection{31.4.3.}{断言}

<cassert>定义了一个assert()宏,接受一个布尔表达式,如果表达式计算为false,会打印一个错误消息并终止程序。如果表达式评估为true,则什么也不做。

\begin{myWarning}{WARNING}
通常,应该避免使用可能终止程序的库函数或宏,assert()宏是一个例外。如果触发一个断言,则某些假设是错误的,或者出现了灾难性的、不可恢复的错误,因此明智的做法是在那一刻终止应用程序,而不是继续运行。
\end{myWarning}

断言允许“强制”程序在错误的确切位置显示错误。如果不在那个点上断言,程序可能会继续使用那些不正确的值,并且错误可能直到很久以后才会出现,断言可以在正常情况下更早地检测到错误。

\begin{myNotic}{NOTE}
标准assert()宏的行为取决于NDEBUG预处理器符号:如果该符号未定义,断言发生;否则,会忽略断言。编译器在编译“发布”构建时通常会定义这个符号。如果想在发布构建中保留断言,可能需要更改编译器设置,或编写自己的assert(),使其不受NDEBUG值的影响。
\end{myNotic}

assert()是一个C风格的宏,其实现取决于客户代码是否定义了NDEBUG预处理器符号。这使得<cassert>成为一个非模块化、不可导入的头文件的例子,正如在第11章中解释的那样。使用\#include <cassert>,而不是import <cassert>,来获取对assert()宏的访问。

可以在代码中使用断言,只要“假设”变量状态即可。例如,调用一个声称永远不会返回nullptr的库函数,可以在函数调用后添加一个assert()来确保指针不是nullptr。

然而,应该尽量少做假设。例如,编写一个库函数,不要假设参数有效。相反,检查参数,并在它们无效时返回一个错误代码或抛出一个异常。

通常,断言应仅用于真正有问题的案例,因此在开发过程中遇到断言时,应修复它,而不是简单地禁用断言。

让我们看看如何使用assert()的两个例子。这是一个process()函数,要求传递给函数的vector中包含三个元素:

\begin{cpp}
void process(const vector<int>& coordinate)
{
    assert(coordinate.size() == 3);
    // ...
}
\end{cpp}

如果调用process()函数时向量包含少于或大于三个元素,断言将失败并生成类似于以下的消息(具体消息取决于使用的编译器):

\begin{shell}
Assertion failed: coordinate.size() == 3, file D:\test\test.cpp, line 12
\end{shell}

如果想要一个自定义的错误消息,可以使用以下技巧,使用逗号运算符和括号:

\begin{cpp}
assert(("A custom message...", coordinate.size() == 3));
\end{cpp}

输出将如下所示:

\begin{shell}
Assertion failed: ("A custom message...", coordinate.size() == 3), file D:\test\
test.cpp, line 106
\end{shell}

如果在代码中某个地方,希望assert总是以特定的错误消息失败,可以使用以下技巧:

\begin{cpp}
assert(!"This should never happen.");
\end{cpp}

\begin{myWarning}{WARNING}
请小心!不要将必须正确执行程序的代码放在断言中。例如,以下行代码可能会带来麻烦:

\begin{cpp}
assert(calculateResult() != 0);
\end{cpp}

如果代码的发布构建去除了断言,那么调用calculateResult()的代码也会被去除!
\end{myWarning}

\mySubsubsection{31.4.4.}{崩溃转储}

确保你的程序能够创建崩溃转储,也称为内存转储或核心转储。崩溃转储是一个在应用程序崩溃时创建的转储文件。包含了崩溃时运行线程的信息、所有线程的调用栈等信息。如何创建这样的转储文件取决于平台,因此应该查阅平台的文档,或者使用一个第三方库来为你处理这个问题。Breakpad(\url{github.com/google/breakpad/})就是一个这样的开源跨平台库的例子,可以编写和处理崩溃转储。

还要确保设置了一个符号服务器和一个版本控制服务器。符号服务器用于存储发布软件的二进制文件的调试符号。这些符号后来用于解释从客户那里收到的崩溃转储。第28章中讨论的版本控制服务器,会存储源代码的所有修订版本。在调试崩溃转储时,这个版本控制服务器用于下载创建崩溃转储软件修订版的正确源码。

分析崩溃转储的确切程序取决于平台和编译器,请查阅它们的文档。

根据我的个人经验,一个崩溃转储往往比一千份错误报告更有价值。

















