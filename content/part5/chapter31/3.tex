编写完全没有错误的代码是不可能的,因此调试技能非常重要。然而,以下几点提示可以尽量减少错误的数量:

\begin{itemize}
\item
阅读本书:深入掌握C++语言,特别是指针和内存管理,将这本书推荐给你的朋友和同事,让他们也避免错误。

\item
先设计再编码:在完全没有考虑设计的情况下就开始编写一个功能的代码,往往会导致难以理解且容易出错的复杂设计,这也使更有可能忽略可能的边界情况和错误条件。因此,在开始编写代码之前,要考虑设计。开始实现时,如果遇到之前没有考虑过的问题,不要害怕改变设计,后期对设计进行更改也完全可以。

\item
进行代码审查:专业环境中,每一行代码都应经过同行审查。有时候需要一个新的视角来发现问题。

\item
多次测试:彻底测试代码,并让其他人测试你的代码!他们更有可能发现你未曾想到的问题。

\item
编写自动化单元测试:单元测试旨在测试独立的功能。应该为所有实现的功能编写单元测试,将这些单元测试作为持续集成设置的一部分自动运行,或者在每次本地编译后自动运行。

\item
预期错误条件,并适当处理:特别是,在处理文件和网络连接时,要计划如何处理错误。

\item
使用智能指针避免资源泄漏:智能指针在不再需要时自动释放资源。

\item
不要忽略编译器警告:配置编译器以高警告级别编译。不要盲目忽略警告。理想情况下,应在编译器中启用一个选项,将警告视为错误。使用GCC或Clang时,你可以传递-Werror给编译器,将所有警告视为错误。在Visual C++中,打开项目属性,转到配置属性-> C/C++ -> 常规,并启用将警告视为错误选项。

\item
使用静态代码分析:静态代码分析器通过分析源代码来定位代码中的问题。理想情况下,静态代码分析应在集成开发环境(IDE)中键入代码时实时进行,以便早期发现问题,也可以设置为由构建过程自动运行。互联网上有许多不同的分析器可供选择,既有免费的,也有商业收费的。

\item
使用良好的编码风格:力求可读性和清晰性,使用有意义的名称,不要使用缩写,添加代码注释(不仅仅是接口注释),使用override和explicit关键字等。使其他人更容易理解你的代码。
\end{itemize}