\noindent
\textbf{WHAT’S IN THIS CHAPTER?}

\begin{itemize}
\item
The fundamental law of debugging, and bug taxonomies

\item
Tips for avoiding bugs

\item
How to plan for bugs

\item
The different kinds of memory errors

\item
How to use a debugger to pinpoint code causing a bug
\end{itemize}

\noindent
\textbf{WILEY.COM DOWNLOADS FOR THIS CHAPTER}

Please note that all the code examples for this chapter are available as part of this chapter’s code download on the book’s website at \url{www.wiley.com/go/proc++6e} on the Download Code tab.

Your code will contain bugs. Every professional programmer would like to write bug-free code, but the reality is that few software engineers succeed in this endeavor. As computer users know, bugs are endemic in computer software. The software that you write is probably no exception.

Therefore, unless you plan to bribe your co-workers into fixing all your bugs, you cannot be a professional C++ programmer without knowing how to debug C++ code. One factor that often distinguishes experienced programmers from novices is their debugging skills.

Despite the obvious importance of debugging, it is rarely given enough attention in courses and books. Debugging seems to be the type of skill that everyone wants you to know, but no one knows how to teach. This chapter attempts to provide concrete debugging guidelines and techniques.

This chapter starts with the fundamental law of debugging and bug taxonomies, followed by tips for avoiding bugs. Techniques for planning for bugs include error logging, debug traces, assertions, and crash dumps. Specific tips are given for debugging the problems that arise, including techniques for reproducing bugs and debugging reproducible bugs, nonreproducible bugs, memory errors, and multithreaded programs. The chapter concludes with a step-by-step debugging example.
