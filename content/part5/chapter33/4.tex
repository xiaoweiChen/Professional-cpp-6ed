前面的部分描述了与工厂相关的两种具体模式:抽象工厂模式和工厂方法模式。

还有其他类型的工厂。例如,工厂也可以在单个类中实现,而不是在类层次结构中。这时,工厂上的单个 create() 成员函数接受一个类型或字符串参数,从中决定创建哪个对象,而不是将这项工作委托给具体的子类。这样的函数通常被称为工厂函数,这种工厂模式不提供依赖倒置,也不允许自定义构建过程。

使用工厂函数的一个例子是在第9章中讨论的 pimpl 习语的替代实现,提供了公共接口和提供的功能的具体实现之间的隔离。使用工厂函数的方式如下。首先,公开暴露的是以下内容,包括一个 create() 工厂函数:

\begin{cpp}
// Public interface (to be included in the rest of the program,
// shared from a library, ...)
class Foo
{
    public:
        virtual ~Foo() = default; // Always a virtual destructor!
        // Omitted defaulted copy/move ctor, copy/move assignment op.
        static unique_ptr<Foo> create(); // Factory function.
        // Public functionality...
        virtual void bar() = 0;
    protected:
        Foo() = default; // Protected default constructor.
};
\end{cpp}

接下来,实现对外部隐藏:

\begin{cpp}
// Implementation
class FooImpl : public Foo
{
    public:
    void bar() override { /* ... */ }
};

unique_ptr<Foo> Foo::create()
{
    return make_unique<FooImpl>();
}
\end{cpp}

需要 Foo 实例的客户端,可以按如下方式创建:

\begin{cpp}
auto fooInstance { Foo::create() };
fooInstance->bar();
\end{cpp}
