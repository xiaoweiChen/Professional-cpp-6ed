迭代器模式提供了一种机制,用于将算法或操作与它们操作的数据结构分离。基本上,迭代器允许算法,在不知道实际数据结构的情况下导航数据结构。这种模式似乎与面向对象编程中的基本原则相矛盾,即将数据及其操作行为封装在类中。虽然在某种程度上这个论点是正确的,但迭代器模式并不提倡从类中移除基本行为,解决了紧密耦合数据和行为时通常会出现的两个问题。

紧密耦合数据和行为的第一问题是,排除了在多种数据结构上工作的通用算法。要编写通用算法,需要某种标准机制来导航/访问数据结构的内容,而不需要了解具体结构。

紧密耦合数据和行为的第二问题是,添加新行为有时会很困难,至少需要访问数据对象的源代码。但如果感兴趣的车厢层次结构是第三方框架或库的一部分,但无法更改,该怎么办呢?如果能够在不修改保存数据的原始类层次结构的情况下,添加一个在数据上工作的算法或操作,那就太好了。

已经在前面的章节中看到了迭代器模式的一个例子,即标准库中的迭代器。从概念上讲,标准库迭代器提供了一种机制,允许操作或算法访问序列中的容器元素。这个名字来自英文单词 iterate,意思是“重复”。适用于迭代器,重复在序列中向前移动以到达每个新元素的动作。标准库中,通用算法使用迭代器来访问容器中的元素。通过定义标准的迭代器接口,标准库允许提供可以在适当接口迭代器容器上工作的算法,可以为单个数据结构提供多个不同的迭代器。这允许算法以不同的方式导航数据,例如:对于树形数据结构,可以采用自顶向下和自底向上的遍历。迭代器可以用来实现通用算法,这些算法可以在不了解数据结构的情况下遍历数据结构的内容。图 33.6 显示了迭代器作为核心协调器;操作依赖于迭代器,数据对象提供迭代器。

\myGraphic{0.7}{content/part5/chapter33/images/6.png}{图 33.6}

第 25 章详细说明了,如何为数据结构实现符合标准库要求的迭代器,所以迭代器可以在标准库的通用算法中使用。




































