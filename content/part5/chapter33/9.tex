
装饰器模式如其名字所述:一个“装饰”类。该模式用于在运行时增强或改变类的行为。装饰器类似于派生类,但能够动态地改变装饰类的行为。与派生类相比,装饰器的优势在于它们不具侵入性;允许适应行为,而不需要更改底层类的代码。装饰器还可以轻松组合以实现所需的行为,而无需为每个组合编写派生类。

例如,正在解析一个数据流,并且遇到代表图像的数据,可以暂时用一个 ImageStream 对象来装饰这个流对象。ImageStream 构造函数将流对象作为参数,并内置了图像解析的知识。图像解析时,就可以继续使用原始对象来解析数据流的其余部分。ImageStream 作为装饰器,因为它为现有对象(一个流)添加了新的功能(图像解析)。

\mySubsubsection{33.9.1.}{示例:为网页定义样式}

网页是用一种简单的基于文本的结构编写的,这种结构称为超文本标记语言(HTML)。在 HTML 中,可以使用样式标签,如 <b> 和 </b> 来实现加粗,以及 <i> 和 </i> 来实现斜体。以下 HTML 代码将消息显示为加粗:

\begin{shell}
<b>A party? For me? Thanks!</b>
\end{shell}

下面的代码将消息显示为加粗和斜体:

\begin{shell}
<i><b>A party? For me? Thanks!</b></i>
\end{shell}

HTML 中的段落用 <p> 和 </p> 标签包裹。例如:

\begin{shell}
<p>This is a paragraph.</p>
\end{shell}

假设正在编写一个 HTML 编辑应用程序,用户应该能够输入文本段落并应用一种或多种样式。可以将每种类型的段落都作为新的派生类,如图 33.7 所示,但这设计既繁琐又随着新样式的增加而呈指数增长。

\myGraphic{0.8}{content/part5/chapter33/images/7.png}{图 33.7}

另一种方法是将样式段落视为装饰过的段落。这导致如图 33.8 所示的情况,其中 ItalicParagraph 操作 BoldParagraph,而 BoldParagraph 操作 Paragraph。对象的递归装饰在代码中嵌套样式,正如它们在 HTML 中嵌套一样。

\myGraphic{0.4}{content/part5/chapter33/images/8.png}{图 33.8}

\mySubsubsection{33.9.2.}{实现装饰器}

需要一个 IParagraph 接口:

\begin{cpp}
class IParagraph
{
    public:
        virtual ~IParagraph() = default; // Always a virtual destructor!
        virtual std::string getHTML() const = 0;
};
\end{cpp}

Paragraph 类实现了这个 IParagraph 接口:

\begin{cpp}
class Paragraph : public IParagraph
{
    public:
        explicit Paragraph(std::string text) : m_text { std::move(text) } {}
        std::string getHTML() const override {return format("<p>{}</p>", m_text); }
    private:
        std::string m_text;
};
\end{cpp}

为了装饰一个 Paragraph 对象,需要一个带有样式装饰的 IParagraph 类,每个类都可以从现有的 IParagraph 对象构建。这样,它们都可以装饰 Paragraph 或者一个带有样式的 IParagraph。BoldParagraph 类从 IParagraph 派生,并实现了 getHTML()。关键在于,由于只打算将其用作装饰器,其唯一的public非复制构造函数接受 IParagraph 的引用。

\begin{cpp}
class BoldParagraph : public IParagraph
{
    public:
        explicit BoldParagraph(const IParagraph& paragraph)
        : m_wrapped { paragraph } { }
        std::string getHTML() const override {
            return format("<b>{}</b>", m_wrapped.getHTML()); }
    private:
        const IParagraph& m_wrapped;
};
\end{cpp}

ItalicParagraph类也类似:

\begin{cpp}
class ItalicParagraph : public IParagraph
{
    public:
        explicit ItalicParagraph(const IParagraph& paragraph)
        : m_wrapped { paragraph } { }
        std::string getHTML() const override {
            return format("<i>{}</i>", m_wrapped.getHTML()); }
    private:
        const IParagraph& m_wrapped;
};
\end{cpp}

\mySubsubsection{33.9.3.}{使用装饰器}

从用户的角度来看,装饰器模式之所以吸引人,是因为其易于应用,可使应用就变得透明。BoldParagraph 表现得就像一个 Paragraph,需要注意的是,由于 BoldParagraph 只包含对 IParagraph 的引用,如果以某种方式改变了 IParagraph 的文本,这个变化也会通过 BoldParagraph 显示出来。

下面是一个创建并输出段落的例子,首先输出加粗的段落,然后输出加粗和斜体的段落:

\begin{cpp}
Paragraph text { "A party? For me? Thanks!" };
// Bold
println("{}", BoldParagraph{text}.getHTML());
// Bold and Italic
println("{}", ItalicParagraph{BoldParagraph{text}}.getHTML());
\end{cpp}

输出结果如下所示:

\begin{shell}
<b><p>A party? For me? Thanks!</p></b>
<i><b><p>A party? For me? Thanks!</p></b></i>
\end{shell}
















