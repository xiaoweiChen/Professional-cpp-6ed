
标准库大量使用了C++的两个特性:模板和运算符重载。

\mySubsubsection{16.1.1.}{使用模板}

模板用于支持泛型编程,这使得编写代码可以适用于各种类型的对象(即使是在编写代码时程序员未知晓的对象)。编写模板代码的开发者有责任指定定义这些对象的类的要求,例如:必须有一个用于比较的运算符或一个复制构造函数,或者认为适当的内容,然后确保代码只使用那些所需的功能。创建对象的开发者需要提供模板所需的运算符和成员函数。

不幸的是,许多开发者认为模板是C++中最难的部分,因此倾向于避免使用它们。然而,即使从不编写自己的模板,也需要理解其语法和能力,才能使用标准库。第12章详细描述了模板。如果你跳过了那一章并且不熟悉模板,我建议你先阅读第12章,然后再回来学习更多关于标准库的内容

\mySubsubsection{16.1.2.}{使用运算符重载}

运算符重载是C++标准库广泛使用的另一个特性。第9章有一个专门介绍运算符重载的部分。确保在处理本章及后续章节之前,阅读并理解了那一部分内容。此外,第15章详细介绍了运算符重载,但那些细节不是理解后续章节所必需的。
