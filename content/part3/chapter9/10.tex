
This chapter, along with Chapter 8, provided all the tools you need to write solid, well-designed classes, and to use objects effectively.

You discovered that dynamic memory allocation in objects presents new challenges: you need to implement a destructor, copy constructor, copy assignment operator, move constructor, and move assignment operator, which properly copy, move, and free your memory. You learned how to prevent pass-by-value and assignment by explicitly deleting the copy constructor and assignment operator. You discovered the copy-and-swap idiom to implement copy assignment operators and the move-andswap idiom to implement move assignment operators, as well as learned about the rule of zero.

You read more about different kinds of data members, including static, const, reference-to-const, and mutable members. You also learned about static, inline, and const member functions, member function overloading, and default arguments. This chapter also described nested class definitions, and friend classes, functions, and member functions.

You encountered operator overloading and learned how to overload the arithmetic and comparison operators, both as global functions and as class member functions. You also discovered how the three-way comparison operator makes adding comparison support to your classes so much easier.

Finally, you learned how to take abstraction to the extreme by providing separate interface and implementation classes.

Now that you’re fluent in the language of object-oriented programming, it’s time to tackle inheritance, which is covered next in Chapter 10, “Discovering Inheritance Techniques.”

