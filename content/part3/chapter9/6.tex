类定义不仅可以包含成员函数和数据成员,还可以包含嵌套类和结构体、类型别名和枚举。在类中声明的内容,都在该类的范围内。如果是public的,可以通过使用类名和作用域解析语法在外部访问。

也可以在另一个类定义中提供类定义。SpreadsheetCell 类实际上是 Spreadsheet 类的一部分,它是 Spreadsheet 类的一部分,不妨将其重命名为 Cell:

\begin{cpp}
export class Spreadsheet
{
    public:
        class Cell
        {
            public:
                Cell() = default;
                Cell(double initialValue);
                // Remainder omitted for brevity
        };

        Spreadsheet(std::size_t width, std::size_t height,
            const SpreadsheetApplication& theApp);
        // Remainder of Spreadsheet declarations omitted for brevity
};
\end{cpp}

Cell 类在 Spreadsheet 类中定义,在 Spreadsheet 类外部引用 Cell 时,必须使用 Spreadsheet:: 作用域限定名称。这也适用于成员函数的定义,例如:Cell 的 double 构造函数为:

\begin{cpp}
Spreadsheet::Cell::Cell(double initialValue)
    : m_value { initialValue }
{}
\end{cpp}

必须对 Spreadsheet 类成员函数的返回类型使用这种语法(但不适用于参数):

\begin{cpp}
const Spreadsheet::Cell& Spreadsheet::getCellAt(size_t x, size_t y) const
{
    verifyCoordinate(x, y);
    return m_cells[x][y];
}
\end{cpp}

直接在 Spreadsheet 类中完全定义嵌套的 Cell 类,会使 Spreadsheet 类的定义变得有些臃肿。可以只在 Spreadsheet 类中包含 Cell 的声明,然后定义 Cell 类来缓解这种情况:

\begin{cpp}
export class Spreadsheet
{
    public:
        class Cell;
        Spreadsheet(std::size_t width, std::size_t height,
            const SpreadsheetApplication& theApp);
        // Remainder of Spreadsheet declarations omitted for brevity
};
class Spreadsheet::Cell
{
    public:
        Cell() = default;
        Cell(double initialValue);
        // Omitted for brevity
};
\end{cpp}

正常的访问控制也适用于嵌套类。如果声明了一个私有的或受保护的嵌套类,只能在外部类内部使用。嵌套类可以访问外部类的所protected和private成员,外部类只能访问嵌套类的public成员。











