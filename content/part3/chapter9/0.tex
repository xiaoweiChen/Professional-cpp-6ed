\noindent
\textbf{WHAT’S IN THIS CHAPTER?}

\begin{itemize}
\item
How to make classes friends of other classes

\item
How to use dynamic memory allocation in objects

\item
What the copy-and-swap idiom is

\item
What rvalues and rvalue references are

\item
How move semantics can improve performance

\item
What the rule of zero means

\item
How to use the constexpr and consteval keywords to perform computations at compile time

\item
The different kinds of data members you can have (static, const, constexpr, reference)

\item
The different kinds of member functions you can implement (static, const, inline)

\item
The details of member function overloading

\item
How to work with default arguments

\item
How to use nested classes

\item
What operator overloading is

\item
How to write separate interface and implementation classes
\end{itemize}

\noindent
\textbf{WILEY.COM DOWNLOADS FOR THIS CHAPTER}

Please note that all the code examples for this chapter are available as part of this chapter’s code download on the book’s website at \url{www.wiley.com/go/proc++6e} on the Download Code tab.

Chapter 8, “Gaining Proficiency with Classes and Objects,” started the discussion on classes and objects. Now it’s time to master their subtleties so you can use them to their full potential. By reading this chapter, you will learn how to manipulate and exploit some of the most powerful aspects of the C++ language to write safe, effective, and useful classes.

Many of the concepts in this chapter arise in advanced C++ programming, especially in the C++ Standard Library. Let’s start the discussion with the concept of friends in the C++ world.














