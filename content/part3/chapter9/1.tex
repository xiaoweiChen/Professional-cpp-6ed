在 C++ 中,可以声明其他类、其他类的成员函数或非成员函数为朋友,并可以访问这些类的受保护(protected)和私有(private)数据成员和成员函数。例如,有两个类,分别称为 Foo 和 Bar,可以指定 Bar 类为 Foo 类的友元:

\begin{cpp}
class Foo
{
    friend class Bar;
    // ...
};
\end{cpp}

现在,Bar 类的所有成员函数都可以访问 Foo 类的private和protected数据成员,以及成员函数。

如果只想使 Bar 类的特定成员函数成为 Foo 类的友元。假设 Bar 类有一个名为 processFoo 的成员函数,其参数为 const Foo\&。要使这个成员函数成为 Foo 类的友元,可以这样写:

\begin{cpp}
class Foo
{
    friend void Bar::processFoo(const Foo&);
    // ...
};
\end{cpp}

独立的函数也可以成为类的友元。例如,编写一个函数,将 Foo 对象的所有数据打印到控制台。可能希望这个函数位于 Foo 类之外,因为打印不是 Foo 的核心功能,但该函数应该能够访问对象内部的数据成员以打印它们。以下是将 printFoo() 作为 Foo 类的友元声明的 Foo 类定义:

\begin{cpp}
class Foo
{
    friend void printFoo(const Foo&);
    // ...
};
\end{cpp}

类中的友元声明作为函数的原型,不需要在其他地方声明(就是声明了也无所谓)。

函数的定义:

\begin{cpp}
void printFoo(const Foo& foo)
{
    // Print all data of foo to the console, including
    // private and protected data members.
}
\end{cpp}

可以在类定义之外编写这个函数,就像编写其他函数一样,可以直接访问 Foo 的私有和受保护成员。在函数定义中不需要重复 friend 关键字。

注意,一个类需要知道哪些其他类、成员函数或函数想要成为它的朋友;一个类、成员函数或函数不能声明自己为某个类的友元,以获取该类的非公共成员的访问权限。

friend 类和函数容易滥用;它们允许通过将类的内部暴露给其他类或函数来违反封装原则。因此,应该在有限的情况下使用友元。









