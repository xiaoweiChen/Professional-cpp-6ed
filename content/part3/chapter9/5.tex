
C++ 为数据成员提供了许多的选择。除了在类中声明简单的数据成员外,还可以创建静态数据成员,所有对象共享这些数据成员、常量成员、引用成员、常量引用成员等。本节将解释这些不同类型的数据成员。

\mySubsubsection{9.5.1.}{静态数据成员}

有时,为类的每个对象提供一个变量的副本,是不必要的或行不通的。数据成员可能与类相关,但不适宜每个对象都有自己的副本。例如,想要为每个电子表格提供一个唯一的数字标识符。需要一个从 0 开始的计数器,每个新对象都可以从中获取其 ID。这个电子表格计数器实际上属于 Spreadsheet 类,但不适合每个 Spreadsheet 对象都有自己的副本,所以需要以某种方式保持所有计数器的同步。C++ 通过静态数据成员提供了一个解决方案。静态数据成员是与类关联,而不是与对象关联的数据成员。可以将静态数据成员视为特定于类的全局变量。以下是 Spreadsheet 类的定义,包括新的静态计数器数据成员:

\begin{cpp}
export class Spreadsheet
{
    // Omitted for brevity
    private:
    static std::size_t ms_counter;
};
\end{cpp}

除了在类定义中列出静态类成员,还需要在源文件中为其分配空间,通常是包含类成员函数定义的源文件。可以同时初始化,但其与普通变量和数据成员不同,默认初始化为 0,静态指针初始化为 nullptr。这里为 ms\_counter 分配空间,并初始化为 0 :

\begin{cpp}
size_t Spreadsheet::ms_counter;
\end{cpp}

静态数据成员默认0初始化,也可以将其显式初始化为 0:

\begin{cpp}
size_t Spreadsheet::ms_counter { 0 };
\end{cpp}

代码在函数或成员函数体之外,像声明一个全局变量,只是 Spreadsheet:: 作用域解析指定它是 Spreadsheet 类的一部分。

与普通数据成员一样,访问控制修饰符也适用于静态数据成员。可以将 ms\_counter 数据成员设置为public,但不推荐定义public的数据成员(const 静态数据成员是一个例外)。应该通过public的 getters 和 setters 授予对数据成员的访问权限。如果想授予对静态数据成员的访问权限,可以提供公共静态的 get/set 成员函数。

\mySamllsection{内联变量}

可以将静态数据成员声明为内联,就不需要在源文件中为其分配空间:

\begin{cpp}
export class Spreadsheet
{
    // Omitted for brevity
    private:
        static inline std::size_t ms_counter { 0 };
};
\end{cpp}

注意 inline 关键字。有了这个类定义,下面这一行代码就可以删除了:

\begin{cpp}
size_t Spreadsheet::ms_counter;
\end{cpp}

\mySamllsection{类成员函数内部访问静态数据成员}

可以在类成员函数内部使用静态数据成员,像使用普通数据成员一样。在 Spreadsheet 类中创建一个 m\_id 数据成员,并在 Spreadsheet 构造函数中从 ms\_counter 初始化。以下是带有 m\_id 成员的 Spreadsheet 类定义:

\begin{cpp}
export class Spreadsheet
{
    public:
        // Omitted for brevity
        std::size_t getId() const;
    private:
        // Omitted for brevity
        static inline std::size_t ms_counter { 0 };
        std::size_t m_id { 0 };
};
\end{cpp}

以下是 Spreadsheet 构造函数的实现,分配初始 ID:

\begin{cpp}
Spreadsheet::Spreadsheet(size_t width, size_t height)
    : m_id { ms_counter++ }, m_width { width }, m_height { height }
{
    // Omitted for brevity
}
\end{cpp}

构造函数可以像普通成员一样访问 ms\_counter,复制构造函数也会分配一个新的 ID。因为 Spreadsheet 复制构造函数委派给非复制构造函数,从而非复制构造函数创建新的 ID。

当为对象分配了 ID,就永远不会变,所以不应该在复制赋值操作符中复制 ID,建议将 m\_id 声明为const数据成员:

\begin{cpp}
export class Spreadsheet
{
    private:
        // Omitted for brevity
        const std::size_t m_id { 0 };
};
\end{cpp}

常量数据成员创建后就不能再更改,所以不能在构造函数体内初始化。这样的数据成员必须在类定义内部直接初始化,或者在构造函数的初始化器中初始化,不能在赋值操作符中为这样的数据成员分配新值。对于 m\_id,这不是问题,因为电子表格有一个ID就不会再改变,所以通常会显式删除赋值操作符。

\mySubsubsection{9.5.2.}{constexpr静态数据成员}

类中的数据成员可以声明为 const 或 constexpr,所以在创建和初始化后不能再修改。当常量仅适用于类时,可以使用静态 constexpr(或 constexpr 静态)数据成员来代替全局常量,这也称为类常量。静态 constexpr 数据成员可以是整数类型和枚举类型,甚至可以在不声明为内联变量的情况下,在类定义内部定义和初始化。例如,想要为电子表格指定最大高度和宽度,当用户尝试构造一个宽度或高度大于最大值的电子表格,则使用最大值。可以将最大高度和宽度作为 Spreadsheet 类的静态 constexpr 成员:

\begin{cpp}
export class Spreadsheet
{
    public:
        // Omitted for brevity
        static constexpr std::size_t MaxHeight { 100 };
        static constexpr std::size_t MaxWidth { 100 };
};
\end{cpp}

可以在构造函数中这样使用这些新常量:

\begin{cpp}
Spreadsheet::Spreadsheet(size_t width, size_t height)
    : m_id { ms_counter++ }
    , m_width { std::min(width, MaxWidth) } // std::min() defined in <algorithm>
    , m_height { std::min(height, MaxHeight) }
{
    // Omitted for brevity
}
\end{cpp}

\begin{myNotic}{NOTE}
可以选择在构造函数中,自动将宽度或高度裁剪到它们的最大值,或者在宽度或高度超过其最大值时抛出异常。当从构造函数中抛出一个异常时,不会调用析构函数,所以要小心使用这种方法。
\end{myNotic}

这些常量也可以用作参数的默认值,不过只能从最右边的参数开始,为参数提供默认值:

\begin{cpp}
export class Spreadsheet
{
    public:
        explicit Spreadsheet(
            std::size_t width = MaxWidth, std::size_t height = MaxHeight);
        // Omitted for brevity
};
\end{cpp}

\mySamllsection{从类成员函数外部访问静态数据成员}

访问控制修饰符也适用于静态数据成员:MaxWidth 和 MaxHeight 为public,可以通过指定变量是 Spreadsheet 类的一部分,使用作用域解析操作符(::),从类成员函数外部访问:

\begin{cpp}
println("Maximum height is: {}", Spreadsheet::MaxHeight);
\end{cpp}

\mySubsubsection{9.5.3.}{引用数据成员}

电子表格和电子表格单元格都定义的很好,它们本身并不能构成一个有用的应用程序。需要代码来控制整个电子表格程序,可以将这两个类打包成一个 SpreadsheetApplication 类,表示电子表格可以存储应用程序对象的引用。SpreadsheetApplication 类的确切定义此刻并不重要,所以下面的代码简单地将它定义成一个空类。修改Spreadsheet 类,使其包含一个新引用数据成员 m\_theApp:

\begin{cpp}
export class SpreadsheetApplication { };

export class Spreadsheet
{
    public:
        Spreadsheet(std::size_t width, std::size_t height,
            SpreadsheetApplication& theApp);
        // Code omitted for brevity.
    private:
        // Code omitted for brevity.
        SpreadsheetApplication& m_theApp;
};
\end{cpp}

这个定义增加了 SpreadsheetApplication 引用作为数据成员,因为电子表格应该始终引用 SpreadsheetApplication,所以这里推荐使用引用而不是指针。

注意,存储应用程序的引用只是为了展示如何将引用作为数据成员使用。不建议以这种方式将 Spreadsheet 和 SpreadsheetApplication 类耦合在一起,而应该使用像第 4 章中介绍的模型-视图-控制器(MVC)这样的方式。

应用程序引用在构造函数中传递给每个电子表格。引用不能在没有指向某个对象的情况下存在,所以 m\_theApp 必须在构造函数的 初始化器 中赋值。

\begin{cpp}
Spreadsheet::Spreadsheet(size_t width, size_t height,
    SpreadsheetApplication& theApp)
    : m_id { ms_counter++ }
    , m_width { std::min(width, MaxWidth) }
    , m_height { std::min(height, MaxHeight) }
    , m_theApp { theApp }
{
    // Code omitted for brevity.
}
\end{cpp}

必须在复制构造函数中初始化引用成员,Spreadsheet 复制构造函数委派给非复制构造函数,后者初始化引用数据成员。

初始化了引用就不能改变它所引用的对象,在赋值操作符中无法为引用分配新值。如果包含引用数据成员,则相应的类应该不提供赋值操作符,所以赋值操作符通常会显式删除。

最后,引用数据成员也可以标记为 const。电子表格应该只对应用程序对象有一个 const 引用,可以简单地将 m\_theApp 改为 const 引用:

\begin{cpp}
export class Spreadsheet
{
    public:
        Spreadsheet(std::size_t width, std::size_t height,
            const SpreadsheetApplication& theApp);
        // Code omitted for brevity.
    private:
        // Code omitted for brevity.
        const SpreadsheetApplication& m_theApp;
};
\end{cpp}



