
正则表达式,定义在 <regex> 中,是标准库中与字符串相关的强大特性。支持一种特殊的迷你语言进行字符串处理,开始可能会显得复杂,但当熟悉了它们,就能让字符串处理变得简单。正则表达式可以用于以下字符串操作:

\begin{itemize}
\item
验证:检查输入字符串是否格式正确。例如,输入的字符串是否是一个有效的电话号码?

\item
决策:检查输入代表哪种类型的字符串。例如,输入的字符串是否是一个 JPEG 或 PNG 文件的名字?

\item
解析:从输入字符串中提取信息。例如,从一个日期中提取年、月、日。

\item
转换:搜索子字符串并将其替换为新格式化的子字符串。例如,搜索所有出现的“C++23”并将其替换为“C++”。

\item
迭代:搜索给定序列 [first, last) 中的所有子字符串出现。例如,从一个输入字符串中提取所有电话号码。

\item
标记化:根据一组分隔符将字符串分割成子字符串。例如,在空格、逗号、点等分隔符处分割字符串,以提取单个单词。
\end{itemize}

可以自定义代码来执行这些字符串操作中的任何一个,但我建议使用正则表达式功能,因为正确且安全的字符串处理代码很麻烦。

详细介绍正则表达式之前,需要了解一些重要的术语。以下术语在整个讨论中都会使用:

\begin{itemize}
\item
模式:实际的正则表达式是一个由字符串表示的模式。

\item
匹配:确定给定的正则表达式是否与给定序列 [first, last) 中的所有字符匹配。

\item
搜索:确定给定序列 [first, last) 中的某个子序列是否与给定的正则表达式匹配。

\item
替换:识别给定序列中的子序列并将其替换为从另一个模式(称为替换模式)计算出的相应的子序列。
\end{itemize}

正则表达式有几种不同的语法。C++ 支持以下语法:

\begin{itemize}
\item
ECMAScript: 基于 ECMAScript 标准的语法。ECMAScript 是一种由 ECMA-262 标准化的脚本语言。JavaScript 的核心、ActionScript、Jscript 等都使用 ECMAScript 语言标准。

\item
basic: 基本的 POSIX 语法。

\item
extended: 扩展的 POSIX 语法。

\item
awk: POSIX awk 实用程序使用的语法。

\item
grep: POSIX grep 实用程序使用的语法。

\item
egrep: POSIX grep 实用程序使用的语法,带参数 -E 。
\end{itemize}

如果已经了解这些正则表达式语法中的一种,可以直接在 C++ 中使用,通过指示正则表达式库使用特定的语法(syntax\_option\_type)。C++ 的默认语法是 ECMAScript,其语法将在下一节详细解释,也是最强大的语法。解释其他正则表达式语法,超出了本书的范围。

\begin{myNotic}{NOTE}
如果是第一次听说正则表达式,请直接使用默认的 ECMAScript 语法。
\end{myNotic}

\mySubsubsection{21.2.1.}{ECMAScript 语法}

正则表达式模式是一系列字符,代表想要匹配的内容。正则表达式中的任何字符都会匹配其本身,除非是以下特殊字符:

\begin{shell}
^ $ \ . * + ? ( ) [ ] { } |
\end{shell}

这些特殊字符将在以下讨论中解释。如果需要匹配这些特殊字符中的一个,需要使用反斜杠 \verb|\| 进行转义:

\begin{shell}
\[ or \. or \* or \\
\end{shell}

\mySamllsection{锚点}

特殊字符 \^{} 和 \$ 称为锚点。\^{} 字符匹配换行符后立即出现的字符位置,而 \$ 匹配换行符的位置。默认情况下,\^{} 和 \$ 分别匹配字符串的开始和结束位置,但这种行为可以禁用。

例如,\^{}test\$ 只匹配字符串 test,而不是包含 test 的任何字符串,如 1test, test2, test abc 等。

\mySamllsection{通配符}

通配符 . 可以用来匹配除换行符外的任何单个字符。例如,正则表达式 a.c 将匹配 abc 和 a5c,但不匹配 ab5c、ac 等。

\mySamllsection{交替}

| 字符可以用来指定“或”关系。例如,a|b 匹配 a 或 b。

\mySamllsection{分组}

括号 () 用于标记子表达式,也称为捕获组。捕获组可以用于多种目的:

\begin{itemize}
\item
捕获组可以用来识别原始字符串的个别子序列;每个标记的子表达式(捕获组)都会在结果中返回。例如,正则表达式 (.)(ab|cd)(.) 有三个标记的子表达式。对这个正则表达式在 1cd4 上执行搜索操作,会得到一个匹配结果,包含四个条目。第一个条目是整个匹配,1cd4,后面跟着三个条目,对应于三个标记的子表达式。这三个条目是 1、cd 和 4。

\item
捕获组可以在匹配过程中在回引用中使用。

\item
捕获组可以在替换操作中用来识别组件。
\end{itemize}

\mySamllsection{量化器}

正则表达式的一部分可以使用四个量化器之一进行重复:

\begin{itemize}
\item
* 匹配前面的部分零次或多次。例如,a*b 匹配 b、ab、aab、aaaaab 等。

\item
+ 匹配前面的部分一次或多次。例如,a+b 匹配 ab、aab、aaaaab 等,但不匹配 b。

\item
? 匹配前面的部分零次或一次。例如,a?b 匹配 b 和 ab,但不匹配其他任何字符。

\item
\{...\} 表示有界量化器。b\{n\} 匹配 b 重复 n 次;b\{n,\} 匹配 b 重复 n 次或更多;而 b\{n,m\} 匹配 b 重复 n 到 m 次(包括 n 和 m)。例如,b\{3,4\} 匹配 bbb 和 bbbb,但不匹配 b、bb、bbbbb 等。
\end{itemize}

这些量化器称为贪婪量化器,因为它们在匹配正则表达式的其余部分的情况下,会找到最长的匹配。要使它们非贪婪,可以在量化器后面添加一个 ?,例如 *?、+?、?? 和 \{...\}?。非贪婪量化器会尽可能少的重复,同时仍然匹配正则表达式的其余部分。

例如,下面的表格展示了在输入序列 aaabbb 上运行贪婪和非贪婪正则表达式的区别,以及运行它们时的子匹配结果:

% Please add the following required packages to your document preamble:
% \usepackage{longtable}
% Note: It may be necessary to compile the document several times to get a multi-page table to line up properly
\begin{longtable}{|l|l|}
\hline
\textbf{正则表示} & \textbf{子匹配字符串} \\ \hline
\endfirsthead
%
\endhead
%
贪婪: (a+)(ab)*(b+)       & "aaa" "" "bbb"      \\ \hline
非贪婪: (a+?)(ab)*(b+)  & "aa" "ab" "bb"      \\ \hline
\end{longtable}

\mySamllsection{优先级}

就像数学公式一样,了解正则表达式元素的优先级也很重要。优先级如下:

\begin{itemize}
\item
基本构建块,像 b 这样的元素是正则表达式。

\item
量化器,如 +、*、? 和 \{...\} 紧密地绑定到左侧的元素上;例如,b+。

\item
连接,如 ab+c 在量化器之后绑定。

\item
交替,如 | 最后绑定。
\end{itemize}

例如,正则表达式 ab+c|d 匹配 abc、abbc、abbbc 等,以及 d。括号可以用来改变这些优先级规则。例如,ab+(c|d) 匹配 abc、abbc、abbbc 等,以及 abd、abbd、abbbd 等。然而,通过使用括号,也会将其标记为子表达式或捕获组。使用 (?:...) 可以改变优先级规则,而不创建新的捕获组。例如,ab+(?:c|d) 与之前的 ab+(c|d) 匹配相同,但没有创建捕获组。

\mySamllsection{字符集匹配}

为了避免编写 (a|b|c|...|z) 这种笨拙且引入捕获组的表达式,可以使用特殊的语法来指定字符集或字符范围。此外,还提供了一种“非”形式的匹配。字符集在方括号之间指定,允许你写 [$c_1$ $c_2$ ... $c_n$],匹配任何 $c_1$、$c_2$、... 或 $c_n$。例如,[abc] 匹配任何字符 a、b 或 c。如果第一个字符是 \^{},表示“除了”:

\begin{itemize}
\item
ab[cde] 匹配 abc、abd 和 abe。

\item
ab[\^{}cde] 匹配 abf、abp 等,但不匹配 abc、abd 和 abe。
\end{itemize}

如果需要匹配 \^{}、[ 或 ] 字符本身,需要转义;例如,[ \verb|\|[ \verb|\|\^{}\verb|\|]] 匹配字符 [、\^{} 或 ]。

如果想匹配所有字母,可以使用一个字符集,如 [abcdefghijklmnopqrstuvwxyz ABCDEFGHIJKLMNOPQRSTUVWXYZ]。但这很笨拙,如果多次这样做,特别是遗漏了一个字母,会显得很别扭。解决这个问题有两个方法。

一种方法是在方括号中使用范围指定;可以这样写 [a-zA-Z],识别范围 a 到 z 和 A 到 Z 中的所有字母。如果需要匹配连字符,需要转义它;例如,[a-zA-Z\verb|\|-]+ 匹配包括连字符在内的任何单词。

另一种方法是使用字符类。这些用于表示特定类型的字符,并以 [:name:] 形式表示。可用的字符类取决于区域设置,但可识别下表中的名称。这些字符类的确切含义也取决于区域设置(假设为标准 C 区域):


% Please add the following required packages to your document preamble:
% \usepackage{longtable}
% Note: It may be necessary to compile the document several times to get a multi-page table to line up properly
\begin{longtable}{|l|l|}
\hline
\textbf{字符类名} &
\textbf{描述} \\ \hline
\endfirsthead
%
\endhead
%
digit &
数字,即0、1、2、3、4、5、6、7、8、9。 \\ \hline
d &
和digit一样。 \\ \hline
xdigit &
\begin{tabular}[c]{@{}l@{}}数字(digit)和十六进制数字中使用的以下字母:a、b、c、d、\\e、f、a、b、c、d、e、f。\end{tabular} \\ \hline
alpha &
\begin{tabular}[c]{@{}l@{}}字母字符。对于C语言环境,这些都是小写字母和大写字母。\end{tabular} \\ \hline
alnum &
alpha类和数字类的组合。 \\ \hline
w &
与alnum一样。 \\ \hline
lower &
小写字母,如果适用于语言环境。 \\ \hline
upper &
大写字母,如果适用于语言环境。 \\ \hline
blank &
\begin{tabular}[c]{@{}l@{}}空白字符,用于分隔文本行中的单词的空白字符。对于C语\\言环境,是空格和\textbackslash{}t (tab)。\end{tabular} \\ \hline
space &
\begin{tabular}[c]{@{}l@{}}空白字符。对于C语言环境,它们是空格、\textbackslash{} t、\textbackslash{} n、\textbackslash{} r、\textbackslash{} v\\和\textbackslash{} f。\end{tabular} \\ \hline
s &
与space一样。\\ \hline
print &
\begin{tabular}[c]{@{}l@{}}可打印字符,占据打印位置(例如,在显示器上),与控制字\\符(control)相反。例如小写字母、大写字母、数字、标点符号和空格字符。\end{tabular} \\ \hline
cntrl &
\begin{tabular}[c]{@{}l@{}}控制字符,与可打印字符(print)相反,不占打印位置,例如:\\在显示器上。C语言环境的示例是\textbackslash{}f、\textbackslash{}n和\textbackslash{}r。\end{tabular} \\ \hline
graph &
\begin{tabular}[c]{@{}l@{}}具有图形表示的字符,这些都是可打印(打印)的字符,除了\\空格字符' '。\end{tabular} \\ \hline
punct &
\begin{tabular}[c]{@{}l@{}}常用的标点符号。对于C语言环境,这些都是图形字符(graph),\\而不是字母数字(alnum)。例如!、\#、@、\}等。\end{tabular} \\ \hline
\end{longtable}

字符类用于字符集中;例如,在英语中 [[:alpha:]]* 与 [a-zA-Z]* 表示相同的意思。

因为某些字符类非常常见,例如数字,所以有精简模式。例如,[:digit:] 和 [:d:] 与 [0-9] 表示相同的意思。有些类甚至使用转义符号有更短的表示。例如,\verb|\|d 表示 [:digit:]。为了识别一个或多个数字的序列,可以写成以下模式:

\begin{itemize}
\item
{}[0-9]+

\item
{}[[:digit:]]+

\item
{}[[:d:]]+

\item
\verb|\|d+
\end{itemize}

以下表格列出了可用于字符类的转义符号:

% Please add the following required packages to your document preamble:
% \usepackage{longtable}
% Note: It may be necessary to compile the document several times to get a multi-page table to line up properly
\begin{longtable}{|l|l|}
\hline
\textbf{转义符号} & \textbf{等价于}              \\ \hline
\endfirsthead
%
\endhead
%
\textbackslash{}d        & {[}{[}:d:{]}{]}                     \\ \hline
\textbackslash{}D        & {[}\textasciicircum{}{[}:d:{]}{]}   \\ \hline
\textbackslash{}s        & {[}{[}:s:{]}{]}                     \\ \hline
\textbackslash{}S        & {[}\textasciicircum{}{[}:s:{]}{]}   \\ \hline
\textbackslash{}w        & {[}\_{[}:w:{]}{]}                   \\ \hline
\textbackslash{}W        & {[}\textasciicircum{}\_{[}:w:{]}{]} \\ \hline
\end{longtable}

这里有些例子:

\begin{itemize}
\item
Test[5-8]匹配 Test5、Test6、Test7 和 Test8。

\item
{}[[:lower:]] 匹配小写字母 a、b 等,但不匹配大写字母 A、B 等。

\item
{}[\^{}[:lower:]] 匹配任何不是小写字母 a、b 等的字符。

\item
{}[[:lower:]5-7] 匹配任何小写字母 a、b 等,以及数字 5、6 和 7。
\end{itemize}

\mySamllsection{词界}

词界可以表示为:

\begin{itemize}
\item
单词的第一个字符,这是单词字符之一,而前一个字符不是单词字符。单词字符是字母、数字或下划线。对于标准 C 区域设置,这等于 [A-Za-z0-9\_]。

\item
单词的末尾,这是一个非单词字符,而前一个字符是单词字符。

\item
源字符串的开始,如果源字符串的第一个字符是单词字符。默认情况下,匹配源字符串的开始是启用的,但可以通过使用 regex\_constants::match\_not\_bow 禁用,其中 bow 表示单词的开始。

\item
源字符串的末尾,如果源字符串的最后一个字符是单词字符。默认情况下,匹配源字符串的末尾是启用的,但可以通过使用 regex\_constants::match\_not\_eow 禁用,其中 eow 表示单词的结束。
\end{itemize}

可以使用 \b 来匹配单词边界,并且可以使用 \verb|\|B 来匹配不是单词边界的内容。

\mySamllsection{回引用}

回引用允许在正则表达式本身内部引用捕获组:\verb|\|n 指的是第 n 个捕获组,其中 n > 0。例如,正则表达式 (\verb|\|d+)-.*-\verb|\|1 匹配具有以下格式的字符串:

\begin{itemize}
\item
一个或多个数字的捕获组 (\verb|\|d+)

\item
后跟一个破折号 -

\item
后跟零个或多个字符 .*

\item
后跟另一个破折号 -

\item
后跟第一个捕获组中捕获的相同数字 \verb|\|1
\end{itemize}

这个正则表达式匹配 123-abc-123、1234-a-1234 等,但不匹配 123-abc-1234、123-abc-321 等。

\mySamllsection{前瞻}

正则表达式支持正前瞻(使用 ?=pattern)和负前瞻(使用 ?!pattern)。前瞻字符必须匹配(正前瞻)或不匹配(负前瞻)前瞻模式,但这些字符尚未消耗。

模式 a(?!b) 包含一个负前瞻,用于匹配不跟随 b 的字母 a。模式 a(?=b) 包含一个正前瞻,用于匹配后面跟着 b 的字母 a,但 b 未消耗,因此不会成为匹配的一部分。

以下是一个更实际的例子。这个正则表达式匹配至少包含一个小写字母、一个或多个大写字母、一个或多个标点字符,并且至少为八个字符长的输入序列。这样的正则表达式可以用于强制密码满足某些标准。

\begin{cpp}
(?=.*[[:lower:]])(?=.*[[:upper:]])(?=.*[[:punct:]]).{8,}
\end{cpp}

本章末尾的练习中,可以实验这个验证密码的正则表达式。

\mySamllsection{正则表达式和原始字符串字面量}

正则表达式经常使用特殊字符,这些特殊字符必须在常规 C++ 字符串字面量中进行转义。例如,在正则表达式中写 \verb|\|d,它会匹配数字。但在 C++ 中 \ 是一个特殊字符,需要在正则表达式字符串字面量中转义它为 \verb|\|\verb|\|d;否则,C++ 编译器会尝试转义 \verb|\|d。如果想要一个正则表达式匹配单个反斜杠字符——“\verb|\|”,因为 \verb|\| 本身就是正则表达式语法中的特殊字符,需要转义它为 \verb|\|\verb|\|。\verb|\| 字符在 C++ 字符串字面量中也是一个特殊字符,所以需要转义,结果为 \verb|\|\verb|\|\verb|\|\verb|\|。

可以使用原始字符串字面量,使复杂的正则表达式在 C++ 源代码中更容易阅读。(原字符串字面量在第2章中介绍过)例如,看看以下的正则表达式:

\begin{cpp}
"( |\\n|\\r|\\\\)"
\end{cpp}

这个正则表达式匹配空格、换行符、回车符和反斜杠,需要大量的转义字符。使用原始字符串字面量,这可以替换为更可读的正则表达式:

\begin{cpp}
R"(( |\n|\r|\\))"
\end{cpp}

原字符串字面量以 R"(" 开头,并以 ")" 结束。之间的所有内容都是正则表达式。当然,仍然需要在结尾处添加一个双反斜杠,因为反斜杠需要在正则表达式中转义。

\mySamllsection{常见的正则表达式}

编写正确的正则表达式并不是那么简单,对于像验证密码、电话号码、社会安全号码、IP 地址、电子邮件地址、信用卡号码、日期等常见模式。当互联网搜索引擎并在线搜索正则表达式时,会发现一些网站上有预定义的模式集合,例如 regexr.com、regex101.com、regextester.com 等。其中很多网站可以在网上测试模式,所以可以验证表达式是否正确,然后再将它们用在自己的代码中。

这就是 ECMAScript 语法的简要描述。接下来的部分将解释,如何在 C++ 代码中使用正则表达式。

\mySubsubsection{21.2.2.}{正则表达式库}

正则表达式库中的一切都在 <regex> 和 std 命名空间中定义,正则表达式库定义的基本模板类型包括:

\begin{itemize}
\item
basic\_regex: 特定正则表达式的对象。

\item
match\_results: 匹配正则表达式的子字符串,包括所有捕获组,是一个子匹配的集合。

\item
sub\_match: 包含输入序列中的迭代器对的对象,这些迭代器表示一个匹配的捕获组。对是表示匹配捕获组第一个字符的迭代器和对后一个字符的迭代器,有一个 str() 成员函数,该函数返回匹配的捕获组作为一个字符串。
\end{itemize}

该库提供了三个关键算法:regex\_match(), regex\_search() 和 regex\_replace()。这些算法都有不同的重载,允许指定源字符串为字符串、C 风格字符串,或者作为开始/结束迭代器对。迭代器可以是以下一种:

\begin{itemize}
\item
const char* 或 const wchar\_t*

\item
string::const\_iterator 或 wstring::const\_iterator
\end{itemize}

实际上,双向迭代器的行为都可以使用。有关迭代器的详细信息,请参阅第 17 章。

该库还定义了以下两种正则表达式迭代器,在源字符串中查找模式的所有出现中起着重要作用:

\begin{itemize}
\item
regex\_iterator: 在源字符串中,遍历所有模式。

\item
regex\_token\_iterator: 在源字符串中,遍历所有模式的捕获组。
\end{itemize}

为了使库更易于使用,标准定义了上述模板的一些类型别名:

\begin{cpp}
using regex = basic_regex<char>;
using wregex = basic_regex<wchar_t>;

using csub_match = sub_match<const char*>;
using wcsub_match = sub_match<const wchar_t*>;
using ssub_match = sub_match<string::const_iterator>;
using wssub_match = sub_match<wstring::const_iterator>;

using cmatch = match_results<const char*>;
using wcmatch = match_results<const wchar_t*>;
using smatch = match_results<string::const_iterator>;
using wsmatch = match_results<wstring::const_iterator>;

using cregex_iterator = regex_iterator<const char*>;
using wcregex_iterator = regex_iterator<const wchar_t*>;
using sregex_iterator = regex_iterator<string::const_iterator>;
using wsregex_iterator = regex_iterator<wstring::const_iterator>;

using cregex_token_iterator = regex_token_iterator<const char*>;
using wcregex_token_iterator = regex_token_iterator<const wchar_t*>;
using sregex_token_iterator = regex_token_iterator<string::const_iterator>;
using wsregex_token_iterator = regex_token_iterator<wstring::const_iterator>;
\end{cpp}

接下来的将介绍regex\_match(), regex\_search() 和 regex\_replace() 算法,以及 regex\_iterator 和 regex\_token\_iterator 类。

\mySubsubsection{21.2.3.}{regex\_match()}

regex\_match() 算法可以用于比较给定的源字符串与正则表达式模式。如果模式匹配整个源字符串,返回 true,否则返回 false。

regex\_match() 算法有七个重载,接受不同类型的参数,具有以下形式:

\begin{cpp}
template<...>
bool regex_match(InputSequence[, MatchResults], RegEx[, Flags]);
\end{cpp}

InputSequence 可以表示为:

\begin{itemize}
\item
源字符串中的开始和结束迭代器

\item
std::string

\item
C 风格字符串
\end{itemize}

可选的 MatchResults 参数是一个对 match\_results 的引用,用于接收匹配结果。如果 regex\_match() 返回 false,只能调用 match\_results::empty() 或 match\_results::size();其他操作都未定义。如果 regex\_match() 返回 true,表示找到了匹配,可以检查 match\_results 对象以查看具体匹配了什么。这在以下子节中会通过示例进行解释。

RegEx 参数是需要匹配的正则表达式,可选的 Flags 参数指定匹配算法的选项。通常,可以保留默认值。更多细节,请参阅标准库手册。

\mySamllsection{regex\_match() 示例}

以下程序要求用户以年份/月份/日期的格式输入日期,其中年份是四位数字,月份是 1 到 12 之间的数字,日期是 1 到 31 之间的数字。使用正则表达式和 regex\_match() 算法来验证用户输入。正则表达式的详细内容,将在代码之后介绍。

\begin{cpp}
regex r { "\\d{4}/(?:0?[1-9]|1[0-2])/(?:0?[1-9]|[1-2][0-9]|3[0-1])" };
while (true) {
    print("Enter a date (year/month/day) (q=quit): ");
    string str;
    if (!getline(cin, str) || str == "q") { break; }

    if (regex_match(str, r)) { println(" Valid date."); }
    else { println(" Invalid date!"); }
}
\end{cpp}

第一行创建了正则表达式。表达式由三个部分组成,由斜杠 (/) 字符分隔:一个部分用于年份,一个用于月份,一个用于日期。以下列表解释了这些部分:

\begin{itemize}
\item
\verb|\|d{4}: 匹配任何四位数字的组合;例如,1234、2024 等。

\item
(?:0?[1-9]|1[0-2]): 这个正则表达式的一部分用括号包围,以确保优先级正确。不需要捕获组,所以使用 (?:...) 形式。内部表达式包含两个部分的交替,由 | 字符分隔。
\begin{itemize}
\item
0?[1-9]: 匹配从 1 到 9 的数字,前面有一个可选的 0。例如,匹配 1、2、9、03、04 等,但不匹配 0、10、11 等。

\item
1[0-2]: 匹配 10、11 或 12,没有其他匹配。

\item
(?:0?[1-9]|[1-2][0-9]|3[0-1]): 这个子部分也被一个非捕获组包围,并包含三个部分的交替。
\begin{itemize}
\item
0?[1-9]: 再次匹配从 1 到 9 的数字,前面有一个可选的 0。

\item
{}[1-2][0-9]: 匹配 10 到 29 之间的任何数字,包括边界,没有其他匹配。

\item
3[0-1]: 匹配 30 或 31,没有其他匹配。
\end{itemize}

\end{itemize}

\end{itemize}

示例进入一个无限循环,要求用户输入日期。每次输入的日期都会传递给 regex\_match() 算法,当 regex\_match() 返回 true 时,用户输入的日期与日期正则表达式模式匹配。

这个例子可以通过 regex\_match() 算法,对返回捕获的表达式在结果对象进行扩展,需要理解捕获组的作用。通过在 regex\_match() 调用中指定 match\_results 对象,例如 smatch,当正则表达式匹配输入字符串时,将填充match\_results 对象的元素。为了能够提取这些子字符串,必须使用括号创建捕获组。

match\_results 对象中的第一个元素 [0] 包含匹配整个模式的字符串。当使用 regex\_match() 且找到匹配时,这是整个源序列。当使用下一节中讨论的 regex\_search() 时,这可能是源序列中匹配正则表达式的子字符串。元素 [1] 是第一个捕获组匹配的子字符串,[2] 是第二个捕获组匹配的子字符串,依此类推。要从 match\_results 对象 m 中获取第 i 个捕获组的字符串表示,可以使用 m[i]或者 m[i].str()。

以下代码将年份、月份和日期的数字提取到三个独立的整数变量中。修订后的正则表达式做了一些小的更改,一个捕获组包围了匹配年份的第一部分,而月份和日期的部分现在也是捕获组。regex\_match() 有一个 smatch 参数,用于接收匹配的捕获组。这里是修改后的示例:

\begin{cpp}
regex r { "(\\d{4})/(0?[1-9]|1[0-2])/(0?[1-9]|[1-2][0-9]|3[0-1])" };
while (true) {
    print("Enter a date (year/month/day) (q=quit): ");
    string str;
    if (!getline(cin, str) || str == "q") { break; }

    if (smatch m; regex_match(str, m, r)) {
        int year { stoi(m[1]) };
        int month { stoi(m[2]) };
        int day { stoi(m[3]) };
        println(" Valid date: Year={}, month={}, day={}", year, month, day);
    } else {
        println(" Invalid date!");
    }
}
\end{cpp}

这个例子中,smatch 结果对象有四个元素:

\begin{itemize}
\item
[0]: 匹配整个正则表达式的字符串,在这个例子中是完整的日期。

\item
[1]: 年份

\item
[2]: 月份

\item
[3]: 日期
\end{itemize}

执行这个例子时,可以得到以下输出:

\begin{shell}
Enter a date (year/month/day) (q=quit): 2024/12/01
    Valid date: Year=2024, month=12, day=1
Enter a date (year/month/day) (q=quit): 24/12/01
    Invalid date!
\end{shell}

\begin{myNotic}{NOTE}
这些日期匹配示例仅检查日期是否包含四位数字的年份、1-12之间的月份和1-31之间的日期。不进行月份天数、闰年等验证。如果需要这些验证,必须编写代码来验证由 regex\_match() 提取的年份、月份和日期的值。如果实现了这样的验证,则正则表达式可以简化为只匹配年份的 4 位数字、月份的 1 或 2 位数字和日期的 1 或 2 位数字。

\begin{cpp}
regex r { "(\\d{4})/(\\d{1,2})/(\\d{1,2})" };
\end{cpp}
\end{myNotic}

\mySubsubsection{21.2.4.}{regex\_search()}

上一节中讨论的 regex\_match() 算法返回 true,如果整个源字符串与正则表达式匹配,否则返回 false。如果想在源字符串中搜索匹配的子字符串,需要使用 regex\_search()。regex\_search() 有七个重载,具有以下形式:

\begin{cpp}
template<...>
bool regex_search(InputSequence[, MatchResults], RegEx[, Flags]);
\end{cpp}

所有重载在输入序列中找到匹配时返回 true,否则返回 false。参数与 regex\_match() 的参数相似。

regex\_search() 的两个重载接受一个开始和结束迭代器作为要处理的输入序列,可能会想使用这个版本的 regex\_search() 在循环中查找源字符串中模式的所有出现,通过为每个 regex\_search() 调用操纵这些开始和结束迭代器。但请不要这样做!如果正则表达式使用锚点(\^{} 或 \$)、词界等,这样做可能会出现问题,可能由于空匹配导致无限循环。本章后面,我们将介绍如何使用 regex\_iterator 或 regex\_token\_iterator 从源字符串中提取出现的所有模式。

\begin{myWarning}{WARNING}
永远不要在循环中使用 regex\_search() 来查找源字符串中所有出现的模式,应该使用 regex\_iterator 或 regex\_token\_iterator。
\end{myWarning}

\mySamllsection{regex\_search() 示例}

regex\_search() 算法可以用于从输入序列中提取匹配的子字符串。例如,以下程序从字符串中提取代码注释。正则表达式搜索以 // 开头的子字符串,后面跟着可选的空格,然后是一个或多个字符的捕获组 (.+)。这个捕获组只捕获注释子字符串。smatch 对象 m 接收搜索结果。如果成功,m[1] 包含找到的注释。可以检查 m[1].first 和 m[1].second 迭代器,来查看注释在源字符串中的确切位置。

\begin{cpp}
regex r { "//\\s*(.+)$" };
while (true) {
    print("Enter a string with optional code comments (q=quit):\n> ");
    string str;
    if (!getline(cin, str) || str == "q") { break; }

    if (smatch m; regex_search(str, m, r)) {
        println(" Found comment '{}'", m[1].str());
    } else {
        println(" No comment found!");
    }
}
\end{cpp}

程序的输出可能如下所示:

\begin{shell}
Enter a string with optional code comments (q=quit):
> std::string str; // Our source string
  Found comment 'Our source string'
Enter a string with optional code comments (q=quit):
> int a; // A comment with // in the middle
  Found comment 'A comment with // in the middle'
Enter a string with optional code comments (q=quit):
> std::vector values { 1, 2, 3 };
  No comment found!
\end{shell}

match\_results 对象还具有 prefix() 和 suffix() 成员函数,分别返回匹配前的字符串和匹配后的字符串。

\mySubsubsection{21.2.5.}{regex\_iterator}

永远不应该在循环中使用 regex\_search(),应该使用 regex\_iterator 或 regex\_token\_iterator,它们的工作方式类似于标准库容器中的迭代器。

\mySamllsection{regex\_iterator 示例}

以下示例要求用户输入源字符串,从字符串中提取每个单词,并打印出所有带引号的单词。这个例子中的正则表达式是 [\verb|\|w]+,搜索一个或多个单词字母。这个例子使用 std::string 作为源,使用 sregex\_iterator 作为迭代器。这里使用了标准迭代器循环,但结束迭代器的处理方式与标准库容器中的结束迭代器略有不同。通常,会为特定的容器指定一个结束迭代器,但对于 regex\_iterator,只有一个“结束”迭代器,可以通过默认构造一个 regex\_iterator 来获得这个结束迭代器。

for 循环创建了一个名为 iter 的开始迭代器,接受源字符串的 begin 和 end 迭代器以及一个正则表达式。循环体对于找到的每个匹配项(在这个例子中是每个单词)都调用。sregex\_iterator 遍历所有匹配项,通过解引用 sregex\_iterator,可以得到一个 smatch 对象。通过访问这个 smatch 对象的第一个元素 [0],可以得到匹配的子字符串:

\begin{cpp}
regex reg { "[\\w]+" };
while (true) {
    print("Enter a string to split (q=quit): ");
    string str;
    if (!getline(cin, str) || str == "q") { break; }

    const sregex_iterator end;
    for (sregex_iterator iter { cbegin(str), cend(str), reg };
    iter != end; ++iter) {
        println("\"{}\"", (*iter)[0].str());
    }
}
\end{cpp}

程序的输出可能如下所示:

\begin{shell}
Enter a string to split (q=quit): This, is a test.
"This"
"is"
"a"
"test"
\end{shell}

正如这个例子,即使是简单的正则表达式,也能执行一些强大的字符串操作!

注意,regex\_iterator 和下一节中讨论的 regex\_token\_iterator 都内部存储了给定正则表达式的指针,所以都明确地删除了接受右值引用正则表达式的构造函数,以避免使用临时正则表达式对象进行构造。以下代码不会编译:

\begin{cpp}
for (sregex_iterator iter { cbegin(str), cend(str), regex { "[\\w]+" } };
    iter != end; ++iter) { ... }
\end{cpp}

\mySubsubsection{21.2.6.}{regex\_token\_iterator}

上一节描述了 regex\_iterator,遍历每个匹配项。在每次迭代中,都会得到一个 match\_results 对象,可以使用它来提取匹配项中的子表达式,这些子表达式由捕获组捕获。

regex\_token\_iterator 可以用来自动遍历所有或选择的捕获组。这里有四个构造函数,具有以下格式:

\begin{cpp}
regex_token_iterator(BidirectionalIterator a,
                     BidirectionalIterator b,
                     const regex_type& re
                     [, SubMatches
                     [, Flags]]);
\end{cpp}

都需要一个输入序列的开始和结束迭代器,以及一个正则表达式。可选的 SubMatches 参数用于指定应该遍历的捕获组。SubMatches 可以用以下四种方式指定:

\begin{itemize}
\item
作为一个整数,表示想要遍历捕获组的索引

\item
作为一个包含整数的vector,表示想要遍历捕获组的索引

\item
作为一个initializer\_list,包含捕获组索引

\item
作为一个 C 风格数组,包含捕获组索引
\end{itemize}

当省略 SubMatches 或当 SubMatches 指定为 0 时,会得到一个迭代器,可遍历所有捕获组索引为 0 的捕获组,即匹配整个正则表达式的子字符串。可选的 Flags 参数指定匹配算法的选项。大多数情况下,可以保留默认值。更多信息,请参阅标准库手册。

\mySubsubsection{21.2.7.}{regex\_token\_iterator 示例}

之前 regex\_iterator 示例可以使用 regex\_token\_iterator 重写。在循环体中,不再需要使用 (*iter)[0].str(),只需使用 iter->str(),因为 token 迭代器默认(索引为 0)自动遍历所有捕获组索引为 0 的捕获组。这段代码的输出与之前 regex\_iterator 示例生成相同的输出。

\begin{cpp}
regex reg { "[\\w]+" };
while (true) {
    print("Enter a string to split (q=quit): ");
    string str;
    if (!getline(cin, str) || str == "q") { break; }

    const sregex_token_iterator end;
    for (sregex_token_iterator iter { cbegin(str), cend(str), reg };
    iter != end; ++iter) {
        println("\"{}\"", iter->str());
    }
}
\end{cpp}

以下示例要求用户输入一个日期,然后使用 regex\_token\_iterator 遍历第二个和第三个捕获组(月份和日期),这些捕获组通过一个整数vector指定。用于日期的正则表达式在本章前面已经解释过。唯一的区别是添加了 \^{} 和 \$ 锚点,因为想要匹配整个源序列。因为 regex\_match() 会自动匹配整个输入字符串,所以这并不是必要的。

\begin{cpp}
regex reg { "^(\\d{4})/(0?[1-9]|1[0-2])/(0?[1-9]|[1-2][0-9]|3[0-1])$" };
while (true) {
    print("Enter a date (year/month/day) (q=quit): ");
    string str;
    if (!getline(cin, str) || str == "q") { break; }

    vector indices { 2, 3 };
    const sregex_token_iterator end;
    for (sregex_token_iterator iter { cbegin(str), cend(str), reg, indices };
    iter != end; ++iter) {
        println("\"{}\"", iter->str());
    }
}
\end{cpp}

这段代码只打印有效日期的月份和日期,示例生成的输出可能如下所示:

\begin{shell}
Enter a date (year/month/day) (q=quit): 2024/1/13
"1"
"13"
Enter a date (year/month/day) (q=quit): 2024/1/32
Enter a date (year/month/day) (q=quit): 2024/12/5
"12"
"5"
\end{shell}

regex\_token\_iterator 也可以用于执行字段拆分或标记化。与使用 C 语言中旧的、不再进一步讨论的 strtok() 函数相比,它是一个更安全、更灵活的替代方案。在 regex\_token\_iterator 构造函数中通过指定 -1 作为要迭代的捕获组索引来启用标记化。标记化模式下,迭代器遍历输入序列中所有不匹配正则表达式的子字符串。以下代码通过在逗号和分号,以及其前后零个或多个空白字符上对字符串进行标记化进行展示。该代码以两种方式演示了标记化:首先直接迭代标记,然后通过创建包含所有标记的新vector,并打印vector的内容:

\begin{cpp}
regex reg { R"(\s*[,;]\s*)" };
while (true) {
    print("Enter a string to split on ',' and ';' (q=quit): ");
    string str;
    if (!getline(cin, str) || str == "q") { break; }

    // Iterate over the tokens.
    const sregex_token_iterator end;
    for (sregex_token_iterator iter { cbegin(str), cend(str), reg, -1 };
    iter != end; ++iter) {
        print("\"{}\", ", iter->str());
    }
    println("");

    // Store all tokens in a vector.
    vector<string> tokens {
        sregex_token_iterator { cbegin(str), cend(str), reg, -1 },
        sregex_token_iterator {} };
    // Print the contents of the tokens vector.
    println("{:n}", tokens);
}
\end{cpp}

这个例子中的正则表达式是用原始字符串字面量指定的,搜索匹配以下模式:

\begin{itemize}
\item
零个或多个空格字符

\item
后面跟着一个 , 或 ; 字符

\item
后面跟着零个或多个空格字符
\end{itemize}

输出可能如下所示:

\begin{shell}
Enter a string to split on ',' and ';' (q=quit): This is, a; test string.
"This is", "a", "test string.",
"This is", "a", "test string."
\end{shell}

正如从输出中看到的,字符串被拆分在 , 和 ; 之间。移除了 , 和 ; 周围的空格字符,因为标记化迭代器遍历所有不匹配正则表达式的子字符串,并且因为正则表达式匹配周围带有空格的 , 和 ;。

\mySubsubsection{21.2.8.}{regex\_replace()}

regex\_replace() 算法需要一个正则表达式和一个格式化字符串,用于替换匹配的子字符串。这个格式化字符串,可以通过使用下表中的转义序列,来引用匹配子字符串的部分。

% Please add the following required packages to your document preamble:
% \usepackage{longtable}
% Note: It may be necessary to compile the document several times to get a multi-page table to line up properly
\begin{longtable}{|l|l|}
\hline
\textbf{转义序列} &
\textbf{替换为} \\ \hline
\endfirsthead
%
\endhead
%
\$n &
\begin{tabular}[c]{@{}l@{}}匹配第 n 个捕获组的字符串;例如,\$1 用于第一个捕获组,\$2 用于第二个,\\依此类推。n 必须大于 0。
\end{tabular} \\ \hline
\$\& &
匹配整个正则表达式的字符串。 \\ \hline
\$` &
\begin{tabular}[c]{@{}l@{}}正则表达式匹配的子字符串出现在输入序列的左侧的部分。\end{tabular} \\ \hline
\$´ &
\begin{tabular}[c]{@{}l@{}}匹配正则表达式的子字符串出现在输入序列的右侧的部分。\end{tabular} \\ \hline
\$\$ &
一个美元符号。 \\ \hline
\end{longtable}

regex\_replace()有六个重载,区别在于参数类型。其中四个重载具有以下格式:

\begin{cpp}
template<...>
string regex_replace(InputSequence, RegEx, FormatString[, Flags]);
\end{cpp}

这四个重载在执行替换后返回结果字符串,InputSequence 和 FormatString 可以是 std::string 或 C 风格字符串,RegEx 参数是需要匹配的正则表达式,可选的 Flags 参数指定替换算法的选项。

regex\_replace() 的两个重载:

\begin{cpp}
OutputIterator regex_replace(OutputIterator,
                             BidirectionalIterator first,
                             BidirectionalIterator last,
                             RegEx, FormatString[, Flags]);
\end{cpp}

这两个重载将结果字符串写入给定的输出迭代器,并返回这个输出迭代器,输入序列由一个开始和结束迭代器给出。其他参数与 regex\_replace() 的其他四个重载相同。

\mySamllsection{regex\_replace() 示例}

作为第一个示例,来看看以下的 HTML 源字符串:

\begin{shell}
<body><h1>Header</h1><p>Some text</p></body>
\end{shell}

以及下面的正则表达式:

\begin{cpp}
<h1>(.*)</h1><p>(.*)</p>
\end{cpp}

下表显示了不同的转义序列,及其将替换的内容:

% Please add the following required packages to your document preamble:
% \usepackage{longtable}
% Note: It may be necessary to compile the document several times to get a multi-page table to line up properly
\begin{longtable}{|l|l|}
\hline
\textbf{转义序列} & \textbf{替换为}         \\ \hline
\endfirsthead
%
\endhead
%
\$1                      & Header                         \\ \hline
\$2                      & Some text                      \\ \hline
\$\& & \textless{}h1\textgreater{}Header\textless{}/h1\textgreater{}\textless{}p\textgreater{}Some text\textless{}/p\textgreater{} \\ \hline
\$`                      & \textless{}body\textgreater{}  \\ \hline
\$´                      & \textless{}/body\textgreater{} \\ \hline
\end{longtable}

下面的代码演示了 regex\_replace() 的使用:

\begin{cpp}
const string str { "<body><h1>Header</h1><p>Some text</p></body>" };
regex r { "<h1>(.*)</h1><p>(.*)</p>" };

const string replacement { "H1=$1 and P=$2" }; // See earlier table.
string result { regex_replace(str, r, replacement) };

println("Original string: '{}'", str);
println("New string : '{}'", result);
\end{cpp}

这个程序的输出如下所示:

\begin{shell}
Original string: '<body><h1>Header</h1><p>Some text</p></body>'
New string     : '<body>H1=Header and P=Some text</body>'
\end{shell}

regex\_replace() 算法接受一些标志来改变其行为。最重要的标志如下表所示:

% Please add the following required packages to your document preamble:
% \usepackage{longtable}
% Note: It may be necessary to compile the document several times to get a multi-page table to line up properly
\begin{longtable}{|l|l|}
\hline
\textbf{标识}       & \textbf{描述}                               \\ \hline
\endfirsthead
%
\endhead
%
format\_default &
\begin{tabular}[c]{@{}l@{}}默认情况下,regex\_replace() 会替换模式匹配的所有地方,并且也会\\将不匹配模式的全部内容复制到输出中。
\end{tabular} \\ \hline
format\_no\_copy &
\begin{tabular}[c]{@{}l@{}}替换模式匹配的所有地方,但不将不匹配模式的内容复制到输出中。
\end{tabular} \\ \hline
format\_first\_only & 只替换模式的第一处。
 \\ \hline
\end{longtable}

之前的代码段中,regex\_replace() 可以修改为使用 format\_no\_copy 标志:

\begin{cpp}
string result { regex_replace(str, r, replacement,
    regex_constants::format_no_copy) };
\end{cpp}

现在输出如下:

\begin{shell}
Original string: '<body><h1>Header</h1><p>Some text</p></body>'
New string     : 'H1=Header and P=Some text
\end{shell}

另一个使用 regex\_replace() 的示例是将字符串中的每个词界替换为新行字符,以便输出包含每行一个单词。以下代码段演示了这一点,不使用循环来处理给定的输入字符串。代码首先创建一个匹配单个单词的正则表达式。当 regex\_replace() 找到匹配时,用 \$1\verb|\|n 替换,其中 \$1 替换为匹配的单词。这里还使用了 format\_no\_copy 标志,以避免将源字符串中的空白和非单词字符复制到输出中。

\begin{cpp}
regex reg { "([\\w]+)" };
const string replacement { "$1\n" };
while (true) {
    print("Enter a string to split over multiple lines (q=quit): ");
    string str;
    if (!getline(cin, str) || str == "q") { break; }

    println("{}", regex_replace(str, reg, replacement,
        regex_constants::format_no_copy));
}
\end{cpp}

这个程序可能的输出:

\begin{shell}
Enter a string to split over multiple lines (q=quit): This is a test.
This
is
a
test
\end{shell}






















