通过解决下面的练习,可以练习本章讨论的内容。所有练习的解决方案都可以在本书的网站\url{www.wiley.com/go/proc++6e}下载到源码。然而,若在练习中卡住了,在从网站上寻找解决方案之前,可以考虑先重读本章的部分内容,试着自己找到答案。

\begin{itemize}
\item
Exercise 21-1: Use an appropriate facet to figure out the decimal separator for formatting numbers according to the user’s environment. Consult a Standard Library reference to learn about the exact member functions of your chosen facet.

\item
Exercise 21-2: Write an application that asks the user to enter a phone number as formatted in the United States. Here’s an example: 202-555-0108. Use a regular expression to validate the format of the phone number, that is, three digits, followed by a dash, three more digits, another dash, and a final four digits. If it’s a valid phone number, print out the three parts on separate lines. For example, for the earlier phone number, the result must be as follows:

\begin{shell}
202
555
0108
\end{shell}

\item
Exercise 21-3: Write an application that asks the user for a piece of source code that can span multiple lines and that can contain // style comments. To signal the end of the input, use a sentinel character, for example @. You can use std::getline() with '@' as delimiter to read in multiple lines of text from the standard input console. Finally, use a regular expression to remove comments from all lines of the code snippet. Make sure your code properly works on a snippet such as the following:

\begin{cpp}
string str; // A comment // Some more comments.
str = "Hello"; // Hello.
\end{cpp}

The result for this input must be as follows:

\begin{shell}
string str;
str = "Hello";
\end{shell}

\item
Exercise 21-4: The section “Lookahead” earlier in this chapter mentioned a password-validation regular expression. Write a program to test this regular expression. Ask the user to enter a password and validate it. Once you’ve verified that the regular expression works, add one more validation rule to it: a password must also consist of at least two digits.
\end{itemize}