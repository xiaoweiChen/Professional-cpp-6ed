This chapter gave you an appreciation for coding with localization in mind. As anyone who has been through a localization effort will tell you, adding support for a new language or locale is infinitely easier if you have planned ahead, for example, by using Unicode characters and being mindful of locales.

The second part of this chapter explained the regular expressions library. Once you know the syntax of regular expressions, it becomes much easier to work with strings. Regular expressions allow you to validate strings, search for substrings inside an input sequence, perform find-and-replace operations, and so on. It is highly recommended that you get to know regular expressions and start using them instead of writing your own string manipulation routines. They will make your life easier.