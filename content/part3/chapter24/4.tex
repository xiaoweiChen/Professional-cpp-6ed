
\CXXTwentythreeLogo{-40}{15}

第1章介绍了 std::optional 的基础知识。C++23 为 optional 添加了三个新的成员函数,统称为 一元(monadic)操作。这些操作允许在不检查 optional 是否有值的情况下,对 optional 进行链式操作。

以下可用的一元操作:

\begin{itemize}
\item
transform(F): 如果 *this 有值,则返回一个包含调用 F 并传递 *this 的值作为参数的结果的 optional;否则,返回一个空的 optional

\item
and\_then(F): 如果 *this 有值,则返回调用 F 并传递 *this 的值作为参数的结果(结果必须是 optional);否则,返回一个空的 optional

\item
or\_else(F): 如果 *this 有值,则返回 *this;否则,返回调用 F 的结果(结果必须是 optional)
\end{itemize}

来看一个例子。以下函数解析给定的字符串以获取一个整数,并返回结果作为 optional。如果字符串不能被解析为整数,则返回一个空的 optional。

\begin{cpp}
optional<int> Parse(const string& str)
{
    try { return stoi(str); }
    catch (...) { return {}; }
}
\end{cpp}

以下循环反复要求用户输入一些内容。Parse() 调用以解析用户的输入,如果输入可以成功解析为整数,则使用 and\_then() 将整数加倍,并使用 transform() 将整数转换回字符串。如果输入无法解析,则使用 or\_else() 返回字符串“No Integer”。由于 一元 操作,无需在应用 Parse() 和 and\_then() 返回的 optional 的下一个操作之前检查它们是否包含值,错误处理已经处理过了。不同的操作可以简单地串联在一起。

\begin{cpp}
while (true) {
    print("Enter an integer (q to stop): ");
    string str;
    if (!getline(cin, str) || str == "q") { break; }

    auto result { Parse(str)
        .and_then([](int value) -> optional<int> { return value * 2; })
        .transform([](int value) { return to_string(value); })
        .or_else([] { return optional<string> { "No Integer" }; }) };
    println(" > Result: {}", *result);
}
\end{cpp}

输出为:

\begin{shell}
Enter an integer (q to stop): 21
    > Result: 42
Enter an integer (q to stop): Test
    > Result: No Integer
\end{shell}

\begin{myNotic}{NOTE}
从 C++23 开始,optional 是一个 constexpr 类(请参阅第9章),可以在编译时使用。
\end{myNotic}

\CXXTwentythreeLogo{-40}{25}

