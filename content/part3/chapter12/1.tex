The main programming unit in the procedural paradigm is the procedure or function. Functions are useful primarily because they allow you to write algorithms that are independent of specific values and can thus be reused for many different values. For example, the sqrt() function in C++ calculates the square root of a value supplied by the caller. A square root function that calculates only the square root of one number, such as the number four, would not be particularly useful! The sqrt() function is written in terms of a parameter, which is a stand-in for whatever value the caller passes. Computer scientists say that functions parameterize values.

The object-oriented programming paradigm adds the concept of objects, which group related data and behaviors, but it does not change the way functions and member functions parameterize values.

Templates take the concept of parameterization a step further to allow you to parameterize on types as well as values. Types in C++ include primitives such as int and double, as well as user-defined classes such as SpreadsheetCell and CherryTree. With templates, you can write code that is independent not only of the values it will be given, but also of the types of those values. For example, instead of writing separate stack classes to store ints, Cars, and SpreadsheetCells, you can write one stack class template definition that can be used for any of those types.

Although templates are an amazing language feature, templates in C++ can be syntactically confusing, and thus, many programmers avoid writing templates themselves. However, every professional C++ programmer needs to know how to write them, and every programmer at least needs to know how to use templates, because they are widely used by libraries, such as the C++ Standard Library.

This chapter teaches you about template support in C++ with an emphasis on the aspects that arise in the Standard Library. Along the way, you will learn about some nifty features that you can employ in your programs aside from using the Standard Library.