The first component of the modern C++ random number generation library is a random number engine, responsible for generating the actual random numbers. As mentioned, everything is defined in <random>.

The following random number engines are available:

\begin{cpp}
random_device
\end{cpp}

\begin{cpp}
linear_congruential_engine
\end{cpp}

\begin{cpp}
mersenne_twister_engine
\end{cpp}

\begin{cpp}
subtract_with_carry_engine
\end{cpp}

The random\_device engine is not a software-based generator; it is a special engine that requires a piece of hardware attached to your computer that generates truly non-deterministic random numbers, for example, by using the laws of physics. A classic mechanism measures the decay of a radioactive isotope by counting alpha-particles-per-time-interval, but there are many other kinds of physics-based random-number generators, including measuring the “noise” of reverse-biased diodes (thus eliminating the concerns about radioactive sources in your computer). The details of these mechanisms fall outside the scope of this book. If no such device is attached to the computer, random\_device is free to use one of the software algorithms. The choice of algorithm is up to the library designer. Luckily, most modern computers have proper support for a true random\_device.

The quality of a random number generator is measured by its entropy. The entropy() member function of the random\_device engine returns 0.0 if it is using a software-based pseudorandom number generator and returns a non-zero value if it is using a hardware device. The non-zero value is an estimate of the entropy of the hardware device.

Using a random\_device engine is straightforward:

\begin{cpp}
random_device rnd;
println("Entropy: {}", rnd.entropy());
println("Min value: {}, Max value: {}", rnd.min(), rnd.max());
println("Random number: {}", rnd());
\end{cpp}

A possible output of this program could be as follows:

\begin{shell}
Entropy: 32
Min value: 0, Max value: 4294967295
Random number: 3590924439
\end{shell}

A random\_device is much slower than a pseudorandom number engine. Therefore, if you need to generate a lot of random numbers, use a pseudorandom number engine and generate a seed for it with a random\_device. This is demonstrated in the section “Generating Random Numbers” later in this chapter.

Next to the random\_device engine, there are three pseudorandom number engines:

\begin{itemize}
\item
Linear congruential engine: Requires a minimal amount of memory to store its state. The state is a single integer containing the last generated random number or the initial seed if no random number has been generated yet. The period of this engine depends on an algorithmic parameter and can be up to $2^{64}$ but is usually less. For this reason, the linear congruential engine should not be used when you need high-quality random numbers.

\item
Mersenne twister: Of the three pseudorandom number engines, this one generates the highest quality of random numbers. The period of a Mersenne twister is a Mersenne prime, which is a prime number one less than a power of two. This period is much bigger than the period of a linear congruential engine. The memory required to store the state of a Mersenne twister also depends on its parameters but is much larger than the single integer state of the linear congruential engine. For example, the predefined Mersenne twister mt19937 has a period of $2^{19937}$−1, while the state contains 625 integers or 2.5 kilobytes. It is also one of the fastest engines.

\item
Subtract with carry engine: Requires a state of around 100 bytes; however, the quality of the generated random numbers is less than that of the numbers generated by the Mersenne twister, and it is also slower than the Mersenne twister.
\end{itemize}

The mathematical details of the engines and of the quality of random numbers fall outside the scope of this book. If you want to know more about these topics, you can consult a reference from the “Random Numbers” section in Appendix B, “Annotated Bibliography.”

The random\_device engine is easy to use and doesn’t require any parameters. However, creating an instance of one of the three pseudorandom number generators requires you to specify a number of mathematical parameters, which can be daunting. The selection of parameters greatly influences the quality of the generated random numbers. For example, the definition of the mersenne\_twister\_engine class template looks like this:

\begin{cpp}
template<class UIntType, size_t w, size_t n, size_t m, size_t r,
                UIntType a, size_t u, UIntType d, size_t s,
                UIntType b, size_t t, UIntType c, size_t l, UIntType f>
    class mersenne_twister_engine {...}
\end{cpp}

It requires 14 parameters. The linear\_congruential\_engine and subtract\_with\_carry\_engine class templates also require a number of such mathematical parameters. For this reason, the standard defines a couple of predefined engines. One example is the mt19937 Mersenne twister, which is defined as follows:

\begin{cpp}
using mt19937 = mersenne_twister_engine<uint_fast32_t, 32, 624, 397, 31,
    0x9908b0df, 11, 0xffffffff, 7, 0x9d2c5680, 15, 0xefc60000, 18,
    1812433253>;
\end{cpp}

These parameters are all magic, unless you understand the details of the Mersenne twister algorithm. In general, you do not want to modify any of these parameters unless you are an expert in the mathematics of pseudorandom number generators. Instead, I recommend using the predefined type aliases such as mt19937. A complete list of predefined engines is given in a later section.






