分布是一种数学公式,描述了在某个范围内数字是如何分布的。随机数生成器库附带了一系列分布,可以与伪随机数引擎一起使用,以定义生成的随机数的分布。每个分布的第一行是名称和类模板参数(如果有),接下来是分布的构造函数,这里只显示了每个分布的一个构造函数。有关每个分布的所有构造函数和成员函数的详细列表,请参阅标准库手册(见附录B)。

以下是统一分布:

\begin{cpp}
template<class IntType = int> class uniform_int_distribution
    uniform_int_distribution(IntType a = 0,
                             IntType b = numeric_limits<IntType>::max());
template<class RealType = double> class uniform_real_distribution
    uniform_real_distribution(RealType a = 0.0, RealType b = 1.0);
\end{cpp}

以下是伯努利分布(第一个生成随机布尔值,最后三个生成随机非负整数值,所有这些都根据离散概率分布):

\begin{cpp}
class bernoulli_distribution
    bernoulli_distribution(double p = 0.5);
template<class IntType = int> class binomial_distribution
    binomial_distribution(IntType t = 1, double p = 0.5);
template<class IntType = int> class geometric_distribution
    geometric_distribution(double p = 0.5);
template<class IntType = int> class negative_binomial_distribution
    negative_binomial_distribution(IntType k = 1, double p = 0.5);
\end{cpp}

以下是泊松分布(根据离散概率分布生成随机非负整数值):

\begin{cpp}
template<class IntType = int> class poisson_distribution
    poisson_distribution(double mean = 1.0);
template<class RealType = double> class exponential_distribution
    exponential_distribution(RealType lambda = 1.0);
template<class RealType = double> class gamma_distribution
    gamma_distribution(RealType alpha = 1.0, RealType beta = 1.0);
template<class RealType = double> class weibull_distribution
    weibull_distribution(RealType a = 1.0, RealType b = 1.0);
template<class RealType = double> class extreme_value_distribution
    extreme_value_distribution(RealType a = 0.0, RealType b = 1.0);
\end{cpp}

以下是正态分布:

\begin{cpp}
template<class RealType = double> class normal_distribution
    normal_distribution(RealType mean = 0.0, RealType stddev = 1.0);
template<class RealType = double> class lognormal_distribution
    lognormal_distribution(RealType m = 0.0, RealType s = 1.0);
template<class RealType = double> class chi_squared_distribution
    chi_squared_distribution(RealType n = 1);
template<class RealType = double> class cauchy_distribution
    cauchy_distribution(RealType a = 0.0, RealType b = 1.0);
template<class RealType = double> class fisher_f_distribution
    fisher_f_distribution(RealType m = 1, RealType n = 1);
template<class RealType = double> class student_t_distribution
    student_t_distribution(RealType n = 1);
\end{cpp}

以下是抽样分布:

\begin{cpp}
template<class IntType = int> class discrete_distribution
    discrete_distribution(initializer_list<double> wl);
template<class RealType = double> class piecewise_constant_distribution
    template<class UnaryOperation>
        piecewise_constant_distribution(initializer_list<RealType> bl,
            UnaryOperation fw);
template<class RealType = double> class piecewise_linear_distribution
    template<class UnaryOperation>
        piecewise_linear_distribution(initializer_list<RealType> bl,
            UnaryOperation fw);
\end{cpp}

每个分布都需要一组参数,但解释所有这些数学参数超出了本书的范围。本节的其余部分包括几个示例,以解释分布对生成的随机数的影响。

查看它们的图形表示,就很容易理解分布。以下代码生成一百万个介于1到99之间的随机数,并跟踪特定数字在直方图中随机生成的次数。计数器存储在映射中,其中键是1到99之间的数字。与键关联的值,是该键随机选择的次数。循环之后,结果写入逗号分隔的值文件(CSV),该文件可以在电子表格应用程序中打开。

\begin{cpp}
const unsigned int Start { 1 };
const unsigned int End { 99 };
const unsigned int Iterations { 1'000'000 };

// Uniform distributed Mersenne Twister.
random_device seeder;
mt19937 engine { seeder() };
uniform_int_distribution<int> distribution { Start, End };
auto generator { bind(distribution, engine) };
map<int, int> histogram;
for (unsigned int i { 0 }; i < Iterations; ++i) {
    int randomNumber { generator() };
    // Search map for a key=randomNumber. If found, add 1 to the value associated
    // with that key. If not found, add the key to the map with value 1.
    ++(histogram[randomNumber]);
}

// Write to a CSV file.
ofstream of { "res.csv" };
for (unsigned int i { Start }; i <= End; ++i) {
    of << i << ";" << histogram[i] << endl;
}
\end{cpp}

随后生成的数据可用于生成图形表示,图 23.1 展示了生成的直方图的图表。

水平轴代表生成随机数的范围。图表清楚地显示,1到99范围内的所有数字大约随机选择了10,000次,并且生成的随机数在整个范围内均匀分布。

\myGraphic{0.7}{content/part3/chapter23/images/1.png}{图 23.1}

示例可以修改为根据正态分布生成随机数,而不是均匀分布。只需要做两个小改动。首先,按以下方式修改分布的创建:

\begin{cpp}
normal_distribution<double> distribution { 50, 10 };
\end{cpp}

因为正态分布使用双精度浮点数,所以还需要修改对generator()的调用:

\begin{cpp}
int randomNumber { static_cast<int>(generator()) };
\end{cpp}

图 23.2 展示了根据此正态分布生成的随机数的图形表示。

\myGraphic{0.7}{content/part3/chapter23/images/2.png}{图 23.2}

图表清楚地显示,大多数生成的随机数都集中在范围的中心。这个例子中,50 大约随机选择了 40,000 次,而像 20 和 80 这样的值只选择了大约 500 次。


















