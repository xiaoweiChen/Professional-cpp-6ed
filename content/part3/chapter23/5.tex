Before you can generate any random number, you first need to create an instance of an engine. If you use a software-based engine, you also need to define a distribution. A distribution is a mathematical formula describing how numbers are distributed within a certain range. The recommended way to create an engine is to use one of the predefined engines discussed in the previous section.

The following example uses the predefined engine called mt19937, using a Mersenne twister engine. This is a software-based generator. Just as with the old rand() generator, a software-based engine must be initialized with a seed. The seed used with srand() was often the current time. In modern C++, it’s recommended to use a random\_device to generate a seed. Here is an example:

\begin{cpp}
random_device seeder;
mt19937 engine { seeder() };
\end{cpp}

As mentioned earlier, most modern systems have a random\_device implementation with proper entropy. If you are not sure that the random\_device implementation on the system your code will be running on has proper entropy, then you can use a time-based seed as a fallback:

\begin{cpp}
random_device seeder;
const auto seed { seeder.entropy() ? seeder() : time(nullptr) };
mt19937 engine { static_cast<mt19937::result_type>(seed) };
\end{cpp}

Next, a distribution is defined. This example uses a uniform integer distribution, for the range 1 to 99. Distributions are explained in detail in the next section, but this uniform distribution is easy enough to use for this example:

\begin{cpp}
uniform_int_distribution<int> distribution { 1, 99 };
\end{cpp}

Once the engine and distribution are defined, random numbers can be generated by calling the function call operator of the distribution and passing the engine as an argument. For this example, this is written as distribution(engine):

\begin{cpp}
println("{}", distribution(engine));
\end{cpp}

As you can see, to generate a random number using a software-based engine, you always need to specify the engine and distribution. The std::bind() utility, introduced in Chapter 19, “Function Pointers, Function Objects, and Lambda Expressions,” can be used to remove the need to specify both the distribution and the engine when generating a random number. The following example uses the same mt19937 engine and uniform distribution as the previous example, but it defines generator by using std::bind() to bind engine as the first argument to distribution(). This way, you can call generator() without any arguments to generate a random number. The example then demonstrates the use of generator() in combination with the constrained ranges::generate() algorithm to fill a vector of ten elements with random numbers. The generate() algorithm is discussed in Chapter 20, “Mastering Standard Library Algorithms.”

\begin{cpp}
auto generator { bind(distribution, engine) };

vector<int> values(10);
ranges::generate(values, generator);

println("{:n}", values);
\end{cpp}

\begin{myNotic}{NOTE}
Remember that the generate() algorithm overwrites existing elements and does not insert new elements. This means you first need to size the vector to hold the number of elements you need and then call the generate() algorithm. The previous example sizes the vector by specifying the size as argument to the constructor.
\end{myNotic}

Even though you don’t know the exact type of generator, it’s still possible to pass generator to another function that wants to use that generator. You have several options: use a parameter of type std::function<int()> or use a function template. The previous example can be adapted to generate random numbers in a function called fillVector(). Here is an implementation using std::function:

\begin{cpp}
void fillVector(vector<int>& values, const function<int()>& generator)
{
    ranges::generate(values, generator);
}
\end{cpp}

Here is a constrained function template variant:

\begin{cpp}
template <invocable T>
void fillVector(vector<int>& values, const T& generator)
{
    ranges::generate(values, generator);
}
\end{cpp}

This can be simplified using the abbreviated function template syntax:

\begin{cpp}
void fillVector(vector<int>& values, const auto& generator)
{
    ranges::generate(values, generator);
}
\end{cpp}

Finally, this function can be used as follows:

\begin{cpp}
vector<int> values(10);
fillVector(values, generator);
\end{cpp}

\begin{myWarning}{WARNING}
The random number generators are not thread safe. If you need to generate random numbers in multiple threads, you should create a generator in each thread and not share one generator among multiple threads. Chapter 27, “Multithreaded Programming with C++,” introduces multithreading.
\end{myWarning}








