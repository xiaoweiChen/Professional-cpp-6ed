通过解决下面的练习,可以练习本章讨论的内容。所有练习的解决方案都可以在本书的网站\url{www.wiley.com/go/proc++6e}下载到源码。然而,若在练习中卡住了,在从网站上寻找解决方案之前,可以考虑先重读本章的部分内容,试着自己找到答案。

\begin{itemize}
\item
Exercise 20-1: Use your favorite Standard Library Reference to look up the parameters for the ranges::fill() algorithm. Ask the user for a number, and then use fill() to fill a vector of 10 integers with the given number. Write the contents of the vector to the standard output for verification. Provide a second solution using the std::fill() algorithm.

\item
Exercise 20-2: Look back at the “Permutation Algorithms” section in Chapter 16, and then use a Standard Library Reference to figure out their parameters. Write a program that asks the user to enter a few numbers, and then use one of the permutation algorithms to print out all possible permutations of those numbers. Provide two solutions, one using only constrained algorithms and a second using legacy, unconstrained algorithms.

\item
Exercise 20-3: Write a function called trim() that removes all whitespace at the beginning and end of a given string and returns the result. Use only constrained algorithms. Tip: to check if a character c is a whitespace character, you can use std::isspace(c), defined in <cctype>. It returns a non-zero value if c is a whitespace character, 0 otherwise. Test your implementation with several strings in your main() function.

\item
Exercise 20-4: Use a constrained algorithm to create a vector containing the numbers 1 to 20. Then, using a single constrained algorithm, copy all even and odd numbers to evens and odds containers without doing any space reservation on those containers, and, still with this single algorithm call, make sure the even numbers are in ascending sequence, while the odd numbers are in descending sequence. Carefully choose the type for the evens and odds containers. Hint: maybe there is something in Chapter 17 that you could use.

\item
Exercise 20-5: The solution for Exercise 20-3 uses only constrained algorithms. Can you do the same using only legacy, unconstrained algorithms?
\end{itemize}












