
处理动态内存使用 new/delete/new[]/delete[] 以及其他底层内存操作,这些操作非常容易出错。很难准确指出哪些情况会导致与内存相关的错误,每个内存泄漏或错误指针都有差别,解决内存问题没有万能的方法。本节讨论了一些常见的问题类别,以及一些可用于检测和解决问题的工具。

\mySubsubsection{7.4.1.}{数据缓冲区分配不足和越界内存访问}

分配不足是C字符串中的一个常见问题,发生在开发者未能为尾部的 '\verb|\|0' 终止符分配额外的字符时。当开发者固定最大大小时,字符串的分配不足也会发生。基本的内置C字符串函数并不遵循固定大小——它们会愉快地将字符串写入未知的内存区域。

以下代码演示了分配不足,从网络连接读取数据并将其放入C字符串中。这是在一个循环中完成的,网络连接一次只能接收少量数据。每次循环中,调用 getMoreData(),返回指向动态分配内存的指针。当 getMoreData() 返回 nullptr 时,表示所有数据都已接收完毕。strcat() 是一个C函数,将第二个参数给定的C字符串连接到第一个参数给定的C字符串的末尾,并期望目标缓冲区足够大。

\begin{cpp}
char buffer[1024] { 0 }; // Allocate a whole bunch of memory.
while (true) {
    char* nextChunk { getMoreData() };
    if (nextChunk == nullptr) {
        break;
    } else {
        strcat(buffer, nextChunk); // BUG! No guarantees against buffer overrun!
        delete [] nextChunk;
    }
}
\end{cpp}

解决这个示例中可能出现分配不足的三种方法,按优先级递减:

\begin{enumerate}
\item
使用C++字符串,会在进行连接时自动处理与连接相关的内存。

\item
将缓冲区分配作为全局变量或栈上的变量,改为在堆上分配。当剩余空间不足时,分配一个新的足够大的缓冲区来至少容纳当前内容加上新的一块,将原始缓冲区复制到新缓冲区,添加新内容,然后删除原始缓冲区。

\item
创建一个版本的 getMoreData(),其接受一个最大计数(包括 '\verb|\|0' 字符)并返回不超过该计数字符的数量,跟踪缓冲区中剩余的空间和当前位置。
\end{enumerate}

数据缓冲区分配不足通常会导致越界内存访问。如果正在用数据填充内存缓冲区,当缓冲区比实际大时,可能会写入到缓冲区之外。直到覆盖重要的内存部分,程序就会崩溃。如果覆盖程序中对象关联的内存会发生什么,那场面可不兴看呀!

越界内存访问还发生在处理丢失了 '\verb|\|0' 终止字符的C风格字符串时,一个不正确终止的字符串传递给以下函数,将用 ‘m’ 字符填充字符串,并愉快地继续用 ‘m’ 填充字符串后面的内存内容,覆盖字符串范围外的内存。

\begin{cpp}
void fillWithM(char* text)
{
    int i { 0 };
    while (text[i] != '\0') {
        text[i] = 'm';
        ++i;
    }
}
\end{cpp}

导致在数组末尾之外写入内存的错误,通常称为缓冲区溢出错误。多种高知名度的恶意软件程序(如病毒和蠕虫)可以利用这些错误,狡猾的黑客可以利用覆盖内存部分的能力将代码注入正在运行的程序。

\begin{myWarning}{WARNING}
避免使用旧的C风格字符串和数组,这些字符串和数组不提供任何保护。使用现代安全的构造,如C++字符串和vector,可管理所有内存。
\end{myWarning}

\mySubsubsection{7.4.2.}{内存泄漏}

现代C++中,没有内存泄漏。所有的内存管理都由更高级的类,如 std::vector、string 等处理。只有手动内存分配和释放时,才会出现内存泄漏。

找到并修复这样的内存泄漏可能会让人感到沮丧。程序终于运行良好,看起来给出了正确的结果。然后,随着程序的运行,程序会吞噬越来越多的内存——程序有内存泄漏。

内存泄漏发生在分配内存之后,忘记释放。起初,这听起来像是由于粗心编程的结果。如果每个新类都有一个对应的 delete,应该就不会有内存泄漏,对吗?实际上,这并不总是正确的。在以下代码中,Simple 类正确地编写以释放它分配的内存。但当调用 doSomething() 时,outSimplePtr 指针改变为另一个 Simple 对象,而没有删除旧的 Simple 对象,出现内存泄漏。当失去了对一个对象的引用,则不可能删除它。

\begin{cpp}
class Simple
{
    public:
    Simple() { m_intPtr = new int{}; }
    ˜Simple() { delete m_intPtr; }
    void setValue(int value) { *m_intPtr = value; }
    private:
    int* m_intPtr;
};

void doSomething(Simple*& outSimplePtr)
{
    outSimplePtr = new Simple{}; // BUG! Doesn't delete the original.
}

int main()
{
    Simple* simplePtr { new Simple{} }; // Allocate a Simple object.
    doSomething(simplePtr);
    delete simplePtr; // Only cleans up the second object.
}
\end{cpp}

\begin{myWarning}{WARNING}
这段代码仅用于演示!生产代码中,m\_intPtr 和 simplePtr 都不应该使用原始指针,应该使用智能指针。
\end{myWarning}

前一个示例中,内存泄漏可能源于开发者之间的沟通不畅或代码文档不足。doSomething() 的调用者可能没有意识到变量通过引用传递,没有理由期望指针会重新分配。如果注意到参数是一个指向非常量指针的引用,可能会怀疑有什么奇怪的事情发生,但是 doSomething() 附近没有注释来解释这种行为。

\mySamllsection{在 Windows 中使用 Visual C++ 查找和修复内存泄漏}

内存泄漏很难追踪,因为无法查看内存并查看哪些对象未使用,以及最初在哪里分配,有一些工具可以做这件事。内存泄漏检测工具有昂贵的专业软件包,也有可免费下载的工具。如果使用 Microsoft Visual C++,其调试库内置了对内存泄漏检测的支持。默认情况下,除非创建 MFC 项目,否则不会启用此功能。要在其他项目中启用它,需要从代码的开头开始包含以下三行。这些使用了 \#define 预处理器宏,会在第 11 章中解释。不过,现在只需这三行即可。

\begin{cpp}
#define _CRTDBG_MAP_ALLOC
#include <cstdlib>
#include <crtdbg.h>
\end{cpp}

顺序按如上所示。接下来,需要将 new 操作符重新定义。这使用了几个其他的预处理器宏,所有这些宏都会在第 11 章中解释。

\begin{cpp}
#ifdef _DEBUG
    #ifndef DBG_NEW
        #define DBG_NEW new ( _NORMAL_BLOCK , __FILE__ , __LINE__ )
        #define new DBG_NEW
    #endif
#endif // _DEBUG
\end{cpp}

\#ifdef \_DEBUG 确保在编译应用程序的调试版本时才重新定义 new。通常情况下,发布版本不会进行内存泄漏检测(有性能惩罚)。

最后,需要在 main() 函数中添加以行作为第一行:

\begin{cpp}
_CrtSetDbgFlag(_CRTDBG_ALLOC_MEM_DF | _CRTDBG_LEAK_CHECK_DF);
\end{cpp}

这告诉 Visual C++ CRT(C 运行时)库在应用程序退出时,将所有检测到的内存泄漏写入调试输出控制台。对于之前的泄漏程序,调试控制台将包含(类似)以下内容的行:

\begin{shell}
Detected memory leaks!
Dumping objects ->
c:\leaky\leaky.cpp(15) : {147} normal block at 0x014FABF8, 4 bytes long.
 Data: < > 00 00 00 00
c:\leaky\leaky.cpp(33) : {146} normal block at 0x014F5048, 4 bytes long.
 Data: <Pa > 50 61 20 01
Object dump complete.
\end{shell}

输出清楚地显示了在哪个文件和哪一行内存分配但从未释放。行号在文件名后面的括号中立即出现。花括号之间的数字是对程序启动以来内存分配的计数器。例如,\{147\} 表示程序自启动以来的第 147 次分配。可以使用 VC++ \_CrtSetBreakAlloc() 函数告诉 VC++ 调试运行时,在执行特定分配时中断调试器。可以在 main() 函数的开头添加以下行,指示调试器在执行第 147 次分配时中断:

\begin{cpp}
_CrtSetBreakAlloc(147);
\end{cpp}

这个泄漏程序中,有两个泄漏:第一个 Simple 对象从未删除(第 33 行)和在堆上创建的整数(第 15 行)。在 Visual C++ 调试器输出窗口中,只需双击其中一个内存泄漏,就会自动跳转到代码中的那一行。

像 Microsoft Visual C++(本节讨论)和 Valgrind(下一节讨论)这样的程序不能修复泄漏。这些工具提供信息,可以使用这些信息来定位实际的问题。通常,这涉及通过代码找出修改指向原始对象的指针,并使其无法释放的位置。大多数调试器都提供“监视点”功能,可以在发生这种情况时中断程序的执行。

\mySamllsection{在 Linux 中使用 Valgrind 查找和修复内存泄漏}

Valgrind 是一个免费的开源工具,适用于 Linux,可以为定位代码中泄漏对象的确切行。

以下输出是由在之前的泄漏程序上运行 Valgrind 生成的,定位了内存分配但从未释放的确切位置。Valgrind 发现了相同的两个内存泄漏——第一个 Simple 对象从未删除,以及其创建的整数对象:

\begin{shell}
==15606== ERROR SUMMARY: 0 errors from 0 contexts (suppressed: 0 from 0)
==15606== malloc/free: in use at exit: 8 bytes in 2 blocks.
==15606== malloc/free: 4 allocs, 2 frees, 16 bytes allocated.
==15606== For counts of detected errors, rerun with: -v
==15606== searching for pointers to 2 not-freed blocks.
==15606== checked 4455600 bytes.
==15606==
==15606== 4 bytes in 1 blocks are still reachable in loss record 1 of 2
==15606==    at 0x4002978F: __builtin_new (vg_replace_malloc.c:172)
==15606==    by 0x400297E6: operator new(unsigned) (vg_replace_malloc.c:185)
==15606==    by 0x804875B: Simple::Simple() (leaky.cpp:4)
==15606==    by 0x8048648: main (leaky.cpp:24)
==15606==
==15606==
==15606== 4 bytes in 1 blocks are definitely lost in loss record 2 of 2
==15606==    at 0x4002978F: __builtin_new (vg_replace_malloc.c:172)
==15606==    by 0x400297E6: operator new(unsigned) (vg_replace_malloc.c:185)
==15606==    by 0x8048633: main (leaky.cpp:20)
==15606==    by 0x4031FA46: __libc_start_main (in /lib/libc-2.3.2.so)
==15606==
==15606== LEAK SUMMARY:
==15606==     definitely lost: 4 bytes in 1 blocks.
==15606==     possibly lost:   0 bytes in 0 blocks.
==15606==     still reachable: 4 bytes in 1 blocks.
==15606==          suppressed: 0 bytes in 0 blocks.
\end{shell}

\begin{myWarning}{WARNING}
强烈建议使用 std::vector、array、string、智能指针(本章后面讨论)和其他现代 C++ 组件来避免内存泄漏。
\end{myWarning}

\mySubsubsection{7.4.3.}{双重释放和无效指针}

使用 delete 释放与指针关联的内存,该内存就可以在程序的其他部分使用。然而,这并不意味着无法继续使用该指针,现在它是一个悬空指针。双重释放也是一个问题。如果第二次使用 delete 对指针进行操作,程序可能会释放已经分配给另一个对象的内存。

双重释放和使用已释放的内存都是难以追踪的问题,因为问题可能不会立即出现。如果两次释放在相对较短的时间内发生,程序可能会无限期地工作,因为相关的内存可能还没有那么快就重新使用。

这种行为是否能够工作或继续工作并没有保证,内存分配器在删除后没有义务保留对象。即使有效,使用已删除对象也是非常糟糕的编程风格。

为了避免双重释放和使用已释放的内存,应该在释放内存后将指针置为 nullptr。

许多内存泄漏检测程序也具备检测双重释放和使用已释放对象的能力。












