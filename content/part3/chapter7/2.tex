
已经了解了指针和数组之间的一些交集,堆分配的数组通过指向其第一个元素的指针来引用。基于栈的数组使用带有变量声明的数组语法([])来引用,然而,这种重叠并没有结束。指针和数组还有着复杂的关系。

\mySubsubsection{7.2.1.}{数组退化为指针}

基于堆的数组可以使用指针来引用数组,还可以使用指针语法来访问基于栈的数组的元素。数组的地址实际上是第一个元素(索引0)的地址。编译器知道,当通过变量名整体引用数组时,实际上是在引用第一个元素的地址,指针就像基于堆的数组一样工作。以下代码在栈上创建了一个零初始化的数组,并使用指针来访问:

\begin{cpp}
int myIntArray[10] {};

int* myIntPtr { myIntArray };

// Access the array through the pointer.
myIntPtr[4] = 5;
\end{cpp}

通过指针引用基于栈的数组的能力,在将数组传递给函数时很有用。以下函数接受一个整数数组作为指针。请注意,因为指针没有说明长度,所以调用者需要显式传入数组的长度,这是应该使用现代容器(如标准库提供的容器)的另一个原因。

\begin{cpp}
void doubleInts(int* theArray, size_t size)
{
    for (size_t i { 0 }; i < size; ++i) { theArray[i] *= 2; }
}
\end{cpp}

此函数的调用者可以传递基于栈的数组或基于堆的数组。堆数组,指针已经存在,并且按值传递给函数。栈数组,调用者可以传递数组变量,编译器自动将数组变量视为指向数组的指针,或者显式传递第一个元素的地址。以下展示了这三种形式:

\begin{cpp}
size_t arrSize { 4 };
int* freeStoreArray { new int[arrSize]{ 1, 5, 3, 4 } };
doubleInts(freeStoreArray, arrSize);
delete [] freeStoreArray;
freeStoreArray = nullptr;

int stackArray[] { 5, 7, 9, 11 };
arrSize = std::size(stackArray); // Since C++17, requires <array>
//arrSize = sizeof(stackArray) / sizeof(stackArray[0]); // Pre-C++17, see Ch1
doubleInts(stackArray, arrSize);

doubleInts(&stackArray[0], arrSize);
\end{cpp}

数组的参数传递语义与指针的非常相似,编译器在将数组传递给函数时,将数组视为指针。接受数组作为参数并更改数组内值的函数,实际上是在更改原始数组,而不是其副本。就像指针一样,传递数组实际上模仿了按引用传递的功能,因为实际传递给函数的是原始数组的地址,而不是副本。以下doubleInts()的实现即使参数是数组而不是指针,也会更改原始数组:

\begin{cpp}
void doubleInts(int theArray[], size_t size)
{
    for (size_t i { 0 }; i < size; ++i) { theArray[i] *= 2; }
}
\end{cpp}

函数原型中theArray后面的方括号内的数字都会忽略。以下三个是相同的:

\begin{cpp}
void doubleInts(int* theArray, size_t size);
void doubleInts(int theArray[], size_t size);
void doubleInts(int theArray[2], size_t size);
\end{cpp}

可能会想知道为什么会这样。为什么编译器在函数定义中使用数组语法时,不直接复制数组呢?这是为了效率——复制数组的元素需要时间,并且可能占用大量的内存。通过始终传递指针,编译器不需要包含复制数组的代码。

有一种方法可以将已知长度的基于栈的数组“按引用”传递给函数,尽管语法不太明显。这不起作用基于堆的数组。例如,以下doubleIntsStack()只接受大小为4的栈数组:

\begin{cpp}
void doubleIntsStack(int (&theArray)[4]);
\end{cpp}

函数模板,可以让编译器自动推导基于栈的数组的大小:

\begin{cpp}
template <size_t N>
void doubleIntsStack(int (&theArray)[N])
{
    for (size_t i { 0 }; i < N; ++i) { theArray[i] *= 2; }
}
\end{cpp}

\begin{myNotic}{NOTE}
建议函数接受类型为std::span的参数,而不是直接传递C风格数组,std::span在第18章中介绍,其包装了指向数组的指针及其大小!
\end{myNotic}

\mySubsubsection{7.2.2.}{不是所有指针都是数组!}

因为编译器允许预期指针的地方传递数组,比如上一节中的 doubleInts() 函数,可能会误以为指针和数组相同。实际上,它们之间有微妙的区别。指针和数组共享许多属性,有时可以互换使用(如前所示),但它们并不相同。

一个单独的指针没有意义,可能指向随机内存、单个对象或数组。可以使用数组语法与指针,但并不总合适的,因为指针并不总是数组。例如,考虑以下代码:

\begin{cpp}
int* ptr { new int };
\end{cpp}

指针 ptr 是一个指针,但它不是一个数组。可以使用数组语法(ptr[0])访问所指的值,但这样做在风格上有问题。实际上,将数组语法用于非数组指针是引发错误的导火索。ptr[1] 处的内存可能是任何内容!

\begin{myWarning}{WARNING}
数组会自动退化为指针,但并非所有指针都是数组。
\end{myWarning}











