
是否在程序中使用异常取决于个人偏好。然而,无论是否使用异常,都强烈建议为程序制定一个正式的错误处理计划。如果使用异常,通常更容易制定一个错误处理方案,但没有异常也不是不可能。一个好的计划最重要的方面是,在程序的所有模块中统一的错误处理。确保项目上的每个开发者,都能理解和遵循错误处理规则。

本节讨论了在异常的背景下最常见的错误处理问题,但这些问题也适用于不使用异常的程序。

\mySubsubsection{14.8.1.}{内存分配错误}

目前为止本书中的所有示例都忽略了这一可能性,但内存分配可能会失败。在64位平台上,这种情况几乎永远不会发生,但在移动设备或有年头的系统中,内存分配可能会失败。在这样的系统中,必须考虑内存分配失败的情况。C++提供了几种不同的方式来处理内存错误。

new和new[]的默认行为是在无法分配内存时,抛出类型为bad\_alloc的异常,其定义在<new>中。代码可以捕获这些异常并适当处理它们。

将所有对new和new[]的调用都包装在try/catch中是不现实的,但至少在尝试分配大量内存时应该这样做。以下示例展示了如何捕获内存分配异常:

\begin{cpp}
int* ptr { nullptr };
size_t integerCount { numeric_limits<size_t>::max() };
println("Trying to allocate memory for {} integers.", integerCount);
try {
    ptr = new int[integerCount];
} catch (const bad_alloc& e) {
    auto location { source_location::current() };
    println(cerr, "{}({}): Unable to allocate memory: {}",
        location.file_name(), location.line(), e.what());
    // Handle memory allocation failure.
    return;
}
// Proceed with function that assumes memory has been allocated.
\end{cpp}

请注意,此代码使用source\_location在错误消息中包含文件名和当前行号,这有助于调试。

当然,可以使用一个try/catch块在程序的高层处理new的失败(如果这对程序有效)。

另一个要考虑的问题是,记录错误可能会尝试分配内存。如果new失败,可能没有足够的内存来记录错误信息。

\mySamllsection{不抛出异常的new}

如果不喜欢异常,可以退回到C模型,其中内存分配例程在无法分配内存时返回nullptr。C++提供了new和new[]的非抛出重载,如果无法分配内存,则返回nullptr,而不是抛出异常。可以通过使用new(nothrow)语法来使用:

\begin{cpp}
int* ptr { new(nothrow) int[integerCount] };
if (ptr == nullptr) {
    auto location { source_location::current() };
    println(cerr, "{}({}): Unable to allocate memory!",
        location.file_name(), location.line());
    // Handle memory allocation failure.
    return;
}
// Proceed with function that assumes memory has been allocated.
\end{cpp}

\begin{myNotic}{NOTE}
我不建议使用不抛出异常的new,在分配失败时抛出的异常无法忽略,而使用非抛出new时很容易忘记检查nullptr。
\end{myNotic}

\mySamllsection{自定义内存分配的失败行为}

C++允许指定一个新分配器回调函数。默认情况下,没有分配器,所以new和new[]只是抛出bad\_alloc异常。如果有分配器,内存分配例程在内存分配失败时,会调用分配器,而非抛出异常。如果分配器返回,内存分配例程会再次尝试分配内存,如果失败,会再次调用分配器。这种循环可能会变成无限循环,除非分配器通过以下三种方式解除循环。实际上,其中一些选项会比其他选项好。

\begin{itemize}
\item
释放更多内存。一种技巧是在程序启动时分配一大块内存,然后在需要时释放它。一个例子是,在遇到分配错误时需要保存用户状态以防止丢失工作。关键是在程序启动时分配足够大的内存块,以便可以进行完整的文档保存操作。当分配器触发时,就释放这个块,保存文档,重新启动应用程序,并让它重新加载保存的文档。

\item
抛出异常。C++标准规定,如果分配器中抛出异常,则必须是一个bad\_alloc异常或派生自bad\_alloc的异常。以下是一些例子:
\begin{itemize}
\item
编写并抛出document\_recovery\_alloc异常(派生自bad\_alloc),这个异常可以在应用程序的某个地方捕获,以触发文档保存操作和应用程序的重新启动。

\item
编写并抛出please\_terminate\_me异常(派生自bad\_alloc),在顶层函数中(例如,main()),可以捕获这个异常并处理它,通过从顶层函数返回来终止程序。建议通过从顶层函数返回来终止程序,而不是调用如exit()之类的函数。
\end{itemize}

\item
设置一个不同的new分配器。理论上,可以有一系列new分配器,每个都尝试创建内存并设置一个新的new分配器(如果分配失败),所以这样的场景会更加复杂。
\end{itemize}

如果在新分配器中不做这些事情,内存分配失败都会导致无限循环。

如果有一些可以失败,但不想调用新分配器的内存分配,可以在这些情况下简单地将new分配器设置回nullptr。

也可以通过调用<new>中声明的set\_new\_handler()来设置分配器。以下是设置分配器的示例,记录错误消息并抛出异常:

\begin{cpp}
class please_terminate_me : public bad_alloc { };

void myNewHandler()
{
    println(cerr, "Unable to allocate memory.");
    throw please_terminate_me {};
}
\end{cpp}

分配器必须不接受任何参数,也不返回任何值。这个分配器抛出一个please\_terminate\_me异常,如前面列表所建议。可以这样激活这个分配器:

\begin{cpp}
int main()
{
    try {
        // Set the new new_handler and save the old one.
        new_handler oldHandler { set_new_handler(myNewHandler) };

        // Generate allocation error.
        size_t numInts { numeric_limits<size_t>::max() };
        int* ptr { new int[numInts] };

        // Reset the old new_handler.
        set_new_handler(oldHandler);
    } catch (const please_terminate_me&) {
        auto location { source_location::current() };
        println(cerr, "{}({}): Terminating program.",
            location.file_name(), location.line());
        return 1;
    }
}
\end{cpp}

new\_handler是一个类型别名,用于set\_new\_handler()函数所接受的函数指针。

\mySubsubsection{14.8.2.}{构造函数中的错误}

在开发者发现异常之前,构造函数中的错误处理会经常困扰着他们。如果构造函数无法正确构造对象怎么办?构造函数没有返回值,所以标准的异常前错误处理机制不起作用。如果没有异常,能做的最好的事情就是在对象中设置一个标志,表明它没有正确构造。可以提供一个成员函数,比如checkConstructionStatus(),返回那个标志的值,并希望客户端在构造对象后记得调用这个成员函数。

异常提供了一个更好的解决方案。可以从构造函数中抛出异常,即使不能返回值(有一个例外,不要从全局对象的构造函数中抛出异常。这样的异常无法捕获,因为这些对象可能在main()开始执行之前就已经构造了)。有了异常,可以很容易地告诉客户对象构造是否成功。然而,有一个主要问题:如果异常离开构造函数,那个对象的析构函数将永远不会调用!因此,允许异常离开构造函数之前,必须小心地在构造函数中清理资源并释放已分配的内存。

本节描述了一个矩阵类模板作为正确处理异常的示例。注意,这个例子使用了一个名为m\_matrix的裸指针来演示。在生产代码中,应该避免使用裸指针。矩阵类模板的定义如下所示:

\begin{cpp}
export template <typename T>
class Matrix final
{
    public:
        explicit Matrix(std::size_t width, std::size_t height);
        ~Matrix();
        // Copy/move ctors and copy/move assignment operators deleted (omitted).
    private:
        void cleanup();

        std::size_t m_width { 0 };
        std::size_t m_height { 0 };
        T** m_matrix { nullptr };
};
\end{cpp}

矩阵类的实现如下。第一个new的调用没有用try/catch块保护。即使第一个new抛出异常,构造函数还没有分配需要释放的其他东西,所以没关系。如果随后的new抛出异常,构造函数必须清理所有已经分配的内存。构造函数不知道T构造函数本身可能会抛出什么异常,所以通过...捕获所有异常,然后将捕获的异常嵌套在bad\_alloc异常中。第一个new调用分配的数组使用\{\}语法零初始化,所以每个元素都是nullptr。因为可以对nullptr使用delete,这使得cleanup()成员函数更容易使用。

\begin{cpp}
template <typename T>
Matrix<T>::Matrix(std::size_t width, std::size_t height)
{
    m_matrix = new T*[width] {}; // Array is zero-initialized!

    // Don't initialize the m_width and m_height members in the ctor-
    // initializer. These should only be initialized when the above
    // m_matrix allocation succeeds!
    m_width = width;
    m_height = height;

    try {
        for (std::size_t i { 0 }; i < width; ++i) {
            m_matrix[i] = new T[height];
        }
    } catch (...) {
        std::println(std::cerr, "Exception caught in constructor, cleaning up...");
        cleanup();
        // Nest any caught exception inside a bad_alloc exception.
        std::throw_with_nested(std::bad_alloc {});
    }
}

template <typename T>
Matrix<T>::~Matrix()
{
    cleanup();
}

template <typename T>
void Matrix<T>::cleanup()
{
    for (std::size_t i { 0 }; i < m_width; ++i) {
        delete[] m_matrix[i];
    }
    delete[] m_matrix;
    m_matrix = nullptr;
    m_width = m_height = 0;
}
\end{cpp}

\begin{myWarning}{WARNING}
记住,如果异常离开构造函数,那对象的析构函数将永远不会调用!
\end{myWarning}

矩阵类模板可以按照以下方式进行测试。为了简洁,省略了在main()中捕获bad\_alloc异常。

\begin{cpp}
class Element
{
    // Kept to a bare minimum, but in practice, this Element class
    // could throw exceptions in its constructor.
    private:
    int m_value;
};

int main()
{
    Matrix<Element> m { 10, 10 };
}
\end{cpp}

将继承混合在一起时,可能会想知道会发生什么。基类构造函数在派生类构造函数之前运行。如果派生类构造函数抛出异常,C++将执行构造基类的析构函数。

\begin{myNotic}{NOTE}
C++保证运行所有构造的“子对象”的析构函数,所以没有异常的构造函数都会使相应的析构函数运行。
\end{myNotic}

\mySubsubsection{14.8.3.}{构造函数使用函数尝试块}

目前为止,本章讨论的异常机制非常适合在函数内部处理异常,但应该如何处理构造函数的初始化器中抛出的异常呢?本节介绍了一个名为函数尝试块的功能,可以捕获这些异常。函数尝试块对普通函数和构造函数都有效。本节主要关注其在构造函数中的使用。即使是经验丰富的C++开发者,大多数C++开发者甚至不知道这个功能的存在(尽管已经引入很长时间了)。

以下伪代码展示了用于构造函数的基本函数尝试块语法:

\begin{cpp}
MyClass::MyClass()
try
: <ctor-initializer>
{
    /* ... constructor body ... */
}
catch (const exception& e)
{
    /* ... */
}
\end{cpp}

try关键字应该在始化器的开始处。catch语句应该在构造函数的闭括号之后,将它们放在构造函数主体之外。使用函数尝试块时,应该记住以下限制和指导原则:

\begin{itemize}
\item
catch语句可捕获初始化器或构造函数主体直接或间接抛出的任何异常。

\item
catch语句必须重新抛出现有异常或抛出新的异常。如果catch语句不做这些事情,运行时会自动重新抛出现有异常。

\item
catch语句可以访问传递给构造函数的参数。

\item
当函数尝试块中的catch语句捕获异常时,将销毁所有在执行catch语句之前构造的基类和对象的成员。

\item
catch语句中,不应该访问数据成员,因为这些成员在执行catch语句之前已经销毁。然而,如果对象包含非类数据成员,例如原始指针,可以在抛出异常之前访问它们。如果拥有原始资源,必须在catch语句中释放。

\item
函数尝试块中的构造函数catch语句不能使用return关键字。
\end{itemize}

基于这些限制,函数尝试块在构造函数中仅在有限的情况下有用:

\begin{itemize}
\item
将初始化器中抛出的异常转换为另一个异常

\item
将消息记录到日志文件

\item
在抛出异常之前,释放在初始化器中分配的资源
\end{itemize}

以下示例演示了如何使用函数尝试块。代码定义了一个名为SubObject的类。它只有一个构造函数,该构造函数抛出一个runtime\_error类型的异常:

\begin{cpp}
class SubObject
{
    public:
        explicit SubObject(int i) {
            throw runtime_error { "Exception by SubObject ctor" }; }
};
\end{cpp}

接下来,MyClass类有一个int*类型的数据成员,另一个是SubObject类型:

\begin{cpp}
class MyClass
{
    public:
        MyClass();
    private:
        int* m_data { nullptr };
        SubObject m_subObject;
};
\end{cpp}

SubObject类没有默认构造函数,所以需要在MyClass的初始化器中初始化m\_subObject。MyClass的构造函数使用函数尝试块,来捕获初始化器中抛出的异常:

\begin{cpp}
MyClass::MyClass()
try
    : m_data { new int[42]{ 1, 2, 3 } }, m_subObject { 42 }
{
    /* ... constructor body ... */
}
catch (const exception& e)
{
    // Cleanup memory.
    delete[] m_data;
    m_data = nullptr;
    println(cerr, "function-try-block caught: '{}'", e.what());
}
\end{cpp}

记住,函数尝试块中的构造函数catch语句,必须重新抛出现有异常或抛出新的异常。前面的catch语句没有抛出任何异常,所以C++运行时会自动重新抛出现有异常。以下是一个简单的函数,仅使用了MyClass:

\begin{cpp}
int main()
{
    try {
        MyClass m;
    } catch (const exception& e) {
        println(cerr, "main() caught: '{}'", e.what());
    }
}
\end{cpp}

输出如下所示:

\begin{shell}
function-try-block caught: 'Exception by SubObject ctor'
main() caught: 'Exception by SubObject ctor'
\end{shell}

请注意,本例中的代码容易出错,不推荐使用。这个例子的情况的适当解决方案是将m\_data成员转换为一个容器(std::vector),或者一个智能指针(unique\_ptr),并移除函数尝试块。

函数尝试块不仅限于构造函数,也可以用于普通函数。对于普通函数,没有使用函数尝试块的必要,因为它们可以很容易地转换为函数体内的简单try/catch块。使用函数尝试块与普通函数与构造函数的一个显著不同之处是,在catch语句中不需要重新抛出现有异常或抛出新的异常,C++运行时也不会自动重新抛出现有异常。在这种catch语句中,可以使用return关键字。

\begin{myWarning}{WARNING}
避免使用函数尝试块!

函数尝试块通常只在拥有原始资源作为数据成员时才需要,原始资源应该通过使用RAII类(如std::vector或unique\_ptr)来避免使用。RAII设计模式将在第32章中介绍。
\end{myWarning}

\mySubsubsection{14.8.4.}{析构函数中的错误}

应该在析构函数处理所有出现的错误,不应该让任何异常从析构函数中抛出,原因有几个:

\begin{enumerate}
\item
客户将采取什么行动?客户不会显式地调用析构函数;析构函数会自动为他们调用。如果从析构函数中抛出异常,客户应该做什么?客户无法采取合理的行动,因此没有理由让这段代码承担异常处理的负担。

\item
析构函数是释放对象使用的内存和资源的机会。如果因为异常而提前退出函数,将永远无法回头释放内存或资源。

\item
析构函数隐式地标记为noexcept,除非显式地标记为noexcept(false)或者noexcept(false)析构函数的子对象。如果从一个noexcept析构函数中抛出异常,C++运行时会调用std::terminate()来终止应用程序。

\item
析构函数可以在处理另一个异常的过程中运行。如果在栈展开过程中从析构函数中抛出异常,C++运行时会调用std::terminate()来终止应用程序。对于勇敢和好奇的人来说,C++确实提供了在析构函数中确定,是否因为正常函数退出或因delete而执行,还是因为栈展开而执行的能力。函数uncaught\_exceptions()在<exception>中声明,返回未捕获的异常数量,即已经抛出但尚未到达匹配的catch的异常数量。如果uncaught\_exceptions()的结果大于零,那么程序处于栈展开过程中。然而,正确使用这个函数是即复杂,又混乱,尽可能避免使用。注意,在C++17之前,这个函数称为uncaught\_exception()(单数形式),并且如果栈在展开中,则返回true。这个单数版本在C++17中已弃用,并已在C++20中删除。
\end{enumerate}

\begin{myWarning}{WARNING}
小心不要让任何异常从析构函数中逃逸!
\end{myWarning}
















