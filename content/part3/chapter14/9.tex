Now that you are fluent in working with exceptions, it’s time to discuss exception safety guarantees. There are several levels of guarantees you can provide for code you write so that users of your code know what can be expected when an exception is thrown. The following exception safety guarantees can be specified for a function:

\begin{itemize}
\item
Nothrow (or nofail) exception guarantee: The function never throws any exceptions.

\item
Strong exception guarantee: If an exception is thrown, all involved objects are rolled back to the state they were in before the function was called. An example of code providing this guarantee is the copy-and-swap idiom introduced in Chapter 9.

\item
Basic exception guarantee: If an exception is thrown, all involved objects remain in a valid state, and no resources are leaked. However, the objects could be in another state than they were before the function was called.

\item
No guarantee: When an exception is thrown, the application can be in any invalid state, resources might be leaked, memory might be corrupted, and so on.
\end{itemize}

\begin{myNotic}{NOTE}
If a function can throw exceptions, then it should provide, at the very least, a basic exception guarantee.
\end{myNotic}