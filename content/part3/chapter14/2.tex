
异常情况在文件输入和输出中经常出现。下面是一个函数,用于打开文件,从文件中读取整数列表,并将整数以std::vector数据结构返回。缺少错误处理需要引起注意:

\begin{cpp}
vector<int> readIntegerFile(const string& filename)
{
    ifstream inputStream { filename };
    // Read the integers one-by-one and add them to a vector.
    vector<int> integers;
    int temp;
    while (inputStream >> temp) {
        integers.push_back(temp);
    }
    return integers;
}
\end{cpp}

以下行从ifstream中不断读取值,直到到达文件末尾或出现错误:

\begin{cpp}
while (inputStream >> temp) {
\end{cpp}

如果>{}>运算符遇到错误,会设置ifstream对象的失败位。这种情况下,bool()转换运算符返回false,while循环终止。在第13章中,已经详细地介绍了流。

可能会这样使用readIntegerFile():

\begin{cpp}
const string filename { "IntegerFile.txt" };
vector<int> myInts { readIntegerFile(filename) };
println("{} ", myInts);
\end{cpp}

本节的其余部分将展示如何使用异常添加错误处理,但首先需要更深入地了解如何抛出和捕获异常。

\mySubsubsection{14.2.1.}{抛出和捕获异常}

异常处理包括在程序中提供两部分:一个try/catch结构来处理异常,以及一个抛出异常的throw语句。两者都必须以某种形式存在,才能使异常处理工作。但在许多情况下,抛出异常发生在一些库(包括C++运行时)的深处,开发者可能永远看不到,但仍需使用try/catch结构来应对。

try/catch结构如下所示:

\begin{cpp}
try {
    // ... code which may result in an exception being thrown
} catch (exception-type1 exception-name) {
    // ... code which responds to the exception of type 1
} catch (exception-type2 exception-name) {
    // ... code which responds to the exception of type 2
}
// ... remaining code
\end{cpp}

异常抛出的代码可以直接包含一个throw,也可能调用一个函数,该函数要么直接抛出异常,要么通过一些未知的调用——调用一个抛出异常的函数。

如果没有抛出异常,catch块中的代码都不会执行,紧随其后的“剩余代码”将接着try块中最后执行的语句执行。

如果抛出了异常,throw之后的代码或导致throw的调用的后续代码都不会执行;相反,控制权会立即转到catch块中,这取决于抛出的异常的类型。

如果catch块没有进行控制转移——例如,通过从函数返回、抛出一个新异常或重新抛出捕获的异常——在catch块的最后一条语句之后将执行“剩余代码”。

演示异常处理最简单的例子是避免除以零。以下示例抛出一个类型为std::invalid\_argument的异常,该异常定义在<stdexcept>中:

\begin{cpp}
double safeDivide(double num, double den)
{
    if (den == 0) { throw invalid_argument { "Divide by zero" }; }
    return num / den;
}

int main()
{
    try {
        println("{}", safeDivide(5, 2));
        println("{}", safeDivide(10, 0));
        println("{}", safeDivide(3, 3));
    } catch (const invalid_argument& e) {
        println("Caught exception: {}", e.what());
    }
}
\end{cpp}

输出如下:

\begin{shell}
2.5
Caught exception: Divide by zero
\end{shell}

throw是C++中的一个关键字,是抛出异常的唯一方式。在代码段中,抛出一个invalid\_argument的实例,这是C++标准库提供的一个标准异常。所有标准库异常形成一个层次结构,这将在本章后面讨论。层次结构中的每个类都支持一个what()成员函数,该函数返回一个描述异常的const char*字符串,这是异常的构造函数中提供的字符串。

\begin{myNotic}{NOTE}
尽管what()的返回类型是const char*,但如果使用UTF-8编码异常,异常可以支持Unicode字符串。有关Unicode字符串的详细信息,参见第21章。
\end{myNotic}

回到readIntegerFile()函数。最可能发生的问题是文件打开失败,这是一个抛出异常的情况。以下代码在文件打开失败时,抛出一个类型为std::exception的异常,该异常定义在<exception>中:

\begin{cpp}
vector<int> readIntegerFile(const string& filename)
{
    ifstream inputStream { filename };
    if (inputStream.fail()) {
        // We failed to open the file: throw an exception.
        throw exception {};
    }

    // Read the integers one-by-one and add them to a vector.
    vector<int> integers;
    int temp;
    while (inputStream >> temp) {
        integers.push_back(temp);
    }
    return integers;
}
\end{cpp}

\begin{myNotic}{NOTE}
始终在函数的代码文档中记录函数可能抛出的异常,因为函数的使用者需要知道可能抛出哪些异常,以便正确处理它们。
\end{myNotic}

如果函数无法打开文件并执行throw exception\{\};语句,则跳过函数的其余部分,并转移到最近的异常处理程序。

在代码中抛出异常的时候,需要编写了处理它们的代码。异常处理是一种“尝试”一个代码块的方式,另一个代码块可指定为对可能发生的问题做出反应。在main()函数中,catch语句通过打印错误消息,来对try块内抛出的exception的异常做出反应。如果try块在没有抛出异常的情况下完成,则跳过catch块。可以将try/catch块视为另一种if语句:如果在try块中抛出了异常,则执行catch块,否则跳过所有catch块。

\begin{cpp}
int main()
{
    const string filename { "IntegerFile.txt" };
    vector<int> myInts;
    try {
        myInts = readIntegerFile(filename);
    } catch (const exception& e) {
        println(cerr, "Unable to open file {}", filename);
        return 1;
    }
    println("{} ", myInts);
}
\end{cpp}

\begin{myNotic}{NOTE}
尽管默认情况下流不会抛出异常,但可以通过调用它们的exceptions()成员函数,告诉流在出现错误条件时抛出异常。然而,大多数编译器在流抛出的异常中,提供的信息是无用的。对于这样的编译器,直接处理流的状态可能比使用异常更好。本书中不使用流异常。
\end{myNotic}

\mySubsubsection{14.2.2.}{异常类型}

可以抛出任何类型的异常。前面的例子抛出了一个std::exception类型的对象,但异常不一定是对象,也可以像这样抛出一个int:

\begin{cpp}
vector<int> readIntegerFile(const string& filename)
{
    ifstream inputStream { filename };
    if (inputStream.fail()) {
        // We failed to open the file: throw an exception.
        throw 5;
    }
    // Omitted for brevity
}
\end{cpp}

然后需要像这样更改catch语句:

\begin{cpp}
try {
    myInts = readIntegerFile(filename);
} catch (int e) {
    println(cerr, "Unable to open file {} (Error Code {})", filename, e);
    return 1;
}
\end{cpp}

或者,可以抛出一个const char* C风格字符串。这种技术有时很有用,因为字符串可以包含有关异常的信息:

\begin{cpp}
vector<int> readIntegerFile(const string& filename)
{
    ifstream inputStream { filename };
    if (inputStream.fail()) {
        // We failed to open the file: throw an exception.
        throw "Unable to open file";
    }
    // Omitted for brevity
}
\end{cpp}

当捕获const char*异常时,可以按如下方式打印结果:

\begin{cpp}
try {
    myInts = readIntegerFile(filename);
} catch (const char* e) {
    println(cerr, "{}", e);
    return 1;
}
\end{cpp}

尽管有前面的例子,请记住以下几点:

\begin{myNotic}{NOTE}
出于两个原因,通常应该抛出对象,而不是其他数据类型作为异常:

\begin{itemize}
\item
对象通过类名传递信息。

\item
对象可以存储各种信息,包括描述异常的字符串。
\end{itemize}
\end{myNotic}

C++标准库定义了许多预定义的异常类,构成了一个类层次结构,本章将在后面介绍。此外,可以编写自己的异常类,并将它们放入标准结构中,这也会在本章后面介绍。

\mySubsubsection{14.2.3.}{捕获异常对象——const的引用}

在前面readIntegerFile()抛出exception类型对象的例子中,catch处理程序为:

\begin{cpp}
} catch (const exception& e) {
\end{cpp}

然而,并没有要求以const引用的方式捕获对象,可以这样按值捕获对象:

\begin{cpp}
} catch (exception e) {
\end{cpp}

或者,可以将对象捕获为非const引用:

\begin{cpp}
} catch (exception& e) {
\end{cpp}

另外,前面const char*的例子中看到的,只要抛出了指向异常的指针,就可以捕获异常指针。

尽管如此,我还是建议遵循以下建议:

\begin{myNotic}{NOTE}
总是以异常的形式抛出对象,并且总是以const引用的方式捕获异常对象!可能会在按值捕获异常对象时发生对象切片(参见第10章)。捕获为非const引用也不推荐,通常不会修改捕获的异常。
\end{myNotic}

\mySubsubsection{14.2.4.}{抛出和捕获多个异常}

无法打开文件不是readIntegerFile()可能遇到的唯一问题。如果文件格式不正确,从文件中读取数据也可能导致错误。这里有一个readIntegerFile()的实现,如果不能打开文件或正确读取数据,就会抛出异常。这次,使用从exception派生的runtime\_error,并在其构造函数中允许指定一个描述性字符串。runtime\_error异常类定义在<stdexcept>中。

\begin{cpp}
vector<int> readIntegerFile(const string& filename)
{
    ifstream inputStream { filename };
    if (inputStream.fail()) {
        // We failed to open the file: throw an exception.
        throw runtime_error { "Unable to open the file." };
    }

    // Read the integers one-by-one and add them to a vector.
    vector<int> integers;
    int temp;
    while (inputStream >> temp) {
        integers.push_back(temp);
    }

    if (!inputStream.eof()) {
        // We did not reach the end-of-file.
        // This means that some error occurred while reading the file.
        // Throw an exception.
        throw runtime_error { "Error reading the file." };
    }

    return integers;
}
\end{cpp}

main()中的代码不需要做太多更改,因为它已经捕获了exception类型的异常,runtime\_error就从其派生。然而,现在可能会在两种不同的情况下抛出异常,所以使用what()成员函数来获取捕获异常的描述:

\begin{cpp}
try {
    myInts = readIntegerFile(filename);
} catch (const exception& e) {
    println(cerr, "{}", e.what());
    return 1;
}
\end{cpp}

或者,可以从readIntegerFile()抛出两种不同类型的异常。这里有一个readIntegerFile()的实现,如果无法打开文件,会抛出invalid\_argument类的异常对象,如果无法读取整数,会抛出runtime\_error类的对象。invalid\_argument和runtime\_error都是在<stdexcept>中定义的C++标准类。

\begin{cpp}
vector<int> readIntegerFile(const string& filename)
{
    ifstream inputStream { filename };
    if (inputStream.fail()) {
        // We failed to open the file: throw an exception.
        throw invalid_argument { "Unable to open the file." };
    }

    // Read the integers one-by-one and add them to a vector.
    vector<int> integers;
    int temp;
    while (inputStream >> temp) {
        integers.push_back(temp);
    }

    if (!inputStream.eof()) {
        // We did not reach the end-of-file.
        // This means that some error occurred while reading the file.
        // Throw an exception.
        throw runtime_error { "Error reading the file." };
    }
    return integers;
}
\end{cpp}

invalid\_argument和runtime\_error没有公共的默认构造函数,只有字符串构造函数,所以必须传递一个字符串作为参数。

现在,main()可以使用两个catch语句捕获invalid\_argument和runtime\_error异常:

\begin{cpp}
try {
    myInts = readIntegerFile(filename);
} catch (const invalid_argument& e) {
    println(cerr, "{}", e.what());
    return 1;
} catch (const runtime_error& e) {
    println(cerr, "{}", e.what());
    return 2;
}
\end{cpp}

如果在try块内抛出异常,编译器会将异常类型与正确的catch处理程序匹配。所以,如果readIntegerFile()无法打开文件并抛出invalid\_argument对象,第一个catch语句将捕获异常。如果readIntegerFile()无法正确读取文件并抛出runtime\_error,那么第二个catch语句会捕获异常。

\mySamllsection{异常匹配和const}

想要捕获的异常类型中指定的const性质对于匹配目的来说没有区别,所以这行代码匹配的异常类型为runtime\_error:

\begin{cpp}
} catch (const runtime_error& e) {
\end{cpp}

以下行也匹配类型为runtime\_error的异常:

\begin{cpp}
} catch (runtime_error& e) {
\end{cpp}

尽管如此,还是建议始终将异常对象作为const引用来捕获。

\mySamllsection{匹配任何异常}

可以编写一个捕获所有块的代码,使用特殊语法匹配任何可能的异常:

\begin{cpp}
try {
    myInts = readIntegerFile(filename);
} catch (...) {
    println(cerr, "Error reading or opening file {}", filename);
    return 1;
}
\end{cpp}

这三个点不是打字错误,它们是一个通配符,匹配任何异常类型。使用捕获所有块的问题是,得不到捕获异常的细节信息。当调用文档记录不良的代码时,这个技术可能很有用,以确保你捕获所有可能的异常。但即便如此,对一个未知异常又能做些什么来恢复呢?

捕获所有块的一个用处是记录抛出了异常,然后重新抛出异常。以下示例显示了如何显式处理invalid\_argument和runtime\_error异常的捕获处理程序,以及如何包含一个捕获所有其他异常的处理程序。任何catch块中,可以使用不带参数的throw关键字重新抛出当前捕获的异常。关于重新抛出异常的内容还有很多,但这需要等到本章后面再介绍。

\begin{cpp}
try {
    // Code that can throw exceptions.
} catch (const invalid_argument& e) {
    // Handle invalid_argument exception.
} catch (const runtime_error& e) {
    // Handle runtime_error exception.
} catch (...) {
    // Handle all other exceptions.
    // Log that an exception occurred...
    throw; // Rethrow the caught exception.
}
\end{cpp}

对抛出的异常集合有完整信息的情况下,不推荐使用捕获所有块,因为它以相同的方式处理每个异常类型。最好显式匹配异常类型并采取适当的、有针对性的行动。

\mySubsubsection{14.2.5.}{未捕获的异常}

如果程序抛出了一个没有被捕获的异常,程序将会终止。可以在main()函数的地方有一个try/catch结构,捕获所有未处理的异常:

\begin{cpp}
try {
    main(argc, argv);
} catch (...) {
    // Issue error message and terminate program.
}
// Normal termination code.
\end{cpp}

然而,这种行为通常不是想要的。异常的目的是让程序有机会处理和纠正不良的或未预期的情况。

\begin{myWarning}{WARNING}
应当尽可能捕获并处理程序中抛出的所有异常。
\end{myWarning}

如果程序遇到未捕获的异常,也可以改变程序的行为。当程序遇到未捕获的异常时,调用内置的terminate()函数,该函数从<cstdlib>调用abort()来杀死程序。可以通过调用set\_terminate()并传递一个指向不接受参数,且不返回值的函数的指针来设置自己的terminate\_handler。terminate()、set\_terminate()和terminate\_handler都在<exception>中声明。以下伪代码展示了它的工作原理:

\begin{cpp}
try {
    main(argc, argv);
} catch (...) {
    if (terminate_handler != nullptr) {
        terminate_handler();
    } else {
        terminate();
    }
}
// Normal termination code.
\end{cpp}

在对这个特性过于兴奋之前,应该知道回调函数必须使用abort()或\_Exit()来终止,不能忽略错误。abort()和\_Exit()都在<cstdlib>中定义,它们在清理资源之前终止应用程序。例如,对象析构函数将不会调用。\_Exit()函数接受一个整数参数,该参数返回给操作系统,可以用来确定进程是如何退出的。值为0或EXIT\_SUCCESS表示程序没有错误地退出;否则,程序异常终止。abort()函数不接受任何参数。此外,还有一个exit()函数,也接受一个整数参数,该参数返回给操作系统,并通过调用析构函数来清理资源,但不建议在terminate\_handler中调用exit()。

一个terminate\_handler可以用来在退出前打印一个有用的错误消息。这里有一个main()函数的例子,不捕获readIntegerFile()抛出的异常。相反,将terminate\_handler设置为一个自定义的回调。这个回调打印一个错误消息,并通过调用\_Exit()来终止进程。注意,这里使用了第1章介绍的[[noreturn]]属性。

\begin{cpp}
[[noreturn]] void myTerminate()
{
    println(cerr, "Uncaught exception!");
    _Exit(1);
}

int main()
{
    set_terminate(myTerminate);

    const string filename { "IntegerFile.txt" };
    vector<int> myInts { readIntegerFile(filename) };
    println("{} ", myInts);
}
\end{cpp}

虽然这个例子中没有显示,但set\_terminate()在设置新terminate\_handler时,返回旧的terminate\_handler。terminate\_handler是程序范围的,所以当完成需要新terminate\_handler的代码时,恢复旧的terminate\_handler是一种好的风格。这个例子中,整个程序需要新的terminate\_handler,所以没有必要恢复。

虽然了解set\_terminate()很重要,但它并不是一个有效的异常处理方法。建议捕获并单独处理每个异常,以提供更精确的错误处理。

\begin{myNotic}{NOTE}
专业编写的软件中,通常会设置一个terminate\_handler来在终止进程之前创建一个崩溃转储。崩溃转储通常包含诸如未捕获异常抛出时,调用堆栈和局部变量的信息。然后可以将这样的崩溃转储加载到调试器中,从而可以了解未捕获的异常是什么,以及是什么原因导致。然而,编写崩溃转储需要依赖于平台,因此本书不再进一步介绍。
\end{myNotic}

\mySubsubsection{14.2.6.}{noexcept指定符}

默认情况下,函数可以抛出任何异常。然而,可以使用noexcept指定符(一个C++关键字)来标记一个函数,以表明它不会抛出异常。例如,下面的函数标记为noexcept,因此它不允许抛出任何异常:

\begin{cpp}
void printValues(const vector<int>& values) noexcept;
\end{cpp}

\begin{myNotic}{NOTE}
标记为noexcept的函数必须不抛出异常。
\end{myNotic}

当一个标记为noexcept的函数抛出异常时,C++会调用terminate()来终止应用程序。

当在一个派生类中覆盖一个虚成员函数时,允许将覆盖的成员函数标记为noexcept,即使基类中的版本不是noexcept。相反的情况则不允许。

\mySubsubsection{14.2.7.}{noexcept(expression)指定符}

noexcept(expression)指定符仅在给定的表达式返回true时,将函数标记为noexcept。换句话说,noexcept等于noexcept(true),而noexcept(false)是noexcept(true)的反义词;也就是说,标记为noexcept(false)的成员函数可以抛出异常,这是默认行为

\mySubsubsection{14.2.8.}{noexcept(expression)运算符}

noexcept(expression)运算符在给定的表达式是noexcept时返回true。这种计算发生在编译时。

这里有一个例子:

\begin{cpp}
void f1() noexcept {}
void f2() noexcept(false) {}
void f3() noexcept(noexcept(f1())) {}
void f4() noexcept(noexcept(f2())) {}

int main()
{
    println("{} {} {} {}", noexcept(f1()),
                           noexcept(f2()),
                           noexcept(f3()),
                           noexcept(f4()));
}
\end{cpp}

这段代码的输出是true false true false:

\begin{itemize}
\item
noexcept(f1())是true,f1()明确标记了noexcept指定符。

\item
noexcept(f2())是false,f2()使用noexcept(expression)指定符明确标记。

\item
noexcept(f3())是true,f3()标记为noexcept,但只有当f1()是noexcept时才是。

\item
noexcept(f4())是false,f4()标记为noexcept,但只有当f2()不是noexcept时才是。
\end{itemize}

\mySubsubsection{14.2.9.}{抛出列表}

旧版本的C++允许指定一个函数打算抛出的异常。这种指定称为抛出列表或指定异常。

\begin{myNotic}{NOTE}
C++11已经弃用了异常指定,C++17已经移除了对异常指定的支持,除了noexcept和throw()。后者等同于noexcept。从C++20开始,对throw()的支持也移除了。
\end{myNotic}

因为C++17已经正式移除了对异常指定的支持,所以本书不再进一步介绍它们。

