By solving the following exercises, you can practice the material discussed in this chapter. Solutions to all exercises are available with the code download on the book’s website at \url{www.wiley.com/go/proc++6e}. However, if you are stuck on an exercise, first reread parts of this chapter to try to find an answer yourself before looking at the solution from the website.

\begin{itemize}
\item
Exercise 14-1: Without compiling and executing, find and correct the errors in the following code:

\begin{cpp}
// Throws a logic_error exception if the number of elements
// in the given dataset is not even.
void verifyDataSize(const vector<int>& data)
{
    if (data.size() % 2 != 0)
        throw logic_error { "Number of data points must be even." };
}

// Throws a logic_error exception if the number of elements
// in the given dataset is not even.
// Throws a domain_error exception if one of the datapoints is negative.
void processData(const vector<int>& data)
{
    // Verify the size of the given dataset.
    try {
        verifyDataSize(data);
    } catch (const logic_error& caughtException) {
        // Write message on standard output.
        println(cerr, "Invalid number of data points in dataset. Aborting.");
        // And rethrow the exception.
        throw caughtException;
    }
    // Verify for negative datapoints.
    for (auto& value : data) {
        if (value < 0)
            throw domain_error { "Negative datapoints not allowed." };
    }
    // Process data ...
}

int main()
{
    try {
        vector data { 1, 2, 3, -5, 6, 9 };
        processData(data);
    } catch (const logic_error& caughtException) {
        println(cerr, "logic_error: {}", caughtException.what());
    } catch (const domain_error& caughtException) {
        println(cerr, "domain_error: {}", caughtException.what());
    }
}
\end{cpp}

\item
Exercise 14-2: Take the code from the bidirectional I/O example from Chapter 13. You can find this in the Ch13\verb|\|22\_Bidirectional folder in the downloadable source code archive. The example implements a changeNumberForID() function. Retrofit the code to use exceptions on all places you deem appropriate. Once your code is using exceptions, do you see a possible change you can make to the changeNumberForID() function header?

\item
Exercise 14-3: Add proper error handling using exceptions to your person database solution of Exercise 13-3.

\item
Exercise 14-4: Take a look at the code in Chapter 9 for the Spreadsheet example that includes support for move semantics using swap(). You can find the entire code in the downloadable source code archive in the folder Ch09\verb|\|07\_SpreadsheetMoveSemantics– WithSwap. Add proper error handling to the code, including handling of memory allocation failures. Add a maximum width and height to the class and include the proper verification checks. Write your own exception class, InvalidCoordinate, which stores both an invalid coordinate and the range of allowed coordinates. Use it in the verifyCoordinate() member function. Write a couple of tests in main() to test various error conditions.
\end{itemize}














