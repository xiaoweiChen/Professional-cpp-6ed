
对象生命周期涉及三个行为:创建、销毁和赋值。了解对象如何以及何时创建、销毁和赋值,以及如何自定义这些行为非常重要。

\mySubsubsection{8.3.1.}{创建}

对象在声明它们(如果在栈上)时创建,或者当使用智能指针、new或new[]显式为它们分配空间时创建。当一个对象创建时,其所有的嵌入对象也会创建。以下是一个例子:

\begin{cpp}
import std;

class MyClass
{
    private:
    std::string m_name;
};

int main()
{
    MyClass obj;
}
\end{cpp}

string对象在main()函数中创建MyClass对象时创建,并在其包含对象销毁时销毁。

声明变量时给出初始值通常很有帮助,如下所示:

\begin{cpp}
int x { 0 };
\end{cpp}

同样,也可以为对象提供初始值。可以通过声明并编写一个特殊的成员函数来实现这个功能,这个函数称为构造函数,可以在其中为对象执行初始化工作。每当创建一个对象时,都会执行造函数。

\begin{myNotic}{NOTE}
C++开发者有时将构造函数称为“tor”(发音为“seetor”)。
\end{myNotic}

\mySamllsection{编写构造函数}

从语法上讲,构造函数通过一个与类名相同的成员函数名来指定。 构造函数没有返回类型,可以有参数,也可以没有参数。一个可以不带参数调用的构造函数称为默认构造函数。这可能是一个没有任何参数的构造函数,或者一个所有参数都有默认值的构造函数。某些情况下,可能需要提供一个默认构造函数,如果没有提供,将收获编译错误。默认构造函数会在本章后面进行讨论。

以下是对SpreadsheetCell类添加构造函数的首次尝试:

\begin{cpp}
export class SpreadsheetCell
{
    public:
        SpreadsheetCell(double initialValue);
        // Remainder of the class definition omitted for brevity
};
\end{cpp}

必须为普通成员函数提供实现一样,也必须为构造函数提供实现:

\begin{cpp}
SpreadsheetCell::SpreadsheetCell(double initialValue)
{
    setValue(initialValue);
}
\end{cpp}

SpreadsheetCell构造函数是SpreadsheetCell类的一个成员,所以C++要求在构造函数名前使用正常的SpreadsheetCell::作用域解析。构造函数名本身也是SpreadsheetCell,所以代码最终呈现出奇怪的SpreadsheetCell::SpreadsheetCell,而实现只调用了setValue()。

\mySamllsection{使用构造函数}

使用构造函数可以创建对象并为其初始化值,可以使用构造函数来进行栈上和堆上的分配。

\mySamllsection{栈上对象的构造函数}

当在栈上分配一个SpreadsheetCell对象时,可以像这样使用构造函数:

\begin{cpp}
SpreadsheetCell myCell(5), anotherCell(4);
println("cell 1: {}", myCell.getValue());
println("cell 2: {}", anotherCell.getValue());
\end{cpp}

或者,可以使用统一初始化语法:

\begin{cpp}
SpreadsheetCell myCell { 5 }, anotherCell { 4 };
\end{cpp}

注意,不需要显式调用SpreadsheetCell构造函数。例如,不要这样做:

\begin{cpp}
SpreadsheetCell myCell.SpreadsheetCell(5); // WILL NOT COMPILE!
\end{cpp}

同样,也不能调用构造函数。这也不正确:

\begin{cpp}
SpreadsheetCell myCell;
myCell.SpreadsheetCell(5); // WILL NOT COMPILE!
\end{cpp}

\mySamllsection{堆区对象的构造函数}

当动态分配一个SpreadsheetCell对象时,可以像这样使用构造函数:

\begin{cpp}
auto smartCellp { make_unique<SpreadsheetCell>(4) };
// ... do something with the cell, no need to delete the smart pointer

// Or with raw pointers, without smart pointers (not recommended)
SpreadsheetCell* myCellp { new SpreadsheetCell { 5 } };
// Or
// SpreadsheetCell* myCellp{ new SpreadsheetCell(5) };
SpreadsheetCell* anotherCellp { nullptr };
anotherCellp = new SpreadsheetCell { 4 };
// ... do something with the cells
delete myCellp; myCellp = nullptr;
delete anotherCellp; anotherCellp = nullptr;
\end{cpp}

注意,可以声明一个SpreadsheetCell对象的指针而不立即调用构造函数,这与栈上的对象不同,栈上的对象在声明时调用构造函数。

请记住始终初始化指针,要么是正确的指针,要么是nullptr。

\mySamllsection{提供多个构造函数}

可以在一个类中提供多个构造函数。所有构造函数都有相同的名称(类的名称),但不同的构造函数必须接受不同数量或不同类型的参数。在C++中,如果有多个具有相同名称的函数,编译器会选择参数类型与调用站点匹配的一个。这称为重载,在第9章中有详细讨论。

在SpreadsheetCell类中,提供两个构造函数是有帮助的:一个接受初始double值,另一个接受初始string值。以下是新的类定义:

\begin{cpp}
export class SpreadsheetCell
{
    public:
        SpreadsheetCell(double initialValue);
        SpreadsheetCell(std::string_view initialValue);
        // Remainder of the class definition omitted for brevity
};
\end{cpp}

这是第二个构造函数的实现:

\begin{cpp}
SpreadsheetCell::SpreadsheetCell(string_view initialValue)
{
    setString(initialValue);
}
\end{cpp}

以下是使用两个不同构造函数的代码:

\begin{cpp}
SpreadsheetCell aThirdCell { "test" }; // Uses string-arg ctor
SpreadsheetCell aFourthCell { 4.4 }; // Uses double-arg ctor
auto aFifthCellp { make_unique<SpreadsheetCell>("5.5") }; // string-arg ctor
println("aThirdCell: {}", aThirdCell.getValue());
println("aFourthCell: {}", aFourthCell.getValue());
println("aFifthCellp: {}", aFifthCellp->getValue());
\end{cpp}

当有多个构造函数时,可能会试图在构造函数初始化器中实现一个构造函数在另一个构造函数的基础上。例如,想从string构造函数调用double构造函数,如下所示:

\begin{cpp}
SpreadsheetCell::SpreadsheetCell(string_view initialValue)
{
    SpreadsheetCell(stringToDouble(initialValue));
}
\end{cpp}

这看起来很有道理,可以在其他成员函数中调用正常的类成员函数。代码将编译、链接并运行,但不会做期望的事情!显式调用SpreadsheetCell构造函数实际上创建了一个新的临时无名SpreadsheetCell对象,并没有调用应该初始化对象的构造函数。

然而,C++确实支持委托构造函数。这些允许在构造函数初始化器中调用同一类的其他构造函数,本章稍后会介绍构造函数初始化器。

\mySamllsection{默认构造函数}

默认构造函数是一个不接受参数的构造函数,也称为无参数构造函数。

\mySamllsection{何时需要默认构造函数}

考虑对象数组。创建对象数组执行两个任务:为所有对象分配连续的内存空间,并在每个对象上调用默认构造函数。C++没有提供语法来直接告诉数组创建代码调用不同的构造函数。例如,如果没有为SpreadsheetCell类定义默认构造函数,以下代码将无法编译:

\begin{cpp}
SpreadsheetCell cells[3]; // FAILS compilation without default constructor
SpreadsheetCell* myCellp { new SpreadsheetCell[10] }; // Also FAILS
\end{cpp}

可以通过使用这些初始化器来规避这个限制:

\begin{cpp}
SpreadsheetCell cells[3] { SpreadsheetCell { 0 }, SpreadsheetCell { 23 },
    SpreadsheetCell { 41 } };
\end{cpp}

然而,如果打算创建SpreadsheetCell类的对象数组,通常需要确保类有一个默认构造函数。如果没有定义自己的构造函数,编译器会自动创建一个默认构造函数。这个编译器生成的构造函数在后面的章节中讨论。

\mySamllsection{如何编写默认构造函数}

这里是带有默认构造函数的SpreadsheetCell类定义的一部分:

\begin{cpp}
export class SpreadsheetCell
{
    public:
        SpreadsheetCell();
        // Remainder of the class definition omitted for brevity
};
\end{cpp}

这是默认构造函数的第一次尝试实现:

\begin{cpp}
SpreadsheetCell::SpreadsheetCell()
{
    m_value = 0;
}
\end{cpp}

如果在类定义中使用了m\_value的内联成员初始化器,则这个默认构造函数中的单个语句可以省略。

\begin{cpp}
SpreadsheetCell::SpreadsheetCell()
{}
\end{cpp}

栈上使用默认构造函数的方式如下所示:

\begin{cpp}
SpreadsheetCell myCell;
myCell.setValue(6);
println("cell 1: {}", myCell.getValue());
\end{cpp}

上述代码创建了一个新的SpreadsheetCell对象myCell,设置其值,并打印出其值。与栈上对象的其它构造函数不同,不需要用函数调用语法调用默认构造函数。根据其他构造函数的语法,可能会像这样调用默认构造函数:

\begin{cpp}
SpreadsheetCell myCell(); // WRONG, but will compile.
myCell.setValue(6); // However, this line will not compile.
println("cell 1: {}", myCell.getValue());
\end{cpp}

不幸的是,尝试调用默认构造函数的行可以编译,但后面的代码无法编译。这个问题通常称为“C++最令人烦恼的解析”,编译器认为第一行实际上是具有零个参数并返回SpreadsheetCell对象的函数声明。当它到达第二行时,认为是在尝试使用一个函数名作为一个对象!

当然,可以使用统一初始化语法,而不是函数调用风格的括号:

\begin{cpp}
SpreadsheetCell myCell { }; // Calls the default constructor.
\end{cpp}

\begin{myWarning}{WARNING}
使用默认构造函数在堆栈上创建对象时,要么使用花括号进行统一初始化语法,要么省略任何括号。
\end{myWarning}

对于堆区对象分配,可以使用以下默认构造函数:

\begin{cpp}
auto smartCellp { make_unique<SpreadsheetCell>() };
// Or with a raw pointer (not recommended)
SpreadsheetCell* myCellp { new SpreadsheetCell { } };
// Or
// SpreadsheetCell* myCellp { new SpreadsheetCell };
// Or
// SpreadsheetCell* myCellp { new SpreadsheetCell() };
// ... use myCellp
delete myCellp; myCellp = nullptr;
\end{cpp}

\mySamllsection{编译器生成的默认构造函数}

本章中第一个SpreadsheetCell类定义如下:

\begin{cpp}
export class SpreadsheetCell
{
    public:
        void setValue(double value);
        double getValue() const;
    private:
        double m_value;
};
\end{cpp}

这个定义没有声明默认构造函数,但接下来的代码仍然可以正常工作:

\begin{cpp}
SpreadsheetCell myCell;
myCell.setValue(6);
\end{cpp}

以下定义与前面的定义相同,只是添加了一个显式构造函数,接受一个double。仍然没有显式声明一个默认构造函数:

\begin{cpp}
export class SpreadsheetCell
{
    public:
        SpreadsheetCell(double initialValue); // No default constructor
        // Remainder of the class definition omitted for brevity
};
\end{cpp}

使用这个定义,以下代码不再编译:

\begin{cpp}
SpreadsheetCell myCell;
myCell.setValue(6);
\end{cpp}

这是怎么回事?原因是没有声明构造函数时,编译器生成了一个不带参数的构造函数。这个编译器生成的默认构造函数,会调用该类所有对象成员的默认构造函数,但不初始化语言原生类型如int和double。尽管如此,它仍然可以创建该类的对象。如果自己声明了构造函数,编译器就不再为生成默认构造函数了。

\begin{myNotic}{NOTE}
默认构造函数与零参数构造函数相同。默认构造函数不仅指如果你没有自己声明构造函数,编译器会为你自动生成的构造函数。它还指如果不需要任何参数,则默认的构造函数。
\end{myNotic}

\mySamllsection{显式声明默认构造函数}

C++11之前,如果类需要几个显式接受参数的构造函数,但也需要一个默认构造函数,该构造函数什么也不做,仍然必须显式编写自己的空默认构造函数,如之前所示。

为了避免手动编写空默认构造函数,C++支持显式默认的默认构造函数的概念。可以像下面这样编写类定义,而不需要为默认构造函数提供空实现:

\begin{cpp}
export class SpreadsheetCell
{
    public:
        SpreadsheetCell() = default;
        SpreadsheetCell(double initialValue);
        SpreadsheetCell(std::string_view initialValue);
        // Remainder of the class definition omitted for brevity
};
\end{cpp}

SpreadsheetCell定义了两个自定义构造函数。然而,编译器仍然生成一个标准的默认构造函数,因为其中一个使用了default关键字显式地进行了默认设置。可以将=default直接放在类定义中,或者放在实现文件中。

\mySamllsection{显式删除默认构造函数}

与显式默认的默认构造函数相反,也可以显式删除默认构造函数,在某些情况下必要。例如,可以定义一个只有静态成员函数的类(请参见第9章),不想编写构造函数,也不希望编译器生成默认构造函数。这种情况下,需要显式删除默认构造函数。

\begin{cpp}
export class MyClass
{
    public:
        MyClass() = delete;
};
\end{cpp}

\begin{myNotic}{NOTE}
如果类的数据成员具有删除的默认构造函数,则该类的默认构造函数也将自动删除。
\end{myNotic}

\mySamllsection{构造函数初始化器(ctor-initializer)}

目前为止,本章在构造函数体内初始化了数据成员,例如:

\begin{cpp}
SpreadsheetCell::SpreadsheetCell(double initialValue)
{
    setValue(initialValue);
}
\end{cpp}

C++提供了一种替代方法来在构造函数中初始化数据成员,称为构造函数初始化器,也称为ctor-initializer或成员初始化列表。以下是使用ctor-initializer语法的相同SpreadsheetCell构造函数的重写:

\begin{cpp}
SpreadsheetCell::SpreadsheetCell(double initialValue)
: m_value { initialValue }
{}
\end{cpp}

ctor-initializer在构造函数参数列表和构造函数体和括号之间具有语法上的位置。列表以冒号开始,由逗号分隔。列表中的每个元素都是使用函数表示法或统一初始化语法初始化数据成员,对基类构造函数的调用(请参见第10章),或者是对委托构造函数的调用,这在本书本章稍后讨论。

使用ctor-initializer初始化数据成员,与在构造函数体内初始化数据成员的行为不同。当C++创建一个对象时,必须在调用构造函数之前创建该对象的所有数据成员。创建这些数据成员的过程中,必须调用自身的构造函数。在构造函数体内为对象分配值时,实际上并没有构造该对象,只是在修改它的值。ctor-initializer允许在创建数据成员时提供初始值,这比稍后分配值更有效率。

如果类有一个默认构造函数的数据成员,那么不需要在ctor-initializer中显式初始化那个对象。例如,如果数据成员是一个std::string,默认构造函数会将字符串初始化为空字符串,所以ctor-initializer中初始化为""是多余的。

另一方面,如果类有一个没有默认构造函数的数据成员,必须使用ctor-initializer来正确构造那个对象。例如,考虑以下SpreadsheetCell类:

\begin{cpp}
export class SpreadsheetCell
{
    public:
        SpreadsheetCell(double d);
};
\end{cpp}

这个类只有一个接受double的构造函数,没有默认构造函数。可以将这个类作为另一个类的数据成员使用,如下所示:

\begin{cpp}
class SomeClass
{
    public:
        SomeClass();
    private:
        SpreadsheetCell m_cell;
};
\end{cpp}

SomeClass构造函数可以实现如下:

\begin{cpp}
SomeClass::SomeClass() { }
\end{cpp}

然而,使用这个实现,代码无法编译。因为编译器不知道如何初始化SomeClass的m\_cell数据成员,所以它没有默认构造函数。

必须使用ctor-initializer在SomeClass的构造函数中初始化m\_cell数据成员:

\begin{cpp}
SomeClass::SomeClass() : m_cell { 1.0 } { }
\end{cpp}

\begin{myNotic}{NOTE}
ctor-initializers允许在创建数据成员时初始化数据成员。
\end{myNotic}

一些开发者更喜欢在构造函数体内分配初始值,尽管这可能效率较低。然而,某些数据类型必须在ctor-initializer中或在类定义中初始化。以下表格做了个总结:

% Please add the following required packages to your document preamble:
% \usepackage{longtable}
% Note: It may be necessary to compile the document several times to get a multi-page table to line up properly
\begin{longtable}{|l|l|}
\hline
\textbf{数据类型} &
\textbf{解释} \\ \hline
\endfirsthead
%
\endhead
%
const data members &
\begin{tabular}[c]{@{}l@{}}You cannot legally assign a value to a const variable after it is\\ created. Any value must be supplied at the time of creation.\end{tabular} \\ \hline
Reference data members &
\begin{tabular}[c]{@{}l@{}}References cannot exist without referring to something, and once\\ created, a reference cannot be changed to refer to something\\ else.\end{tabular} \\ \hline
\begin{tabular}[c]{@{}l@{}}Object data members for\\ which there is no default\\ constructor\end{tabular} &
\begin{tabular}[c]{@{}l@{}}C++ attempts to initialize member objects using a default\\ constructor. If no default constructor exists, it cannot initialize the\\ object, and you must tell it explicitly which constructor to call.\end{tabular} \\ \hline
\begin{tabular}[c]{@{}l@{}}Base classes without default\\ constructors\end{tabular} &
These are covered in Chapter 10. \\ \hline
\end{longtable}

There is one important caveat with ctor-initializers: they initialize data members in the order that they appear in the class definition, not their order in the ctor-initializer! Take the following definition for a class called Foo. Its constructor simply stores a double value and prints out the value to the console.

\begin{cpp}
class Foo
{
    public:
        Foo(double value);
    private:
        double m_value { 0 };
};

Foo::Foo(double value) : m_value { value }
{
    println("Foo::m_value = {}", m_value);
}
\end{cpp}

Suppose you have another class, MyClass, that contains a Foo object as one of its data members.

\begin{cpp}
class MyClass
{
    public:
        MyClass(double value);
    private:
        double m_value { 0 };
        Foo m_foo;
};
\end{cpp}

Its constructor could be implemented as follows:

\begin{cpp}
MyClass::MyClass(double value) : m_value { value }, m_foo { m_value }
{
    println("MyClass::m_value = {}", m_value);
}
\end{cpp}

The ctor-initializer first stores the given value in m\_value and then calls the Foo constructor with m\_value as argument. You can create an instance of MyClass as follows:

\begin{cpp}
MyClass instance { 1.2 };
\end{cpp}

Here is the output of the program:

\begin{cpp}
Foo::m_value = 1.2
MyClass::m_value = 1.2
\end{cpp}

So, everything looks fine. Now make one tiny change to the MyClass definition; just reverse the order of the m\_value and m\_foo data members. Nothing else is changed.

\begin{cpp}
class MyClass
{
    public:
        MyClass(double value);
    private:
        Foo m_foo;
        double m_value { 0 };
};
\end{cpp}

The output of the program now depends on your system. It could, for example, be as follows:

\begin{shell}
Foo::m_value = -9.255963134931783e+61
MyClass::m_value = 1.2
\end{shell}

This is far from what you expected. You might assume, based on your ctor-initializer, that m\_value is initialized before using m\_value in the call to the Foo constructor. But C++ doesn’t work that way. The data members are initialized in the order they appear in the definition of the class, not the order in the ctor-initializer! So, in this case, the Foo constructor is called first with an uninitialized m\_value.

Note that some compilers issue a warning when the order in the ctor-initializer does not match the order in the class definition.

For this example, there is an easy fix. Don’t pass m\_value to the Foo constructor, but simply pass the value parameter:

\begin{cpp}
MyClass::MyClass(double value) : m_value { value }, m_foo { value }
{
    println("MyClass::m_value = {}", m_value);
}
\end{cpp}

\begin{myWarning}{WARNING}
Ctor-initializers initialize data members in their declared order in the class definition, not their order in the ctor-initializer list.
\end{myWarning}

\mySamllsection{Copy Constructors}

There is a special constructor in C++ called a copy constructor that allows you to create an object that is an exact copy of another object. Here is the declaration for a copy constructor in the SpreadsheetCell class:

\begin{cpp}
export class SpreadsheetCell
{
    public:
        SpreadsheetCell(const SpreadsheetCell& src);
        // Remainder of the class definition omitted for brevity
};
\end{cpp}

The copy constructor takes a reference-to-const to the source object. Like other constructors, it does not return a value. The copy constructor should copy all the data members from the source object. Technically, of course, you can do whatever you want in a copy constructor, but it’s generally a good idea to follow expected behavior and initialize the new object to be a copy of the old one. Here is an example implementation of the SpreadsheetCell copy constructor. Note the use of the ctorinitializer.

\begin{cpp}
SpreadsheetCell::SpreadsheetCell(const SpreadsheetCell& src)
: m_value { src.m_value }
{}
\end{cpp}

If you don’t write a copy constructor yourself, C++ generates one for you that initializes each data member in the new object from its equivalent data member in the source object. For object data members, this initialization means that their copy constructors are called. Given a set of data members, called m1, m2, . . . mn, this compiler-generated copy constructor can be expressed as follows:

\begin{cpp}
classname::classname(const classname& src)
: m1 { src.m1 }, m2 { src.m2 }, ... mn { src.mn } { }
\end{cpp}

Therefore, in most circumstances, there is no need for you to specify a copy constructor!

\begin{myNotic}{NOTE}
The SpreadsheetCell copy constructor is shown only for demonstration purposes. In fact, in this case, the copy constructor can be omitted because the default compiler-generated one is good enough. However, under certain conditions, this compiler-generated copy constructor is not sufficient. These conditions are covered in Chapter 9.
\end{myNotic}

\mySamllsection{When the Copy Constructor Is Called}

The default semantics for passing arguments to functions in C++ is pass-by-value. That means that the function receives a copy of the value or object. Thus, whenever you pass an object to a function, the compiler calls the copy constructor of the new object to initialize it. For example, suppose you have the following printString() function accepting an std::string parameter by value:

\begin{cpp}
void printString(string value)
{
    println("{}", value);
}
\end{cpp}

Recall that std::string is actually a class, not a built-in type. When your code makes a call to printString() passing a string argument, the string parameter value is initialized with a call to its copy constructor. The argument to the copy constructor is the string you passed to printString(). In the following example, the string copy constructor is executed for the value object in printString() with name as its argument:

\begin{cpp}
string name { "heading one" };
printString(name); // Copies name
\end{cpp}

When the printString() member function finishes, value is destroyed. Because it was only a copy of name, name remains intact. Of course, you can avoid the overhead of copy constructors by passing parameters as references-to-const, discussed in an upcoming section.

When returning objects by value from a function, the copy constructor might also get called. This is discussed in the section “Objects as Return Values” later in this chapter.

\mySamllsection{Calling the Copy Constructor Explicitly}

You can use the copy constructor explicitly as well. It is often useful to be able to construct one object as an exact copy of another. For example, you might want to create a copy of a SpreadsheetCell object like this:

\begin{cpp}
SpreadsheetCell myCell1 { 4 };
SpreadsheetCell myCell2 { myCell1 }; // myCell2 has the same values as myCell1
\end{cpp}

\mySamllsection{Passing Objects by Reference}

To avoid copying objects when you pass them to functions, you should declare that the function takes a reference to the object. Passing objects by reference is usually more efficient than passing them by value, because only the address of the object is copied, not the entire contents of the object.

Additionally, pass-by-reference avoids problems with dynamic memory allocation in objects, which is discussed in Chapter 9.

When you pass an object by reference, the function using the object reference could change the original object. When you are only using pass-by-reference for efficiency, you should preclude this possibility by declaring the object const as well. This is known as passing objects by reference-to-const and has been done in examples throughout this book.

\begin{myNotic}{NOTE}
For performance reasons, it is best to pass objects by reference-to-const instead of by value. Chapter 9 slightly modifies this rule after the introduction of move semantics, allowing pass-by-value of objects in certain cases.
\end{myNotic}

Note that the SpreadsheetCell class has a number of member functions accepting an std::string\_view as parameter. As discussed in Chapter 2, a string\_view is basically just a pointer and a length. So, it is cheap to copy and is usually passed by value.

Also primitive types, such as int, double, and so on, should just be passed by value. You don’t gain anything by passing such types as reference-to-const.

The doubleToString() member function of the SpreadsheetCell class always returns a string by value because the implementation of the member function creates a local string object that at the end of the member function is returned to the caller. Returning a reference to this string wouldn’t work because the string to which it refers to will be destroyed when the function exits.

\mySamllsection{Explicitly Defaulted and Deleted Copy Constructors}

Just as you can explicitly default or delete a compiler-generated default constructor for a class, you can also explicitly default or delete a compiler-generated copy constructor as follows:

\begin{cpp}
SpreadsheetCell(const SpreadsheetCell& src) = default;
\end{cpp}

or

\begin{cpp}
SpreadsheetCell(const SpreadsheetCell& src) = delete;
\end{cpp}

By deleting the copy constructor, the object cannot be copied anymore. This can be used to disallow passing the object by value, as discussed in Chapter 9.

\begin{myNotic}{NOTE}
If a class has data members that have a deleted or private copy constructor, then the copy constructor for the class is automatically deleted as well, even if you explicitly default one.
\end{myNotic}

\mySamllsection{Initializer-List Constructors}

An initializer-list constructor is a constructor with an std::initializer\_list<T> (see Chapter 1) as the first parameter and without any additional parameters or with additional parameters having default values. The initializer\_list<T> class template is defined in <initializer\_list>. The following class demonstrates its use. The class accepts only an initializer\_list<double> with an even number of elements; otherwise, it throws an exception. Chapter 1 introduces exceptions.

\begin{cpp}
class EvenSequence
{
    public:
        EvenSequence(initializer_list<double> values)
        {
            if (values.size() % 2 != 0) {
                throw invalid_argument { "initializer_list should "
                    "contain even number of elements." };
            }
            m_sequence.reserve(values.size());
            for (const auto& value : values) {
                m_sequence.push_back(value);
            }
        }

        void print() const
        {
            for (const auto& value : m_sequence) {
                std::print("{}, ", value);
            }
            println("");
        }
    private:
        vector<double> m_sequence;
};
\end{cpp}

Inside the initializer-list constructor you can access the elements of the initializer-list with a rangebased for loop. You can get the number of elements in an initializer-list with the size() member function.

The EvenSequence initializer-list constructor uses a range-based for loop to copy elements from the given initializer\_list<T>. You can also use the assign() member function of vector. The different member functions of vector, including assign(), are discussed in detail in Chapter 18, “Standard Library Containers.” As a sneak preview, to give you an idea of the power of a vector, here is the initializer-list constructor using assign():

\begin{cpp}
EvenSequence(initializer_list<double> values)
{
    if (values.size() % 2 != 0) {
        throw invalid_argument { "initializer_list should "
            "contain even number of elements." };
    }
    m_sequence.assign(values);
}
\end{cpp}

EvenSequence objects can be constructed as follows:

\begin{cpp}
try {
    EvenSequence p1 { 1.0, 2.0, 3.0, 4.0, 5.0, 6.0 };
    p1.print();
    EvenSequence p2 { 1.0, 2.0, 3.0 };
} catch (const invalid_argument& e) {
    println("{}", e.what());
}
\end{cpp}

The construction of p2 throws an exception because it has an odd number of elements in the initializer-list.

The Standard Library has full support for initializer-list constructors. For example, the std::vector container can be initialized with an initializer-list:

\begin{cpp}
vector<string> myVec { "String 1", "String 2", "String 3" };
\end{cpp}

Without initializer-list constructors, one way to initialize this vector is by using several push\_back() calls:

\begin{cpp}
vector<string> myVec;
myVec.push_back("String 1");
myVec.push_back("String 2");
myVec.push_back("String 3");
\end{cpp}

Initializer lists are not limited to constructors and can also be used with normal functions as explained in Chapter 1.

\begin{myNotic}{NOTE}
When a class has both an initializer-list constructor and another singleargument constructor, then you should be careful to call the correct one. For example, std::vector has an initializer-list constructor to initialize a vector with a given set of elements. It also has a constructor accepting a single argument, the desired size of the new vector. You call the initializer-list constructor using a braced initializer, \{\}, and call the other single-argument constructor using parentheses, (). For example:

\begin{cpp}
vector<int> v1 { 6 }; // Constructs a vector with a single element, 6.
vector<int> v2 ( 6 ); // Constructs a vector with 6 default-
                      // initialized elements.
\end{cpp}
\end{myNotic}

\mySamllsection{Delegating Constructors}

Delegating constructors allow constructors to call another constructor from the same class. However, this call cannot be placed in the constructor body; it must be in the ctor-initializer, and it must be the only member-initializer in the list. The following is an example:

\begin{cpp}
SpreadsheetCell::SpreadsheetCell(string_view initialValue)
: SpreadsheetCell { stringToDouble(initialValue) }
{}
\end{cpp}

When this string\_view constructor (the delegating constructor) is called, it first delegates the call to the target constructor, which is the double constructor in this example. When the target constructor returns, the body of the delegating constructor is executed.

Make sure you avoid constructor recursion while using delegating constructors. Here is an example:

\begin{cpp}
class MyClass
{
    MyClass(char c) : MyClass { 1.2 } { }
    MyClass(double d) : MyClass { 'm' } { }
};
\end{cpp}

The first constructor delegates to the second constructor, which delegating back to the first one. The behavior of such code is undefined by the standard and depends on your compiler.

\mySamllsection{Converting Constructors and Explicit Constructors}

The current set of constructors for SpreadsheetCell is as follows:

\begin{cpp}
export class SpreadsheetCell
{
    public:
        SpreadsheetCell() = default;
        SpreadsheetCell(double initialValue);
        SpreadsheetCell(std::string_view initialValue);
        SpreadsheetCell(const SpreadsheetCell& src);
    // Remainder omitted for brevity
};
\end{cpp}

The single-parameter double and string\_view constructors can be used to convert a double or a string\_view into a SpreadsheetCell. Such constructors are called converting constructors. The compiler can use such constructors to perform implicit conversions for you. Here’s an example:

\begin{cpp}
SpreadsheetCell myCell { 4 };
myCell = 5;
myCell = "6"sv; // A string_view literal (see Chapter 2).
\end{cpp}

This might not always be the behavior you want. You can prevent the compiler from doing such implicit conversions by marking constructors as explicit. The explicit keyword goes only in the class definition. Here’s an example:

\begin{cpp}
export class SpreadsheetCell
{
    public:
        SpreadsheetCell() = default;
        SpreadsheetCell(double initialValue);
        explicit SpreadsheetCell(std::string_view initialValue);
        SpreadsheetCell(const SpreadsheetCell& src);
    // Remainder omitted for brevity
};
\end{cpp}

With this change, a line as follows no longer compiles:

\begin{cpp}
myCell = "6"sv; // A string_view literal (see Chapter 2).
\end{cpp}

Prior to C++11, converting constructors could have only a single parameter, as in the SpreadsheetCell example. Since C++11, converting constructors can have multiple parameters because of support for list initialization. Let’s look at an example. Suppose you have the following class:

\begin{cpp}
class MyClass
{
    public:
        MyClass(int) { }
        MyClass(int, int) { }
};
\end{cpp}

This class has two constructors, and since C++11, both are converting constructors. The following example shows that the compiler automatically converts arguments such as 1, \{1\}, and \{1,2\}, to instances of MyClass using these converting constructors:

\begin{cpp}
void process(const MyClass& c) { }

int main()
{
    process(1);
    process({ 1 });
    process({ 1, 2 });
}
\end{cpp}

To prevent these implicit conversions, both converting constructors can be marked as explicit:

\begin{cpp}
class MyClass
{
    public:
    explicit MyClass(int) { }
    explicit MyClass(int, int) { }
};
\end{cpp}

With this change, you have to perform these conversions explicitly; here’s an example:

\begin{cpp}
process(MyClass{ 1 });
process(MyClass{ 1, 2 });
\end{cpp}

It is possible to pass a Boolean argument to explicit to turn it into a conditional explicit. The syntax is as follows:

\begin{cpp}
explicit(true) MyClass(int);
\end{cpp}

Of course, just writing explicit(true) is equivalent to explicit, but it becomes more useful in the context of generic template code using type traits. With type traits you can query certain properties of given types, such as whether a certain type is convertible to another type. The result of such a type trait can be used as argument to explicit(). Type traits allow for writing advanced generic code and are discussed in Chapter 26, “Advanced Templates.”

\begin{myNotic}{NOTE}
It is recommended to mark at least any constructor that can be called with a single argument as explicit to avoid accidental implicit conversions. If there is a real use case for implicit conversions, you can mark the constructor with explicit(false). Doing so explains to users of your class that the implicit conversion is consciously allowed.
\end{myNotic}

\mySamllsection{Summary of Compiler-Generated Constructors}

The compiler can automatically generate a default constructor and a copy constructor for every class. However, the constructors that the compiler automatically generates depend on the constructors that you define yourself according to the rules in the following table:

% Please add the following required packages to your document preamble:
% \usepackage{longtable}
% Note: It may be necessary to compile the document several times to get a multi-page table to line up properly
\begin{longtable}{|l|l|l|}
\hline
\textbf{IF YOU DEFINE. . .} &
\textbf{\begin{tabular}[c]{@{}l@{}}THEN THE COMPILER\\ GENERATES. . .\end{tabular}} &
\textbf{AND YOU CAN CREATE AN OBJECT. . .} \\ \hline
\endfirsthead
%
\endhead
%
{[}no constructors{]} &
\begin{tabular}[c]{@{}l@{}}A default\\ constructor\\ A copy constructor\end{tabular} &
\begin{tabular}[c]{@{}l@{}}With no arguments: SpreadsheetCell a;\\ As a copy: SpreadsheetCell b\{a\};\end{tabular} \\ \hline
\begin{tabular}[c]{@{}l@{}}A default\\ constructor only\end{tabular} &
A copy constructor &
\begin{tabular}[c]{@{}l@{}}With no arguments: SpreadsheetCell a;\\ As a copy: SpreadsheetCell b\{a\};\end{tabular} \\ \hline
\begin{tabular}[c]{@{}l@{}}A copy constructor\\ only\end{tabular} &
No constructors &
\begin{tabular}[c]{@{}l@{}}Theoretically, as a copy of another object\\ (practically, you can’t create any objects, because\\ there are no non-copy constructors)\end{tabular} \\ \hline
\begin{tabular}[c]{@{}l@{}}A single- or multi\\ argument non-copy\\ constructor only\end{tabular} &
A copy constructor &
\begin{tabular}[c]{@{}l@{}}With arguments: SpreadsheetCell a\{6\};\\ As a copy: SpreadsheetCell b\{a\};\end{tabular} \\ \hline
\begin{tabular}[c]{@{}l@{}}A default\\ constructor as well\\ as a single- or multi\\ argument non-copy\\ constructor\end{tabular} &
A copy constructor &
\begin{tabular}[c]{@{}l@{}}With no arguments: SpreadsheetCell a;\\ With arguments: SpreadsheetCell b\{5\};\\ As a copy: SpreadsheetCell c\{a\};\end{tabular} \\ \hline
\end{longtable}

Note the lack of symmetry between the default constructor and the copy constructor. As long as you don’t define a copy constructor explicitly, the compiler creates one for you. On the other hand, as soon as you define any constructor, the compiler stops generating a default constructor.

As mentioned before in this chapter, the automatic generation of a default constructor and a default copy constructor can be influenced by defining them as explicitly defaulted or explicitly deleted.

\begin{myNotic}{NOTE}
A final type of constructor is a move constructor, required to implement move semantics. Move semantics is used to increase performance in certain situations and is discussed in detail in Chapter 9.
\end{myNotic}

\mySubsubsection{8.3.2.}{Object Destruction}

When an object is destroyed, two events occur: the object’s destructor member function is called, and the memory it was taking up is freed. The destructor is your chance to perform any cleanup work for the object, such as freeing dynamically allocated memory or closing file handles. If you don’t declare a destructor, the compiler writes one for you that does recursive member-wise destruction and allows the object to be deleted. A destructor of a class is a member function with as name the name of the class prefixed with a tilde (~). A destructor does not return anything and does not have any parameters. Here is an example of a destructor that simply writes something to standard output:

\begin{cpp}
export class SpreadsheetCell
{
    public:
        ~SpreadsheetCell(); // Destructor.
        // Remainder of the class definition omitted for brevity
};

SpreadsheetCell::~SpreadsheetCell()
{
    println("Destructor called.");
}
\end{cpp}

Objects on the stack are destroyed when they go out of scope, which means whenever the current function or other execution block ends. In other words, whenever the code encounters an ending curly brace, any objects created on the stack within those curly braces are destroyed. The following program shows this behavior:

\begin{cpp}
int main()
{
    SpreadsheetCell myCell { 5 };
    if (myCell.getValue() == 5) {
        SpreadsheetCell anotherCell { 6 };
    } // anotherCell is destroyed as this block ends.
    println("myCell: {}", myCell.getValue());
} // myCell is destroyed as this block ends.
\end{cpp}

Objects on the stack are destroyed in the reverse order of their declaration (and construction). For example, in the following code fragment, myCell2 is created before anotherCell2, so anotherCell2 is destroyed before myCell2 (note that you can start a new code block at any point in your program with an opening curly brace):

\begin{cpp}
{
    SpreadsheetCell myCell2 { 4 };
    SpreadsheetCell anotherCell2 { 5 }; // myCell2 constructed before anotherCell2
} // anotherCell2 destroyed before myCell2
\end{cpp}

This ordering also applies to objects that are data members of other objects. Recall that data members are initialized in the order of their declaration in the class. Thus, following the rule that objects are destroyed in the reverse order of their construction, data member objects are destroyed in the reverse order of their declaration order in the class.

Objects allocated on the free store without the help of smart pointers are not destroyed automatically. You must call delete on the object pointer to call its destructor and free its memory. The following program shows this behavior.

\begin{myWarning}{WARNING}
Do not write programs like the next example where cellPtr2 is not deleted. Make sure you always free dynamically allocated memory by calling delete or delete[] depending on whether the memory was allocated using new or new[]. Or better yet, use smart pointers as discussed earlier!
\end{myWarning}

\begin{cpp}
int main()
{
    SpreadsheetCell* cellPtr1 { new SpreadsheetCell { 5 } };
    SpreadsheetCell* cellPtr2 { new SpreadsheetCell { 6 } };
    println("cellPtr1: {}", cellPtr1->getValue());
    delete cellPtr1; // Destroys cellPtr1
    cellPtr1 = nullptr;
} // cellPtr2 is NOT destroyed because delete was not called on it.
\end{cpp}


\mySubsubsection{8.3.3.}{Assigning to Objects}

Just as you can assign the value of one int to another in C++, you can assign the value of one object to another. For example, the following code assigns the value of myCell to anotherCell:

\begin{cpp}
SpreadsheetCell myCell { 5 }, anotherCell;
anotherCell = myCell;
\end{cpp}

You might be tempted to say that myCell is “copied” to anotherCell. However, in the world of C++, “copying” occurs only when an object is being initialized. If an object already has a value that is being overwritten, the more accurate term is “assigned to.” Note that the facility that C++ provides for copying is the copy constructor. Because it is a constructor, it can only be used for object creation, not for later assignments to the object.

Therefore, C++ provides another member function in every class to perform assignment. This member function is called the assignment operator. Its name is operator= because it is actually an overload of the = operator for that class. In the preceding example, the assignment operator for anotherCell is called, with myCell as the argument.

\begin{myNotic}{NOTE}
The assignment operator as explained in this section is sometimes called the copy assignment operator because data is copied from the right-hand side object to the left-hand side object. Chapter 9 discusses another kind of assignment operator, the move assignment operator, in which data is moved instead of copied, which improves performance for certain use cases.
\end{myNotic}

As usual, if you don’t write your own assignment operator, C++ writes one for you to allow objects to be assigned to one another. The default C++ assignment behavior is almost identical to its default copying behavior: it recursively assigns each data member from the source to the destination object.

\mySamllsection{Declaring an Assignment Operator}

Here is the assignment operator for the SpreadsheetCell class:

\begin{cpp}
export class SpreadsheetCell
{
    public:
        SpreadsheetCell& operator=(const SpreadsheetCell& rhs);
        // Remainder of the class definition omitted for brevity
};
\end{cpp}

The assignment operator usually takes a reference-to-const to the source object, like the copy constructor. In this case, the source object is called rhs, which stands for right-hand side of the equal sign, but you are of course free to call it whatever you want. The object on which the assignment operator is called is the left-hand side of the equal sign.

Unlike a copy constructor, the assignment operator returns a reference to a SpreadsheetCell object. The reason is that assignments can be chained, as in the following example:

\begin{cpp}
myCell = anotherCell = aThirdCell;
\end{cpp}

When that line is executed, the first thing that happens is the assignment operator for anotherCell is called with aThirdCell as its “right-hand side” argument. Next, the assignment operator for myCell is called. However, its argument is not anotherCell; its right-hand side is the result of the assignment of aThirdCell to anotherCell. The equal sign is simply just shorthand for what is really a member function call. When you look at the line in its full functional syntax shown here, you can see the problem:

\begin{cpp}
myCell.operator=(anotherCell.operator=(aThirdCell));
\end{cpp}

Now, you can see that the operator= call from anotherCell must return a value, which is passed to the operator= call for myCell. The correct value to return is a reference to anotherCell itself, so it can serve as the source for the assignment to myCell.

\begin{myWarning}{WARNING}
You could actually declare the assignment operator to return whatever type you wanted, including void. However, you should always return a reference to the object on which it is called because that’s what clients expect.
\end{myWarning}

\mySamllsection{Defining an Assignment Operator}

The implementation of the assignment operator is similar to that of a copy constructor but with several important differences. First, a copy constructor is called only for initialization, so the destination object does not yet have valid values. An assignment operator can overwrite the current values in an object. This consideration doesn’t really come into play until you have dynamically allocated resources, such as memory, in your objects. See Chapter 9 for details.

Second, it’s legal in C++ to assign an object to itself. For example, the following code compiles and runs:

\begin{cpp}
SpreadsheetCell cell { 4 };
cell = cell; // Self-assignment
\end{cpp}

Your assignment operator needs to take the possibility of self-assignment into account. In the SpreadsheetCell class, this is not important, as its only data member is a primitive type, double. However, when your class has dynamically allocated memory or other resources, it’s paramount to take self-assignment into account, as is discussed in detail in Chapter 9. To prevent problems in such cases, the first thing assignment operators usually do is check for self-assignment and return immediately if that’s the case.

Here is the start of the definition of the assignment operator for the SpreadsheetCell class:

\begin{cpp}
SpreadsheetCell& SpreadsheetCell::operator=(const SpreadsheetCell& rhs)
{
    if (this == &rhs) {
\end{cpp}

This first line checks for self-assignment, but it might be a bit cryptic. Self-assignment occurs when the left-hand side and the right-hand side of the equal sign are the same. One way to tell whether two objects are the same is if they occupy the same memory location—more explicitly, if pointers to them are equal. Recall that this is a pointer to an object accessible from any member function called on the object. Thus, this is a pointer to the left-hand side object. Similarly, \&rhs is a pointer to the righthand side object. If these pointers are equal, the assignment must be self-assignment, but because the return type is SpreadsheetCell\&, a correct value must still be returned. All assignment operators return *this as follows, and the self-assignment case is no exception:

\begin{cpp}
        return *this;
    }
\end{cpp}

this is a pointer to the object on which the member function executes, so *this is the object itself. The compiler returns a reference to the object to match the declared return type. Now, if it is not selfassignment, you have to do an assignment to every member:

\begin{cpp}
    m_value = rhs.m_value;
    return *this;
}
\end{cpp}

Here the member function copies the values, and finally, it returns *this, as explained earlier.

Astute readers will notice there’s some code duplication between the copy assignment operator and the copy constructor; they both need to copy all data members. Chapter 9 introduces the copy-andswap idiom to prevent such code duplication.

\begin{myNotic}{NOTE}
The SpreadsheetCell assignment operator is shown only for demonstration purposes. In fact, in this case, the assignment operator can be omitted because the default compiler-generated one is good enough; it does simple member-wise assignments of all data members. However, under certain conditions, this compilergenerated assignment operator is not sufficient. These conditions are covered in Chapter 9.
\end{myNotic}

\begin{myWarning}{WARNING}
If your class requires special handling for copy operations, always implement both the copy constructor and the copy assignment operator.
\end{myWarning}

\mySamllsection{Explicitly Defaulted and Deleted Assignment Operator}

You can explicitly default or delete a compiler-generated assignment operator as follows:

\begin{cpp}
SpreadsheetCell& operator=(const SpreadsheetCell& rhs) = default;
\end{cpp}

or

\begin{cpp}
SpreadsheetCell& operator=(const SpreadsheetCell& rhs) = delete;
\end{cpp}

\mySubsubsection{8.3.4.}{Compiler-Generated Copy Constructor and Copy Assignment Operator}

C++11 deprecated the generation of a copy constructor if the class has a user-declared copy assignment operator or destructor. If you still need a compiler-generated copy constructor in such a case, you can explicitly default one:

\begin{cpp}
MyClass(const MyClass& src) = default;
\end{cpp}

C++11 also deprecated the generation of a copy assignment operator if the class has a user-declared copy constructor or destructor. If you still need a compiler-generated copy assignment operator in such a case, you can explicitly default one:

\begin{cpp}
MyClass& operator=(const MyClass& rhs) = default;
\end{cpp}

\mySubsubsection{8.3.5.}{Distinguishing Copying from Assignment}

It is sometimes difficult to tell when objects are initialized with a copy constructor rather than assigned to with the assignment operator. Essentially, things that look like a declaration are going to be using copy constructors, and things that look like assignment statements are handled by the assignment operator. Consider the following code:

\begin{cpp}
SpreadsheetCell myCell { 5 };
SpreadsheetCell anotherCell { myCell };
\end{cpp}

AnotherCell is constructed with the copy constructor. Now consider the following:

\begin{cpp}
SpreadsheetCell aThirdCell = myCell;
\end{cpp}

aThirdCell is also constructed with the copy constructor, because this is a declaration. The operator= is not called for this line! This syntax is just another way to write SpreadsheetCell aThirdCell\{myCell\};. However, consider the following code:

\begin{cpp}
anotherCell = myCell; // Calls operator= for anotherCell
\end{cpp}

Here, anotherCell has already been constructed, so the compiler calls operator=.

\mySamllsection{Objects as Return Values}

When you return objects from functions, it is sometimes difficult to see exactly what copying and assigning is happening. For example, the implementation of SpreadsheetCell::getString() looks like this:

\begin{cpp}
string SpreadsheetCell::getString() const
{
    return doubleToString(m_value);
}
\end{cpp}

Now consider the following code:

\begin{cpp}
SpreadsheetCell myCell2 { 5 };
string s1;
s1 = myCell2.getString();
\end{cpp}

When getString() returns the string, the compiler actually creates an unnamed temporary string object by calling a string copy constructor. When you assign this result to s1, the assignment operator is called for s1 with the temporary string as a parameter. Then, the temporary string object is destroyed. Thus, the single line of code could invoke the copy constructor and the assignment operator (for two different objects).

In case you’re not confused enough, consider this code:

\begin{cpp}
SpreadsheetCell myCell3 { 5 };
string s2 = myCell3.getString();
\end{cpp}

In this case, getString() still creates a temporary unnamed string object when it returns. But now, s2 gets its copy constructor called, not its assignment operator.

With move semantics, the compiler can use a move constructor or move assignment operator instead of a copy constructor or copy assignment operator to return the string from getString(). This can be more efficient in certain cases and is discussed in Chapter 9. However, even better, compilers are free to (and often even required to) implement copy elision to optimize away costly copy operations or move operations when returning values; see Chapter 1.

If you ever forget the order in which these things happen or which constructor or operator is called, you can easily figure it out by temporarily including helpful output in your code or by stepping through your code with a debugger.

\mySamllsection{Copy Constructors and Object Members}

You should also note the difference between assignment operator and copy constructor calls in constructors. If an object contains other objects, the compiler-generated copy constructor calls the copy constructors of each of the contained objects recursively. When you write your own copy constructor, you can provide the same semantics by using a ctor-initializer, as shown previously. If you omit a data member from the ctor-initializer, the compiler performs default initialization on it (a call to the default constructor for objects) before executing your code in the body of the constructor. Thus, by the time the body of the constructor executes, all object data members have already been initialized.

For example, you could write the SpreadsheetCell copy constructor like this:

\begin{cpp}
SpreadsheetCell::SpreadsheetCell(const SpreadsheetCell& src)
{
    m_value = src.m_value;
}
\end{cpp}

However, when you assign values to data members in the body of the copy constructor, you are using the assignment operator on them, not the copy constructor, because they have already been initialized.

If you write the copy constructor as follows, then m\_value is initialized using the copy constructor:

\begin{cpp}
SpreadsheetCell::SpreadsheetCell(const SpreadsheetCell& src)
: m_value { src.m_value }
{}
\end{cpp}



















