\noindent
\textbf{WHAT’S IN THIS CHAPTER?}

\begin{itemize}
\item
How to write your own classes with member functions and data members

\item
How to control access to your member functions and data members

\item
How to use objects on the stack and on the free store

\item
What the life cycle of an object is

\item
How to write code that is executed when an object is created or destroyed

\item
How to write code to copy or assign objects
\end{itemize}

\noindent
\textbf{WILEY.COM DOWNLOADS FOR THIS CHAPTER}

Please note that all the code examples for this chapter are available as part of this chapter’s code download on the book’s website at \url{www.wiley.com/go/proc++6e} on the Download Code tab.

As an object-oriented language, C++ provides facilities for using objects and for writing object blueprints, called classes. You can certainly write programs in C++ without classes and objects, but by doing so, you do not take advantage of the most fundamental and useful aspect of the language; writing a C++ program without classes is like traveling to Paris and eating at McDonald’s. To use classes and objects effectively, you must understand their syntax and capabilities.

Chapter 1, “A Crash Course in C++ and the Standard Library,” reviewed the basic syntax of class definitions. Chapter 5, “Designing with Classes,” introduced the object-oriented approach to programming in C++ and presented specific design strategies for classes and objects. This chapter describes the fundamental concepts involved in using classes and objects, including writing class definitions, defining member functions, using objects on the stack and the free store, writing constructors, default constructors, compiler-generated constructors, constructor initializers (also known as ctor-initializers), copy constructors, initializer-list constructors, destructors, and assignment operators. Even if you are already comfortable with classes and objects, you should skim this chapter because it contains various tidbits of information with which you might not yet be familiar.























