By solving the following exercises, you can practice the material discussed in this chapter. Solutions to all exercises are available with the code download on the book’s website at \url{www.wiley.com/go/proc++6e}. However, if you are stuck on an exercise, first reread parts of this chapter to try to find an answer yourself before looking at the solution from the website.

\begin{itemize}
\item
Exercise 8-1: Implement a Person class storing a first and last name as data members. Add a single constructor accepting two parameters, the first and last name. Provide appropriate getters and setters. Write a small main() function to test your implementation by creating a Person object on the stack and on the free store.

\item
Exercise 8-2: With the set of member functions implemented in Exercise 8-1, the following line of code does not compile:

\begin{cpp}
Person persons[3];
\end{cpp}

Can you explain why this does not compile? Modify the implementation of your Person class to make this work.

\item
Exercise 8-3: Add the following member functions to your Person class implementation: a copy constructor, a copy assignment operator, and a destructor. In all of these member functions, implement what you think is necessary, and additionally, output a line of text to the console so you can trace when they are executed. Modify your main() function to test these new member functions. Note: technically, these new member functions are not strictly required for this Person class, because the compiler-generated versions are good enough, but this exercise is to practice writing them.

\item
Exercise 8-4: Remove the copy constructor, copy assignment operator, and destructor from your Person class, because the default compiler-generated versions are exactly what you need for this simple class. Next, add a new data member to store the initials of a person, and provide a getter and setter. Add a new constructor that accepts three parameters, a first and last name, and a person’s initials. Modify the original two-parameter constructor to automatically generate initials for a given first and last name, and delegate the actual construction work to the new three-parameter constructor. Test this new functionality in your main() function.
\end{itemize}