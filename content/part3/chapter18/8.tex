通过解决下面的练习,可以练习本章讨论的内容。所有练习的解决方案都可以在本书的网站\url{www.wiley.com/go/proc++6e}下载到源码。若在练习中卡住了,可以考虑先重读本章的部分内容,试着自己找到答案,再在从网站上寻找解决方案。

\begin{itemize}
\item
\textbf{练习18-1}:这个练习是为了练习使用vector。创建一个包含整数2和5的vector变量values。接下来,实现以下操作:
\begin{enumerate}
\item
使用insert()在values中正确位置插入数字3和4。

\item
创建一个初始化为0和1的第2个整数vector,然后将这个新vector的内容插入到values的开头。

\item
创建一个第3个整数vector。从values的元素开始逆序遍历,并将它们插入到这个第3个vector中。

\item
使用println()打印第3个vector的内容。

\item
使用基于范围的for循环打印第3个vector的内容。
\end{enumerate}

\item
\textbf{练习18-2}:基于练习15-4中实现的Person类,添加一个名为phone\_book的新模块,定义一个PhoneBook类,用于存储一个人一个或多个电话号码(字符串)。提供添加和删除个人电话号码的成员函数,还提供一个返回给定人所有电话号码vector的成员函数。在main()函数中测试你的实现。测试中,使用在练习15-4中开发的用户定义的person字面量。

\item
\textbf{练习18-3}:练习15-1中,我们开发了自己的AssociativeArray。修改那个练习中的main()函数中的测试代码,使用标准库容器。

\item
\textbf{练习18-4}:编写一个average()函数(不是函数模板)来计算一个double值序列的平均值,确保它能够处理vector或数组中的序列或子序列。在main()函数中使用vector和数组测试代码。

\item
\textbf{附加练习}:能将average()函数转换为函数模板吗?该函数模板只能与整数或浮点类型实例化,对main()函数中的测试代码有什么影响?
\end{itemize}














