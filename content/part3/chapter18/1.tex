
标准库中的容器为通用数据结构,非常适合存储数据集合。使用标准库时,很少需要使用标准C语言风格的数组、定义链表或设计栈。容器实现为类模板,因此可以为满足下一节所述基本条件的类型进行实例化。除了数组和bitset之外,大多数标准库容器的大小都很灵活,可以自动增长或缩小,以容纳更多或更少的元素。与旧式的、大小固定的C语言风格数组相比,这是一个巨大的优势。由于C语言风格数组大小固定的特性,更容易受到越界的影响,因为数据破坏,会导致程序崩溃,但最糟糕的情况下,可能会成为漏洞,可以对程序进行到某些类型的攻击。通过使用标准库容器,可以确保程序对这些类型问题的抵抗力更强。

第16章提供了标准库提供的不同容器、容器适配器和顺序视图的概述。下表进行了总结:

\begin{itemize}
\item
顺序容器
\begin{itemize}
\item
vector (动态数组)

\item
deque

\item
list

\item
forward\_list

\item
array
\end{itemize}

\item
顺序视图
\begin{itemize}
\item
span

\item
mdspan
\end{itemize}

\item
容器适配器
\begin{itemize}
\item
queue

\item
priority\_queue

\item
stack
\end{itemize}

\item
有序关联容器
\begin{itemize}
\item
map / multimap

\item
set / multiset
\end{itemize}

\item
无序关联容器或散列表
\begin{itemize}
\item
unordered\_map / unordered\_multimap

\item
unordered\_set / unordered\_multiset
\end{itemize}

\item
平面集合和平面映射关联容器适配器
\begin{itemize}
\item
flat\_map / flat\_multimap

\item
flat\_set / flat\_multiset
\end{itemize}
\end{itemize}

此外,C++字符串和流也可以用作标准库容器,而bitset可用于存储固定数量的位。

标准库中的所有内容都在std命名空间中,本书中的示例会在源文件中使用using namespace std;指令(永远不要在头文件中使用!),但在自己的程序中,可以更选择性地的使用。

\mySubsubsection{18.1.1.}{元素的要求}

标准库容器对元素使用值语义,存储给定元素的副本,使用赋值操作符分配给元素,并使用析构函数销毁元素。当定义与标准库一起使用的类时,需要确保它们可以复制。当从容器读取元素时,将返回对存储副本的引用。

如果喜欢引用语义,可以使用指向元素的指针,而不是将元素本身存储在容器中。当容器复制指针,结果仍然引用相同的元素。另一种选择是在容器中存储std::reference\_wrapper。reference\_wrapper是为了使引用可复制,可以使用std::ref()和cref()辅助函数创建。reference\_wrapper类模板,以及ref()和cref()函数模板定义在<functional>中。

在容器中可以存储仅移动类型,即不可复制的类型。但这样,容器上的一些操作可能无法编译。一个典型仅移动类型的例子即为std::unique\_ptr。

\begin{myWarning}{WARNING}
需要在容器中存储指针,如果容器需要成为指向对象的所有者,请使用unique\_ptr。如果容器与其他所有者共享所有权,请使用shared\_ptr。不要在容器中使用旧的已删除的auto\_ptr类。
\end{myWarning}

标准库容器的模板类型参数之一是分配器,容器使用这个分配器为元素分配和释放内存。分配器类型参数有一个默认值,所以可以忽略它。例如,vector类模板如下所示:

\begin{cpp}
template <typename T, typename Allocator = std::allocator<T>> class vector;
\end{cpp}

一些容器,如map,还接受比较器作为模板类型参数,比较器用于对元素进行排序。也有一个默认值,这个默认值是使用operator<比较元素。map类模板如下所示:

\begin{cpp}
template <typename Key, typename T, typename Compare = std::less<Key>,
    typename Allocator = std::allocator<std::pair<const Key, T>>> class map;
\end{cpp}

本章后面将详细介绍分配器和比较器模板类型参数。

使用默认分配器和默认比较器的容器中元素的具体要求,如下表所示:

% Please add the following required packages to your document preamble:
% \usepackage{longtable}
% Note: It may be necessary to compile the document several times to get a multi-page table to line up properly
\begin{longtable}{|l|l|l|}
\hline
\textbf{成员函数} &
\textbf{描述} &
\textbf{说明} \\ \hline
\endfirsthead
%
\endhead
%
复制构造函数 &
\begin{tabular}[c]{@{}l@{}}创建一个新元素,该元素与旧元素\\“相等”,可以安全地析构,而不会\\影响旧元素。
\end{tabular} &
\begin{tabular}[c]{@{}l@{}}每次插入元素时使用,\\除了使用emplace成员\\函数的情况。
\end{tabular} \\ \hline
移动构造函数 &
\begin{tabular}[c]{@{}l@{}}通过将源元素的所有内容移动到新\\元素来创建一个新元素。
\end{tabular} &
\begin{tabular}[c]{@{}l@{}}当源元素是一个右值,\\并且在构造新元素后\\销毁时使用;也用于\\vector增长时。移动\\构造函数应该是\\noexcept的;否则,\\不会使用!
\end{tabular} \\ \hline
赋值操作符 &
\begin{tabular}[c]{@{}l@{}}用源元素的副本替换元素的内容。
\end{tabular} &
每次修改元素时使用。
 \\ \hline
\begin{tabular}[c]{@{}l@{}}移动赋值操作符\end{tabular} &
\begin{tabular}[c]{@{}l@{}}通过将源元素的所有内容移动过来,\\替换元素的内容。
\end{tabular} &
\begin{tabular}[c]{@{}l@{}}当源元素是一个右值,\\并且在赋值操作后将\\销毁时使用。移动赋\\值操作符应该是\\noexcept的;否则,\\不会使用!
\end{tabular} \\ \hline
析构函数 &
清理元素。 &
\begin{tabular}[c]{@{}l@{}}每次移除元素时使用,\\或者当vector增长时\\使用。
\end{tabular} \\ \hline
默认构造函数 &
\begin{tabular}[c]{@{}l@{}}构造一个没有任何参数的元素。
\end{tabular} &
\begin{tabular}[c]{@{}l@{}}仅对某些操作是必需\\的,例如带有单个参\\数的vector::resize()\\成员函数,以及对\\map::operator{[}{]}的访问。
\end{tabular} \\ \hline
operator== &
\begin{tabular}[c]{@{}l@{}}比较两个元素是否相等。
\end{tabular} &
\begin{tabular}[c]{@{}l@{}}对于无序关联容器中的\\键,以及两个容器的\\operator==等操作是必\\需存在的。
\end{tabular} \\ \hline
operator< &
\begin{tabular}[c]{@{}l@{}}确定一个元素是否小于另一个元素。
\end{tabular} &
\begin{tabular}[c]{@{}l@{}}对于有序关联容器和\\扁平关联容器适配器\\中的键,以及两个容\\器上的operator<等某\\些操作是必需存在的。
\end{tabular} \\ \hline
operator>, <=, >=, != &
比较两个元素。
 &
\begin{tabular}[c]{@{}l@{}}比较两个容器时是必\\需存在的。
\end{tabular} \\ \hline
\end{longtable}

第9章解释了如何编写这些成员函数。

\begin{myWarning}{WARNING}
标准库容器经常移动或复制元素,为了获得最佳性能,请确保存储在容器中的对象类型支持移动语义,参见第9章。如果移动语义不确定,请尽可能高效的实现复制构造函数和复制赋值操作符。
\end{myWarning}

\mySubsubsection{18.1.2.}{异常和错误检查}

标准库容器提供有限的错误检查,客户端应确保他们的使用是有效的,但某些容器成员函数在某些条件下会抛出异常,例如:越界索引。当然,不可能详尽地列出这些成员函数可能抛出的异常,它们会对用户指定类型在执行操作时,抛出未知的异常。本章在适当的地方介绍了异常。

有关每个成员函数可能抛出的异常的列表,请参阅标准库参考(参见附录B)。












