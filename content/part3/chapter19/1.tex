
C++中虽然不会考虑函数在内存中的位置,但每个函数实际上都驻留在特定的地址。C++可以将函数当作数据使用,C++具有一等函数,可以获取函数的地址,并像使用变量一样使用。

函数指针的类型取决于与之兼容的函数参数类型和返回类型。下面是一个变量定义的例子,该变量名为fun,能够指向返回布尔值并接受两个整数参数的函数:

\begin{cpp}
bool (*fun)(int, int);
\end{cpp}

不要忘记*fun周围的括号;否则,这个声明将不是一个变量,而是一个名为fun的函数原型,该函数接受两个整数并返回一个指向布尔值的指针。fun函数指针未初始化,应避免使用未初始化的数据,所以可以将fun初始化为nullptr:

\begin{cpp}
bool (*fun)(int, int) { nullptr };
\end{cpp}

\mySubsubsection{19.1.1.}{findMatches()使用函数指针}

处理函数指针的另一种方法是使用类型别名,类型别名允为具有给定特征的函数族分配一个类型名称。例如,下面的定义了一个名为Matcher的类型,代表指向具有两个整数参数,并返回布尔值的函数指针:

\begin{cpp}
using Matcher = bool(*)(int, int);
\end{cpp}

下面的类型别名定义了一个名为MatchHandler的类型,用于接受一个size\_t和两个整数,并返回空的函数:

\begin{cpp}
using MatchHandler = void(*)(size_t, int, int);
\end{cpp}

现在这些类型已经定义,可以编写一个函数,该函数接受两个回调参数:Matcher和MatchHandler。接受其他函数作为参数的函数,或者返回函数的函数,称为高阶函数。例如,下面的函数接受两个整数范围的span,以及一个Matcher和一个MatchHandler。并行遍历两个span,并对两个span中对应位置的元素调用Matcher。如果Matcher返回true,则调用MatchHandler,第一个参数是匹配的位置,第二个和第三个参数是导致Matcher返回true的值。注意,尽管Matcher和MatchHandler作为变量传递,但它们可以像普通函数一样调用:

\begin{cpp}
void findMatches(span<const int> values1, span<const int> values2,
Matcher matcher, MatchHandler handler)
{
    if (values1.size() != values2.size()) { return; } // Must be same size.
    for (size_t i { 0 }; i < values1.size(); ++i) {
        if (matcher(values1[i], values2[i])) {
            handler(i, values1[i], values2[i]);
        }
    }
}
\end{cpp}

注意,这个实现要求两个span具有相同数量的元素。要调用findMatches()函数,需要的是符合定义的Matcher类型函数——即任何接受两个整数并返回布尔值的函数——以及符合MatchHandler类型的函数。下面是一个Matcher的例子,如果两个参数相等则返回true:

\begin{cpp}
bool intEqual(int value1, int value2) { return value1 == value2; }
\end{cpp}

下面是一个简单的MatchHandler示例,只打印出匹配项:

\begin{cpp}
void printMatch(size_t position, int value1, int value2)
{
    println("Match found at position {} ({}, {})", position, value1, value2);
}
\end{cpp}

intEqual()和printMatch()函数可以作为参数传递给findMatches(),如下所示:

\begin{cpp}
vector values1 { 2, 5, 6, 9, 10, 1, 1 };
vector values2 { 4, 4, 2, 9, 0, 3, 1 };
println("Calling findMatches() using intEqual():");
findMatches(values1, values2, &intEqual, &printMatch);
\end{cpp}

回调函数通过取其地址传递给findMatches()。从技术上讲,\&是可选的——如果省略,只放置函数名,编译器也会知道是要取其地址。输出如下:

\begin{shell}
Calling findMatches() using intEqual():
Match found at position 3 (9, 9)
Match found at position 6 (1, 1)
\end{shell}

函数指针的好处在于,findMatches()是一个通用函数,可以比较两个vector中并行的值。之前的例子中,基于相等性比较值。因为接受一个函数指针,所以可以根据其他标准比较值。例如,下面的函数也符合Matcher的定义:

\begin{cpp}
bool bothOdd(int value1, int value2) { return value1 % 2 == 1 && value2 % 2 == 1; }
\end{cpp}

以下代码段说明,bothOdd()也可以在调用findMatches()时使用:

\begin{cpp}
println("Calling findMatches() using bothOdd():");
findMatches(values1, values2, bothOdd, printMatch);
\end{cpp}

输出如下:

\begin{shell}
Calling findMatches() using bothOdd():
Match found at position 3 (9, 9)
Match found at position 5 (1, 3)
Match found at position 6 (1, 1)
\end{shell}

通过使用函数指针,单个函数findMatches()可以根据作为参数传递的函数/回调进行定制。

\mySubsubsection{19.1.2.}{findMatches()作为函数模板}

不需要使用显式的函数指针参数来让findMatches()接受回调参数,而可以将findMatches()转换为一个函数模板。所需的更改是删除Matcher和MatchHandler类型别名,并使findMatches()成为一个函数模板。更改如下所示:

\begin{cpp}
template <typename Matcher, typename MatchHandler>
void findMatches(span<const int> values1, span<const int> values2,
    Matcher matcher, MatchHandler handler)
{ /* ... */ }
\end{cpp}

更好的是下面的受限函数模板。Matcher模板类型参数受到predicate<int,int>的限制(参见第12章),以确保用户提供的回调可以接受两个int参数并返回布尔值。类似地,MatchHandler模板类型参数可以接受一个size\_t参数和两个int参数,并返回空的回调的限制。

\begin{cpp}
template <predicate<int, int> Matcher, invocable<size_t, int, int> MatchHandler>
void findMatches(span<const int> values1, span<const int> values2,
    Matcher matcher, MatchHandler handler)
{ /* ... */ }
\end{cpp}

这两个findMatches()的实现都需要两个模板类型参数,即Matcher和MatchHandler回调的类型,但由于函数模板参数推导,调用版本与早期版本相同。

使用简化的函数模板语法,findMatches()函数模板可以更优雅地进行编写。请注意,不再有显式的模板头,template<...>。

\begin{cpp}
void findMatches(span<const int> values1, span<const int> values2,
    auto matcher, auto handler)
{ /* ... */ }
\end{cpp}

matcher和handler参数可以再次受到限制:

\begin{cpp}
void findMatches(span<const int> values1, span<const int> values2,
    predicate<int, int> auto matcher, invocable<size_t, int, int> auto handler)
{ /* ... */ }
\end{cpp}

\begin{myNotic}{NOTE}
findMatches()函数模板或简化函数模板实际上是实现函数的首选方式,而不是使用显式的函数指针参数。
\end{myNotic}

现在应该很清楚,回调允许编写非常通用和可配置的代码。正是这种回调的使用,使得许多标准库算法(在第20章中介绍)如此强大。

\mySubsubsection{19.1.3.}{Windows的DLL和函数指针}

使用函数指针的一个常见用例是获取动态链接库中函数的指针,以下示例获取Microsoft Windows动态链接库(DLL)中函数的指针。DLL基本上是一个包含代码和数据的库,可以被任何程序使用。Windows DLL的一个具体例子是User32 DLL,提供了大量的功能,包括在屏幕上显示消息框的函数。Windows DLL的详细信息超出了本书关于平台无关C++的范围,但对Windows开发者来说非常重要,因此值得简要讨论,并且是函数指针的一般好例子。

User32.dll中显示消息框的一个函数名为MessageBoxA()。假设只想在需要显示消息框时加载这个库。在运行时加载库可以使用Windows的LoadLibraryA()函数(需要包含<Windows.h>):

\begin{cpp}
HMODULE lib { ::LoadLibraryA("User32.dll") };
\end{cpp}

这个调用的结果是库句柄,如果出错则为NULL。在从库中加载函数之前,需要知道函数的原型。MessageBoxA()的原型如下所示:

\begin{cpp}
int MessageBoxA(HWND, LPCSTR, LPCSTR, UINT);
\end{cpp}

第一个参数是拥有消息框的窗口(可以为NULL),第二个是要显示为消息的字符串,第三个是窗口的标题,第四个是消息框的配置标志,例如显示哪些按钮和哪个图标。

现在可以为具有所需原型的函数指针,定义一个类型别名MessageBoxFunction:

\begin{cpp}
using MessageBoxFunction = int(*)(HWND, LPCSTR, LPCSTR, UINT);
\end{cpp}

成功加载库并为函数指针定义了类型别名后,可以按以下方式获取库中函数的指针:

\begin{cpp}
MessageBoxFunction messageBox {
    (MessageBoxFunction)::GetProcAddress(lib, "MessageBoxA") };
\end{cpp}

如果失败,messageBox将为nullptr。如果成功,可以按以下方式调用加载的函数。MB\_OK是在消息框中,只显示一个OK按钮的标志。

\begin{cpp}
messageBox(NULL, "Hello World!", "ProC++", MB_OK);
\end{cpp}











