
可以重载类的函数调用操作符,使得该类的对象可以代替函数指针,这样的对象称为函数对象,简称函子。使用函数对象的好处是,函数对象可以保持状态。

\mySubsubsection{19.3.1.}{第一个函数对象}

如第15章所述,要使类成为一个函数对象,只需提供一个函数调用操作符的重载:

\begin{cpp}
class IsLargerThan
{
    public:
        explicit IsLargerThan(int value) : m_value { value } {}
        bool operator()(int value1, int value2) const {
            return value1 > m_value && value2 > m_value;
        }
    private:
        int m_value;
};

int main()
{
    vector values1 { 2, 500, 6, 9, 10, 101, 1 };
    vector values2 { 4, 4, 2, 9, 0, 300, 1 };

    findMatches(values1, values2, IsLargerThan { 100 }, printMatch);
}
\end{cpp}

IsLargerThan类的函数调用操作符标记为const,这并不是必要的,但对于大多数标准库算法,谓词函数调用操作符必须为const。

从C++23开始,如果重载的operator()不需要访问非静态数据成员和成员函数,可以标记为static。这样做可以让编译器更好地优化代码。

\CXXTwentythreeLogo{-40}{15}

\mySubsubsection{19.3.2.}{标准库中的函数对象}

下一章中介绍的许多标准库算法,如find\_if()、accumulate()等,接受回调作为参数,以定制算法的行为。C++提供了一些预定义的函数对象类,定义在<functional>中,执行最常用的回调操作。本节概述了这些预定义的函数对象。

<functional>可能还包含像bind1st()、bind2nd()、mem\_fun()、mem\_fun\_ref()和ptr\_fun()这样的函数。这些函数在C++17中已经被官方移除,因此本书中不再介绍,读者们也应该避免使用。

\mySamllsection{算术函数对象}

C++提供了五个二元算术操作符(加、减、乘、除、取模)的函数对象类模板,以及一个一元取反的函数对象。这些类根据操作数的类型参数化,是实际操作符的包装器。接受模板类型的一个或两个参数,执行操作,并返回结果。以下是一个使用plus类模板的例子:

\begin{cpp}
plus<int> myPlus;
int res { myPlus(4, 5) };
println("{}", res);
\end{cpp}

这个例子有些蠢,可以直接使用operator+时,没有理由使用plus类模板。算术函数对象的好处在于,可以将它们作为回调传递给其他函数,这是无法直接使用算术操作符能够完成的。以下代码段定义了一个受约束的accumulateData()函数模板,接受Operation作为最后一个参数。几何平均值的实现调用了accumulateData(),使用预定义的multiplies函数对象实例:

\begin{cpp}
template <input_iterator Iter, copy_constructible StartValue,
          invocable<const StartValue&, const StartValue&> Operation>
auto accumulateData(Iter begin, Iter end,
    const StartValue& startValue, Operation op)
{
    auto accumulated { startValue };
    for (Iter iter { begin }; iter != end; ++iter) {
        accumulated = op(accumulated, *iter);
    }
    return accumulated;
}

double geometricMean(span<const int> values)
{
    auto mult {accumulateData(cbegin(values), cend(values), 1, multiplies<int>{})};
    return pow(mult, 1.0 / values.size()); // pow() is defined in <cmath>
}
\end{cpp}

表达式multiplies<int>\{\}创建了一个multiplies函数对象类的实例,其类型为int。

其他算术函数对象的行为类似。

\begin{myWarning}{WARNING}
算术函数对象只是算术操作符的包装器。要使用它们对特定类型的对象进行操作,必须确保这些类型实现了适当的操作符,如operator*或operator+。
\end{myWarning}

\mySamllsection{透明操作符}

C++支持透明操作符函数对象,允许省略模板类型参数。可以简单地指定multiplies<>\{\},即multiplies<void>\{\}的简写,而不是multiplies<int>\{\}:

\begin{cpp}
double geometricMeanTransparent(span<const int> values)
{
    auto mult { accumulateData(cbegin(values), cend(values), 1, multiplies<>{}) };
    return pow(mult, 1.0 / values.size());
}
\end{cpp}

些透明操作符的一个重要特性是异构,不仅比非透明函数对象更简洁,而且具有真正的功能优势。以下代码使用透明操作符函数对象multiplies<>\{\},并使用1.1,一个double,作为起始值,而vector包含整数。accumulateData()将结果计算为double,结果将是6.6。

\begin{cpp}
vector<int> values { 1, 2, 3 };
double result {accumulateData(cbegin(values), cend(values), 1.1, multiplies<>{})};
\end{cpp}

这段代码使用非透明操作符函数对象multiplies<int>\{\},accumulateData()将结果计算为整数,结果将是6。当编译这段代码时,编译器会给出有关可能数据丢失的警告。

\begin{cpp}
vector<int> values { 1, 2, 3 };
double result {
    accumulateData(cbegin(values), cend(values), 1.1, multiplies<int>{}) };
\end{cpp}

最后,使用透明操作符可以提高性能。

\begin{myNotic}{NOTE}
建议使用透明操作符。
\end{myNotic}

\mySamllsection{比较函数对象}

除了算术函数对象类,所有标准的比较操作也作为函数对象提供:equal\_to、not\_equal\_to、less、greater、less\_equal和greater\_equal。已经在第18章中看到less作为优先队列、有序关联容器和扁平关联容器适配器的默认比较器。以下是一个使用默认比较器std::less的优先队列的例子:

\begin{cpp}
priority_queue<int> myQueue;
myQueue.push(3);
myQueue.push(4);
myQueue.push(2);
myQueue.push(1);
while (!myQueue.empty()) {
    print("{} ", myQueue.top());
    myQueue.pop();
}
\end{cpp}

程序的输出如下所示:

\begin{shell}
4 3 2 1
\end{shell}

队列中的元素按照less比较器以降序移除。可以通过指定比较器模板类型参数来改变比较器,以升序排列元素。优先队列模板定义如下:

\begin{cpp}
template <typename T, typename Container = vector<T>, typename Compare = less<T>>;
\end{cpp}

Compare类型参数是最后一个,所以要指定,还需要指定容器类型。如果想使用一个按升序排序元素的优先队列,需要将优先队列的定义更改为以下形式:

\begin{cpp}
priority_queue<int, vector<int>, greater<>> myQueue;
\end{cpp}

输出为:

\begin{shell}
1 2 3 4
\end{shell}

myQueue定义了一个透明操作符greater<>,对于接受比较器类型的标准库容器,推荐使用透明操作符。与非透明操作符相比,使用透明操作符可以提高性能。如果一个set<string>使用非透明操作符(这是默认的),对一个字符串字面量给出的键进行查询会有不必要的复制,因为必须使用字符串字面量,构造一个string实例:

\begin{cpp}
set<string> mySet;
auto i1 { mySet.find("Key") }; // string constructed, allocates memory!
//auto i2 { mySet.find("Key"sv) }; // Compilation error!
\end{cpp}

使用透明操作符时,可以避免这种复制,这称为异构查找。以下是异构查找的示例:

\begin{cpp}
set<string, less<>> mySet;
auto i1 { mySet.find("Key") }; // No string constructed, no memory allocated.
auto i2 { mySet.find("Key"sv) }; // No string constructed, no memory allocated.
\end{cpp}

C++ 23为异构删除和提取添加了支持,使用erase()和extract()。

无序关联容器,如unordered\_map和unordered\_set,也支持透明操作符。与有序关联容器相比,使用透明操作符与无序关联容器稍微复杂一些。需要实现一个自定义的哈希函数对象,其中包含一个定义为void的is\_transparent类型别名:

\begin{cpp}
class Hasher
{
    public:
        using is_transparent = void;
        size_t operator()(string_view sv) const { return hash<string_view>{}(sv); }
};
\end{cpp}

使用这个自定义的哈希函数对象时,还需要指定透明equal\_to<>函数对象作为键等价性模板类型参数的类型:

\begin{cpp}
unordered_set<string, Hasher, equal_to<>> mySet;
auto i1 { mySet.find("Key") }; // No string constructed, no memory allocated.
auto i2 { mySet.find("Key"sv) }; // No string constructed, no memory allocated.
\end{cpp}

\mySamllsection{逻辑函数对象}

对于三个逻辑操作符!、\&\&和||,C++提供了以下函数对象类:logical\_not、logical\_and和logical\_or。这些逻辑运算只处理真假值,位运算函数对象将在下一节介绍。

逻辑函数对象可以用于实现一个allTrue()函数,该函数检查容器中所有布尔标志是否为true:

\begin{cpp}
bool allTrue(const vector<bool>& flags)
{
    return accumulateData(begin(flags), end(flags), true, logical_and<>{});
}
\end{cpp}

logical\_or函数对象可以用于实现一个anyTrue()函数,该函数如果容器中至少有一个布尔标志为true,则返回true:

\begin{cpp}
bool anyTrue(const vector<bool>& flags)
{
    return accumulateData(begin(flags), end(flags), false, logical_or<>{});
}
\end{cpp}

\begin{myNotic}{NOTE}
allTrue()和anyTrue()函数只是作为示例给出的。标准库提供了std::all\_of()和any\_of()算法(见第20章),其执行相同的操作,但具有短路(见第1章)特性,因此性能更好。
\end{myNotic}

\mySamllsection{位运算函数对象}

C++有bit\_and、bit\_or、bit\_xor和bit\_not函数对象,对应于位操作符\&、|、\^{}和$\sim$。这些位运算函数对象可以与第20章介绍的transform()算法一起使用,对容器中的所有元素执行位运算。

\mySamllsection{适配器函数对象}

使用标准库提供的基本函数对象时,经常会感觉像是在尝试将方钉放入圆孔中。如果想要使用标准函数对象,但其签名与需求不完全匹配,则可以使用适配器函数对象来尝试解决签名不匹配的问题。允许适配函数对象、函数指针,基本上是任何可调用对象。适配器提供了一些对功能组合的支持,即组合函数以创建所需要的行为。

\mySamllsection{绑定器}

绑定器可以用来将可调用对象的参数,绑定到特定值。第一个绑定器是std::bind(),定义在<functional>中,允许以灵活的方式绑定可调用对象的参数。可以将参数绑定到固定值,甚至可以重新排列参数的顺序。

最好通过示例来解释。假设有一个名为func()的函数,接受两个参数:

\begin{cpp}
void func(int num, string_view str)
{
    println("func({}, {})", num, str);
}
\end{cpp}

以下代码展示了如何使用bind(),将func()的第二个参数绑定到固定值myString,结果存储在f1()中。使用auto关键字是因为bind()的返回类型在C++标准中未指定,由实现指定。未绑定到特定值的参数应指定为\_1、\_2、\_3等,这些在std::placeholders命名空间中定义。在f1()的定义中,\_1指定了f1()调用func()时第一个参数的位置。结果:f1()可以只用一个整数参数调用:

\begin{cpp}
string myString { "abc" };
auto f1 { bind(func, placeholders::_1, myString) };
f1(16);
\end{cpp}

输出如下:

\begin{shell}
func(16, abc)
\end{shell}

bind()也可以用来重新排列参数,如下所示。\_2指定了f2()调用func()时第二个参数的位置,f2()的绑定意味着f2()的第一个参数将成为func()的第二个参数,f2()的第二个参数将成为func()的第一个参数:

\begin{cpp}
auto f2 { bind(func, placeholders::_2, placeholders::_1) };
f2("Test", 32);
\end{cpp}

输出如下:

\begin{shell}
func(32, Test)
\end{shell}

<functional>定义了std::ref()和cref()辅助函数模板,这些可以用来绑定非常量引用和常量引用:

\begin{cpp}
void increment(int& value) { ++value; }
\end{cpp}

如果这样调用这个函数,index的值将变为1:

\begin{cpp}
int index { 0 };
increment(index);
\end{cpp}

如果使用bind(),因为会创建index的副本,index的值不会增加,并且对这一副本的引用会绑定到increment()函数的第一个参数上:

\begin{cpp}
auto incr { bind(increment, index) };
incr();
\end{cpp}

使用std::ref()正确传递引用,可以正确增加index:

\begin{cpp}
auto incr { bind(increment, ref(index)) };
incr();
\end{cpp}

绑定参数与重载函数结合使用时,存在一个小问题。假设有以下两个重载函数:

\begin{cpp}
void overloaded(int num) {}
void overloaded(float f) {}
\end{cpp}

如果想使用bind()与这些重载函数结合使用,需要显式指定想要绑定的哪个重载函数。以下代码将无法编译:

\begin{cpp}
auto f3 { bind(overloaded, placeholders::_1) }; // ERROR
\end{cpp}

如果想绑定接受浮点数参数的重载函数,则需要以下语法:

\begin{cpp}
auto f4 { bind((void(*)(float))overloaded, placeholders::_1) }; // OK
\end{cpp}

另一个使用 bind() 的例子是将 findMatches()(在本章前面定义)与类的成员函数一起作为 MatchHandler 使用。例如,有以下 Handler 类:

\begin{cpp}
class Handler
{
    public:
        void handleMatch(size_t position, int value1, int value2)
        {
            println("Match found at position {} ({}, {})",
                position, value1, value2);
        }
};
\end{cpp}

如何将 handleMatch() 成员函数作为最后一个参数传递给 findMatches()?这里的问题是,成员函数必须始终在对象上下文中调用。类的每个成员函数都有一个隐式的第一个参数,包含一个指向对象实例的指针,在成员函数体中可以通过名称 this 访问。因此,存在签名不匹配的问题,MatchHandler 类型只接受三个参数:一个 size\_t 和两个 int。解决方案是将这个隐式的第一个参数绑定如下:

\begin{cpp}
Handler handler;
findMatches(values1, values2, intEqual, bind(&Handler::handleMatch, &handler,
            placeholders::_1, placeholders::_2, placeholders::_3));
\end{cpp}

还可以使用 bind() 来绑定标准函数对象的参数。可以绑定 greater\_equal 的第二个参数,以便始终与固定值进行比较:

\begin{cpp}
auto greaterEqualTo100 { bind(greater_equal<>{}, placeholders::_1, 100) };;
\end{cpp}

标准库还提供了两个更多的绑定器函数对象:std::bind\_front() 和 bind\_back()。后者在 C++23 中引入,都包装了一个可调用的 f。当调用 bind\_front() 包装器时,f 的前 n 个参数绑定到一组给定的值。对于 bind\_back(),绑定f 的最后 n 个参数。这里有两个例子:

\begin{cpp}
auto f5 { bind_front(func, 42) };
f5("Hello");

auto f6 { bind_back(func, "Hello")};
f6(42);
\end{cpp}

将生成以下输出:

\begin{shell}
func(42, Hello)
func(42, Hello)
\end{shell}

\begin{myNotic}{NOTE}
C++11 之前,有 bind2nd() 和 bind1st()。自从 C++17 标准以来,两者都已移除。请使用本章后面讨论的 Lambda 表达式,或者使用 bind()、bind\_front() 或 bind\_back() 代替删除的函数。
\end{myNotic}

\mySamllsection{否定器}

not\_fn()是一个否定器,类似于一个绑定器,补充了可调用对象的返回结果。例如,使用findMatches()来查找非相等的值对,可以像这样将not\_fn()否定器适配器应用于intEqual()的结果:

\begin{cpp}
findMatches(values1, values2, not_fn(intEqual), printMatch);
\end{cpp}

not\_fn()函数对象补充了作为参数,接收每个可调用对象的调用结果。

\begin{myNotic}{NOTE}
std::not\_fn()适配器自C++17起可用。在C++17之前,可以使用std::not1()和not2()适配器。而自C++17起,not1()和not2()都被弃用,并在C++20中被移除。因此,不再进一步介绍它们,读者们也应该避免使用它们。
\end{myNotic}

\mySamllsection{调用成员函数}

可能希望将类成员函数的指针,作为回调传递给算法。有以下算法,该算法打印出符合特定条件的容器中的字符串。Matcher 模板类型参数受到 predicate<const string\&> 的约束,以确保用户提供的回调可以接受字符串参数并返回布尔值。

\begin{cpp}
template <predicate<const string&> Matcher>
void printMatchingStrings(const vector<string>& strings, Matcher matcher)
{
    for (const auto& string : strings) {
        if (matcher(string)) { print("'{}' ", string); }
    }
}
\end{cpp}

可以使用这个算法来打印所有非空字符串,通过使用 string 的 empty() 成员函数。如果只是将 string::empty() 的指针作为第二个参数传递给 printMatchingStrings(),算法无法知道接收的是成员函数指针,而非普通函数指针或函数对象。调用成员函数指针的代码与调用普通函数指针的代码不同,因为前者必须在对象上下文中调用。

C++ 提供了一个名为 mem\_fn() 的转换函数,可以将成员函数指针传递给算法之前调用它。下面的代码,将其与 not\_fn() 结合使用来反转 mem\_fn() 的结果。注意,必须将成员函数指针指定为 \&string::empty。\&string:: 部分是必填的。

\begin{cpp}
vector<string> values { "Hello", "", "", "World", "!" };
printMatchingStrings(values, not_fn(mem_fn(&string::empty)));
\end{cpp}

输出如下:

\begin{shell}
'Hello' 'World' '!'
\end{shell}

not\_fn(mem\_fn()) 生成一个函数对象,作为 printMatchingStrings() 的回调。每次调用时,都会在其参数上调用 empty() 成员函数,并反转结果。

\begin{myNotic}{NOTE}
mem\_fn() 不是实现所需行为的直观方式。我建议使用本章后面讨论的 Lambda 表达式,以更优雅、更易读的方式进行实现。
\end{myNotic}






