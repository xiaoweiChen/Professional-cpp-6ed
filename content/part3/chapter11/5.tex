
可以使用特性测试宏来检测编译器支持哪些核心语言特性,这些宏都以\_\_cpp\_或\_\_has\_cpp\_\_开头。以下是一些例子。请查询正在使用的C++编译器参考手册,以获取所有核心语言特性测试宏的完整列表。

\begin{itemize}
\item
\_\_cpp\_range\_based\_for

\item
\_\_cpp\_binary\_literals

\item
\_\_cpp\_char8\_t

\item
\_\_cpp\_generic\_lambdas

\item
\_\_cpp\_consteval

\item
\_\_cpp\_coroutines

\item
. . .

\item
\_\_has\_cpp\_attribute([attribute\_name])

\item
. . .
\end{itemize}

这些宏的值是一个数字,代表特定功能被添加或更新的月份和年份。日期格式为 YYYYMM。例如,\_\_cpp\_binary\_literals 的值是 201304,即二进制字面量是在 2013 年 4 月引入的。另一个例子,\_\_has\_cpp\_attribute(nodiscard) 的值可以是 201603,即 [[nodiscard]] 属性是在 2016 年 3 月首次引入的。或者它可以是 201907,即该属性在 2019 年 7 月更新,允许指定一个“原因”,如 [[nodiscard("Reason")]]。

所有这些核心语言功能测试宏,都可以在不包含特定头文件的情况下使用:

\begin{cpp}
int main()
{
#ifdef __cpp_range_based_for
    println("Range-based for loops are supported!");
#else
    println("Bummer! Range-based for loops are NOT supported!");
#endif
}
\end{cpp}

第 16 章介绍了标准库功能也有类似的功能测试宏。

\begin{myNotic}{NOTE}
通常不需要这些功能测试宏,除非正在编写跨平台和跨编译器的代码。这种情况下,可能想知道编译器支持是否支持某些功能,以便在缺少功能时提供备选代码。第 34 章中,介绍了跨平台开发。
\end{myNotic}



