\noindent
\textbf{WHAT’S IN THIS CHAPTER?}

\begin{itemize}
\item
What modules are and how to author and consume them

\item
Details of preprocessor directives

\item
What preprocessor macros are and why they are dangerous

\item
What linkage is and how it affects accessibility of named entities

\item
The meaning of the one definition rule, ODR

\item
Details of header files

\item
What feature-test macros for core language features are

\item
The different uses of the static keyword

\item
C-style variable-length argument lists
\end{itemize}

\noindent
\textbf{WILEY.COM DOWNLOADS FOR THIS CHAPTER}

Please note that all the code examples for this chapter are available as part of this chapter’s code download on the book’s website at \url{www.wiley.com/go/proc++6e} on the Download Code tab.

This chapter starts with a detailed discussion on how modules allow you to write reusable components and contrasts this against old-style header files. It also explains what preprocessor directives are and gives some examples of why C-style preprocessor macros are dangerous.

The chapter then explains the concept of linkage, which specifies where named entities can be accessed from, and explains the one definition rule. The final part of the chapter discusses the different uses of the static and extern keywords, as well as C-style variable-length argument lists.






























