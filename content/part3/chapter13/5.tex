目前为止,本章讨论了输入流和输出流作为两个相关的类。然而,有一种流可以执行输入和输出:双向流。

双向流从iostream派生,iostream又从istream和ostream派生,因此它是一个有用的多重继承的例子,双向流支持>{}>运算符和<{}<运算符,以及输入流和输出流的成员函数。

fstream类提供了一个双向文件流。fstream适用于需要替换文件中的数据的程序,可以在找到正确位置后立即切换到写入。例如,有一个程序,将ID号和电话号码之间的映射存储在一个数据文件中,可能使用以下格式的数据文件:

\begin{shell}
123 408-555-0394
124 415-555-3422
263 585-555-3490
100 650-555-3434
\end{shell}

一个合理的做法是在程序打开时读取整个数据文件,并在程序关闭时修改重写文件。如果数据集非常大,可能无法将其全部保存在内存中。使用iostreams,不必这样做。可以轻松地扫描文件以找到一个记录,并通过以追加模式打开文件来添加新记录。要修改现有记录,可以使用双向流,如下所示的函数将给定ID的电话号码更改为新号码:

\begin{cpp}
bool changeNumberForID(const string& filename, int id, string_view newNumber)
{
    fstream ioData { filename };
    if (!ioData) {
        println(cerr, "Error while opening file {}.", filename);
        return false;
    }

    // Loop until the end of file.
    while (ioData) {
        // Read the next ID.
        int idRead;
        ioData >> idRead;
        if (!ioData) { break; }

        // Check to see if the current record is the one being changed.
        if (idRead == id) {
            // Seek the write position to the current read position.
            ioData.seekp(ioData.tellg());
            // Output a space, then the new number.
            ioData << " " << newNumber;
            break;
        }

        // Read the current number to advance the stream.
        string number;
        ioData >> number;
    }
    return true;
}
\end{cpp}

当然,这种方法只有在数据大小固定时才能正确工作。当程序从前一个读取切换到写入时,输出的数据会覆盖文件中的其他数据。为了保持文件格式,避免覆盖下一个记录,数据必须相同。

也可以通过stringstream类,以双向方式访问字符串流。

\begin{myNotic}{NOTE}
双向流有单独的指针用于读取位置和写入位置。切换读取和写入时,需要移动到适当的位置。
\end{myNotic}















