
\CXXTwentythreeLogo{-40}{12}

C++23引入了基于span的流,定义在<spanstream>中,允许使用流的概念对可用的固定内存缓冲区进行操作,这个缓冲区是如何分配内存的并不重要。这个上下文中最重要的是ispanstream用于输入,ospanstream用于输出,以及spanstream用于输入和输出。从技术上讲,这些是basic\_ispanstream、basic\_ospanstream和basic\_spanstream类的char实例化。还有宽字符,wchar\_t实例化,称为wispanstream、wospanstream和wspanstream。宽字符在本章早些时候提到,并将在第21章中更详细地介绍。本节将给出非宽字符类的示例,其他类的工作方式非常相似。

基于span的流类的构造函数需要一个std::span。第18章详细介绍了span,并解释了为什么以及何时使用它,这些细节对于本节并不重要。基于span流的上下文中使用span非常直接,span允许对连续内存块进行查看。有点类似于std::string\_view允许你创建对任何类型的字符串的只读视图,在第2章介绍过。区别在于,span可以是一个只读视图,但也可以是一个可写的视图,允许对底层缓冲区进行修改。

以下是使用ispanstream,来解析存储在固定内存缓冲区fixedBuffer中的数据的示例。要在此缓冲区上创建一个span,只需使用span构造函数,并传递缓冲区的位置。

\begin{cpp}
char fixedBuffer[] { "11 2.222 Hello" };
ispanstream stream { span { fixedBuffer } };
int i; double d; string str;
stream >> i >> d >> str;
println("Parsed data: int: {}, double: {}, string: {}", i, d, str);
\end{cpp}

输出如下:

\begin{shell}
Parsed data: int: 11, double: 2.222, string: Hello
\end{shell}

使用ospanstream也很简单。以下代码创建了一个32个字符的固定缓冲区,构造了一个可写的ospanstream视图,使用标准流插入操作将一些数据输出到缓冲区,最后打印出结果:

\begin{cpp}
char fixedBuffer[32] {};
ospanstream stream { span { fixedBuffer } };
stream << "Hello " << 2.222 << ' ' << 11;
println("Buffer contents: \"{}\"", fixedBuffer);
\end{cpp}

输出如下:

\begin{shell}
Buffer contents: "Hello 2.222 11"
\end{shell}






