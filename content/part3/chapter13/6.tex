
C++标准库包括一个文件系统库,定义在<filesystem>中,位于std::filesystem命名空间中,允许编写可移植的代码来操作文件系统。可以用它来查询某物是否是目录或文件,遍历目录的内容,操作路径,以及获取文件信息,如大小、扩展名、创建时间等。库的两大重要部分——路径和目录条目——将在下一节中介绍。

\mySubsubsection{13.6.1.}{路径}

库的基本组件是路径。路径可以是绝对的或相对的,可以包含可选的文件名。以下代码定义了几个路径。注意,使用原字符串字面量,这在第2章中引入,以避免转义反斜杠:

\begin{cpp}
path p1 { R"(D:\Foo\Bar)" };
path p2 { "D:/Foo/Bar" };
path p3 { "D:/Foo/Bar/MyFile.txt" };
path p4 { R"(..\SomeFolder)" };
path p5 { "/usr/lib/X11" };
\end{cpp}

通过调用string()可以将路径转换为运行代码的系统本机格式。这是一个例子:

\begin{cpp}
println("{}", p1.string());
println("{}", p2.string());
\end{cpp}

在Windows上,支持前后斜杠,输出如下:

\begin{shell}
D:\Foo\Bar
D:/Foo/Bar
\end{shell}

可以使用append()成员函数或operator/=向路径追加一个组件,自动插入平台相关的路径分隔符:

\begin{cpp}
path p { "D:\\Foo" };
p.append("Bar");
p /= "Bar";
println("{}", p.string());
\end{cpp}

Windows上的输出为D:\verb|\|Foo\verb|\|Bar\verb|\|Bar。

使用concat()或operator+=将字符串拼接到现有路径,这不会插入任何路径分隔符!

\begin{cpp}
path p { "D:\\Foo" };
p.concat("Bar");
p += "Bar";
println("{}", p.string());
\end{cpp}

Windows上的输出为D:\verb|\|FooBarBar。

\begin{myWarning}{WARNING}
append()和operator/=会自动插入平台相关的路径分隔符,而concat()和operator+=不会。
\end{myWarning}

可以使用基于范围的for循环来遍历路径的不同组件:

\begin{cpp}
path p { R"(C:\Foo\Bar)" };
for (const auto& component : p) {
    println("{}", component.string());
}
\end{cpp}

Windows上的输出如下:

\begin{cpp}
C:
\
Foo
Bar
\end{cpp}

路径接口支持如remove\_filename()、replace\_filename()、replace\_extension()、root\_name()、parent\_path()、extension()、stem()、filename()、has\_extension()、is\_absolute()、is\_relative()等操作。以下代码片段演示了其中一些:

\begin{cpp}
path p { R"(C:\Foo\Bar\file.txt)" };
println("{}", p.root_name().string());
println("{}", p.filename().string());
println("{}", p.stem().string());
println("{}", p.extension().string());
\end{cpp}

Windows上,这段代码产生的结果如下所示:

\begin{shell}
C:
file.txt
file
.txt
\end{shell}

有关所有可用功能的完整列表,请参考标准库参考手册。

\mySubsubsection{13.6.2.}{目录}

路径仅代表文件系统上的目录或文件,路径可能指向不存在的目录或文件。如果想查询实际的目录或文件,需要从路径构造一个directory\_entry。directory\_entry接口支持,诸如exists()、is\_directory()、is\_regular\_file()、file\_size()、last\_write\_time()等操作。

以下示例从一个路径构建directory\_entry来查询文件的大小:

\begin{cpp}
path myPath { "c:/windows/win.ini" };
directory_entry dirEntry { myPath };
if (dirEntry.exists() && dirEntry.is_regular_file()) {
    println("File size: {}", dirEntry.file_size());
}
\end{cpp}

\mySubsubsection{13.6.3.}{辅助函数}

这里也有一整套辅助函数可供使用,可以使用copy()复制文件或目录,create\_directory()在文件系统上创建新的目录,exists()查询给定的目录或文件是否存在,file\_size()获取文件的大小,last\_write\_time()获取文件最后修改的时间,remove()删除文件,temp\_directory\_path()获取适合存储临时文件的目录,space()查询文件系统的可用空间等。请参阅标准库参考(详见附录B)以获取完整列表。

以下示例打印出文件系统的容量和剩余空间:

\begin{cpp}
space_info s { space("c:\\") };
println("Capacity: {}", s.capacity);
println("Free: {}", s.free);
\end{cpp}

可以在以下关于目录迭代的章节中,找到这些辅助函数的更多示例。

\mySubsubsection{13.6.4.}{目录迭代}

如果想递归地遍历给定目录中的所有文件和子目录,可以使用recursive\_directory\_iterator。为了开启迭代过程,需要一个指向directory\_entry的迭代器。为了知道何时停止迭代,还需要一个结束迭代器。要创建开始迭代器,构造一个recursive\_directory\_iterator并传递要迭代的目录路径。要构造结束迭代器,使用默认构造函数创建一个recursive\_directory\_iterator。要获取迭代器所指的directory\_entry的访问权限,使用解引用操作符*。通过简单地使用自加操作符递增迭代器,直到达到结束迭代器,就可以遍历集合中所有元素。结束迭代器不再属于集合,不再指向有效的directory\_entry,并且不能解引用。

\begin{cpp}
void printDirectoryStructure(const path& p)
{
    if (!exists(p)) { return; }

    recursive_directory_iterator begin { p };
    recursive_directory_iterator end { };
    for (auto iter { begin }; iter != end; ++iter) {

        const string spacer(iter.depth() * 2, ' ');

        auto& entry { *iter }; // Dereference iter to access directory_entry.

        if (is_regular_file(entry)) {
            println("{}File: {} ({} bytes)",
            spacer, entry.path().string(), file_size(entry));
        } else if (is_directory(entry)) {
            println("{}Dir: {}", spacer, entry.path().string());
        }
    }
}
\end{cpp}

此函数可以按以下方式调用:

\begin{cpp}
path p { R"(D:\Foo\Bar)" };
printDirectoryStructure(p);
\end{cpp}

也可以使用directory\_iterator迭代目录的内容,并自行实现递归。以下示例与前一个示例相同,这里使用了directory\_iterator,而非recursive\_directory\_iterator:

\begin{cpp}
void printDirectoryStructure(const path& p, unsigned level = 0)
{
    if (!exists(p)) { return; }

    const string spacer(level * 2, ' ');

    if (is_regular_file(p)) {
        println("{}File: {} ({} bytes)", spacer, p.string(), file_size(p));
    } else if (is_directory(p)) {
        println("{}Dir: {}", spacer, p.string());
        for (auto& entry : directory_iterator { p }) {
            printDirectoryStructure(entry, level + 1);
        }
    }
}
\end{cpp}









