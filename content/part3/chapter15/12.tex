通过解决下面的练习,可以练习本章讨论的内容。所有练习的解决方案都可以在本书的网站\url{www.wiley.com/go/proc++6e}下载到源码。若在练习中卡住了,可以考虑先重读本章的部分内容,试着自己找到答案,再在从网站上寻找解决方案。

\begin{itemize}
\item
\textbf{练习 15-1:} 实现一个 AssociativeArray 类模板。这个类应该在向量中存储一些元素,每个元素由一个键和一个值组成。键总是字符串,而值的类型可以使用模板类型参数指定。提供重载的下标操作符,以便可以根据键检索元素。在 main() 函数中测试实现。注意:这里只是为了练习使用非整数索引实现下标操作符。实际上,应该使用标准库提供的 std::map 类模板,这在第18章中有讨论。

\item
\textbf{练习 15-2}: 使用在第13章练习2中实现的 Person 类,为其添加插入和提取运算符的实现。确保提取运算符,可以读取插入运算符写出的内容。

\item
\textbf{练习 15-3}: 在第15-2练习的解决方案中添加一个字符串转换运算符,该运算符简单地返回一个由人的名字和姓氏组成的字符串。

\item
\textbf{练习 15-4:} 基于第15-3练习的解决方案,添加一个用户定义的字面量运算符 \_p,它从一个字符串字面量构造一个 Person 对象。支持姓氏中的空格,但不支持名字中的空格。例如,"Peter Van Weert"\_p 应该是一个 Person 对象,其名字为 Peter,姓氏为 Van Weert。
\end{itemize}












