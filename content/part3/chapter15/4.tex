C++中不仅用操作符进行算术运算,还用于从和向流中读写。例如,向 cout 写入整数和字符串时,使用插入操作符 <{}<:

\begin{cpp}
int number { 10 };
cout << "The number is " << number << endl;
\end{cpp}

当从流中读取内容时,可以使用提取操作符 >{}>:

\begin{cpp}
int number;
string str;
cin >> number >> str;
\end{cpp}

也可以为类写插入和提取操作符,可以对其进行读写:

\begin{cpp}
SpreadsheetCell myCell, anotherCell, aThirdCell;
cin >> myCell >> anotherCell >> aThirdCell;
cout << myCell << " " << anotherCell << " " << aThirdCell << endl;
\end{cpp}

写插入和提取操作符之前,需要决定如何输出类,以及如何读取。这个例子中,SpreadsheetCell 简单地读取和写入一个双精度浮点数值。

提取或插入操作符左边的对象是一个 istream 或 ostream(如 cin 或 cout),而不是 SpreadsheetCell 对象。不能向 istream 或 ostream 类添加成员函数,所以必须将提取和插入操作符写为全局函数。这些函数的声明如下:

\begin{cpp}
export std::ostream& operator<<(std::ostream& ostr, const SpreadsheetCell& cell);
export std::istream& operator>>(std::istream& istr, SpreadsheetCell& cell);
\end{cpp}

通过使插入操作符以 ostream 类型的引用作为其第一个参数,可以让它用于文件输出流、字符串输出流、cout、cerr、clog 等。关于流的详细信息,请参见第13章。通过使提取操作符以 istream 类型的引用作为其第一个参数,可以使其适用于任何输入流,如文件输入流、字符串输入流和 cin。

当想要读取或写入时,提取操作符的第二个参数是对 SpreadsheetCell 对象的引用。插入操作符不会改变它写入的 SpreadsheetCell,所以参数类型是引用到常量。提取操作符,但会修改 SpreadsheetCell 对象,需要参数是非常量的引用。

这两个操作符都返回作为第一个参数给出的流的引用,以便可以嵌套调用。操作符语法是调用全局 operator>{}> 或 operator<{}< 函数的简写。来看看这行代码:

\begin{cpp}
cin >> myCell >> anotherCell >> aThirdCell;
\end{cpp}

可简写为:

\begin{cpp}
operator>>(operator>>(operator>>(cin, myCell), anotherCell), aThirdCell);
\end{cpp}

第一个调用 operator>{}> 的返回值可用作下一个调用的输入。必须返回流引用,以便可以在下一次嵌套调用中使用。否则,嵌套编译不会通过。

以下是 SpreadsheetCell 类的 operator<{}< 和 operator>{}> 实现:

\begin{cpp}
ostream& operator<<(ostream& ostr, const SpreadsheetCell& cell)
{
    ostr << cell.getValue();
    return ostr;
}

istream& operator>>(istream& istr, SpreadsheetCell& cell)
{
    double value;
    istr >> value;
    cell.set(value);
    return istr;
}
\end{cpp}










