
可以重载三个解引用操作符:*、-> 和 ->*。先忽略 ->*(稍后再回来讨论),先考虑 * 和 -> 的内置含义。* 操作符解引用指针以直接访问其值,而 -> 是 * 解引用后紧跟一个 . 成员选择的缩写。以下代码展示了等价性:

\begin{cpp}
SpreadsheetCell* cell { new SpreadsheetCell };
(*cell).set(5); // Dereference plus member selection.
cell->set(5); // Shorthand arrow dereference and member selection together.
\end{cpp}

可以为类重载解引用操作符,使对象的行为像指针一样。这种能力的主要用途是实现智能指针。也用于迭代器,标准库广泛使用。迭代器在第17章中介绍。本章将介绍如何在简单智能指针类模板的上下文中,重载相关操作符的基本机制。

\begin{myWarning}{WARNING}
C++ 有两个标准智能指针:std::unique\_ptr 和 std::shared\_ptr。应该使用这些标准智能指针,而不是自己创建。这里的例子只是为了展示如何编写解引用操作符。
\end{myWarning}

这里是智能指针类模板定义的一个例子,尚未填写解引用操作符:

\begin{cpp}
export template <typename T> class Pointer
{
    public:
        explicit Pointer(T* ptr) : m_ptr { ptr } {}
        virtual ~Pointer() { reset(); }
        // Disallow copy constructor and copy assignment.
        Pointer(const Pointer& src) = delete;
        Pointer& operator=(const Pointer& rhs) = delete;
        // Allow move construction.
        Pointer(Pointer&& src) noexcept : m_ptr{ std::exchange(src.m_ptr, nullptr)}
        { }
        // Allow move assignment.
        Pointer& operator=(Pointer&& rhs) noexcept
        {
            if (this != &rhs) {
                reset();
                m_ptr = std::exchange(rhs.m_ptr, nullptr);
            }
            return *this;
        }

        // Dereferencing operators will go here...
    private:
        void reset()
        {
            delete m_ptr;
            m_ptr = nullptr;
        }
        T* m_ptr { nullptr };
};
\end{cpp}

这个智能指针应尽可能简单,所做的只是存储一个原始的裸指针,当智能指针销毁时,所指向的存储也应删除。实现同样简单:构造函数接受一个裸指针,作为类中唯一的数据成员存储。析构函数释放指针所引用的存储。

能够像这样使用智能指针类模板:

\begin{cpp}
Pointer<int> smartInt { new int };
*smartInt = 5; // Dereference the smart pointer.
println("{} ", *smartInt);

Pointer<SpreadsheetCell> smartCell { new SpreadsheetCell };
smartCell->set(5); // Dereference and member select the set() member function.
println("{} ", smartCell->getValue());
\end{cpp}

这个例子中,可以了解到需要为这个类提供 operator* 和 operator-> 的实现。这些将在接下来的两个部分中实现。

\begin{myWarning}{WARNING}
很少(如果不是从不)只重载 operator* 或 operator-> 的实现,两个操作符应该一起重载。如果没有提供这两个操作符,会使用户对这个类感到困惑。
\end{myWarning}

\mySubsubsection{15.7.1.}{实现 operator*}

当解引用一个指针时,期望能够访问指针所指向的内存。如果那个内存包含一个简单的类型,如 int,就能够直接改变它的值。如果内存包含一个更复杂的类型,如一个对象,应该能够使用点操作符 . 来访问它的数据成员或成员函数。

为了提供这些语义,应该在 operator* 中返回一个引用。对于 Pointer 类,实现是这样的:

\begin{cpp}
export template <typename T> class Pointer
{
    public:
        // Omitted for brevity
        T& operator*() { return *m_ptr; }
        const T& operator*() const { return *m_ptr; }
        // Omitted for brevity
};
\end{cpp}

operator* 返回底层原始指针所指向的对象或变量的引用。就像重载下标操作符一样,提供两个版本的成员函数(const 和非 const)是有用的,分别返回引用到常量和引用到非常量。

\mySubsubsection{15.7.2.}{实现 operator–>}

箭头操作符稍微复杂一些,箭头操作符的结果应该是一个对象的成员或成员函数。这需要实现 operator* 后的 operator; 的等效操作。C++ 不允许重载 operator. 是有原因的:不可能写出一个单一的原型来捕获可能的成员或成员函数。因此,C++ 将 operator-> 视为特殊情况。

看看这行代码:

\begin{cpp}
smartCell->set(5);
\end{cpp}

C++ 将这行翻译为:

\begin{cpp}
(smartCell.operator->())->set(5);
\end{cpp}

C++ 对从重载的 operator-> 返回的东西应用另一个 operator->。必须要返回一个指针:

\begin{cpp}
export template <typename T> class Pointer
{
    public:
        // Omitted for brevity
        T* operator->() { return m_ptr; }
        const T* operator->() const { return m_ptr; }
        // Omitted for brevity
};
\end{cpp}

聪明的读者应该发现了 operator* 和 operator-> 的不对称,但看多了就会习惯。

\mySubsubsection{15.7.3.}{什么是 operator.* 和 operator->*?}

C++ 中获取类数据成员和成员函数的地址,以获取指针完全合法。没有对象就无法访问非静态数据成员或调用非静态成员函数,类数据成员和成员函数的目的是以每个对象为基础的存在,所以想通过指针调用成员函数或访问数据成员时,必须在该对象的上下文中解引用指针。使用 operator.* 和 ->* 的语法细节将在第19章中介绍(需要知道如何定义函数指针)。

C++ 不允许重载 operator.*(就像不能重载 operator.),但可以重载 operator->*。然而,这很复杂,考虑到大多数 C++ 开发者可能都不知道,可以通过指针访问成员函数和数据成员,这样的重载可能得不偿失。例如,标准库中的 shared\_ptr 智能指针就没有重载 operator->*。





















