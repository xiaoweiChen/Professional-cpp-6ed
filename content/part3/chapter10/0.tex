\noindent
\textbf{WHAT’S IN THIS CHAPTER?}

\begin{itemize}
\item
How to extend a class through inheritance

\item
How to employ inheritance to reuse code

\item
How to build interactions between base classes and derived classes

\item
How to use inheritance to achieve polymorphism

\item
How to work with multiple inheritance

\item
How to deal with unusual problems in inheritance

\item
How to cast one type to another type
\end{itemize}

\noindent
\textbf{WILEY.COM DOWNLOADS FOR THIS CHAPTER}

Please note that all the code examples for this chapter are available as part of this chapter’s code download on the book’s website at \url{www.wiley.com/go/proc++6e} on the Download Code tab.

Without inheritance, classes would simply be data structures with associated behaviors. That alone would be a powerful improvement over procedural languages, but inheritance adds an entirely new dimension. Through inheritance, you can build new classes based on existing ones. In this way, your classes become reusable and extensible components. This chapter teaches you the different ways to leverage the power of inheritance. You will learn about the specific syntax of inheritance as well as sophisticated techniques for making the most of inheritance.

The portion of this chapter relating to polymorphism draws heavily on the spreadsheet example discussed in Chapter 8, “Gaining Proficiency with Classes and Objects,” and Chapter 9, “Mastering Classes and Objects.” This chapter also refers to the object-oriented methodologies described in Chapter 5, “Designing with Classes.” If you have not read that chapter and are unfamiliar with the theories behind inheritance, you should review Chapter 5 before continuing.






























