By solving the following exercises, you can practice the material discussed in this chapter. Solutions to all exercises are available with the code download on the book’s website at www.wiley.com/go/ proc++6e. However, if you are stuck on an exercise, first reread parts of this chapter to try to find an answer yourself before looking at the solution from the website.

\begin{itemize}
\item
Exercise 10-1: Take the Person class from Exercise 9-2 and add a derived class called Employee. You can omit the overload of operator<=> from Exercise 9-2. The Employee class adds one data member, an employee ID. Provide an appropriate constructor. From Employee, derive two more classes called Manager and Director.

Put all your classes, including the Person class, in a namespace called HR. Note that you can export everything in a namespace from a module as follows:

\begin{cpp}
export namespace HR { /* ... */ }
\end{cpp}

\item
Exercise 10-2: Continuing with your solution from Exercise 10-1, add a toString() member function to the Person class returning a string representation of a person. Override this member function in the Employee, Manager, and Director classes to build up a complete string representation by delegating part of their work to parent classes.

\item
Exercise 10-3: Practice polymorphic behavior of the classes in your Person hierarchy from Exercise 10-2. Define a vector to store a mix of persons, employees, managers, and directors, and fill it with some test data. Finally, use a single range-based for loop to call toString() on all of the elements in the vector.

\item
Exercise 10-4: In real companies, employees can get promoted to manager or director positions, and managers can get promoted to director. Do you see a way you can add support for this to your class hierarchy of Exercise 10-3?
\end{itemize}

