
标准库提供了四种流迭代器。这些类似于迭代器的类模板,可以将输入和输出流视为输入和输出迭代器。使用这些流迭代器,可以适配输入和输出流,能够分别作为各种标准库算法的源和目标。以下是可以使用的流迭代器:

\begin{itemize}
\item
ostream\_iterator: 写入 basic\_ostream 的输出迭代器

\item
istream\_iterator: 从 basic\_istream 读取的输入迭代器

\item
ostreambuf\_iterator: 写入 basic\_streambuf 的输出迭代器

\item
istreambuf\_iterator:  从 basic\_streambuf 读取的输入迭代器
\end{itemize}

\mySubsubsection{17.2.1.}{输出流迭代器: ostream\_iterator}

ostream\_iterator 是一个输出流迭代器,是一个类模板,接受元素类型作为模板类型参数。构造函数接受一个输出流和一个分隔字符串,用于在每个元素之后写入流。ostream\_iterator 类使用 operator<{}< 写入元素。

来看一个例子,有以下的 myCopy() 函数模板,将给定开始和结束迭代器的范围复制到一个给定开始迭代器的目标范围。第二个模板类型参数限制为 std::iter\_reference\_t<InputIter> 类型的值的输出迭代器,这是给定 InputIter 指向的值的类型。

\begin{cpp}
template <input_iterator InputIter,
    output_iterator<iter_reference_t<InputIter>> OutputIter>
void myCopy(InputIter begin, InputIter end, OutputIter target)
{
    for (auto iter { begin }; iter != end; ++iter, ++target) { *target = *iter; }
}
\end{cpp}

myCopy() 的前两个参数是要复制的范围的开始和结束迭代器,第三个参数是目标范围的迭代器。必须确保目标范围足够大,能够容纳来自源范围的所有元素。使用 myCopy() 函数模板,会直接将一个vector的元素复制到另一个vector。

\begin{cpp}
vector myVector { 1, 2, 3, 4, 5, 6, 7, 8, 9, 10 };
// Use myCopy() to copy myVector to vectorCopy.
vector<int> vectorCopy(myVector.size());
myCopy(cbegin(myVector), cend(myVector), begin(vectorCopy));
\end{cpp}

通过使用一个 ostream\_iterator,myCopy() 函数模板也可以用来仅用一行代码打印容器元素。以下代码段打印 myVector 和 vectorCopy 的内容:

\begin{cpp}
// Use the same myCopy() to print the contents of both vectors.
myCopy(cbegin(myVector), cend(myVector), ostream_iterator<int> { cout, " " });
println("");
myCopy(cbegin(vectorCopy), cend(vectorCopy), ostream_iterator<int> { cout, " " });
println("");
\end{cpp}

输出如下:

\begin{shell}
1 2 3 4 5 6 7 8 9 10
1 2 3 4 5 6 7 8 9 10
\end{shell}

\mySubsubsection{17.2.2.}{输入流迭代器: istream\_iterator}

可以使用输入流迭代器 istream\_iterator 从输入流中读取值,利用迭代器抽象。它是一个类模板,接受元素类型作为模板类型参数,其构造函数接受一个输入流作为参数。元素通过 operator>{}> 读取,可以将 istream\_iterator 用作算法和容器成员函数的源。

假设有一个 sum() 函数模板,用于计算给定普通范围内所有元素的和:

\begin{cpp}
template <input_iterator InputIter>
auto sum(InputIter begin, InputIter end)
{
    auto sum { *begin };
    for (auto iter { ++begin }; iter != end; ++iter) { sum += *iter; }
    return sum;
}
\end{cpp}

可以使用 istream\_iterator 从控制台读取整数,直到到达流的末尾。在 Windows 上,这发生在按下 Ctrl+Z 后跟 Enter 的情况下,而在 Linux 上则是按下 Enter 后跟 Ctrl+D。sum() 函数用于计算所有整数的和,默认构造的 istream\_iterator 表示结束迭代器。

\begin{cpp}
println("Enter numbers separated by whitespace.");
println("Press Ctrl+Z followed by Enter to stop.");
istream_iterator<int> numbersIter { cin };
istream_iterator<int> endIter;
int result { sum(numbersIter, endIter) };
println("Sum: {}", result);
\end{cpp}


\mySubsubsection{17.2.3.}{输入流迭代器: istreambuf\_iterator}

istreambuf\_iterator 输入流迭代器是用单个语句,读取整个文件的内容。默认构造的 istreambuf\_iterator 表示结束迭代器:

\begin{cpp}
ifstream inputFile { "some_data.txt" };
if (inputFile.fail()) {
    println(cerr, "Unable to open file for reading.");
    return 1;
}
string fileContents {
    istreambuf_iterator<char> { inputFile },
    istreambuf_iterator<char> { }
};
println("{}", fileContents);
\end{cpp}














