This chapter is a continuation of the template discussion from Chapter 12. These chapters show you how to use templates for generic programming and template metaprogramming for compile-time computations. Ideally you have gained an appreciation for the power and capabilities of these features and an idea of how you can apply these techniques to your own code. Don’t worry if you didn’t understand all the syntax, or didn’t follow all the examples, on your first reading. The techniques can be difficult to grasp when you are first exposed to them, and the syntax is tricky whenever you want to write more complicated templates. When you actually sit down to write a class or function template, you can consult this chapter and Chapter 12 for a reference on the proper syntax.