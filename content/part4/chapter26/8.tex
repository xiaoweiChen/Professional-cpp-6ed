通过解决下面的练习,可以练习本章讨论的内容。所有练习的解决方案都可以在本书的网站\url{www.wiley.com/go/proc++6e}下载到源码。然而,若在练习中卡住了,在从网站上寻找解决方案之前,可以考虑先重读本章的部分内容,试着自己找到答案。

\begin{itemize}
\item
Exercise 26-1: In Exercise 12-2, you wrote a full specialization of a KeyValuePair class template for const char* keys and values. Replace that full specialization with a partial specialization where the values are of type const char* but the keys can be of any type.

\item
Exercise 26-2: Calculate the nth number in the Fibonacci series at compile time using template recursion. The Fibonacci series starts with 0 and 1, and any subsequent value is the sum of the two previous values, so: 0, 1, 1, 2, 3, 5, 8, 13, 21, 34, 55, 89, and so on.

Can you also provide a variable template to make your recursive Fibonacci template easier to use?

\item
Exercise 26-3: Take your solution for Exercise 26-2 and modify it so that the calculation still happens at compile time, but without the use of any template or function recursion.

\item
Exercise 26-4: Write a variadic function template called push\_back\_values() accepting a reference to a vector and a variable number of values. The function should use a fold expression to push all the values into the given vector. Then, write an insert\_values() function template doing the same thing but in terms of vector::insert(initializer\_list<value\_type>). What’s the difference with the push\_back\_values() implementation?

\item
Exercise 26-5: Write a multiply() non-abbreviated function template accepting two template type parameters T1 and T2. Use a type trait to verify that both types are arithmetic. If they are, perform the multiplication and return the result. If they are not, throw an exception containing the names of both types.

\item
Exercise 26-6: Advanced. Transform your solution for Exercise 26-5 to use an abbreviated function template.
\end{itemize}




















