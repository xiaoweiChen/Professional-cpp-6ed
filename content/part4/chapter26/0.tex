\noindent
\textbf{WHAT’S IN THIS CHAPTER?}

\begin{itemize}
\item
The different kinds of template parameters

\item
How to use partial specialization

\item
How to write recursive templates

\item
How to write variadic templates

\item
How to write type-safe variable argument functions using variadic templates

\item
What constexpr if statements are

\item
What fold-expressions are and how to use them

\item
What metaprogramming is and how to use it

\item
What type traits are and what they can be used for

\item
What SFINAE means
\end{itemize}

\noindent
\textbf{WILEY.COM DOWNLOADS FOR THIS CHAPTER}

Please note that all the code examples for this chapter are available as part of this chapter’s code download on the book’s website at \url{www.wiley.com/go/proc++6e} on the Download Code tab.

Chapter 12, “Writing Generic Code with Templates,” covers the most widely used features of class and function templates. If you are interested in only a basic knowledge of templates so that you can better understand how the Standard Library works, or perhaps write your own simple class and function templates, you can skip this chapter on advanced templates. However, if templates interest you and you want to uncover their full power, continue reading this chapter to learn some obscure, but fascinating, details.


















