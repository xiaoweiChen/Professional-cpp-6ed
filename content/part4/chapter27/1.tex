

有三种主要执行多个任务的方式:

\begin{itemize}
\item
顺序执行:每个任务依次执行。

\item
并发执行:多个任务看似同时执行,但实际上是因为操作系统为每个任务分配了时间片来执行工作,然后给另一个任务分配时间片执行,如此循环直到所有任务完成。这种任务切换持续进行,直到任务完成。

\item
并行执行:多个任务真正同时执行,例如在多个处理器单元上。
\end{itemize}

多线程编程允许你并发执行多个任务(可能甚至是并行执行),可以利用当今几乎所有系统中的多个处理器单元。二十年前,处理器市场在追求最高频率,这对于单线程应用程序来说是完美的。大约在2005年,这场竞赛停止了,原因是一系列电源和热量管理问题。从那时起,处理器市场开始向单个处理器芯片上最多的内核发展。四核和八核处理器很常见,也有高达128核甚至更多核的处理器。

同样,如果了解图形卡上的处理器,也就是图形处理单元(GPU),会发现它们是大规模并行处理器。今天,高端图形卡有超过16,000个核心,这个数字正在迅速增长!这些图形卡不仅用于游戏,还用于执行计算密集型任务,如人工智能、机器学习、图像和视频处理、蛋白质折叠(有助于发现新药物)、作为搜寻地外智能(SETI)项目的一部分处理信号等。

C++98/03没有支持多线程编程,必须求助于第三方库或目标操作系统的多线程API。自从C++11引入了标准多线程库后,编写跨平台多线程应用程序变得更加容易。然而,当前的C++标准只针对CPU,而不是GPU,这在未来可能会发生变化。

开始编写多线程代码有两个原因。首先,有一个计算问题,并且能够将其分割成可以独立并行运行的小块,当在多个处理器单元上运行时,可以期望获得巨大的性能提升。其次,可以沿着正交轴对计算进行模块化。例如,可以在一个工作线程中执行长时间计算,而不是阻塞UI线程,这样在后台进行长时间计算时,用户界面仍可保持响应。

图 27.1 展示了一个非常适合并行运行的情况,通过一个不需要了解相邻像素信息的算法来处理图像的像素。该算法可以将图像分割成四个部分。在单核处理器上,每个部分将按顺序处理;在双核处理器上,两个部分将并行处理;在四核处理器上,四个部分将并行处理,从而使得性能几乎与核心数成线性增长。

\myGraphic{0.9}{content/part4/chapter27/images/1.png}{图 27.1}

当然,并不总是可能将问题分割成可以独立并行执行的部分,但通常可以部分地并行化,从而提高性能。多线程编程的一个困难部分是使算法并行化,这高度依赖于算法的类型。其他困难包括竞争条件、死锁、撕裂和伪共享。这些将在接下来的章节中讨论。代码线程安全的选项包括:

\begin{itemize}
\item
不可变的数据:常量数据本质上可以由多个线程安全访问。

\item
原子操作:自动提供线程安全操作的低层类型。

\item
互斥和其他同步机制:这些用于协调多个线程对共享数据的访问。

\item
线程局部存储:标记为 thread\_local 的变量是线程局部变量;其他线程没有访问它们的权限(至少默认情况下没有),因此线程安全。
\end{itemize}

本章将涉及这些内容。

\begin{myNotic}{NOTE}
为了避免多线程问题,需要设计程序,使多个线程不需要读写共享内存。或者,使用同步机制(“互斥”)或原子操作(“原子操作库”)。
\end{myNotic}

\mySubsubsection{27.1.1.}{竞争条件}

当多个线程想要访问任何类型的共享资源时,可能会发生竞争条件。多个线程共享内存的竞争条件称为数据竞争。当多个线程访问同一个变量,并且至少有一个线程对其进行写操作时,可能会发生数据竞争。

例如,有一个共享变量,一个线程在另一个线程对其进行递减操作的同时对其进行递增操作。递增和递减值意味着当前值需要从内存中检索,然后递增或递减,并存储回内存。大多数处理器都有 INC 和 DEC 指令来执行这些操作。在现代 x86 处理器上,这些指令不是原子性的,所以其他指令可以在操作中间执行,这可能导致代码检索错误的值。

下面的表格显示了当递增操作完成在递减操作开始之前,并且假设初始值为 1 时的结果:

% Please add the following required packages to your document preamble:
% \usepackage{longtable}
% Note: It may be necessary to compile the document several times to get a multi-page table to line up properly
\begin{longtable}{|l|l|}
\hline
\textbf{线程 1 (增加)} & \textbf{线程 2 (减少)} \\ \hline
\endfirsthead
%
\endhead
%
load value (value = 1)        &                               \\ \hline
increment value (value = 2)   &                               \\ \hline
store value (value = 2)       &                               \\ \hline
& load value (value = 2)        \\ \hline
& decrement value (value = 1)   \\ \hline
& store value (value = 1)       \\ \hline
\end{longtable}

内存中存储的最终值是 1。当递减线程在递增线程开始之前完成时,最终值也是 1,如下表所示:

% Please add the following required packages to your document preamble:
% \usepackage{longtable}
% Note: It may be necessary to compile the document several times to get a multi-page table to line up properly
\begin{longtable}{|l|l|}
\hline
\textbf{线程 1 (增加)} & \textbf{线程 2 (减少)} \\ \hline
\endfirsthead
%
\endhead
%
& load value (value = 1)        \\ \hline
& decrement value (value = 0)   \\ \hline
& store value (value = 0)       \\ \hline
load value (value = 0)        &                               \\ \hline
increment value (value = 1)   &                               \\ \hline
store value (value = 1)       &                               \\ \hline
\end{longtable}

然而,当指令交错执行时,结果会有所不同,如下表所示:

% Please add the following required packages to your document preamble:
% \usepackage{longtable}
% Note: It may be necessary to compile the document several times to get a multi-page table to line up properly
\begin{longtable}{|l|l|}
\hline
\textbf{线程 1 (增加)} & \textbf{线程 2 (减少)} \\ \hline
\endfirsthead
%
\endhead
%
oad value (value = 1)         &                               \\ \hline
increment value (value = 2)   &                               \\ \hline
& load value (value = 1)        \\ \hline
& decrement value (value = 0)   \\ \hline
store value (value = 2)       &                               \\ \hline
& store value (value = 0)       \\ \hline
\end{longtable}

这种情况下,最终结果是 0。换句话说,递增操作的效果丢失了。这就是数据竞争。

\mySubsubsection{27.1.2.}{撕裂}

撕裂是数据竞争的一种特定情况或后果。撕裂有两种类型:撕裂读和撕裂写。如果一个线程将数据的一部分写入内存,而另一部分尚未写入,那么在那一刻读取该数据的任何其他线程都会看到不一致的数据:一个撕裂读。如果两个线程同时写入数据,一个线程可能已经写入了数据的一部分,而另一个线程可能已经写入了数据的其他部分。最终结果将是不一致的:一个撕裂写。

\mySubsubsection{27.1.3.}{死锁}

如果选择通过使用同步机制(如互斥)来解决竞争条件,可能会遇到多线程编程的另一个常见问题:死锁。两个线程如果在等待对方执行某些操作,则会进入死锁状态。这可以扩展到超过两个线程。例如,两个线程都想获取访问共享资源的权限,需要请求访问该资源的权限。如果其中一个线程当前持有访问该资源的权限,但由于其他原因而永久阻塞,那么另一个线程在尝试获取同一资源的权限时,也会永久阻塞。

获取共享资源权限的一种机制称为互斥对象,或简称为 mutex,在本章稍后详细讨论。例如,有两个线程和两个资源,每个资源都由一个 mutex 保护,分别是 A 和 B。两个线程都获取了两个资源的权限,但获取权限的顺序不同。以下表格展示了这种情况的伪代码:

% Please add the following required packages to your document preamble:
% \usepackage{longtable}
% Note: It may be necessary to compile the document several times to get a multi-page table to line up properly
\begin{longtable}{|l|l|}
\hline
\textbf{线程 1} &
\textbf{线程 2} \\ \hline
\endfirsthead
%
\endhead
%
\begin{tabular}[c]{@{}l@{}}Acquire A\\ Acquire B\\ // . . . compute\\ Release B\\ Release A\end{tabular} &
\begin{tabular}[c]{@{}l@{}}Acquire B\\ Acquire A\\ // . . . compute\\ Release A\\ Release B\end{tabular} \\ \hline
\end{longtable}

现在,想象一下这两个线程的代码以以下顺序执行:

\begin{itemize}
\item
线程 1:获取 A(成功)

\item
线程 2:获取 B(成功)

\item
线程 1:获取 B(等待/阻塞,因为线程 2 持有 B)

\item
线程 2:获取 A(等待/阻塞,因为线程 1 持有 A)
\end{itemize}

两个线程现在都在死锁中永久等待,图 27.2 展示了死锁的图形表示。线程 1 已经获取了资源 A 的权限,并等待获取资源 B 的权限。线程 2 已经获取了资源 B 的权限,并等待获取资源 A 的权限。这个图形表示中,会看到一个循环,描绘了死锁。两个线程都将进入永久等待。

\myGraphic{0.5}{content/part4/chapter27/images/2.png}{图 27.2}

最好总是以相同的顺序获取,以避免这种死锁。还可以在程序中包含某些机制,来打破死锁。一种可能的解决方案是在一定时间内尝试获取资源的权限,如果在指定时间间隔内无法获取权限,线程可以停止等待,并可能释放它目前持有的其他权限。线程可以稍后尝试再次获取它需要的所有资源。这个机制给了其他线程获取必要权限,并继续执行的机会。这个机制是否有效取决于死锁的具体情况。

上一段描述了避免死锁的解决方案,这些解决方案已在标准库中通过 std::lock() 实现,该函数在本章稍后的“互斥”部分详细讨论。这个函数一次调用获取多个资源的权限,而不存在死锁的风险。应该使用 std::lock() ,而不是重新实现同样的解决方案。然而,最好从一开始就避免进入这样的情况,通过避免一次性获取多个锁。理想情况下,尝试避免使用锁。

\mySubsubsection{27.1.4.}{伪共享}

大多数缓存器都采用缓存行工作。对于现代 CPU,缓存行通常是 64 字节。如果需要写入一个缓存行,整个缓存行都需要被锁定。如果数据结构设计不当,这可能会对多线程代码造成严重的性能惩罚。例如,两个线程使用两个不同的数据块,但这些数据块共享一个缓存行,那么当一个线程写入某些内容时,会阻塞另一个线程,因为整个缓存行都锁定了。图 27.3 图形化地展示了两个线程在共享一个缓存行的情况下,很明显地写入两个不同的内存块。

\myGraphic{0.5}{content/part4/chapter27/images/3.png}{图 27.3}

可以通过使用显式内存对齐来优化数据结构,确保多个线程工作的数据不共享缓存行。为了以可移植的方式做到这一点,可以使用一个名为 hardware\_destructive\_interference\_size 的常量,该常量在 <new> 中定义,返回两个同时访问的对象之间的最小推荐偏移量,以避免缓存行共享。可以将这个值与 alignas 关键字结合使用,以正确对齐数据。







