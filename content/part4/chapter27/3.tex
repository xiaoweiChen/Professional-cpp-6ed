
原子类型允许原子访问,可以并发地读写,而不需要额外的同步。如果没有原子操作,增加一个变量会破坏线程安全性。编译器首先从内存加载值到寄存器,然后增加它,最后将结果存储回内存。另一个线程可能在增加操作期间访问相同的内存,这会导致数据竞争。例如,以下代码不是线程安全的,并且包含数据竞争。

\begin{cpp}
int counter { 0 }; // Global variable
...
++counter; // Executed in multiple threads
\end{cpp}

可以使用定义在<atomic>中的std::atomic类模板,以不显式使用同步机制的方式来使这段代码线程安全。这里是使用的代码与原子整数相同:

\begin{cpp}
atomic<int> counter { 0 } ; // Global variable
...
++counter; // Executed in multiple threads
\end{cpp}

<atomic>还定义了所有原始类型命名的整数原子类型。以下表格列出了其中的一部分:

% Please add the following required packages to your document preamble:
% \usepackage{longtable}
% Note: It may be necessary to compile the document several times to get a multi-page table to line up properly
\begin{longtable}{|l|l|}
\hline
\textbf{命名的原子类型} & \textbf{等效的std::atomic类型}              \\ \hline
\endfirsthead
%
\endhead
%
atomic\_bool               & atomic\textless{}bool\textgreater{}               \\ \hline
atomic\_char               & atomic\textless{}char\textgreater{}               \\ \hline
atomic\_uchar              & atomic\textless{}unsigned char\textgreater{}      \\ \hline
atomic\_int                & atomic\textless{}int\textgreater{}                \\ \hline
atomic\_uint               & atomic\textless{}unsigned int\textgreater{}       \\ \hline
atomic\_long               & atomic\textless{}long\textgreater{}               \\ \hline
atomic\_ulong              & atomic\textless{}unsigned long\textgreater{}      \\ \hline
atomic\_llong              & atomic\textless{}long long\textgreater{}          \\ \hline
atomic\_ullong             & atomic\textless{}unsigned long long\textgreater{} \\ \hline
atomic\_wchar\_t           & atomic\textless{}wchar\_t\textgreater{}           \\ \hline
atomic\_flag               & (none)                                            \\ \hline
\end{longtable}

可以使用原子类型,而不显式使用任何同步机制。在底层,对特定类型的原子操作可能使用同步机制,例如互斥锁。当针对的硬件缺乏执行操作的原子指令,可以使用原子类型的is\_lock\_free()成员函数来查询它是否支持无锁操作,即所有操作是否在没有显式同步机制的底层运行。还有一个静态常量称为atomic<T>::is\_always\_lock\_free,如果atomic<T>总是无锁的,则为true,否则为false。

std::atomic类模板可以与所有类型的数据一起使用,不仅限于整数类型。例如,可以创建一个atomic<double>,或者一个atomic<MyType>,但只有当MyType是简单的复制类型时才可行。根据指定类型的尺寸,这些操作在底层可能需要显式的同步机制。在下面的例子中,Foo和Bar都是简单的复制类型,即std::is\_trivially\_copyable\_v对于两者都为true。然而,由于Foo的大小,atomic<Foo>不是无锁的,而atomic<Bar>是有锁。

\begin{cpp}
struct Foo { int m_array[123]; };
struct Bar { int m_int; };

int main()
{
    atomic<Foo> f;
    println("{} {}", is_trivially_copyable_v<Foo>, f.is_lock_free()); // true false

    atomic<Bar> b;
    println("{} {}", is_trivially_copyable_v<Bar>, b.is_lock_free()); // true true
}
\end{cpp}

当从多个线程访问数据时,原子类型还能解决与内存序、编译器优化等方面的问题。基本上,如果没有使用原子类型或显式同步机制,不可能安全地从多个线程读写同一块数据!

\begin{myNotic}{NOTE}
内存序是指访问内存的顺序。在没有原子类型和其他同步机制的情况下,编译器和硬件允许对内存访问重新排序,只要这不会影响结果。这称为as-if规则,但在多线程环境中可能会导致问题。
\end{myNotic}

atomic\_flag是一个原子布尔类型,总是无锁的,这是C++标准保证的。与atomic<bool>的区别在于,不提供赋值运算符,而提供了一些成员函数,如clear()、test()和test\_and\_set()。使用atomic\_flag的示例将在本章后面的互斥部分中给出,用于实现一个自旋锁。

\mySubsubsection{27.3.1.}{Atomic Operations}

C++标准定义了一系列对atomic<T>的特殊操作,本节描述了其中的一些操作。要查看完整的列表,请参考标准库参考(见附录B)。

我们的第一个原子操作示例是:

\begin{cpp}
bool atomic<T>::compare_exchange_strong(T& expected, T desired);
\end{cpp}

这个操作原子地执行的逻辑如下,以伪代码表示:

\begin{cpp}
if (*this == expected) {
    *this = desired;
    return true;
} else {
    expected = *this;
    return false;
}
\end{cpp}

虽然这个逻辑在初次看到时可能相当奇怪,但这个操作是进行原子操作的关键构建块。下面是一个原子地乘以给定数字atomic<int>的示例:

\begin{cpp}
void atomicallyMultiply(atomic<int>& a, int n)
{
    int expected { a.load() };
    int desired { n * expected };
    while (!a.compare_exchange_strong(expected, desired)) {
        desired = n * expected;
    }
}

int main()
{
    atomic<int> value { 10 };
    println("Value = {}", value.load());
    atomicallyMultiply(value, 3);
    println("Result = {}", value.load());
}
\end{cpp}

第二个例子是atomic<T>::fetch\_add()。获取原子类型的当前值,将给定的增量加到原子值上,并返回原始的非增加值。下面是一个示例:

\begin{cpp}
atomic<int> value { 10 };
println("Value = {}", value.load());
int fetched { value.fetch_add(4) };
println("Fetched = {}", fetched);
println("Value = {}", value.load());
\end{cpp}

如果没有其他线程正在访问fetched和value变量的内容,输出如下:

\begin{shell}
Value = 10
Fetched = 10
Value = 14
\end{shell}

原子整数类型支持以下原子操作: fetch\_add(), fetch\_sub(), fetch\_and(), fetch\_or(), fetch\_xor(), ++, -{}-, +=, -=, \&=, \^{}= 和 |=。 原子指针类型支持fetch\_add(), fetch\_sub(), ++, --, +=, and -=。 原子浮点数类型支持fetch\_add() and fetch\_sub()。

大多数原子操作可以接受一个参数,用于指定内存序。例如:

\begin{cpp}
T atomic<T>::fetch_add(T value, memory_order = memory_order_seq_cst);
\end{cpp}

可以更改默认的memory\_order。C++标准提供了memory\_order\_relaxed、memory\_order\_consume、memory\_order\_acquire、memory\_order\_release、memory\_order\_acq\_rel和memory\_order\_seq\_cst,这些都在std命名空间中定义。然而,除非是这个领域的专家,否则很少会想使用非默认值。尽管根据某些指标,另一个内存序可能比默认值表现得更好,但如果以不正确的方式使用,仍然会引入数据竞争或其他难以调试的问题。如果确实想了解更多关于内存序的信息,请参阅附录B中的多线程参考资料。

\mySubsubsection{27.3.2.}{智能原子指针}

支持atomic<std::shared\_ptr<T>{}>。共享指针的控制块,用于存储引用计数等,是线程安全的,这保证了指向的对象删除的次数恰好是一次。然而,共享指针的其他内容不都是线程安全的。如果从多个线程同时使用同一个共享指针实例,就会在调用非常量成员函数时发生数据竞争,比如:调用reset()、assign()、swap()等。另一方面,当从多个线程使用同一个atomic<shared\_ptr<T>{}>实例时,即使是调用非常量共享指针成员函数也是线程安全的。需要注意的是,调用共享指针所指向对象的成员函数仍然不是线程安全的,并且需要手动同步。

\mySubsubsection{27.3.3.}{原子引用}

std::atomic\_ref基本上与std::atomic相同,甚至具有相同的接口,但它与引用一起工作,而atomic总是提供一个提供的值的副本。atomic\_ref实例本身的生命周期应该比它引用的对象短。atomic\_ref是可复制的,可以创建尽可能多的atomic\_ref实例,指向同一个对象。通过atomic\_ref实例进行的加载和存储都是原子性的,并且不会相互竞争。而没有通过atomic\_ref实例进行的并发加载和存储仍然可能会与这些原子访问竞争。atomic\_ref<T>类模板可以与简单的复制类型T一起使用,就像std::atomic一样。此外,标准库还提供了以下内容:

\begin{itemize}
\item
针对指针类型的偏特化,支持fetch\_add()和fetch\_sub()

\item
针对整数类型的全特化,支持fetch\_add()、fetch\_sub()、fetch\_and()、fetch\_or()和fetch\_xor()

\item
针对浮点数类型的全特化,支持fetch\_add()和fetch\_sub()
\end{itemize}

以下部分给出了如何使用atomic\_ref的一个示例。

\mySubsubsection{27.3.4.}{使用原子类型}

本节更详细地解释了为什么使用原子类型。假设有一个名为increment()的函数,在循环中递增一个整数引用参数。

\begin{cpp}
void increment(int& counter)
{
    for (int i { 0 }; i < 100; ++i) {
        ++counter;
        this_thread::sleep_for(1ms);
    }
}
\end{cpp}

现在,希望并行运行多个线程,每个线程都执行这个increment()函数,共享一个计数器变量。如果不使用原子类型或不使用线程同步,将会引入数据竞争。以下代码启动了10个increment()线程,然后通过调用每个线程的join()等待所有线程完成,并打印结果:

\begin{cpp}
int main()
{
    int counter { 0 };
    vector<jthread> threads;
    for (int i { 0 }; i < 10; ++i) {
        threads.emplace_back(increment, ref(counter));
    }
    for (auto& t : threads) { t.join(); }
    println("Result = {}", counter);
}
\end{cpp}

由于increment()将计数器参数递增100次,并且启动了10个线程,每个线程都执行increment()函数来共享同一个计数器,可能期望最终结果是1,000。如果多次执行这个程序,可能会得到以下输出,但值会有所不同:

\begin{shell}
Result = 982
Result = 977
Result = 984
\end{shell}

这段代码展示了数据竞争:多个线程并发地写入计数器,没有同步。这个例子中,可以使用原子类型来修复代码:

\begin{cpp}
void increment(atomic<int>& counter)
{
    for (int i { 0 }; i < 100; ++i) {
        ++counter;
        this_thread::sleep_for(1ms);
    }
}

int main()
{
    atomic<int> counter { 0 };
    vector<jthread> threads;
    for (int i { 0 }; i < 10; ++i) {
        threads.emplace_back(increment, ref(counter));
    }
    for (auto& t : threads) { t.join(); }
    println("Result = {}", counter);
}
\end{cpp}

唯一的修改是将共享计数器的类型从int更改为std::atomic<int>。当运行这个修改版本时,结果总会是1,000:

\begin{shell}
Result = 1000
Result = 1000
Result = 1000
\end{shell}

没有显式添加任何同步机制的情况下,现在代码是线程安全的,没有数据竞争,因为原子类型上的++counter操作是一个原子操作,加载、递增和存储值,这无法中断。

使用atomic\_ref,可以如下解决数据竞争:

\begin{cpp}
void increment(int& counter)
{
    atomic_ref<int> atomicCounter { counter };
    for (int i { 0 }; i < 100; ++i) {
        ++atomicCounter;
        this_thread::sleep_for(1ms);
    }
}

int main()
{
    int counter { 0 };
    vector<jthread> threads;
    for (int i { 0 }; i < 10; ++i) {
        threads.emplace_back(increment, ref(counter));
    }
    for (auto& t : threads) { t.join(); }
    println("Result = {}", counter);
}
\end{cpp}

然而,这两种修改后的实现都引入了一个新的问题:性能问题。应尽量减少同步量,无论是原子同步还是显式同步,因为这会降低性能。对于这个简单的例子,最佳和推荐的方法是让increment()在局部变量中计算其结果,并在循环结束后才将其添加到计数器引用中。即使这样,因为从多个线程写入计数器,仍然需要使用原子或atomic\_ref类型。

\begin{cpp}
void increment(atomic<int>& counter)
{
    int result { 0 };
    for (int i { 0 }; i < 100; ++i) {
        ++result;
        this_thread::sleep_for(1ms);
    }
    counter += result;
}
\end{cpp}

\mySubsubsection{27.3.5.}{等待原子变量}

对于std::atomic和atomic\_ref,以下是与等待相关的成员函数可用,以高效地等待修改原子变量:

% Please add the following required packages to your document preamble:
% \usepackage{longtable}
% Note: It may be necessary to compile the document several times to get a multi-page table to line up properly
\begin{longtable}{|l|l|}
\hline
\textbf{成员函数} & \textbf{描述}                                   \\ \hline
\endfirsthead
%
\endhead
%
wait(oldValue) &
\begin{tabular}[c]{@{}l@{}}阻塞线程,直到另一个线程调用notify\_one()或notify\_all(),且原子变量的值已\\更改,即不再等于oldValue。如果当前值已经不等于oldValue,则该函数根本不\\会阻塞。\end{tabular} \\ \hline
notify\_one()            & 唤醒因调用wait()而阻塞的一个线程。 \\ \hline
notify\_all()            & 唤醒因调用wait()而阻塞的所有线程。        \\ \hline
\end{longtable}

以下是示例

\begin{cpp}
atomic<int> value { 0 };

jthread job { [&value] {
    println("Thread starts waiting.");
    value.wait(0);
    println("Thread wakes up, value = {}", value.load());
} };

this_thread::sleep_for(2s);

println("Main thread is going to change value to 1.");
value = 1;
value.notify_all();
\end{cpp}

输出如下:

\begin{shell}
Thread starts waiting.
Main thread is going to change value to 1.
Thread wakes up, value = 1
\end{shell}




























