动态地创建和删除线程,可以创建一个线程池,需要线程时使用。这种技术通常用于处理某种事件的程序,理想的线程数等于处理核心数。如果线程数多于核心数,线程将挂起,以允许其他线程运行,这最终会增加开销。注意,虽然理想的线程数等于核心数,但这仅适用于线程是计算密集型,且不能因为其他原因(包括I/O)而阻塞的情况。当线程可以阻塞时,通常可以运行比核心数更多的线程。这种情况下确定最佳线程数是困难的,可能需要测量吞吐量。

由于处理并不完全相同,线程池中的线程通常会接收一个可调用对象作为其输入的一部分,该可调用对象代表要执行的计算。

由于线程池中的线程预先存在,因此操作系统调度线程池中的线程运行,比响应输入创建线程要高效得多。使用线程池可以管理创建的线程数量,可以根据平台创建一个或成千上万的线程。

有几个库可以实现线程池,包括Intel Threading Building Blocks (TBB)和Microsoft Parallel Patterns Library (PPL)等。建议使用这样的库来管理线程池,而不是自己编写实现。如果确实想自己实现线程池,可以以类似对象池的方式实现。第29章给出了一个对象池的示例实现。























