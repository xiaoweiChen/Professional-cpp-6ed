栅栏是一种可重用的线程协调机制,由一系列阶段组成。多个线程在栅栏点处阻塞。每次达到预定数量的线程到达栅栏时,都会执行一个阶段完成回调,释放所有阻塞的线程,重置线程计数器,并开始下一个阶段。每次阶段中,可以调整下一个阶段的预期线程数。栅栏非常适合在循环之间进行同步。例如,有多个线程以并发方式运行,并在循环中执行一些计算。当这些计算完成,需要在线程开始新迭代之前对结果进行处理。对于这样的场景,非常适合栅栏。当线程完成其工作时,会在栅栏处阻塞。当它们都到达栅栏时,阶段完成回调处理线程的结果,然后释放所有线程以开始它们的新迭代。

栅栏由模板类std::barrier实现,定义在<barrier>中。栅栏最重要的成员函数是arrive\_and\_wait(),递减计数器,然后阻塞线程,直到当前阶段完成。有关其他可用成员函数的完整描述,请参阅标准库参考。

以下代码片段演示了栅栏的使用,模拟的是一个使用机器人的生产环境。对于每次迭代,所有机器人都需要执行一些工作。当一个机器人完成时,会等待直到所有其他机器人都完成。当所有机器人都完成工作,就会准备下一个迭代,机器人会开始工作。

\begin{cpp}
constexpr unsigned numberOfRobots { 2 };
constexpr unsigned numberOfIterations { 3 };
unsigned iterationCount { 1 };
vector<jthread> robots;

auto completionCallback { [&] () noexcept {
    if (iterationCount == numberOfIterations) {
        println("Finished {} iterations, stopping robots.", numberOfIterations);
        for (auto& robot : robots) { robot.request_stop(); }
    } else {
        ++iterationCount;
        println("All robots finished. Preparing iteration {}.", iterationCount);
        this_thread::sleep_for(1s);
        println("Iteration {} ready to start. Waking up robots.", iterationCount);
    }
} };

barrier robotSynchronization { numberOfRobots, completionCallback };

auto robotThreadFunction { [&](stop_token token, string_view name) {
        println(" Thread for robot {} started.", name);
        while (!token.stop_requested()) {
            this_thread::sleep_for(1s);
            println(" {} finished.", name);
            robotSynchronization.arrive_and_wait();
        }
        println(" {} shutting down.", name);
} };

println("Preparing first iteration. Creating {} robot threads.", numberOfRobots);

for (unsigned i { 0 }; i < numberOfRobots; ++i) {
    robots.emplace_back(robotThreadFunction, format("Robot_{}", i));
}

for (auto& robot : robots) { robot.join(); }
println("Done with all work.");
\end{cpp}

输出为:

\begin{shell}
Preparing first iteration. Creating 2 robot threads.
    Thread for robot Robot_0 started.
    Thread for robot Robot_1 started.
    Robot_1 finished.
    Robot_0 finished.
All robots finished. Preparing iteration 2.
Iteration 2 ready to start. Waking up robots.
    Robot_1 finished.
    Robot_0 finished.
All robots finished. Preparing iteration 3.
Iteration 3 ready to start. Waking up robots.
    Robot_1 finished.
    Robot_0 finished.
Finished 3 iterations, stopping robots.
    Robot_0 shutting down.
    Robot_1 shutting down.
Done with all work.
\end{shell}

本章末尾的一个练习中,将改进这个机器人模拟,以便主线程启动所有机器人线程,等待所有机器人开始,准备第一次迭代,并指示所有等待的机器人开始工作。


















