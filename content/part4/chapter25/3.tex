This chapter introduced you to the concept of allocators that allow you to customize how memory is allocated and deallocated for containers. It also showed you how to write your own algorithms that can work with data from Standard Library containers. Finally, the major part of this chapter showed almost the complete development of a Standard Library–compliant directed\_graph container. Thanks to its iterator support, directed\_graph is compatible with Standard Library algorithms.

In the process of reading this chapter, you ideally gained an appreciation for the steps involved in developing algorithms and containers. Even if you never write another Standard Library algorithm or container, you understand better the Standard Library’s mentality and capabilities, and you can put it to better use.

This chapter concludes the tour of the C++ Standard Library. Even with all the details given in this book, some features are still omitted. If this material excited you, and you would like more information, consult some of the resources in Appendix B. Don’t feel compelled to use all the features discussed in these chapters. Forcing them into your programs without a true need will just complicate your code. However, I encourage you to consider incorporating aspects of the Standard Library into your programs where they make sense. Start with the containers, maybe throw in an algorithm or two, and before you know it, you’ll be a convert!