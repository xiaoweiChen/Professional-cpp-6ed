\noindent
\textbf{WHAT’S IN THIS CHAPTER?}

\begin{itemize}
\item
What allocators are

\item
How to write Standard Library–compliant custom algorithms, containers, and iterators
\end{itemize}

\noindent
\textbf{WILEY.COM DOWNLOADS FOR THIS CHAPTER}

Please note that all the code examples for this chapter are available as part of this chapter’s code download on the book’s website at \url{www.wiley.com/go/proc++6e} on the Download Code tab.

Chapters 16, “Overview of the C++ Standard Library,” 18, “Standard Library Containers,” and 20, “Mastering Standard Library Algorithms,” show that the Standard Library contains a powerful general-purpose collection of containers and algorithms. The information covered so far should be sufficient for most applications. However, those chapters show only the functionality of the library that is available out of the box. The Standard Library can be customized and extended however you like. For example, you can write your own Standard Library–compliant containers, algorithms, and iterators, compatible with existing Standard Library functionality. You can even specify your own memory allocation schemes for containers to use. This chapter provides a taste of these advanced features, primarily through the development of a find\_all() algorithm and a directed\_graph container.

\begin{myNotic}{NOTE}
Customizing and extending the Standard Library is rarely necessary. If you’re happy with the existing Standard Library containers and algorithms, you can skip this chapter. However, if you really want to understand the Standard Library, not just use it, give this chapter a chance. You should be comfortable with the operator-overloading material from Chapter 15, “Overloading C++ Operators,” and because this chapter uses templates extensively, you should also be comfortable with the template material from Chapter 12, “Writing Generic Code with Templates,” before continuing.
\end{myNotic}


