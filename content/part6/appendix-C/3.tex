以下头文件定义了标准库中可用的算法、迭代器、分配器,以及范围库的功能:

% Please add the following required packages to your document preamble:
% \usepackage{longtable}
% Note: It may be necessary to compile the document several times to get a multi-page table to line up properly
\begin{longtable}{|l|l|}
\hline
\textbf{头文件} &
\textbf{内容} \\ \hline
\endfirsthead
%
\endhead
%
\textless{}algorithm\textgreater{} &
\begin{tabular}[c]{@{}l@{}}标准库中大部分算法的原型,以及min(),max(),minmax(),和clamp()。\\参见第20章。\end{tabular} \\ \hline
\textless{}bit\textgreater{} &
\begin{tabular}[c]{@{}l@{}}定义了端序类枚举,参见第34章,并提供了在位序列上执行低级操作的\\函数原型,如bit\_ceil(),rotl(),countl\_zero()等,参见第16章。\end{tabular} \\ \hline
\textless{}execution\textgreater{} &
\begin{tabular}[c]{@{}l@{}}定义了用于与标准库算法一起使用的执行策略类型。参见第20章。\end{tabular} \\ \hline
\textless{}functional\textgreater{} &
\begin{tabular}[c]{@{}l@{}}定义了内置函数对象,否定器,绑定器,和适配器。参见第19章。\end{tabular} \\ \hline
\textless{}iterator\textgreater{} &
\begin{tabular}[c]{@{}l@{}}iterator\_traits,迭代器标签,iterator,reverse\_iterator,插入迭代器(如\\back\_insert\_iterator)和流迭代器的定义。参见第17章。\end{tabular} \\ \hline
\textless{}memory\textgreater{} &
\begin{tabular}[c]{@{}l@{}}定义了默认分配器和处理容器内未初始化内存的函数原型。还提供了\\unique\_ptr,shared\_ptr,weak\_ptr,make\_unique(),和make\_shared(),在\\第7章中介绍。\end{tabular} \\ \hline
\textless{}memory\_resource\textgreater{} &
\begin{tabular}[c]{@{}l@{}}定义了多态分配器和内存资源。参见第25章。\end{tabular} \\ \hline
\textless{}numeric\textgreater{} &
\begin{tabular}[c]{@{}l@{}}一些数值算法的原型:accumulate(),inner\_product(),partial\_sum(),\\adjacent\_difference(),gcd(),lcm()等。参见第20章。\end{tabular} \\ \hline
\textless{}ranges\textgreater{} &
提供了范围库的所有功能。参见第17章。 \\ \hline
\textless{}scoped\_allocator\textgreater{} &
\begin{tabular}[c]{@{}l@{}}一个可以与嵌套容器一起使用的分配器,如vector<string>或vector<map>。\end{tabular} \\ \hline
\end{longtable}























