标准库在多个不同的头文件中包含了一些通用的工具:


% Please add the following required packages to your document preamble:
% \usepackage{longtable}
% Note: It may be necessary to compile the document several times to get a multi-page table to line up properly
\begin{longtable}{|l|l|}
\hline
\textbf{头文件} &
\textbf{内容} \\ \hline
\endfirsthead
%
\endhead
%
\textless{}any\textgreater{} &
\begin{tabular}[c]{@{}l@{}}定义了any类。\\参见第24章。 \end{tabular}\\ \hline
\textless{}charconv\textgreater{} &
\begin{tabular}[c]{@{}l@{}}定义了chars\_format枚举,from\_chars()和to\_chars()函数,\\及相关结构体。\\参见第2章。\end{tabular} \\ \hline
\textless{}chrono\textgreater{} &
\begin{tabular}[c]{@{}l@{}}定义了chrono库。\\参见第22章。 \end{tabular}\\ \hline
\textless{}codecvt\textgreater{} &
\begin{tabular}[c]{@{}l@{}}为各种字符编码提供代码转换组件,此头文件自C++17起废弃。\end{tabular} \\ \hline
\textless{}compare\textgreater{} &
\begin{tabular}[c]{@{}l@{}}提供了支持三向比较的类和函数。\\参见第1章和第9章。\end{tabular} \\ \hline
\textless{}concepts\textgreater{} &
\begin{tabular}[c]{@{}l@{}}提供了如same\_as,convertible\_to,integral,movable等标准\\概念。\\参见第12章。\end{tabular} \\ \hline
\textless{}expected\textgreater (C++23) &
\begin{tabular}[c]{@{}l@{}}定义了expected和unexpected类模板,\\bad\_expected\_access异常,及unexpect\_t和\\unexpect标记。\\参见第24章。\end{tabular} \\ \hline
\textless{}filesystem\textgreater{} &
\begin{tabular}[c]{@{}l@{}}定义了所有可用于操作文件系统的类和函数。\\参见第13章。\end{tabular} \\ \hline
\textless{}format\textgreater{} &
\begin{tabular}[c]{@{}l@{}}为格式库提供了所有功能,如format(),format\_to()等。\\参见第2章。\end{tabular} \\ \hline
\textless{}initializer\_list\textgreater{} &
\begin{tabular}[c]{@{}l@{}}定义了initializer\_list类模板。\\参见第1章。\end{tabular} \\ \hline
\textless{}limits\textgreater{} &
\begin{tabular}[c]{@{}l@{}}定义了numeric\_limits类模板,及大多数内置类型的特化。\\参见第1章。\end{tabular} \\ \hline
\textless{}locale\textgreater{} &
\begin{tabular}[c]{@{}l@{}}定义了locale类,use\_facet()和has\_facet()函数\\模板,及各种组件族。\\参见第21章。\end{tabular} \\ \hline
\textless{}new\textgreater{} &
\begin{tabular}[c]{@{}l@{}}定义了bad\_alloc异常和set\_new\_handler()函数。此\\头文件还定义了所有六种形式的operator new和operator\\delete的原型。\\参见第15章。\end{tabular} \\ \hline
\textless{}optional\textgreater{} &
\begin{tabular}[c]{@{}l@{}}定义了optional类模板。\\参见第1章。\end{tabular} \\ \hline
\textless{}print\textgreater (C++23) &
\begin{tabular}[c]{@{}l@{}}定义了print(),println(),vprint\_unicode()和\\vprint\_nonunicode()函数。\\参见第1和2章。\end{tabular} \\ \hline
\textless{}random\textgreater{} &
\begin{tabular}[c]{@{}l@{}}定义了随机数生成库。\\参见第23章。\end{tabular} \\ \hline
\textless{}ratio\textgreater{} &
\begin{tabular}[c]{@{}l@{}}定义了ratio库,用于处理编译时有理数。\\参见第22章。\end{tabular} \\ \hline
\textless{}regex\textgreater{} &
\begin{tabular}[c]{@{}l@{}}定义了正则表达式库。\\参见第21章。\end{tabular} \\ \hline
\textless{}source\_location\textgreater{} &
\begin{tabular}[c]{@{}l@{}}提供了source\_location类。\\参见第14章。\end{tabular} \\ \hline
\textless{}stacktrace\textgreater (C++23) &
\begin{tabular}[c]{@{}l@{}}提供了stacktrace类。\\参见第14章。\end{tabular} \\ \hline
\textless{}string\textgreater{} &
\begin{tabular}[c]{@{}l@{}}定义了basic\_string类模板和string和wstring类型别名。\\参见第2章。\end{tabular} \\ \hline
\textless{}string\_view\textgreater{} &
\begin{tabular}[c]{@{}l@{}}定义了basic\_string\_view类模板和string\_view和\\wstring\_view类型别名。\\参见第2章。\end{tabular} \\ \hline
\textless{}system\_error\textgreater{} &
定义了错误类别和错误码。 \\ \hline
\textless{}tuple\textgreater{} &
\begin{tabular}[c]{@{}l@{}}定义了tuple类模板,作为pair类模板的一般化。\\参见第24章。\end{tabular} \\ \hline
\textless{}type\_traits\textgreater{} &
\begin{tabular}[c]{@{}l@{}}定义了用于模板元编程的类型特征。\\参见第26章。\end{tabular} \\ \hline
\textless{}typeindex\textgreater{} &
\begin{tabular}[c]{@{}l@{}}定义了type\_info的简单包装,可用于关联容器中的索引类型。\end{tabular} \\ \hline
\textless{}typeinfo\textgreater{} &
\begin{tabular}[c]{@{}l@{}}定义了bad\_cast和bad\_typeid异常。定义了type\_info类,\\其对象由typeid运算符返回。\\有关typeid的详细信息,请参见第10章。\end{tabular} \\ \hline
\textless{}utility\textgreater{} &
\begin{tabular}[c]{@{}l@{}}定义了pair类模板和make\_pair()(参见第1章)。此头\\文件还定义了如swap(),exchange(),move(),as\_const()\\等工具函数。\end{tabular} \\ \hline
\textless{}variant\textgreater{} &
\begin{tabular}[c]{@{}l@{}}定义了variant类模板。\\参见第24章。\end{tabular} \\ \hline
\textless{}version\textgreater{} &
\begin{tabular}[c]{@{}l@{}}提供了关于您正在使用的C++标准库的实现依赖信息,并公开了所有\\标准库功能测试宏。\\参见第16章。\end{tabular} \\ \hline
\end{longtable}


