C++ provides some facilities for numeric processing. These capabilities are not described in detail in this book; for details, consult one of the Standard Library references listed in Appendix B, “Annotated Bibliography”:

% Please add the following required packages to your document preamble:
% \usepackage{longtable}
% Note: It may be necessary to compile the document several times to get a multi-page table to line up properly
\begin{longtable}{|l|l|}
\hline
\textbf{HEADER}                  & \textbf{CONTENTS}                                                          \\ \hline
\endfirsthead
%
\endhead
%
\textless{}complex\textgreater{} & Defines the complex class template for working with complex numbers.       \\ \hline
\textless{}numbers\textgreater{} & Provides several mathematical constants, such as pi, phi, log2e, and more. \\ \hline
\textless{}stdfloat\textgreater (C++23) &
\begin{tabular}[c]{@{}l@{}}Provides the float16\_t, float32\_t, float64\_t, float128\_t, and bfloat16\_t\\ fixed-width floating-point types. See Chapter 1.\end{tabular} \\ \hline
\textless{}valarray\textgreater{} &
\begin{tabular}[c]{@{}l@{}}Defines valarray and related classes and class templates for working with\\ mathematical vectors and matrices.\end{tabular} \\ \hline
\end{longtable}











