\begin{itemize}
\item
罗伯特·C·马丁(Robert C. Martin)。《敏捷软件开发:原则、模式与实践》(Agile Software Development, Principles, Patterns, and Practices)。培生(Pearson),2003年。ISBN:978-1292025940。

\hspace*{\fill}

面向“前线”的软件工程师编写,本书着重于技术——即帮助软件工程师有效管理,日益复杂的操作系统和应用程序的原则、模式和过程。

\hspace*{\fill}

\item
迈克·科恩(Mike Cohn)。《敏捷的成功:使用Scrum进行软件开发》(Succeeding with Agile: Software Development Using Scrum)。Addison-Wesley Professional,2009年。ISBN:0-321-57936-4。

\hspace*{\fill}

学习Scrum方法的极佳指南。

\hspace*{\fill}

\item
大卫·托马斯(David Thomas)和安德鲁·亨特(Andrew Hunt),《实用程序员:通往精通之路》(The Pragmatic Programmer, Your Journey to Mastery),第二版。Addison-Wesley Professional,2019年。ISBN:978-0135957059。

\hspace*{\fill}

一本经典著作的新版本,每位软件工程师必读。二十年后,第一版的建议仍然非常贴切,探讨了核心流程——作为一个个体和一个团队,如果想创建易于操作且对用户友好的软件,应该怎么做。

\hspace*{\fill}

\item
巴里·W·博厄姆(Barry W. Boehm),TRW防御系统集团(TRW Defense Systems Group)。《软件开发和增强的螺旋模型》(A Spiral Model of Software Development and Enhancement)。IEEE Computer,21(5): 61–72,1988年。

\hspace*{\fill}

一篇具有里程碑意义的论文,描述了当时软件开发的状态,并提出了螺旋模型。

\hspace*{\fill}

\item
肯特·贝克(Kent Beck)和辛西娅·安德雷斯(Cynthia Andres)。《解释极端编程:拥抱变化》,第二版。Addison-Wesley Professional,2004年。ISBN:0-321-27865-8。

\hspace*{\fill}

一系列推广极限编程作为软件开发新途径的书籍。

\hspace*{\fill}

\item
罗伯特·T·富特尔(Robert T. Futrell)、唐纳德·F·夏弗(Donald F. Shafer)和琳达·伊莎贝尔·夏弗(Linda Isabell Shafer)。《质量软件项目管理》(Quality Software Project Management)。普伦蒂斯·霍尔(Prentice Hall),2002年。ISBN:0-130-91297-2。

\hspace*{\fill}

负责软件开发过程管理的人都可以参考的指南。

\hspace*{\fill}

\item
罗伯特·L·格拉斯(Robert L. Glass)。《软件工程的事实与谬误》(Facts and Fallacies of Software Engineering)。Addison-Wesley Professional,2002年。ISBN:0-321-11742-5。

\hspace*{\fill}

讨论了软件开发过程的各个方面,并揭露了隐藏的陈词滥调。

\hspace*{\fill}

\item
菲利普·克鲁钦(Philippe Kruchten)。《理性的统一过程:简介》,第三版。Addison-Wesley Professional出版社,2003年。ISBN:0-321-19770-4。

\hspace*{\fill}

提供了RUP的概览,包括其使命和过程。

\hspace*{\fill}

\item
爱德华·尤登(Edward Yourdon)。《死亡行军》,第二版。普伦蒂斯·霍尔(Prentice Hall),2003年。ISBN:0-131-43635-X。

\hspace*{\fill}

一本关于软件开发的政治和现实的深刻启发性书籍。

\hspace*{\fill}

\item
维基百科贡献者。Scrum。维基百科,自由的百科全书,\url{en.wikipedia.org/wiki/Scrum_(software_development)}(访问日期为2023年8月15日)。

\hspace*{\fill}

关于Scrum方法的详细讨论。

\hspace*{\fill}

\item
敏捷软件开发宣言。\url{agilemanifesto.org}(访问日期为2023年8月15日)。

\hspace*{\fill}

完整的敏捷软件开发宣言。

\hspace*{\fill}

\item
维基百科贡献者。版本控制。维基百科,自由的百科全书,\url{en.wikipedia.org/wiki/Version_control}(访问日期为2023年8月15日)。

\hspace*{\fill}

解释了版本控制系统背后的概念以及可用的解决方案类型。
\end{itemize}