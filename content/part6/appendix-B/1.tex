

\mySubsubsection{B.1.1.}{Beginning C++ Without Previous Programming Experience}

\begin{itemize}
\item
Ivor Horton and Peter Van Weert. Beginning C++23: From Beginner to Pro, 7th ed. Apress, 2023. ISBN: 978-1484293423.

This book starts with the basics and progresses through step-by-step examples to become a proficient C++ programmer. This edition includes new features from C++23. There is no assumption of prior programming knowledge.

\item
Bjarne Stroustrup. Programming: Principles and Practice Using C++, 2nd ed. AddisonWesley Professional, 2014. ISBN: 0-321-99278-4.

An introduction to programming in C++ by the inventor of the language. This book assumes no previous programming experience, but even so, it is also a good read for experienced programmers.

\item
Steve Oualline. Practical C++ Programming, 2nd ed. O’Reilly Media, 2003. ISBN: 0-596-00419-2.

An introductory C++ text that assumes no prior programming experience.

\item
Walter Savitch. Problem Solving with C++, 10th ed. Pearson, 2017. ISBN: 978- 0134448282.

Assumes no prior programming experience. This book is often used as a textbook in introductory programming courses.
\end{itemize}

\mySubsubsection{B.1.2.}{Beginning C++ with Previous Programming Experience}

\begin{itemize}
\item
Bjarne Stroustrup. A Tour of C++, 3rd ed. Addison-Wesley Professional, 2022. ISBN: 978- 0136816485.

A quick (about 320 pages) tutorial-based overview of the entire C++ language and Standard Library at a moderately high level for people who already know C++ or are at least experienced programmers. This book includes C++20 features.

\item
Stanley B. Lippman, Josée Lajoie, and Barbara E. Moo. C++ Primer, 5th ed. Addison-Wesley Professional, 2012. ISBN: 0-321-71411-3.

A thorough introduction to C++ that covers just about everything in the language in an accessible format and in great detail.

\item
Andrew Koenig and Barbara E. Moo. Accelerated C++: Practical Programming by Example. Addison-Wesley Professional, 2000. ISBN: 0-201-70353-X.

Covers the same material as C++ Primer, but in much less space, because it assumes that the reader has programmed in another language before.

\item
Bruce Eckel. Thinking in C++, Volume 1: Introduction to Standard C++, 2nd ed. Prentice Hall, 2000. ISBN: 0-139-79809-9.

An excellent introduction to C++ programming that expects the reader to know C already.
\end{itemize}

\mySubsubsection{B.1.3.}{General C++}

\begin{itemize}
\item
The C++ Programming Language. isocpp.org (accessed August 15, 2023).

The home of Standard C++ on the web, containing news, status, and discussions about the C++ standard on all compilers and platforms.

\item
The C++ Super-FAQ. isocpp.org/fag (accessed August 15, 2023).

A huge collection of frequently asked questions about C++.

\item
Klaus Iglberger. C++ Software Design, O’Reilly Media, Inc, 2022. ISBN: 9781098113162.

Excellent book on good software design. With this book, experienced C++ developers will get a thorough, practical, and unparalleled overview of software design.

\item
Marius Bancila. Modern C++ Programming Cookbook, 2nd ed. Packt, 2020. ISBN: 9781800208988.

This book is organized in the form of practical recipes covering a wide range of problems faced by C++ developers. This 2nd edition comes with 30 new or updated recipes for C++20.

\item
Paul Deitel, Harvey Deitel. C++20 for Programmers, 3rd ed. O’Reilly, 2020. ISBN: 9780136905776.

This book is an introductory-through-intermediate-level tutorial-based presentation of programming in the C++ programming language, covers C++20.

\item
Scott Meyers. Effective Modern C++: 42 Specific Ways to Improve Your Use of C++11 and C++14. O’Reilly, 2014. ISBN: 1-491-90399-6.

\item
Scott Meyers. Effective C++: 55 Specific Ways to Improve Your Programs and Designs, 3rd ed. Addison-Wesley Professional, 2005. ISBN: 0-321-33487-6.

\item
Scott Meyers. More Effective C++: 35 New Ways to Improve Your Programs and Designs. Addison-Wesley Professional, 1996. ISBN: 0-201-63371-X.

Three books that provide excellent tips and tricks on commonly misused and misunderstood features of C++.

\item
Bjarne Stroustrup. The C++ Programming Language, 4th ed. Addison-Wesley Professional, 2013. ISBN: 0-321-56384-0.

The “bible” of C++ books, written by the designer of C++. Every C++ programmer should own a copy of this book, although it can be a bit obscure in places for the C++ novice.

\item
Herb Sutter. Exceptional C++: 47 Engineering Puzzles, Programming Problems, and Solutions. Addison-Wesley Professional, 1999. ISBN: 0-201-61562-2.

Presented as a set of puzzles, with one of the best, most thorough discussions of proper resource management and exception safety in C++ through resource acquisition is initialization (RAII). This book also includes in-depth coverage of a variety of topics, such as the pimpl idiom, name lookup, good class design, and the C++ memory model.

\item
Herb Sutter. More Exceptional C++: 40 New Engineering Puzzles, Programming Problems, and Solutions. Addison-Wesley Professional, 2001. ISBN: 0-201-70434-X.

Covers additional exception safety topics not covered in Exceptional C++: 47 Engineering Puzzles, Programming Problems, and Solutions. This book also discusses effective objectoriented programming and correct use of certain aspects of the Standard Library.

\item
Herb Sutter. Exceptional C++ Style: 40 New Engineering Puzzles, Programming Problems, and Solutions. Addison-Wesley Professional, 2004. ISBN: 0-201-76042-8.

Discusses generic programming, optimization, and resource management. This book also has an excellent exposition of how to write modular code in C++ by using nonmember functions and the single responsibility principle.

\item
Stephen C. Dewhurst. C++ Gotchas: Avoiding Common Problems in Coding and Design. Addison-Wesley Professional, 2002. ISBN: 0-321-12518-5.

Provides 99 specific tips for C++ programming.

\item
Bruce Eckel and Chuck Allison. Thinking in C++, Volume 2: Practical Programming. Prentice Hall, 2003. ISBN: 0-130-35313-2.

The second volume of Eckel’s book, which covers more advanced C++ topics.

\item
Ray Lischner. C++ in a Nutshell. O’Reilly, 2003. ISBN: 0-596-00298-X.

A C++ reference covering everything from the basics to more-advanced material.

\item
Stephen Prata, C++ Primer Plus, 6th ed. Addison-Wesley Professional, 2011. ISBN: 0-321-77640-2.

One of the most comprehensive C++ books available.

\item
The C++ Reference. cppreference.com (accessed August 15, 2023).

An excellent reference of C++98, C++03, C++11, C++14, C++17, C++20, and C++23.

\item
The C++ Resources Network. cplusplus.com (accessed August 15, 2023).

A website containing a lot of information related to C++, with a complete reference of the language, including C++23.
\end{itemize}

\mySubsubsection{B.1.4.}{I/O Streams and Strings}

\begin{itemize}
\item
Cameron Hughes and Tracey Hughes. Stream Manipulators and Iterators in C++. \url{www.informit.com/articles/article.aspx?p=171014} (accessed August 15, 2023).

A well-written article that takes the mystery out of defining custom stream manipulators in C++.

\item
Philip Romanik and Amy Muntz. Applied C++: Practical Techniques for Building Better Software. Addison-Wesley Professional, 2003. ISBN: 0-321-10894-9.

A unique blend of software development advice and C++ specifics, as well as a very good explanation of locale and Unicode support in C++.

\item
Joel Spolsky. The Absolute Minimum Every Software Developer Absolutely, Positively Must Know About Unicode and Character Sets (No Excuses!). \url{www.joelonsoftware.com/articles/Unicode.html} (accessed August 15, 2023).

A treatise by Joel Spolsky on the importance of localization. After reading this, you’ll want to check out the other entries on his Joel on Software website.

\item
The Unicode Consortium. The Unicode Standard 5.0, 5th ed. Addison-Wesley Professional, 2006. ISBN: 0-321-48091-0.

The definitive book on Unicode, which all developers using Unicode must have.

\item
Unicode, Inc. Where is my Character? \url{www.unicode.org/standard/where} (accessed August 15, 2023).

The best resource for finding Unicode characters, charts, and tables.

\item
Wikipedia contributors. Universal Coded Character Set. Wikipedia, The Free Encyclopedia, \url{en.wikipedia.org/wiki/Universal_Character_Set} (accessed August 15, 2023).

An explanation of what the Universal Character Set (UCS) is, including the Unicode standard.
\end{itemize}

\mySubsubsection{B.1.5.}{The C++ Standard Library}

\begin{itemize}
\item
Peter Van Weert and Marc Gregoire. C++17 Standard Library Quick Reference. Apress, 2019. ISBN: 978-1-4842-4922-2.

This quick reference is a condensed guide to all essential data structures, algorithms, and functions provided by the C++17 Standard Library.

\item
Rainer Grimm. The C++ Standard Library: What Every Professional C++ Programmer Should Know about the C++ Standard Library. Independently published, 2023. ISBN: 979- 8386658595.

The goal of this book is to provide a concise reference of the C++23 Standard Library in about 350 pages. This book assumes that you are familiar with C++.

\item
Nicolai M. Josuttis. The C++ Standard Library: A Tutorial and Reference, 2nd ed. AddisonWesley Professional, 2012. ISBN: 0-321-62321-5.

Covers the entire Standard Library, including I/O streams and strings as well as the containers and algorithms. This book is an excellent reference.

\item
Scott Meyers. Effective STL: 50 Specific Ways to Improve Your Use of the Standard Template Library. Addison-Wesley Professional, 2001. ISBN: 0-201-74962-9.

Written in the same spirit as the author’s Effective C++ books. This book provides targeted tips for using the Standard Library but is not a reference or tutorial.

\item
Stephan T. Lavavej. Standard Template Library (STL). \url{learn.microsoft.com/en-us/shows/c9-lectures-stephan-t-lavavej-standard-template-library-stl-} (accessed August 15, 2023).

An interesting video lecture series on the C++ Standard Library.

\item
David R. Musser, Gillmer J. Derge, and Atul Saini. STL Tutorial and Reference Guide: Programming with the Standard Template Library, 2nd ed. Addison-Wesley Professional, 2001. ISBN: 0-321-70212-3.

Similar to the Josuttis text, but covering only parts of the Standard Library, such as containers and algorithms.
\end{itemize}

\mySubsubsection{B.1.6.}{C++ Templates}

\begin{itemize}
\item
Herb Sutter. “Sutter’s Mill: Befriending Templates.” C/C++ User’s Journal. \url{www.drdobbs.com/befriending-templates/184403853} (accessed August 15, 2023).

An excellent explanation of making function templates friends of classes.

\item
David Vandevoorde, Nicolai M. Josuttis, and Douglas Gregor. C++ Templates: The Complete Guide, 2nd ed. Addison-Wesley Professional, 2017. ISBN: 0-321-71412-1.

Everything you ever wanted to know (or didn’t want to know) about C++ templates. This book assumes significant background in general C++.

\item
David Abrahams and Aleksey Gurtovoy. C++ Template Metaprogramming: Concepts, Tools, and Techniques from Boost and Beyond. Addison-Wesley Professional, 2004. ISBN: 0-321-22725-5.

Delivers practical metaprogramming tools and techniques into the hands of the everyday programmer.
\end{itemize}

\mySubsubsection{B.1.7.}{C++11/C++14/C++17/C++20/C++23}

\begin{itemize}
\item
C++ Standards Committee Papers. \url{www.open-std.org/jtc1/sc22/wg21/docs/papers} (accessed August 15, 2023).

A wealth of papers written by the C++ standards committee.

\item
Nicolai M. Josuttis. C++20 - The Complete Guide. NicoJosuttis, 2022. ISBN: 978- 3967300208.

A book explaining all C++20 features with a focus on how these features impact day-to-day programming, what effect combining features can have, and how you can benefit from them in practice.

\item
Wikipedia contributors. C++11. Wikipedia, The Free Encyclopedia, \url{en.wikipedia.org/wiki/C%2B%2B11} (accessed August 15, 2023).

\item
Wikipedia contributors. C++14. Wikipedia, The Free Encyclopedia, \url{en.wikipedia.org/wiki/C%2B%2B14} (accessed August 15, 2023).

\item
Wikipedia contributors. C++17. Wikipedia, The Free Encyclopedia, \url{en.wikipedia.org/wiki/C%2B%2B17} (accessed August 15, 2023).

\item
Wikipedia contributors. C++20. Wikipedia, The Free Encyclopedia, \url{en.wikipedia.org/wiki/C%2B%2B20} (accessed August 15, 2023).

\item
Wikipedia contributors. C++23. Wikipedia, The Free Encyclopedia, \url{en.wikipedia.org/wiki/C%2B%2B23} (accessed August 15, 2023).

Five Wikipedia articles with a description of new features added to C++11, C++14, C++17, C++20, and C++23.

\item
Scott Meyers. Presentation Materials: Overview of the New C++ (C++11/14). Artima, 2013. \url{www.artima.com/shop/overview_of_the_new_cpp} (accessed August 15, 2023).

A document containing the presentation materials from a Scott Meyers’ training course. This is an excellent reference to get a list of all C++11 and select C++14 features.

\item
ECMAScript 2017 Language Specification. \url{www.ecma-international.org/publications/files/ECMA-ST/ECMA-262.pdf} (accessed August 15, 2023).

One of the syntaxes of the regular expressions in C++ is the same as the regular expressions in the ECMAScript language, as described in this specification document.
\end{itemize}













