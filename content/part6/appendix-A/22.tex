
Potential employers value strong testing abilities. Because your résumé probably doesn’t indicate your testing skills, unless you have explicit quality assurance (QA) experience, you might face interview questions about testing.

\mySubsubsection{A.22.1.}{Things to Remember}

\begin{itemize}
\item
The difference between black-box and white-box testing

\item
The concept of unit testing, integration testing, system testing, and regression testing

\item
Techniques for higher-level tests

\item
Testing and QA environments in which you’ve worked before: what worked and what didn’t?
\end{itemize}

\mySubsubsection{A.22.2.}{Types of Questions}

An interviewer could ask you to write some tests during the interview, but it’s unlikely that a program presented during an interview would contain the depth necessary for interesting tests. It’s more likely that you will be asked high-level testing questions. Be prepared to describe how testing was done at your last job and what you liked and didn’t like about it. Again, be careful not to disclose any confidential information. After you’ve answered the interviewer’s questions about testing, a good question for you to ask the interviewer is to ask how testing is done at their company. It might start a conversation about testing and give you a better idea of the environment at your potential job.


