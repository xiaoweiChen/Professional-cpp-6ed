
Few programmers submit résumés that list only a single language or technology, and few large applications rely on only a single language or technology. Even if you’re only interviewing for a C++ position, the interviewer may still ask questions about other languages, especially as they relate to C++.

\mySubsubsection{A.26.1.}{Things to Remember}

\begin{itemize}
\item
The ways in which platforms can differ (architecture, integer sizes, and so on)

\item
The fact that you should try to find a cross-platform library to accomplish a certain task, instead of starting to implement the functionality yourself for different kinds of platforms

\item
The fact that C++ can interoperate with other languages, such as C\#, Java, scripting languages, and so on
\end{itemize}

\mySubsubsection{A.26.2.}{Types of Questions}

The most popular cross-language question is to compare and contrast two different languages. You should avoid saying only positive or negative things about a particular language, even if you really love or hate that language. The interviewer wants to know that you are able to see trade-offs and make decisions based on them.

Cross-platform questions are more likely to be asked while discussing previous work. If your résumé indicates that you once wrote C++ applications that ran on a custom hardware platform, you should be prepared to talk about the compiler you used and the challenges of that platform.






