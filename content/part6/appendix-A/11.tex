
作为C++中最晦涩的部分之一,模板是面试官用来区分C++新手和专业人士的好方法。虽然大多数面试官会原谅你记不住一些高级模板语法,但应该在面试前了解基础知识。

\mySubsubsection{A.11.1.}{需要注意的事项}

\begin{itemize}
\item
如何使用类或函数模板

\item
如何编写一个简单的类或函数模板

\item
函数模板的缩写语法

\item
函数模板参数推导

\item
类模板参数推导(CTAD)

\item
别名模板是什么,以及为什么使用它比typedef更好

\item
概念背后的理念,以及其基本用途

\item
什么是变长模板和折叠表达式

\item
模板元编程背后的理念

\item
类型特征,以及可以用于什么
\end{itemize}

\mySubsubsection{A.11.2.}{问题的类型}

许多面试问题始于一个简单的问题,然后逐渐增加复杂性。通常,面试官准备无限增加复杂性,他们只是想看看你能走多远。例如,面试官可能会从一个问题开始,要求你创建一个类,该类提供对固定数量int的顺序访问。然后,该类需要扩展以适应任意数量的元素。接着,它需要与任意数据类型一起工作,这就是模板发挥作用的地方。面试官可以沿着模板路径提出多种问题,比如要求你使用运算符重载提供数组样式的语法,或者继续沿着模板路径,要求为模板类型参数提供默认类型,或者对它们施加类型约束,但大多数面试官都理解模板语法可能很难,他们会原谅语法错误。

面试官可能会问你一些与模板元编程相关的高层次问题,以了解你是否听说过它。在解释时,可以给出一个小的示例,比如在编译时计算一个数的阶乘。不用担心语法是否完全正确,只要你解释它应该做什么就没问题。









