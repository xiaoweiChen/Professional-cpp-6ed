
面试中,你可能需要执行比简单重载运算符更复杂的事情。一些面试官喜欢准备一个高级问题,他们并不真的期望任何人能正确回答。运算符重载的复杂性使它成为一个很好的、几乎不可能的问题,因为很少有开发者能不查资料就正确地编写语法。所以这个领域需要在面试前复习好。

\mySubsubsection{A.14.1.}{需要注意的事项}

\begin{itemize}
\item
重载流运算符,因为它们是常见的重载运算符,在概念上独特

\item
什么是函数对象(可调用对象)以及如何创建一个

\item
使一个类的函数调用运算符静态的好处 (C++23)

\item
在成员函数运算符和全局函数之间进行选择

\item
某些运算符可以用其他运算符表示(例如,operator<=可以通过取operator>的结果的负数来编写)

\item
多维下标索引运算符及其用途 (C++23)

\item
可以定义自己的用户定义字面量,但没有语法细节
\end{itemize}

\mySubsubsection{A.14.2.}{问题的类型}

不可能预测你会得到的具体问题,但运算符的数量是有限的。只要你见过每个可以重载的运算符的重载示例,就能做得很好!

可能的问题是编写一个简单的类,例如一个用于存储数学分数的Fraction类。面试官可能会要求你添加一些运算符的支持,例如:加法和减法。如果不确定如何加减分数,可以询问面试官,因为这不是问题的重点;重点是编写重载运算符。

除了要求实现一个重载运算符之外,可能会被问到关于运算符重载的高层次问题。一个寻找错误的问题可能会包含一个运算符,重载以执行与该特定运算符概念上不正确的事情。除了语法之外,还要记住运算符重载的使用案例和理论。

