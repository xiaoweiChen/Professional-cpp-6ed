
在软件中生成好的随机数是一个复杂的话题,面试官也知道这一点。他们不会问语法细节,但需要知道<random>库背后的基本概念,这是C++标准库的一部分。

\mySubsubsection{A.17.1.}{需要注意的事项}

\begin{itemize}
\item
使用<random>库作为生成随机数的首选技术

\item
随机数引擎和分布如何共同生成随机数

\item
什么是种子,以及为什么很重要
\end{itemize}

\mySubsubsection{A.17.2.}{问题的类型}

面试官可能展示一段使用C函数rand()和srand()生成随机数的代码,并让你对代码进行评论。应该解释说那些C函数不再推荐使用,以及为什么使用<random>提供的功能更好。

涉及随机数的问题中,重要的是要解释真随机数和伪随机数之间的区别。如果解释说你可以使用random\_device为伪随机数生成器生成一个真正的随机种子,以及为什么总是不使用random\_device,而使用伪随机数生成器,会得到额外分数。

