

Strings are important and are used in almost every kind of application. An interviewer will most likely ask at least one question related to string handling in C++.

\mySubsubsection{A.2.1.}{Things to Remember}

\begin{itemize}
\item
The std::string and string\_view classes

\item
To prefer const string\& or string as function return type instead of string\_view

\item
Differences between the C++ std::string class and C-style (char*) strings, including why C-style strings should be avoided

\item
Conversion of strings to numeric types such as integers and floating-point numbers, and vice versa

\item
String formatting using std::format()

\item
String printing using std::print() and println() (C++23)

\item
How to format and print entire ranges at once (C++23)

\item
Raw string literals

\item
The importance of localization

\item
Ideas behind Unicode

\item
The high-level concepts of locales and facets

\item
What regular expressions are
\end{itemize}

\mySubsubsection{A.2.2.}{Types of Questions}

An interviewer could ask you to explain how you can append two strings together. With this question, the interviewer wants to find out whether you are thinking as a C++ programmer or as a C programmer. If you get such a question, you should explain the std::string class and show how to use it to append two strings. It’s also worth mentioning that the string class handles all memory management for you automatically, and contrasting this to C-style strings.

Most interviewers won’t ask specific details about localization. If you do receive a question about your experience with localization, be sure to mention the importance of considering worldwide use from the beginning of the project.

You might also be asked about the general idea behind locales and facets. Most likely, you will not have to explain the exact syntax, but you should explain that they allow you to format text and numbers according to the rules of a certain language or country.

You might get a question about Unicode, but almost certainly it will be a question to explain the ideas and the basic concepts behind Unicode instead of implementation details. So, make sure you understand the high-level concepts of Unicode and that you can explain their use in the context of localization. You should also know about the different options for encoding Unicode characters, such as UTF-8 and UTF-16, without specific details.

As discussed in Chapter 21, regular expressions can have a daunting syntax. It is unlikely that an interviewer will ask you about little details of regular expressions. However, you should be able to explain the concept of regular expressions and what kind of string manipulations you can do with them.





