
You can be sure that an interviewer will ask you some questions related to memory management, including your knowledge of smart pointers. Besides smart pointers, you will also get more low-level questions. The goal is to determine whether the object-oriented aspects of C++ have distanced you too much from the underlying implementation details. Memory management questions will give you a chance to prove that you know what’s really going on.

\mySubsubsection{A.7.1.}{Things to Remember}

\begin{itemize}
\item
Know how to draw the stack and the free store; this can help you understand what’s going on.

\item
Avoid using low-level memory allocation and deallocation functions. In modern C++, there should be no calls to new, delete, new[], delete[], malloc(), free(), and so on. Instead, use smart pointers.

\item
Understand smart pointers; use std::unique\_ptr by default, shared\_ptr for shared ownership.

\item
Use std::make\_unique() to create a unique\_ptr.

\item
Use std::make\_shared() to create a shared\_ptr.

\item
Never use auto\_ptr; it has been removed since C++17.

\item
If you do need to use low-level memory allocation functions, use new, delete, new[], and delete[], never malloc() and free().

\item
If you have an array of pointers to objects, you still need to allocate memory for each individual pointer and delete the memory—the array allocation syntax doesn’t take care of pointers.

\item
Be aware of the existence of memory allocation problem detectors, such as Valgrind, to expose memory problems.
\end{itemize}

\mySubsubsection{A.7.2.}{Types of Questions}

Find-the-bug questions often contain memory issues, such as double deletion, new/delete/new[]/ delete[] mix-up, and memory leaks. When you are tracing through code that makes heavy use of pointers and arrays, you should draw and update the state of the memory as you process each line of code.

Another good way to find out if a candidate understands memory is to ask how pointers and arrays differ. At this point, the question might catch you off guard for a moment. If that’s the case, skim Chapter 7 again for the discussion on pointers and arrays.

When answering questions about memory allocation, it’s always a good idea to mention the concept of smart pointers and their benefits for automatically cleaning up memory and other resources. You should also mention that it’s much better to use Standard Library containers, such as std::vector, instead of C-style arrays, because the Standard Library containers handle memory management for you automatically.







