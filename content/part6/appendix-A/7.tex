
你可以肯定面试官会问你一些与内存管理相关的问题,包括对智能指针的了解。除了智能指针,还会得到更多底层的问题。目标是为了确定C++的面向对象方面是否让你与底层实现细节太过疏远。内存管理问题将证明知道真正发生什么的机会。

\mySubsubsection{A.7.1.}{需要注意的事项}

\begin{itemize}
\item
了解如何绘制栈和堆区;这有助于理解正在发生的事情。

\item
避免使用低层的内存分配和释放函数。现代C++中,不应该有对new、delete、new[]、delete[]、malloc()、free()等的调用。相反,使用智能指针。

\item
理解智能指针;默认使用std::unique\_ptr,shared\_ptr用于共享所有权。

\item
使用std::make\_unique()创建一个unique\_ptr。

\item
使用std::make\_shared()创建一个shared\_ptr。

\item
永远不要使用auto\_ptr;自C++17起已被移除。

\item
如果确实需要使用低层的内存分配函数,使用new、delete、new[]和delete[],而不是malloc()和free()。

\item
如果有一个指向对象的指针数组,仍然需要为每个单独的指针分配内存并释放内存——数组分配语法不会处理指针。

\item
要了解内存分配问题检测器的存在,如Valgrind,以暴露内存问题。
\end{itemize}

\mySubsubsection{A.7.2.}{问题的类型}

寻找错误的问题经常包含内存问题,如重复删除、new/delete/new[]/delete[]的混淆和内存泄漏。当跟踪使用大量指针和数组的代码时,应该绘制并更新处理每一行代码时的内存状态。

了解候选人是否理解内存的另一个好方法是,询问指针和数组之间的区别,问题可能会让你措手不及。如果是这样,请再次快速浏览第7章关于指针和数组的讨论。

在回答关于内存分配的问题时,总是提到智能指针的概念,及其自动清理内存和其他资源的好处是个好主意。还应该提到,使用标准库容器,如std::vector,而不是C风格数组,因为标准库容器为你自动处理内存管理。







