
UML 2支持四种类型的交互图表:序列图、通信图、时间图和交互概览图。本书仅使用序列图,接下来的部分将对其进行简要讨论。

\mySubsubsection{D.3.1.}{序列图}

序列图以图形方式表示不同对象之间发送的消息,以及发送的顺序。一个序列图由以下组件组成:

\begin{itemize}
\item
对象:参与交互的对象实例。

\item
生命线:图形化表示对象的生命周期。

\item
消息:一个对象向另一个对象发送的消息。

\item
回复:当一个对象从另一个对象接收消息时,会发送回复。

\item
自我消息:一个对象发送给自己的消息。

\item
替代:表示替代流程,类似于if-then-else语句中的分支。
\end{itemize}

图 D.9 显示了一个序列图的例子。这是第4章中图表的简化版,但这次带有标签,指示图表重要部分的意义。

\myGraphic{0.9}{content/part6/appendix-D/images/9.png}{图 D.9}








