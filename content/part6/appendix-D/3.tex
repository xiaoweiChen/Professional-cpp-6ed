
UML 2 supports four types of interaction diagrams: sequence, communication, timing, and interaction overview diagrams. This book uses only sequence diagrams, briefly discussed in the following section.

\mySubsubsection{D.3.1.}{Sequence Diagrams}

A sequence diagram graphically represents which messages are sent between different objects and the order in which these are sent. A sequence diagram consists of the following components:

\begin{itemize}
\item
Objects: Object instances involved in the interactions.

\item
Lifelines: Graphically represent the lifetime of objects.

\item
Messages: Messages are sent from one object to another object.

\item
Replies: When an object receives a message from another object, it sends a reply.

\item
Self-messages: Messages an object sends to itself.

\item
Alternatives: Represent alternative flows, similar to the branching in an if-then-else statement.
\end{itemize}

Figure D.9 shows an example of a sequence diagram. It’s a simplified version of the diagram from Chapter 4, “Designing Professional C++ Programs,” but this time with labels indicating the meaning of the important parts of the diagram.

\myGraphic{0.5}{content/part6/appendix-D/images/9.png}{FIGURE D.9}








